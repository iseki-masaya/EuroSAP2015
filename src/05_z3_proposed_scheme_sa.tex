\begin{proof}
 The proof proceeds in a sequence of games. \vspace{10pt}\\
 {\bfseries Game 0.} This game equals the \textit{server authentication} experiment in Def.~\ref{def:rsacce-sa}.\\
 \begin{equation}
  \Adv_0 = \Adv^{\rsaccesa}_{P}(A)
 \end{equation}%
%
%
 \textbf{Game 1.} We try to guess which client oracle will be the first oracle to break \textit{server authentication}. If our guess is wrong, i.e. if there is another client oracle that breaks \textit{server authentication} before, then we abort the game.

 Technically, the game is identical to Game 0, except for the following. The challenger guesses three random indexes $(c^{\ast}, i^{\ast}, \ell^{\ast}) \xleftarrow{\$} [\nclient] \times [\noracle] \times [n_{\ell}]$. If there exists a client oracle $\pi^c_{i,\ell}$ that breaks server authentication, and $(c, i, \ell) \neq c^{\ast}, i^{\ast}, \ell^{\ast})$, then the challenger aborts the game. Note that if the first oracle $\pi^c_{i,\ell}$ that breaks server authentication, then with probability $1/(\nclient \noracle n_{\ell})$ we have $(c,i,\ell) = (c^{\ast}, i^{\ast}, \ell^{\ast})$, and thus
 \begin{equation}
  \Adv_0 \leq \nclient \noracle n_{\ell} \Adv_1.
 \end{equation}%
 Note that in this game the attacker can only break the security of the protocol, if oracle $\pi^{c^{\ast}}_{i^{\ast},\ell^{\ast}}$ is the first oracle that breaks server authentication, as otherwise the game is aborted.
\vspace{10pt}\\%
%
%
 \textbf{Game 2.} Again the challenger proceeds as before, but we add an abort rule. We want to make sure that $\pi^{c^\ast}_{i^{\ast},0}$ receives as input exactly the server's DH public key $T_s$ in $\SCFG_{pub}$ that was selected by some other uncorrupted server oracle.

 Technically, we abort and raise event $\abort_\SIG$, if oracle $\pi^{c^{\ast}}_{i^{\ast},0}$ ever receives as input a message $\cert_s$ indicating the intended partner $\peer = s$ and server config $\SCFG_{pub} = (\SCID, T_s, \expy, \sigma_s, \cert_s)$ such taht $\sigma_s$ is a valid signature over $\SCID \| T_s \| \expy \| \cert_s$, however there exists no oracle $\pi^s_{j,0}$ which has previously output $\sigma_s$. Clearly we have
 \begin{equation}
  |\Adv_2 - \Adv_1| = \Pr[\abort_{\SIG}].
 \end{equation}%

 Note that the experiment is aborted, if $\pi^{c^{\ast}}_{i^{\ast},0}$ satisfies server authentication, due to Game 1. This means that server $\Server$ with an identity number $s$ must be $\tau_s$-corrupted with $\tau_s = \infty$ (i.e. not corrupted) when $\pi^{c^{\ast}}_{i^{\ast},0}$ accepts (as otherwise $\pi^{c^{\ast}}_{i^{\ast},0}$ satisfies server authentication). To show that $\Pr[\abort_{\SIG}] \leq \ell \epsilon_{\SIG}$, we construct a signature forger as follows. The forger receives as input a public key $pk^{\ast}$ and simulates the challenger for $\mathcal{A}$. It guesses an index $\phi \xleftarrow{\$}[\nserver]$, sets $pk_{\phi} = pk^{\ast}$, and generates all long-term public/secret keys as before. Then it proceeds as the challenger in Game 3, except for that it uses its chosen message oracle to generate a signature under $pk_{\phi}$ when necessary.

 If $\phi = s$, which happens with probability $1/\nserver$, then the forger can use the signature received by $\pi^{c^{\ast}}_{i^{\ast},\ell^{\ast}}$ to break the EUF-CMA security of the signature scheme with success probability $\epsilon_{\SIG}$, so $\Pr[\abort_{\SIG}]/\ell \leq \epsilon_{\SIG}$. Therefore if $\Pr[\abort_{\SIG}]$ is not negligible, then $\epsilon_{\SIG}$ is not negligible as well and we have
 \begin{equation}
  |\Adv_2 - \Adv_1| = \nserver \epsilon_{\SIG}.
 \end{equation}%

 Note that in Game 2 oracle $\pi^{c^{\ast}}_{i^{\ast},0}$ receives as input a server's DH public key $T_s$ such that $T_s$ was chosen by another oracle, but not by the attacker.
\vspace{10pt}\\%
%
%
 \textbf{Game 3.} In this game we want to make sure that we know which oracle $\pi^s_{j,0}$ will issue the signature $\sigma_s$ that $\cOracleAstFull$ receives. The challenger in this game proceeds as before, however additionally guesses two indices $(s^{\ast}, j^{\ast}) \xleftarrow{\$} [\nserver] \times [\noracle]$.

 We know that there must exists at least one oracle that outputs $\sigma_s$ such that $\sigma_s$ is forwarded to $\cOracleAstFull$, due to Game 2. Thus we have
 \begin{equation}
  \Adv_3 \leq \nserver \noracle \Adv_4
 \end{equation}%
 Note that in this game we know exactly that oracle $\sOracleAstFull$ chooses the DH public key $T_s$ that $\cOracleAstFull$ uses to compute its premaster secret.
 \vspace{10pt}\\
%
%
 \textbf{Game 4.} Let $T_{\cIndexAstRes} = g^u$ denote the DH public key chosen by $\cOracleAstRes$, let $T_{\sIndexAstRes} = g^v$ denote the share chosen by its partner $\sOracleAstRes$, and let $ik_{\cIndexAstRes}$ and $k_{\cIndexAstRes}$ are the key computed by $\cOracleAstRes$, let $pms_{\ik}$ and $ms_{\ik}$ denote a premaster secret and master secret for an initial key, let $pms_{k}$ and $ms_{k}$ denote a premaster secret and master secret for session key. Thus, both oracles compute the premaster secret as $pms = g^{uv}$.

 The challenger in this game proceeds as before, however replaces the premaster secret $pms_{\ik}$ of $\cOracleAstRes$ and $\sOracleAstRes$ with a random group element $\widetilde{pms_{\ik}} = g^w$, $w \xleftarrow{\$} \Zset_p$. Note that both $g^u$ and $g^v$ are chosen by oracles $\cOracleAstRes$ and $\sOracleAstRes$, respectively, as otherwise $\cOracleAstRes$ would not have a matching conversations for an initial key with $\sOracleAstRes$ and the game would be aborted.

 Distinguish Game 4 from Game 3 implies an algorithm solving the decisional Diffie-Hellman problem, thus
 \begin{equation}
  |\Adv_{4} - \Adv_{3}| \leq \epsilon_{\ddh}
 \end{equation}%
%
%
 \textbf{Game 5.} In this game we replace the value $ms_{\ik} = \PRF(\widetilde{pms_{\ik}}, \NONC)$ with a random value $\widetilde{ms_{\ik}}$.

 Distinguishing Game 5 from Game 4 implies an algorithm breaking the security of the pseudo random function $\PRF$, thus
 \begin{equation}
  |\Adv_{5} - \Adv_{4}| \leq \epsilon_{\prf}
 \end{equation}%
%
%
 \textbf{Game 6.} In this game we replace the function $\PRF(\widetilde{ms_{\ik}},\cdot)$ with a random function. If $\sOracleAstRes$ uses the same master secret $\widetilde{ms_{\ik}}$ as $\cOracleAstRes$, then the function $\PRF(\widetilde{ms_{\ik}},\cdot)$ used by $\sOracleAstRes$ is replaced as well. Of course the same random function is used for both oracles sharing the same $\widetilde{ms_{\ik}}$.

 Distinguishing Game 6 from Game 5 implies an algorithm breaking the security of the pseudo random function $\PRF$, thus
 \begin{equation}
  |\Adv_6 - \Adv_5| \leq \epsilon_{\prf}.
 \end{equation}%
%
%
 \textbf{Game 7.} Now we use that the key $\ik$ used by $\cOracleAstRes$ and $\sOracleAstRes$ in the symmetric encryption scheme uniformly at random and independent of all QUIC handshake messages.

 The adversary have to make ciphertext $c$ such that $\SE$.$\Dec$ ( $\ik$, $\iv$, H, $c$) $\neq \perp$ without knowing the key $k$. It implies an algorithm breaking the LHAE security of the symmetric encryption scheme, we have
 \begin{equation}
  \Adv_7 \leq 1/2 + \epsilon_{\LHAE}.
 \end{equation}%
\end{proof}