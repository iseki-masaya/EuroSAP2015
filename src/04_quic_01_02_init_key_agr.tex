%=====================================================
\subsubsection{Initial Key Agreement}
%=====================================================
In this phase, the client sends an inchoate client
hello (inchoateCHLO) which contains connection id
and some information such as server name, protocol
version, and user agent id. In our definition,
the some informations are omitted.
\\
\noindent
\underline{$\inchoateCHLO()$:} \\
 \setcounter{nombre}{0}%
 $\prob.\quad \cid \xleftarrow{\$} \{0,1\}^{64} $ \\
 $\prob.\quad \pInfo = (IP_c, IP_s, port_c, port_s)$ \\
 $\prob.\quad \return\ (\pInfo, \cid)$ \\
%
After the server receives inchoate client hello, it
sends a rejection (REJ). The rejection (REJ) contains
source address token (STK), server config (SCFG),
a certificate, and a signature of the server config generated
by the server long term secret key. The client use
STK in future queries to demonstrate ownership of their
source IP address.
\\
\noindent
\underline{$\REJ(m, \SCFG_{pub})$:} \\
 \setcounter{nombre}{0}%
 $\prob.\quad (\pInfo, \cid) = m$ \\
 $\prob.\quad \STK = \makeSTK()$ \\
 $\prob.\quad \pInfo = (IP_s, IP_c, port_s, port_c)$ \\
 $\prob.\quad \return\ (\pInfo, \cid, \SCFG_{pub}, \STK)$ \\
\underline{$\makeSTK()$:} \\
 \setcounter{nombre}{0}%
 $\prob.\quad \iv_{\STK} \xleftarrow{\$} \{0,1\}^{96}$ \\
 $\prob.\quad \plaintext = IP_c \| currentTime$ \\
 $\prob.\quad \STK \leftarrow \iv_{\STK} \|
        \SE.\Enc(k_{\STK}, len,\iv_{\STK},
        \plaintext)$ \\
 $\prob.\quad \return\ \STK$ \\
%
After the client receives a rejection (REJ), the client
checks the server config and generate NONC which consists
of a random value and current time, and ephemeral
Diffie-Hellman values.
The client sends a client hello (CHLO) to the server.
We define a client hello in 1-RTT connection as initialCHLO.
\\
\noindent
\underline{$\initialCHLO(m)$:} \\
 \setcounter{nombre}{0}%
 $\prob.\quad (\pInfo, \cid, \SCFG_{pub}, \STK) = m$ \\
 $\prob.\quad \pInfo = (IP_c, IP_s, port_c, port_s)$ \\
 $\prob.\quad t_c \xleftarrow{\$} \Zset_{q}^{\ast}$ \\
 $\prob.\quad T_c = g^{t_c}$ \\
 $\prob.\quad r \xleftarrow{\$} \{0,1\}^{160}$ \\
 $\prob.\quad \NONC \leftarrow currentTime \| r$ \\
 $\prob.\quad \return\ (\pInfo, \cid, \STK, \SCID, \NONC, T_c)$ \\
\underline{$\checkSCFG(\SCFG_{pub})$:} \\
 \setcounter{nombre}{0}%
 $\prob.\quad (\SCID, T_s, \expy, \sigma_s, \cert_s) = \SCFG_{pub}$ \\
 $\prob.\quad \text{If } \expy \leq currentTime$ \\
 $\prob.\quad \quad \Lambda = \text{'reject' and abort}$ \\
 $\prob.\quad pk_s = \getPK(cert_s)$ \\
 $\prob.\quad str = \text{ QUIC server config signature }$ \\
 $\prob.\quad \doc = str \| 0x00 \| \SCID \| T_s \| \expy$ \\
 $\prob.\quad \text{If } \SIG.\Vfy(pk_s, \sigma_s, \doc) = \perp$ \\
 $\prob.\quad \quad \Lambda = \text{'reject' and abort}$ \\
%
After the server receives a initial client hello, the
server check this query. $\STK$ is an opaque byte string
from the client's point of view. From the server's point
of view it's an authenticated-encryption block that
contains, at least, the client's IP address and a time
stamp by the server. The server decrypt $\STK$ and
validate time and source IP address. $\NONC$ is
concatenation of time and a random value. The server
register the time and random value and validate each
$\NONC$ to ensure that it does not process the same
connection twice. This checks prevent a part of replay
attacks.
\\
\noindent
\underline{$\checkQuery(\STK, k_{\STK}, \NONC, IP_c)$:} \\
 \setcounter{nombre}{0}%
 $\prob.\quad (\iv_{\STK}, c) = \STK$ \\
 $\prob.\quad (IP_c^{\prime}, time_{\STK}) = \SE.\Dec(k_{\STK}, \iv_{\STK}, c)$ \\
 $\prob.\quad (time_{\NONC}, r) = \NONC$ \\
 $\prob.\quad \text{If } (IP_c^{\prime}, currentTime) = \perp$, or \\
 $\prob.\quad \quad IP_c^{\prime} \neq IP_c$, or $time_{\STK} \leq time_{allowed}$\\
 $\prob.\quad \quad r \in \strike$, or $time_{\NONC} \not\in \strike_{rng}$ \\
 $\prob.\quad \quad \quad \Lambda = \text{'reject' and abort}$ \\
%
After the server validate initial client hello or the
client sends initial client hello, they calculate initial
key.
\noindent
\underline{$\getKey_c(\shareInfo, m, \init)$:} \\
 \setcounter{nombre}{0}%
 $\prob.\quad (\NONC, \cid ,T_s, t_c) = \shareInfo$ \\
 $\prob.\quad pms = T_s^{t_c}$ \\
 $\prob.\quad \return\ \extractKey(pms, \NONC, \cid, m, 40, \init)$ \\
\underline{$\getKey_s(\shareInfo, m, \init)$:} \\
 \setcounter{nombre}{0}%
 $\prob.\quad (\NONC, \cid ,T_c, t_s) = \shareInfo$ \\
 $\prob.\quad pms = T_c^{t_s}$ \\
 $\prob.\quad \return\ \extractKey(pms, \NONC, \cid, m, 40, \init)$ \\
\underline{$\extractKey(pms, \NONC, \cid, m, \ell, \init)$:}\\
 \setcounter{nombre}{0}%
 $\prob.\quad ms = \PRF(pms, \NONC)$ \\
 $\prob.\quad \text{If } \init = 1$ \\
 $\prob.\quad \quad str = \text{ QUIC key expansion }$ \\
 $\prob.\quad \text{Else }$ \\
 $\prob.\quad \quad str = \text{ QUIC forward secure expansion }$ \\
 $\prob.\quad \info = str \| 0x00 \| \cid \| m \| \SCFG_{pub}$ \\
 $\prob.\quad \return\ \text{the first $\ell$ octets (i.e. bytes) of T = }$ \\
 $\quad \quad \text{(T(1),T(2), ...), where for all $i \in \Nset$, $T(i) = $} $\\
 $\quad \quad \text{$\PRF(ms, T(i-1) \| \info \| 0x0i)$ and $T(0) = \epsilon$} $\\