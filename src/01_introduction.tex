%=====================================================
\section{Introduction} \label{sec:intro}
%=====================================================
Quick UDP Internet Connections (QUIC for short) is a
new transport layer network protocol recently proposed
by Google \cite{QUIC}, which is experimentally
implemented in Google Chrome.
The main purpose of developing QUIC is to provide an
alternative equivalence of TLS wrapping TCP, with much
reduced latency and better SPDY support.
Transport Layer Security (TLS) starts with a three-move
TCP handshake before initiating the TLS Handshake
Protocol.
In contrast, QUIC uses UDP and starts with its own
handshake, which reduces the total number of
interactions.
The cryptographic core of QUIC is specified in the QUIC
crypto protocol~\cite{QUIC:Crypto}, which consists of a
handshake protocol and a record layer protocol, as does
TLS.
Similarly to TLS, QUIC has two types of handshake
connections.
One is called a full handshake -- a handshake
``from scratch" between a client and a server.
The other is called a resumption -- an abbreviated
handshake, which occurs when a client and a server have
once established a full handshake session and want to
establish a new session between them in an abbreviate
way.
Unlike TLS, QUIC only supports the elliptic-curve
Diffie-Hellman key-exchange (ECDHE) cipher suites and
server authentication.
%
One of the good features of QUIC is that it can
establish an abbreviate session with $0$-RTT
connectivity overhead.
Namely, in the QUIC resumption, a client can send
encrypted data to a server, concurrently with resuming
a new session.
We provide the abstract model of the full handshake and
resumption protocols of QUIC in Fig.~\ref{fig:quic}.

%=====================================================
\subsection{Prior Security Analyses and Some Security Concern} \label{sec:concern}
%=====================================================
To the best of our knowledge, there are only two
security analyses on QUIC~\cite{FG14:QUIC,LJBN15:QUIC}.
Both papers define new security models and show that
QUIC is secure in that model.
In~\cite{FG14:QUIC}, they formalized a secure
authenticated key-exchange as an extension of the
Bellare-Rogaway model~\cite{BR93:AKE} and analysed the
security of QUIC (with resumption).
However, the QUIC protocol analysed in \cite{FG14:QUIC}
is slightly different from the protocol given in the
source codes.
As described in Fig.~\ref{fig:quic}, the QUIC protocol
makes a server send a ciphertext (using authenticated
encryption) in the full handshake protocol, which cannot
preserve \textit{key-indistinguishability}.
Therefore, the authenticated and channel confidentiality
establishment (ACCE) model~\cite{JKSS12:ACCE} is more
suitable to analyse QUIC.
Another important security issue is that in~\cite{FG14:QUIC},
an adversary is allowed to send a ``test" query only to
a client oracle (to receive either a real session-key or
a random key from the client oracle),  when a protocol
is server-only authenticated.
Apparently, the restriction is appropriate, because
an adversary can establish a session with a honest
server (due to the lack of client's certificate) to
share a session key.
However, if resumption is provided, we should consider
the attack that, after a honest client and a honest
server establish a full handshake session, an adversary
might hijack a resumption session -- it might
impersonate the initial client and share a session key
with the server.
To protect the attack, we should allow an adversary to
send test queries to \textit{server} oracles in resumption
sessions (including the full handshake session), as long
as the initial full handshake session is established
between a honest client and a honest server.
We can consider a weaker attack: The adversary cannot
share a session key with the server, but it can make the
server accept in a resumption session. (Note that in a
full handshake session, it is a ``trivial" attack, because
an adversary can always do so.)
Protection of this attack guarantees that only parties
that establish the initial full handshake session can
resume and establish a new resumption session.
Without this protection, we would possibly have an actual
inconvenience.
The aim of resumption is to resume a new session more
rapidly than a full handshake session.
In QUIC, it actually achieves $0$-RTT.
However, if this attack succeeds, the client and server
will have inconsistent cache data and they cannot resume
a new session, which means that they must establish a
full handshake session again.

In~\cite{LJBN15:QUIC}, they formalized a secure
authenticated key-exchange as an extension of the ACCE
model.
We also formalized it as an extension of the ACCE model.
However, their security model cannot prevent the attack
in resumption.

%=====================================================
\subsection{Related Work} \label{sec:Related Work}
%=====================================================
There are a huge body of works on authenticated key
exchange protocols (See~\cite{CK01:AKE} for survey).
An important stream of research dates back to Bellare
and Rogaway~\cite{BR93:AKE}, followed by~\cite{DB96,
Blei98,JMDP00,JB02,EK09,KK05:TLS,KCRE08,SMOAJ08,KTT11,
Kraw01}.
However, as mentioned above, the QUIC full handshake
protocol does not satisfy key-indistinguishability as
in the Bellare-Rogaway like model, because a server
sends a ciphertext (using authenitcated encryption) in
the full handshake protocol, as does TLS.
TLS Handshake Protocol is recently analysed in various
security models, e.g., ~\cite{JKSS12:ACCE,KPW13:SACCE,
FS13:ACCE,GKS13:RACCE,BDKSS14:SSH,BFKPSB14:TLS}.
Still, the security model for analysing a server-only
authenticated connection of TLS, i.e., Server-Only
Authenticated and Confidential Channel Establishment
(SACCE)~\cite{KPW13:SACCE}, does not capture our
security issues.
This is because, besides not treating the resumption
originally, the SACCE security model only concerns,
which is more essential, \textit{server} authentication
and a \textit{client's} message confidentiality.
However, these issues appear in (some kind of)
\textit{client} authentication and \textit{server}
confidentiality.

%=====================================================
\subsection{Our Results} \label{sec:proposal}
%=====================================================

To treat the security issues above, we introduce a new
security model, what we call \textit{Resumable} SACCE
(RSACCE) security, where we consider a server's message
confidentiality, as well as a client's message
confidentiality, where an adversary is allowed to send
an encryption query to a \textit{server} (to break a
server's message confidentiality) both in the full
handshake session and its successor resumption sessions,
as far as the server establishes the initial full
handshake session with a \textit{honest} client.
We also provide a stronger model, called strong RSACCE
security, where we ensure that a sever can establish a
resumption session only with the \textit{same} client
as initially connected in the full handshake session.
We require in that model forward secrecy among all
independent sessions.
For resumption to be effective, we compromise but still
require some level of forward secrecy among related
resumption sessions.

We propose a more efficient and secure protocol with
the spirit of QUIC.
In~\cite{LJBN15:QUIC}, they find the five attacks for
QUIC.
Our proposal scheme can prevent these attacks.

We analyse QUIC as it is, and prove that QUIC meets
RSACCE security, but it does not meet the strong
RSACCE one.
We also analyse an optional version of QUIC with CETV,
QUIC with an optional client encrypted tag value (CETV)
mechanism, and show that it meets strong RSACCE security.
Finally, we present a more efficient protocol than QUIC,
which satisfies strong RSACCE security.