%=====================================================
\section{Introduction} \label{sec:intro}
%=====================================================
Quick UDP Internet Connections (QUIC for short) is a
new transport layer network protocol recently proposed by Google \cite{QUIC,QUICDraft}, 
which is experimentally implemented in Google Chrome. 
The main purpose of developing QUIC is to provide an alternative equivalence of TLS wrapping TCP, 
with much reduced latency and better SPDY and HTTP/2 support. 
Transport Layer Security (TLS) starts with a three-move TCP handshake 
before initiating the TLS Handshake Protocol.
In contrast, QUIC uses UDP and starts with its own handshake, 
which reduces the total number of interactions.
The cryptographic core of QUIC is specified in the QUIC crypto protocol~\cite{QUIC:Crypto}, 
which consists of a handshake protocol and a record layer protocol, as does TLS.
Similarly to TLS, QUIC has two types of handshake connections: 
One is called \textit{the full handshake} -- a handshake connection 
``from scratch" between a client and a server. 
The other is called \textit{the abbreviate handshake} -- a handshake connection 
which takes place when a client who has already established a full handshake connection with a server, 
wants to establish a new connection with the same server in an abbreviate way. 
Unlike TLS, QUIC does not support mutual authentication, but server authentication only. 
In addition, QUIC only supports the elliptic-curve version of Diffie-Hellman (DH) key-exchange cipher suite 
in the handshake protocol,  which yields a unique \textit{two-key} structure in the handshake protocol   
to achieve reduced latency:  
The first round of the QUIC full handshake executes the \textit{static} DH key exchange of  
a server's \textit{static} (in some time period) DH public-key $T_s$ 
with a client's ephemeral DH public-key $T_c$. 
Then, the second round executes the \textit{ephemeral} DH key exchange of 
the client's ephemeral DH public-key $T_c$ (in the first round) with 
the server's \textit{ephemeral} DH public-key $T'_s$.
Here, the client may send a ciphertext encrypted by the first static DH key after the first round. 
He may also send a ciphertext encrypted by the second ``strong'' ephemeral DH key 
instead of the first ``weak" static DH key, after the full handshake is established. 
The abbreviate handshake starts by client sending an ephemeral DH public-key $T'_c$. 
The server sends back its ephemeral DH public-key $T''_s$ and the abbreviate handshake is established. 
Here, we note that the client can send an encrypted data to the server, 
concurrently with the first message with $T'_c$, where 
the ciphertext is encrypted by the DH key derived from  
the client's ephemeral $T'_c$ and the server's static $T_s$. 
After the abbreviate handshake is established, 
the client may send a ciphertext encrypted by the second ``strong'' ephemeral DH key 
derived from client's ephemeral $T'_c$ and server's ephemeral $T''_s$.

This two-key structure is a remarkable feature of QUIC. 
By this, QUIC requires only  $1$-RTT and $0$-RTT connectivity overhead in the full and 
abbreviate handshake connections, respectively.


\if0
%
an abbreviate handshake session is called
the 0-RTT session. On one hand, 
a client can send encrypted data just after executing one round interaction with a server.
Hence, a full handshake session is called the 1-RTT session.
$1$-RTT connectivity overhead in the full handshake connection and 
$0$-RTT one in the abbreviate handshake connection.
Namely, in the QUIC abbreviate handshake, a client can send
encrypted data to a server, concurrently with a new session.
% We provide the abstract model of the full handshake and
% abbreviate handshake protocols of QUIC in
% Fig.~\ref{fig:quic_abst_1rtt_init},~\ref{fig:quic_abst_1rtt_last}, and ~\ref{fig:quic_abst_0rtt}.
By this property, an abbreviate handshake session is called
the 0-RTT session. On one hand, 
a client can send encrypted data just after executing one round interaction with a server.
Hence, a full handshake session is called the 1-RTT session.
\fi
%=========================================================================
\subsection{Prior Works and Security Concerns} \label{sec:concern}
%=========================================================================
To the best of our knowledge, there are two security analyses on QUIC~\cite{FG14:QUIC,LJBN15:QUIC}.
Both papers define new security models and have analyzed the security of QUIC. 
Both provide positive conclusions, but 
\cite{LJBN15:QUIC} points out some security concerns at the same time. 

In~\cite{FG14:QUIC}, Fischlin and Gunther focus on the two-key structure in the QUIC handshake 
and formalize a secure authenticated key-exchange protocol with such a two-key structure, 
in an extension  of the Bellare-Rogaway model~\cite{BR93:AKE}. 
%and analyze the security of QUIC (with abbreviate handshakes).
However, the actual QUIC protocol is slightly different from the protocol defined in \cite{FG14:QUIC}, 
because a server always sends an authenticated encryption in the last message of 
the QUIC full handshake and   
it makes impossible to obtain \textit{key-indistinguishability}, as with the case of TLS. 
Therefore, the authenticated and channel confidentiality
establishment (ACCE) model~\cite{JKSS12:ACCE} is more
suitable to analyze QUIC.

In~\cite{LJBN15:QUIC}, Lychev et al.~also formalize a new security model, called QUIC ACCE (QACCE) model,  
based on ACCE model to analyze full and abbreviate handshake connections of QUIC.
Although they conclude that QUIC is secure in their model as it is, 
they also point out vulnerabilities, which appear in 
(1) the server-config replay attack, (2) the source-address token replay attack, 
(3) the connection ID manipulation attack, (4) the source-address token manipulation attack, 
and (5) the crypto stream offset attack. 
Each attack affects the increase of server loads and makes it easy that 
an adversary executes the distributed denial of service (DDoS) attacks. 
The security model QACCE does not prevent an adversary from doing these attacks.


%=====================================================
\subsection{Other Related Works} \label{sec:Related Work}
%=====================================================
There are a huge body of literature on authenticated key exchange protocols 
(See~\cite{CK01:AKE,JKSS12:ACCE,KPW13:SACCE} for survey). 
An important stream of research dates back to Bellare and Rogaway~\cite{BR93:AKE}, 
followed by~\cite{DB96,Blei98,JMDP00,Kraw01,JB02,KK05:TLS,KCRE08,SMOAJ08,EK09,KTT11,}.
However, as mentioned above, the QUIC full handshake protocol does not satisfy key-indistinguishability,  
because a server always sends an authenticated encryption in a full handshake connection, as does TLS.
TLS Handshake Protocol is recently analyzed in various security models, 
e.g., ~\cite{JKSS12:ACCE,KPW13:SACCE,FS13:ACCE,GKS13:RACCE,BDKSS14:SSH,BFKPSB14:TLS}. 
Our proposed security model lies in the line of 
the ACCE security model~\cite{JKSS12:ACCE,KPW13:SACCE}. 
Although QUIC is a server-only authenticated protocol, 
it is not appropriate to apply it directly to 
the server-only authenticated and confidential channel establishment
(SACCE) model~\cite{KPW13:SACCE}, defined for analyzing 
a server-only authenticated connection of TLS. 
Firstly, QUIC has a two-key mechanism mentioned above, 
in which a client and a server firstly share an initial 
key $\ik$ in the middle of a connection and finally take a final key $\key$ when the connection is established.
The initial key is used to encrypt messages before the final key is shared. 
This mechanism allows QUIC to enjoy $1$-RTT and $0$-RTT connections in the full and abbreviate handshakes  
respectively, but it prevents adaptation to the SACCE model. 
Secondly, QUIC has full and abbreviate handshake connections, where 
an abbreviate handshake is a descendent  of some full handshake. 
When such two related handshake connections exist, 
it is natural to consider some binding property between them. 
Otherwise, an adversary might hijack an abbreviate handshake connection, 
which actually causes DDoS attacks.
TLS also has an abbreviate handshake, called the resumption, but 
it is not treated in the (S)ACCE model. 

\if0
In \cite{LJBN15:QUIC}, 
 does not
consider the consistency of the client between a full
handshake and abbreviate handshakes.
% The second reasons is important in order to prevent the
% attacks \cite{LJBN15:QUIC}.
\fi
%=====================================================
\subsection{Our Contribution} \label{sec:proposal}
%=====================================================
We introduce a new security model, what we call \textit{resumable} SACCE
(RSACCE) to analyze the securty of QUIC and its variants. 
RSACCE is a stronger security notion than QACCE. 
Unfortunately, the original QUIC is not secure in the RSACCE model. 
We present a new protocol, a ``variant'' of QUIC, which is secure in the RSACCE model. 
Here the DDoS attacks mentioned in \cite{LJBN15:QUIC} do not effectively work against the new protocol.

Our motivation is to enhance QUIC without harming its concepts, and to prevent an adversary from executing the DDoS attacks mentioned above. 
Observing the original QUIC, we noticed  that 
a client cannot be aware of the presence of a man-in-the-middle adversary to make inconsistent views 
in the right and left interactions.
In addition, an adversary may easily hijack an abbreviate connection to pretend an initial client 
if the adversary can see the initial full handshake connection. 
The DDoS attacks mentioned in \cite{LJBN15:QUIC} take advantage of these vulnerabilities of QUIC. 
Such vulnerabilities cannot be allowed to a protocol secure in the RSACCE model. 

We want to enhance QUIC secure in the RSACCE model, without changing its design concepts.
What features define QUIC and what should be preserved, then. 
We consider that the following features represent the concept of QUIC.

\begin{itemize}
\item The two-key structure based on the DH key exchange protocol. 
\item Repeated usage of server-configuration $\SCFG$ (with static DH public-key $T_s$) in some time period. 
\item Server-only authentication. 
\item No server cash data.
\end{itemize}

The  first and second properties enables 
$1$-round trip time (RTT) and $0$-RTT connective overheads in the full and abbreviate handshakes, respectively. 
The third and forth properties represent some policy of the design of QUIC. 
In addition, we have not changed any server's (long-term and middle-term) secret in our proposal.   
So, we think our proposed protocol retains the spirit of QUIC. 

In the RSACCE security model, we consider three security requirements, 
server authentication, channel confidentiality, and channel binding. 
The server authentication prevents an adversary from making inconsistent views, 
and the channel  binding prevents an adversary from pretending the client in the initial handshake connection.
The channel confidentiality guarantees message confidentiality.   
Unlike \cite{KPW13:SACCE,FG14:QUIC,LJBN15:QUIC}, 
we consider a server's message confidentiality, as well as a client's message
confidentiality, where an adversary is allowed to send 
an encryption query, i.e., a ``test" query, to a \textit{server} 
both in the full and abbreviate handshake connections, 
as far as the server has established the initial full handshake session with a \textit{honest} client.
In QACCE, an adversary 
is also allowed to send a ``test" query to a server oracle in an
abbreviate handshake.
However, an adversary is only allowed
to send it to the server oracle with a matching
conversation with a client oracle.
In this case, server's message confidentiality is 
equivalent to client's message confidentiality.  
% An adversary needs to forge the query to make a server accept
% and the server does not have a matching conversations with any
% client oracle.
\if0
By this restriction, QACCE cannot take care of the attack where 
an adversary makes a server accept.
In QUIC, a server does not make sure consistencies of a client
between 0-RTT connections and 1-RTT connections and
an adversary can establish a connection with the server in 0-RTT connections.
This property allows one of the attacks found in~\cite{LJBN15:QUIC} that
the adversary can establish the connection spoofing IP address.
This assists the adversary to do Distributed Denial of Service
(DDoS) attack.
Our security model guarantees that only parties
that establish the initial full handshake session can
establish a new abbreviate handshake session.
% Our security model also prevent other attacks.
% In QUIC, the client and server share two keys which are an initial
% key $\ik$ and a final key $\key$.
Other attacks~\cite{LJBN15:QUIC} make a client and a server share
a different initial key $\ik$.
The server impersonation advantage in QACCE does not
cover this case because this advantage consider only forward
secure key $\key$.
Our security model also consider the security of an initial
key $\ik$ and it prevents the other attacks
~\cite{LJBN15:QUIC}.
\fi

\if0
Our security model also consider the consistency of the client between
0-RTT connections and 1-RTT connections.
The consistency of the client between 0-RTT connections
and 1-RTT connections prevent one of the attacks.

We consider a security model, which would be appropriate for QUIC, but 
Our contributions are two-holds:
\begin{itemize}
 \item{
 We propose a security model, which would be appropriate for a server-only authenticated key exchange 
 equipped with the two-key structure and the abbreviate handshake connections, such as QUIC. 
 In particular, we consider a binding property between a full handshake connection 
 and its successive abbreviate handshake, which is not considered in QACCE~\cite{LJBN15:QUIC}, but 
 the property is 
 }

 \item{A new scheme which is more secure and efficient
 than original QUIC.}
\end{itemize}



Here, the long-term secret of a server is a signing key and  
a secret key for an authenticated encryption, to create a source address token (STK)  
described later. The middle-term secret is a DH secret key $t_s$ with respects to a static (in some period) 
DH public key $T_s$.


We also propose a new scheme which is more secure and more efficient
than the original QUIC and it satisfy RSACCE secure.
In the our proposed scheme, a client sends its Diffie-Hellman public value in
a first query while in original QUIC the client sends its Diffie-Hellman
public value in a second query.
For this scheme, the server can generate MAC for important parameters in the
response and it prevents the attacks.
% In~\cite{LJBN15:QUIC}, they found the five attacks for
% QUIC and the our proposed scheme can prevent the four attacks.


%%%%%%%%
that with a replay attack on some
public parameters exchanged during the handshake, an
adversary could easily prevent QUIC from achieving
minimal latency advantages either by having it fall back
to TCP or by causing the client and the server to have an
inconsistent view of their handshake leading to a failure
to complete the connection.
The adversary can increase server loads using
these attacks.

Another important security issue is that in~\cite{FG14:QUIC},
an adversary is allowed to send a ``test" query only to
a client oracle (to receive either a real session-key or
a random key from the client oracle), when a protocol
is server-only authenticated.
Apparently, the restriction is appropriate, because
an adversary can establish a session with a honest
server (due to the lack of client's certificate) to
share a session key.
However, if an abbreviate handshake is provided, we should consider
the attack that, after a honest client and a honest
server establish a full handshake session, an adversary
might hijack an abbreviate handshake session -- it might
impersonate the initial client and share a session key
with the server.
To prevent the attack, we should allow an adversary to
send test queries to \textit{server} oracles in abbreviate
handshake sessions (including the full handshake session), as long
as the initial full handshake session is established
between a honest client and a honest server.
We can consider an attack: The adversary can
share a session key with the server and it can make the
server accept in an abbreviate handshake session. (Note that in a
full handshake session, it is a ``trivial" attack, because
an adversary can always do so.)
%%%%%%%%%%%%
\fi