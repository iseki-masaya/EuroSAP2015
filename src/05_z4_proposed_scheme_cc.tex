\begin{proof}
 The proof proceeds in a sequence of games. \vspace{10pt}\\
 \textbf{Game 0.} This game equals the \textit{channel confidetility} security experiment.
 \begin{equation}
  \Adv_0 = \Adv^{\rsaccecc}_{P}(A)
 \end{equation}%
%
%
 \textbf{Game 1.} The challenger in this game proceeds as before, however it aborts and chooses $b^{\prime}$ uniformly random, if there exists any oracle that breaks server authentication. Thus we have
 \begin{equation}
  |\Adv_1 - \Adv_0| = \Adv^{\rsaccesa}_{P}(A).
 \end{equation}%
 Note that if there exists the oracle which breaks \textit{server authentication}, the adversary easily breaks \textit{channel confidentiality}. If the target oracle breaks \textit{server authentication}, the adversary or unrelated oracle (i.e. it is not an intended partner of the oracle) can establish the session with the oracle. The adversary can issue $\Reveal$-query to unrelated oracle which is out of the restriction of \textit{channel confidentiality}. Then the adversary can know the session key of the target oracle.
\vspace{10pt}\\%
%
%
 \textbf{Game 2.} The challenger in this game proceeds as before, however in addition guesses indices $(p^{\ast}, i^{\ast}, \ell^{\ast}) \xleftarrow{\$} [n_s + n_c] \times [n_o] \times [n_{\ell}]$. It aborts and chooses $b^{\prime}$ at random, if the attacker issues a $\Encrypt$-query with $(p,i,\ell) \neq (p^{\ast}, i^{\ast}, \ell^{\ast})$. With probability $1/((n_s+n_c)n_o n_{\ell})$ we have $(p,i,\ell) = (p^{\ast}, i^{\ast}, \ell^{\ast})$, and thus
 \begin{equation}
  \Adv_1 = (n_s + n_c) n_o n_{\ell}\Adv_2.
 \end{equation}%
 Note that in Game 2 we know that $\mathcal{A}$ will issue a $\Encrypt$-query to oracle $\pOracleAstRes$. Note also that $\pOracleAstRes$ has a unique partner due to Game 1. In the sequel we denote with $\qOracleAstRes$ the unique oracle such that $\pOracleAstRes$ has a matching conversations for an initial key or a final key with $\qOracleAstRes$, and say that $\qOracleAstRes$ is the intended partner of $\pOracleAstRes$.
\vspace{10pt}\\%
%
%
 \textbf{Game 3 + $4\ell$.} We repeat the game until randomizing the session between $\pOracleAstRes$ and $\qOracleAstRes$. Initially $\ell = 0$. The reason of repeating is that the adversary always return correct $b^{\prime}$ if the adversary can obtain $k_{mac}$ of $\pOracleAstEll$ or $\qOracleAstEll$. We need to prevent the adversary obtaining all $k_{mac}$ in $ 0 \leq \ell \leq \ell^{\ast}$.
 Let $T_{\pindexell} = g^u$ denote the DH public key chosen by $\pOracleAstEll$, let $T_{\qindexell} = g^v$ denote the share chosen by its partner $\qOracleAstEll$, and let $\ik_{\pindexell}$ and $k_{\pindexell}$ are the key computed by $\pOracleAstEll$, let $pms_{\ik}$ and $ms_{\ik}$ denote a premaster secret and master secret for an initial key, let $pms_{k}$ and $ms_{k}$ denote a premaster secret and master secret for session key. Thus, both oracles compute the premaster secret as $pms = g^{uv}$.

 The challenger in this game proceeds as before, however replaces the premaster secret $pms_{\ik}$ of $\pOracleAstEll$ and $\qOracleAstEll$ with a random group element $\widetilde{pms_{\ik}} = g^w$, $w \xleftarrow{\$} \Zset_p$. Note that both $g^u$ and $g^v$ are chosen by oracles $\pOracleAstEll$ and $\qOracleAstEll$, respectively, as otherwise $\pOracleAstEll$ would not have a matching conversations for an initial key with $\qOracleAstEll$ and the game would be aborted.

 Suppose that there exists an algorithm $\mathcal{A}$ distinguishes Game 3 + $4\ell$ from Game 2 + $4\ell$. Then we can construct an algorithm $\mathcal{B}$ solving the DDH problem as follows. $\mathcal{B}$ receives as input $(g,g^u,g^v,g^w)$. The challenger defines $T_{\pindexell} := g^u$ and $T_{\qindexell} := g^v$, and the premaster secret of $\pOracleAstEll$ and $\qOracleAstEll$ equal to $pms_{\ik} := g^w$. Note that $\mathcal{B}$ can simulate all messages exchanged between $\pOracleAstEll$ and $\qOracleAstEll$ properly. Since all other oracles are not modified, $\mathcal{B}$ can simulate these oracles properly as well.

 If $w=uv$, then the view of $\mathcal{A}$ when interacting with $\mathcal{B}$ is identical to Game 2 + $4\ell$, while if $w \xleftarrow{\$}\Zset_p$ then it is identical to Game 3 + $4\ell$. Thus, the DDH assumption implies that
 \begin{equation}
  |\Adv_{3 + 4\ell} - \Adv_{2 + 4\ell}| \leq \epsilon_{\ddh}
 \end{equation}%
%
%
 \textbf{Game 4 + $4\ell$.} In Game 4 + 4$\ell$ we make use of the fact that the premaster secret $\widetilde{pms_{\ik}}$ of $\pOracleAstEll$ and $\qOracleAstEll$ is chosen uniformly random. We thus replace the value $ms_{\ik} = \PRF(\widetilde{pms_{\ik}}, \NONC)$ with a random value $\widetilde{ms_{\ik}}$.

 Distinguish Game 4 + 4$\ell$ from Game 3 + 4$\ell$ implies an algorithm breaking the security of the pseudo random function $\PRF$, thus
 \begin{equation}
  |\Adv_{4 + 4\ell} - \Adv_{3 + 4\ell}| \leq \epsilon_{\prf}
 \end{equation}%
%
%
 \textbf{Game 5 + $4\ell$.} In this game we replace the function $\PRF(\widetilde{ms_{\ik}}, \cdot)$ used by $\pOracleAstEll$ and $\qOracleAstEll$ with a random function $F_{\widetilde{ms_{\ik}}}$. Of course the same random function is used for both oracles $\pOracleAstEll$ and $\qOracleAstEll$. Distinguishing Game 5 + $4\ell$ from Game 4 + $4\ell$ again implies an algorithm breaking the security of the pseudo random function $\PRF$.
 \begin{equation}
  |\Adv_{5 + 4\ell} - \Adv_{4 + 4\ell}| \leq \epsilon_{\prf}
 \end{equation}%

 Note that the adversary cannot obtain the DH public key and $k_{mac}$ because the adversary obtain nothing from the transcription due to randomization of the initial key $\ik$. These changes prevent trivially attack. If the adversary obtain the MAC key $k_{mac}$, the adversary can hijack the session following way: the adversary generate $T_c$, $\NONC$ by himself and calculate MAC generated by MAC key $k_{mac}$ and send these value with $\SCID$. The server cannot reject this query made by the adversary because MAC is valid. The adversary can share the secret key with the server and always return correct $b^{\prime}$.
\vspace{10pt}\\%
%
%
 \textbf{Game 6 + $4\ell$.} Now we use that the key $\ik_{\pindexell}$ and $\ik_{\qindexell}$ in the symmetric encryption scheme uniformly at random and independent of all QUIC handshake messages. In this we replace the value $k_{mac}$ with another random value $\widetilde{k_{mac}}$.

 Suppose that there exists an algorithm $\mathcal{A}$ distinguishes Game 6 + 4 $\ell$ from Game 5 + 4 $\ell$. Then we can construct an algorithm $\mathcal{B}$ breaking $\LHAE$ secure. By assumption, the simulator $\mathcal{B}$ is given access to an encryption oracle $\Encrypt$ and a decryption oracle $\Decrypt$.

 Since by assumption any attacker has advantage at most $\epsilon_{\LHAE}$ in breaking the $\LHAE$ security of the symmetric encryption scheme, we have
 \begin{equation}
  |\Adv_{6 + 4\ell} - \Adv_{5 + 4\ell}| \leq \epsilon_{\LHAE}.
 \end{equation}%
 After this game, we add $\ell = \ell + 1$. If $\ell \leq \ell^{\ast}$ we repeat the game.
\vspace{10pt}\\%
%
%
 For next step, there are two cases. The first case is that the adversary issue $\Encrypt$-query $\pOracleAstRes$ or $\qOracleAstRes$ whose state $\Lambda = \preaccept$. The second case is that the adversary issue $\Encrypt$-query $\pOracleAstRes$ or $\qOracleAstRes$ whose state $\Lambda = \accept$.

 We define the first case as Game$_{a}$ and the second case as Game$_{b}$
\vspace{10pt}\\%
%
%
 \textbf{Game$_a$ 7 + 4$\ell^{\ast}$.} Now we use that the key $\ik_{\pindexell}$ and $\ik_{\qindexell}$ which is independent of all QUIC handshake messages.

 In this game we construct a simulator $\mathcal{B}$ that uses a RSACCE attacker $\mathcal{A}$ to break the security of the underlying $\LHAE$ secure symmetric encryption scheme. By assumption, the simulator $\mathcal{B}$ is given access to an encryption oracle $\Encrypt$ and a decryption oracle $\Decrypt$. $\mathcal{B}$ embeds that $\LHAE$ experiment by simply forwarding all $\Encrypt(\pOracleAstRes,\cdot)$ queries to $\Encrypt$, and all $\Decrypt(\qOracleAstRes,\cdot)$ queries to $\Decrypt$. Otherwise it proceeds as the challenger in Game 6.

 Observe that the values generated in this game are exactly distributed as in the previous game. We thus have
 \begin{equation}
  \Adv_{7 + 4\ell^{\ast}} = \Adv_{6 + 4\ell^{\ast}}
 \end{equation}%
 If $\mathcal{A}$ outputs a quad $\pindexout$, then $\mathcal{B}$ forwards $b^{\prime}$ to the $\LHAE$ challenger. Otherwise it outputs a random bit. Since the simulator essentially relays all messages it is easy to see that an attacker $\mathcal{A}$ having advantage $\epsilon^{\prime}$ yields an attacker $\mathcal{B}$ against the $\LHAE$ security of the encryption scheme with success probability at least $1/2 + \epsilon^{\prime}$.

 Since by assumption any attacker has advantage at most $\epsilon_{\LHAE}$ in breaking the $\LHAE$ security of the symmetric encryption scheme, we have
 \begin{equation}
  \Adv_{7 + 4\ell^{\ast}} \leq \frac{1}{2} + \epsilon_{\LHAE}.
 \end{equation}%
%
%
 \textbf{Game$_b$ 7 + 4$\ell^{\ast}$.} The challenger in this game proceeds as before, however replaces the premaster secret $pms_{k}$ of $\pOracleAstRes$ and $\qOracleAstRes$ with a random group element $\widetilde{pms_{k}} = g^w$, $w \xleftarrow{\$} \Zset_p$. Note that both $g^u$ and $g^v$ are chosen by oracles $\pOracleAstRes$ and $\qOracleAstRes$, respectively, as otherwise $\pOracleAstRes$ would not have a matching conversations for a final key with $\qOracleAstRes$ and the game would be aborted.

 Distinguish Game$_b$ 7 + 4$\ell^{\ast}$ from Game 6 + 4$\ell^{\ast}$ implies an algorithm breaking the security of the decisional Diffie-Hellman problem, thus
 \begin{equation}
  |\Adv_{7 + 4\ell^{\ast}} - \Adv_{6 + 4\ell^{\ast}}| \leq \epsilon_{\ddh}
 \end{equation}%
%
%
 \textbf{Game$_b$ 8 + 4$\ell^{\ast}$.} In this game we make use of the fact that the premaster secret $\widetilde{pms_{k}}$ of $\pOracleAstRes$ and $\qOracleAstRes$ is chosen uniformly random. We thus replace the value $ms_{k} = \PRF(\widetilde{pms_{k}}, \NONC)$ with a random value $\widetilde{ms_{k}}$.

 Distinguish Game$_b$ 8 + 4$\ell^{\ast}$ from Game$_b$ 7 + 4$\ell^{\ast}$ implies an algorithm breaking the security of the pseudo random function $\PRF$, thus
 \begin{equation}
  |\Adv_{8 + 4\ell^{\ast}} - \Adv_{7 + 4\ell^{\ast}}| \leq \epsilon_{\prf}
 \end{equation}%
%
%
 \textbf{Game$_b$ 9 + 4$\ell^{\ast}$} In this game we replace the all function $\PRF(\widetilde{ms_{k}}, \cdot)$ used by $\pOracleAstRes$ and $\qOracleAstRes$ with a random function $F_{\widetilde{ms_{k}}}$. Of course the same random function is used for both oracles $\pOracleAstRes$ and $\qOracleAstRes$. Distinguishing Game$_b$ 9 + 4$\ell^{\ast}$ from Game$_b$ 8 + 4$\ell^{\ast}$ again implies an algorithm breaking the security of the pseudo random function $\PRF$.
 \begin{equation}
  |\Adv_{9 + 4\ell^{\ast}} - \Adv_{8 + 4\ell^{\ast}}| \leq \epsilon_{\prf}
 \end{equation}%
%
%
 \textbf{Game$_b$ 10 + 4$\ell^{\ast}$.} Now we use that the key $k_{\pindexell}$ and $k_{\qindexell}$ which is independent of all QUIC handshake messages.

 In this game we construct a simulator $\mathcal{B}$ that uses a RSACCE attacker $\mathcal{A}$ to break the security of the underlying $\sLHAE$ secure symmetric encryption scheme. By assumption, the simulator $\mathcal{B}$ is given access to an encryption oracle $\Encrypt$ and a decryption oracle $\Decrypt$. $\mathcal{B}$ embeds that $\sLHAE$ experiment by simply forwarding all $\Encrypt(\pOracleAstRes,\cdot)$ queries to $\Encrypt$, and all $\Decrypt(\qOracleAstRes,\cdot)$ queries to $\Decrypt$.

 Observe that the values generated in this game are exactly distributed as in the previous game. We thus have
 \begin{equation}
  \Adv_{10 + 4\ell^{\ast}} = \Adv_{9 + 4\ell^{\ast}}
 \end{equation}%
 If $\mathcal{A}$ outputs a quad $\pindexout$, then $\mathcal{B}$ forwards $b^{\prime}$ to the $\sLHAE$ challenger. Otherwise it outputs a random bit. Since the simulator essentially relays all messages it is easy to see that an attacker $\mathcal{A}$ having advantage $\epsilon^{\prime}$ yields an attacker $\mathcal{B}$ against the $\LHAE$ security of the encryption scheme with success probability at least $1/2 + \epsilon^{\prime}$.

 Since by assumption any attacker has advantage at most $\epsilon_{\LHAE}$ in breaking the $\LHAE$ security of the symmetric encryption scheme, we have
 \begin{equation}
  \Adv_{10 + 4\ell^{\ast}} \leq \frac{1}{2} + \epsilon_{\LHAE}.
 \end{equation}%
\end{proof}