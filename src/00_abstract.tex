\begin{abstract}
We study the security of Quick UDP Internet Connections (QUIC for short) --
an experimental transport layer network protocol recently developed by Google
-- and show some security concern,
when abbreviated handshakes, aka ``resumptions", are established.
To explain our concern, we propose a new security model,
extending server-only authenticated and channel confidentiality establishment (SACCE),
so that authentication and channel confidentiality can be evaluated
including abbreviated handshake (resumption) sessions.
We then show that QUIC meets the weaker version of our security notion,
but not the stronger one.
On one hand, QUIC with an optional client encrypted tag value (CETV) mechanism,
satisfies the stronger one.
We finally present a more efficient protocol with the spirit of QUIC,
because QUIC with CETV increases communication and computational costs.
\end{abstract}