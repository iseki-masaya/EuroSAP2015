\documentclass[conference]{IEEEtran}
\usepackage[cmex10]{amsmath}
\usepackage{url}

%% for debug
\usepackage{color}

\newtheorem{definition}{Definition}
\newtheorem{lemma}{Lemma}
\newtheorem{note}{Note}
\newtheorem{remark}{Remark}
\newtheorem{theorem}{Theorem}
\newtheorem{proof}{Proof}

\newcounter{nombre}
\renewcommand{\thenombre}{\arabic{nombre}}
\newcommand{\prob}[1][]{\refstepcounter{nombre}\thenombre}

% %2000/9/27�@by E. Fujisaki
% Def., Thm., Lemma, etc.

\spnewtheorem{assumption}[theorem]{Assumption}{\bfseries}{\itshape}

%% \qqed
%-- proof-endings 
\def\blackslug
{\hbox{\hskip 1pt\vrule width 8pt height 8pt depth 1.5pt\hskip 1pt}}
\def\qed{\quad\blackslug\lower 8.5pt\null\par}
\def\qqed{$\Box$}

% Proof and Sketch of Proof
%\newenvironment{proof}{\par{\textbf{Proof.}} 
%\smallskip}{\nopagebreak[4]\par} 

\newenvironment{sproof}{\par{\textbf{Proof (Sketch).}} 
\smallskip}{\nopagebreak[4]\par} 


\usepackage{amsmath,amssymb,mathrsfs}
%\usepackage{txfonts}
%\usepackage{epic,eepic}
\long\def\comment#1{}

\NeedsTeXFormat{LaTeX2e}
% \newcommand*{\keywords}[1]{\par\addvspace\baselineskip\noindent\keywordname\enspace\ignorespaces#1}
% MagicWand
%\begin{enumerate}
%\magicwand
%\item ..
%\end{enumerate}
\def\magicwand{\itemsep=0pt\parskip=0pt\topskip=0pt}
% For commentout
\newcommand*{\ignore}[1]{}
%%%%%%%%%%%%%%%%%%%%%%%%%%%%
% Utilities
%%%%%%%%%%%%%%%%%%%%%%%%%%%%
\newcommand*{\etal}{{et~al.}}
\newcommand*{\hs}[1]{\hspace*{ #1 mm}}
\newcommand*{\vs}[1]{\vspace*{ #1 mm}}

%%%%%%%%%%
%Def
%%%%%%%%%%
\newcommand*{\defeq}{\stackrel{def}{=}}
\newcommand*{\checks}{\stackrel{\rm ?}{=}}
\newcommand{\sets}{:=}
%\newcommand*{\eqdef}{\stackrel{def}{=}}

%%%%%%%%%%%%%%%%%%%%%%%%%%%%%%
% Math:
%%%%%%%%%%%%%%%%%%%%%%%%%%%%%%
\def\vector#1{\mbox{\boldmath $#1$}}
%\usepackage[b]{esvect} %for \vv
\newcommand*{\vv}[1]{\overrightarrow{#1}}
%
%\newcommand*{\VEC}[1]{\check{\vec{#1}}}
%\newcommand*{\mat}[1]{\boldsymbol{#1}}
%\newcommand*{\func}[1]{\mathop{\mathrm{#1}}\nolimits}
%\newcommand*{\pol}[1]{\mathbf{#1}}
%
%----------------------------------
% Sets
%----------------------------------
\newcommand*{\Nset}{\mathbb{N}}
\newcommand*{\Zset}{\mathbb{Z}}
\newcommand*{\Qset}{\mathbb{Q}}
\newcommand*{\Rset}{\mathbb{R}}
\newcommand*{\Tset}{\mathbb{T}}
\newcommand*{\Cset}{\mathbb{C}}
\newcommand*{\Fset}{\mathbb{F}}
\newcommand*{\Gset}{\mathbb{G}}
\newcommand*{\GF}{\func{GF}}
\newcommand*{\Z}{\mathbb{Z}}
\newcommand*{\N}{\mathbb{N}}
\newcommand*{\R}{\mathbb{R}}

%----------------------------------
%complexity
%----------------------------------
\newcommand{\ZPP}{\mathcal{ZPP}}
\newcommand{\BPP}{\mathcal{BPP}}
\newcommand{\DP}{\mathcal{P}}
\newcommand{\NP}{\mathcal{NP}}
\newcommand{\IP}{\mathcal{IP}}
\newcommand{\AM}{\mathcal{AM}}

%---------------------------------
% Algebraic Sets
%---------------------------------
\newcommand*{\F}{\mathbb{F}}
\newcommand*{\G}{\mathbb{G}}
\newcommand*{\Gx}{{\G}^{\times}}
\newcommand*{\ZnZ}[1]{{\Z}/{#1}{\Z}} %%%% usage $\ZnZ{p3}$
\newcommand*{\ZnZx}[1]{({\Z}/{#1}{\Z})^{\times}}
\newcommand*{\Fx}[1]{{\F}_{#1}^{\times}}
%\newcommand*{\Fx}{{\F}^{\times}}
\newcommand*{\Zp}{{\Z}/p{\Z}}
\newcommand*{\Zpt}{({\Z}/p{\Z})^{+}}
\newcommand*{\Rpt}{{\R}^{+}}
\newcommand*{\Zpx}{({\Z}/p{\Z})^{\times}}
\newcommand*{\Zq}{{\Z}/q{\Z}}
\newcommand*{\Zqx}{({\Z}/q{\Z})^{\times}}
\newcommand*{\Zpp}{{\Z}/p^{2}{\Z}}
\newcommand*{\Zppx}{({\Z}/p^{2}{\Z})^{\times}}
\newcommand*{\Zpq}{{\Z}/{pq}{\Z}}
\newcommand*{\Zpqx}{({\Z}/{pq}{\Z})^{\times}}
\newcommand*{\Fp}{{\F}_{p}}
\newcommand*{\Fpx}{{\F}_{p}^{\times}}
\newcommand*{\Fqx}{{\F}_{q}^{\times}}
\newcommand*{\Fq}{{\F}_{q}}
\newcommand*{\Zn}{{\Z}/n{\Z}}
\newcommand*{\Znx}{({\Z}/n{\Z})^{\times}}
\newcommand*{\cycl}[1]{\langle{#1}\rangle} %%%% usage $\cycl{g}$
\newcommand*{\angles}[1]{\langle{#1}\rangle}
\newcommand*{\isom}{\cong} %%% isomorphism
\newcommand*{\ord}[1]{\ensuremath{\mathsf{order}({#1})}}


%%%%%%%%%%%%%%%%%%%%%%%%%%%%%%%%%%%%%%%%%%%%%%%%%%%%%%%%%%%%
%========================================================
% Distribution and Indistinguishability
%========================================================
\newcommand*{\View}{\mathop{\mathrm{View}}\nolimits}
\newcommand*{\view}{\ensuremath{\mathsf{view}}}
%\newcommand*{\View}{\textsc{View}}
\newcommand*{\Dist}{\ensuremath{\mathsf{Dist}}}
\newcommand*{\dist}{\ensuremath{\mathsf{dist}}}
\newcommand*{\indp}{\stackrel{\mathrm p}{\approx}}
\newcommand*{\inds}{\stackrel{\mathrm s}{\approx}}
\newcommand*{\indc}{\stackrel{\mathrm c}{\approx}}

%sampling
%\newcommand*{\inr}{\in_{\mbox{\tiny R}}}
\newcommand*{\getsr}{\gets_{\mbox{\tiny R}}}
\newcommand*{\getsd}{\xleftarrow{\$}}
\newcommand*{\bits}{\{0,1\}}
\newcommand{\rsets}{\stackrel{\scriptscriptstyle{\sf R}}{\leftarrow}}
\newcommand{\usets}{\stackrel{\scriptscriptstyle{\sf U}}{\leftarrow}}

%%% ensembles
\newcommand*{\ensembleX}{\mathcal{X}}
\newcommand*{\ensembleY}{\mathcal{Y}}
%%%
%%%%%%%%%%%%%%%%%%%%%%%%%%%%%%%%%%%%%%%%%%%%%%%%%%%%%%%%%%%%%%%%%%

%----------------------------------
% Operations
%----------------------------------
\newcommand*{\lsb}{\mathop{\mathsf{lsb}}\nolimits}
\newcommand*{\msb}{\mathop{\mathsf{msb}}\nolimits}
\newcommand*{\round}[1]{\mathrm{round}\left( #1 \right)}
\newcommand*{\frc}[1]{\mathrm{frc}\left( #1 \right)}
\newcommand*{\frcq}[1]{\mathrm{frc}_{q}\left( #1 \right)}
\newcommand*{\near}[1]{\left\lfloor{#1}\right\rceil}
\newcommand*{\floor}[1]{\left\lfloor{#1}\right\rfloor}
\newcommand*{\ceil}[1]{\left\lceil{#1}\right\rceil}
\newcommand*{\abs}[1]{\left\lvert{#1}\right\rvert}  %% |x|
\newcommand*{\leng}[1]{\left\lvert{#1}\right\rvert} %% |x|
\newcommand*{\card}[1]{\left\lvert{#1}\right\rvert} %{\#{#1}}
\newcommand*{\norm}[1]{\left\lVert{#1}\right\rVert}
\newcommand*{\ideal}[1]{\langle{#1}\rangle}
\newcommand*{\diag}{\mathop{\mathrm{diag}}\nolimits}
\newcommand*{\concat}{\mathbin{\|}} %concatination
%
%\newcommand*{\eqeq}{\stackrel{\mathrm{?}}{=}}

%%%%%%%%%%%%%%%%%%%%%%%%%%%%%%%%%%%%%%%%%%%%%%%%%%%%%%%
%% parenthesis
%
\newcommand{\pare}[1]{\left( #1 \right)}
\newcommand{\Pare}[1]{\Bigl( #1 \Bigr)}
\newcommand{\PARE}[1]{\Biggl( #1 \Biggr)}

%% blacket
\newcommand{\brak}[1]{\left[ #1 \right]}
\newcommand{\Brak}[1]{\Bigl[ #1 \Bigr]}
\newcommand{\BRAK}[1]{\Biggl[ #1 \Biggr]}

%% brace
\newcommand*{\brae}[1]{\left\{ #1 \right\}}
\newcommand*{\Brae}[1]{\Bigl\{ #1 \Bigr\}}
\newcommand*{\BRAE}[1]{\Biggl\{ #1 \Biggr\}}

%% ceiling
\newcommand{\ce}[1]{\lceil #1 \rceil}
%\newcommand{\ceil}[1]{\left\lceil #1 \right\rceil}
\newcommand{\Ceil}[1]{\Bigl\lceil #1 \Bigr\rceil}
\newcommand{\CEIL}[1]{\Biggl\lceil #1 \Biggr\rceil}

%% flooring
\newcommand{\fl}[1]{\lfloor #1 \rfloor}
%\newcommand{\floor}[1]{\left\lfloor #1 \right\rfloor}
\newcommand{\Floor}[1]{\Bigl\lfloor #1 \Bigr\rfloor}
\newcommand{\FLOOR}[1]{\Biggl\lfloor #1 \Biggr\rfloor}

%%%%%%%%%%%%%%%%%%%%%%%%%%%%%%%%%%%%%%%%%%%%%%%%%%%%%%



%%%
\if0
\newcommand*{\Exp}{\mathop{\mathrm{Exp}}}
\newcommand*{\EF}{\mathop{\mathrm{EF}}\nolimits}
\newcommand*{\Rot}{\mathop{\mathrm{Rot}}\nolimits}
\newcommand*{\Rotf}{\Rot_{\pol{f}}}
\newcommand*{\rot}{\mathop{\mathrm{rot}}\nolimits}
\newcommand*{\rotf}{\rot_{\pol{f}}}
\newcommand*{\ROT}{\mathop{\mathnormal{ROT}}\nolimits}
\newcommand*{\rev}{\mathop{\mathrm{rev}}\nolimits}
\newcommand*{\rec}{\mathop{\mathrm{rec}}\nolimits}
%\newcommand*{\conc}{\circ}
%\newcommand*{\conv}{\otimes}
%\newcommand*{\concat}{\mathbin{\|}}
\def\mapstofill@{\arrowfill@{\mapstochar\relbar}\relbar\rightarrow}
\newcommand*\xmapsto[2][]{\ext@arrow 0395\mapstofill@{#1}{#2}}
%\newcommand*{\eqdef}{\stackrel{def}{=}}
%\newcommand*{\eqeq}{\stackrel{\mathrm{?}}{=}}
%\newcommand*{\inr}{\in_{\mbox{\tiny R}}}
%\newcommand*{\getsr}{\gets_{\mbox{\tiny R}}}
%\newcommand*{\bits}{\{0,1\}}
%\newcommand*{\poly}{\mathop{\mathrm{poly}}\nolimits}
\newcommand*{\volume}{\mathop{\mathrm{vol}}\nolimits}
\newcommand*{\diam}{\mathop{\mathrm{diam}}\nolimits}
\newcommand*{\Part}{\mathop{\mathrm{part}}\nolimits}
\newcommand*{\Span}{\mathop{\mathrm{span}}\nolimits}
\newcommand*{\Round}{\mathop{\mathrm{round}}\nolimits}
\newcommand*{\View}{\mathop{\mathrm{View}}\nolimits}
\newcommand*{\Hom}{\mathop{\mathrm{Hom}}\nolimits}
\newcommand*{\Ker}{\mathop{\mathrm{Ker}}\nolimits}
\newcommand*{\Img}{\mathop{\mathrm{Im}}\nolimits}
\newcommand*{\rank}{\mathop{\mathrm{rank}}\nolimits}
\newcommand*{\SD}[2]{\Delta\left(#1,#2\right)}
\def\tr#1{\mathord{\mathopen{{\vphantom{#1}}^t}#1}}
\fi
%%%

%%%%%%%%%%%%%%%%%%%%%%%%%%%%
% Utilities
%%%%%%%%%%%%%%%%%%%%%%%%%%%%
\newcommand*{\descr}{\ensuremath{\mathsf{descr}}}
%\newcommand*{\desc}[1]{\left\langle{#1}\right\rangle}
\newcommand*{\keys}[1]{\mathit{#1}}
\newcommand*{\orac}[1]{\text{\sc #1}}
\newcommand*{\scheme}[1]{\ensuremath{\mathsf{#1}}}
\newcommand*{\algo}[1]{\ensuremath{\mathsf{#1}}}
\newcommand*{\events}[1]{\mathrm{#1}}
\newcommand*{\attack}[1]{\ensuremath{\mathsf{#1}}}
\newcommand*{\ahyph}{\attack{\mathchar`-}}
\newcommand*{\security}[1]{\ensuremath{\textsc{#1}}}
\newcommand*{\shyph}{\security{-}}
%----------------------------------
%Index/parameter
%----------------------------------
\newcommand*{\Index}{\mathcal{I}}
\newcommand*{\spar}{\kappa}
%---------------------------------
%advantages, experiments
%---------------------------------
\newcommand*{\Adv}{\ensuremath{\mathsf{Adv}}}
\newcommand*{\Succ}{\ensuremath{\mathsf{Succ}}}
\newcommand*{\EXEC}{\ensuremath{\mathsf{EXEC}}}
\newcommand*{\Expt}{\ensuremath{\mathsf{Expt}}}
%---------------------------------
%growth of functions
%---------------------------------
\newcommand*{\order}{{\frak o}} %%% o()
\newcommand*{\poly}{\mathop{\mathrm{poly}}\nolimits}
\newcommand*{\negl}{\ensuremath{\mathsf{negl}}}
\newcommand*{\nonnegl}{\mbox\ensuremath{\mathsf{non-negl}}}

%-------------------------------------------
%entropy
%-------------------------------------------
\newcommand*{\minH}{{H}_{\infty}}
\newcommand*{\aveminH}{\widetilde{H}_{\infty}}
\newcommand{\expect}{\mathop{\mathbf{E}}\displaylimits}

%--------------------------
%Standard Protocols
%-------------------------
%generator
\newcommand*{\Gen}{\algo{Gen}}
\newcommand*{\KGen}{\algo{KGen}}
\newcommand*{\EGen}{\algo{EGen}}
\newcommand*{\SGen}{\algo{SGen}}
%\newcommand*{\CRSGen}{\algo{CRSGen}}

%space
\newcommand*{\KSP}{\ensuremath{\mathsf{KSP}}}
\newcommand*{\MSP}{\ensuremath{\mathsf{MSP}}}
\newcommand*{\COIN}{\ensuremath{\mathsf{COIN}}}

%enc
\newcommand{\Enc}{\algo{Enc}}
\newcommand{\Dec}{\algo{Dec}}
\newcommand*{\K}{\mathbf{K}}
\newcommand*{\E}{\mathbf{E}}
\newcommand*{\D}{\mathbf{D}}

%sig
\newcommand{\Sig}{\algo{Sig}}
\newcommand{\Sign}{\algo{Sign}}
\newcommand{\Vrfy}{\algo{Vrfy}}

%otsig
\newcommand{\otKGen}{\algo{otKGen}}
\newcommand{\otSign}{\algo{otSign}}
\newcommand{\otsig}{\algo{otsig}}
\newcommand{\otVrfy}{\algo{otVrfy}}

%---------------------------------------
% Commitment, WI, and ZK, IP
%\newcommand{\Rewind}{{\sf Rewind}}
%---------------------------------------
\newcommand*{\ip}[2]{\langle{#1},{#2}\rangle}
\newcommand*{\Com}{\algo{Com}}
\newcommand*{\com}{\algo{com}}
\newcommand*{\Decom}{\algo{Decom}}
\newcommand*{\decom}{\algo{decom}}
\newcommand*{\vrfy}{\algo{vrfy}}
\newcommand*{\accept}{\algo{accept}}

%-------------------------------------------
%chameleon hash function/trapdoor commitment
%-------------------------------------------
\newcommand*{\CH}{\mathcal{CH}}
\newcommand*{\CHGen}{\algo{CHGen}}
\newcommand*{\CHEval}{\algo{CHEval}}
\newcommand*{\CHColl}{\algo{CHColl}}
\newcommand*{\TCom}{\algo{TCom}}
\newcommand*{\TColl}{\algo{TColl}}

\newcommand{\HashH}{\mathcal{H}}
\newcommand{\TCH}{\mathsf{TCH}}
\newcommand{\PRF}{\mathsf{PRF}}

%----------------------------------
% Standard Security Notions
%----------------------------------

\newcommand*{\Goal}{\security{Goal}}
\newcommand*{\OT}{\security{OT}}
\newcommand*{\IND}{\security{IND}}
\newcommand*{\INDPA}{\security{INDPA}}
\newcommand*{\OW}{\security{OW}}
\newcommand*{\NM}{\security{NM}}
\newcommand*{\SNM}{\security{SNM}}
\newcommand*{\CNM}{\security{CNM}}
\newcommand*{\EUF}{\security{EUF}}
\newcommand*{\sEUF}{\security{sEUF}}
%
\newcommand*{\ATK}{\security{ATK}}
\newcommand*{\CPA}{\security{CPA}}
\newcommand*{\CCA}{\security{CCA}}
\newcommand*{\CCAa}{\security{CCA1}}
\newcommand*{\CCAb}{\security{CCA2}}
\newcommand*{\CMA}{\security{CMA}}
%
\newcommand*{\goal}{\attack{goal}}
\newcommand*{\ot}{\attack{ot}}
\newcommand*{\sot}{\attack{sot}}
\newcommand*{\ow}{\attack{ow}}
%
\newcommand*{\ind}{\attack{ind}}
\newcommand*{\indpa}{\attack{indpa}}
\newcommand*{\nm}{\attack{nm}}
\newcommand*{\snm}{\attack{snm}}
\newcommand*{\cnm}{\attack{cnm}}
%
\newcommand*{\sig}{\attack{sig}}
\newcommand*{\euf}{\attack{euf}}
\newcommand*{\seuf}{\attack{seuf}}
%
\newcommand*{\atk}{\attack{atk}}
\newcommand*{\cpa}{\attack{cpa}}
\newcommand*{\cca}{\attack{cca}}
\newcommand*{\ccaa}{\attack{cca1}}
\newcommand*{\ccab}{\attack{cca2}}
%
\newcommand*{\cma}{\attack{cma}}
%----------------------------------
%----------------------------------
%One-Way Function, Hardcore
\newcommand*{\owf}{\attack{owf}}
\newcommand*{\hc}{\attack{hc}}

%Collision Resistance
\newcommand*{\clr}{\attack{cr}}

%PRG, PRF
\newcommand*{\prg}{\attack{prg}}
\newcommand*{\prf}{\attack{prf}}
% OW
\newcommand*{\owatk}{\ow\ahyph\atk}
\newcommand*{\owcpa}{\ow\ahyph\cpa}
\newcommand*{\owccaa}{\ow\ahyph\ccaa}
\newcommand*{\owccab}{\ow\ahyph\ccab}
%
\newcommand*{\OWATK}{\OW\shyph\ATK}
\newcommand*{\OWCPA}{\OW\shyph\CPA}
\newcommand*{\OWCCAa}{\OW\shyph\CCAa}
\newcommand*{\OWCCAb}{\OW\shyph\CCAb}
% IND
\newcommand*{\indatk}{\ind\ahyph\atk}
\newcommand*{\indcpa}{\ind\ahyph\cpa}
\newcommand*{\indcca}{\ind\ahyph\cca}
\newcommand*{\indccaa}{\ind\ahyph\ccaa}
\newcommand*{\indccab}{\ind\ahyph\ccab}
%
\newcommand*{\INDATK}{\IND\shyph\ATK}
\newcommand*{\INDCPA}{\IND\shyph\CPA}
\newcommand*{\INDCCA}{\IND\shyph\CCA}
\newcommand*{\INDCCAa}{\IND\shyph\CCAa}
\newcommand*{\INDCCAb}{\IND\shyph\CCAb}
% NM
\newcommand*{\nmatk}{\nm\ahyph\atk}
\newcommand*{\nmcpa}{\nm\ahyph\cpa}
\newcommand*{\nmccaa}{\nm\ahyph\ccaa}
\newcommand*{\nmccab}{\nm\ahyph\ccab}
%
\newcommand*{\NMATK}{\NM\shyph\ATK}
\newcommand*{\NMCPA}{\NM\shyph\CPA}
\newcommand*{\NMCCAa}{\NM\shyph\CCAa}
\newcommand*{\NMCCAb}{\NM\shyph\CCAb}
% SNM
\newcommand*{\snmatk}{\snm\ahyph\atk}
\newcommand*{\snmcpa}{\snm\ahyph\cpa}
\newcommand*{\snmccaa}{\snm\ahyph\ccaa}
\newcommand*{\snmccab}{\snm\ahyph\ccab}
%
\newcommand*{\SNMATK}{\SNM\shyph\ATK}
\newcommand*{\SNMCPA}{\SNM\shyph\CPA}
\newcommand*{\SNMCCAa}{\SNM\shyph\CCAa}
\newcommand*{\SNMCCAb}{\SNM\shyph\CCAb}
% CNM
\newcommand*{\cnmatk}{\cnm\ahyph\atk}
\newcommand*{\cnmcpa}{\cnm\ahyph\cpa}
\newcommand*{\cnmccaa}{\cnm\ahyph\ccaa}
\newcommand*{\cnmccab}{\cnm\ahyph\ccab}
%
\newcommand*{\CNMATK}{\CNM\shyph\ATK}
\newcommand*{\CNMCPA}{\CNM\shyph\CPA}
\newcommand*{\CNMCCAa}{\CNM\shyph\CCAa}
\newcommand*{\CNMCCAb}{\CNM\shyph\CCAb}

%Sign
\newcommand*{\eufcma}{\euf\ahyph\cma}
\newcommand*{\seufcma}{\seuf\ahyph\cma}
\newcommand*{\EUFCMA}{\EUF\shyph\CMA}
\newcommand*{\sEUFCMA}{\sEUF\shyph\CMA}



%\newcommand{\Client}{\mathcal{C}}
%\newcommand{\Server}{\mathcal{S}}
%\newcommand*{\accept}{\algo{accept}}
\newcommand*{\reject}{\algo{reject}}
\newcommand*{\peer}{\textsc{peer}}
\newcommand*{\preaccept}{\algo{preaccept}}

%%%%%%%%%%%%%%%%%%%%%%%%%%%%%%%%%%%%%%%%%%%%%%%%%
%%%%%%%%%%%%%%%%%%%%%%%%%%%%%%%%%%%%%%%%%%%%%%%%%
%% The function of the protocol in handshake
%%%%%%%%%%%%%%%%%%%%%%%%%%%%%%%%%%%%%%%%%%%%%%%%%
%%%%%%%%%%%%%%%%%%%%%%%%%%%%%%%%%%%%%%%%%%%%%%%%%
\newcommand*{\scfgGen}{\ensuremath{\mathsf{scfgGen}}}
\newcommand*{\inchoateCHLO}{\ensuremath{\mathsf{inchoateCHLO}}}
\newcommand*{\initialCHLO}{\ensuremath{\mathsf{initialCHLO}}}
\newcommand*{\initialSHLO}{\ensuremath{\mathsf{initialSHLO}}}
\newcommand*{\checkQuery}{\ensuremath{\mathsf{checkQuery}}}
\newcommand*{\receiveSHLO}{\ensuremath{\mathsf{receiveSHLO}}}
\newcommand*{\receiveCHLO}{\ensuremath{\mathsf{receiveCHLO}}}
\newcommand*{\checkSCFG}{\ensuremath{\mathsf{checkSCFG}}}
\newcommand*{\checkSHLO}{\ensuremath{\mathsf{checkSHLO}}}
\newcommand*{\makeSTK}{\ensuremath{\mathsf{makeSTK}}}
\newcommand*{\CHLO}{\ensuremath{\mathsf{CHLO}}}
\newcommand*{\REJ}{\ensuremath{\mathsf{REJ}}}
\newcommand*{\SHLO}{\ensuremath{\mathsf{SHLO}}}
\newcommand*{\zCHLO}{\ensuremath{\mathsf{zCHLO}}}
\newcommand{\getPK}{\mathsf{getPK}}
\newcommand{\Hash}{\mathsf{Hash}}


%%%%%%%%%%%%%%%%%%%%%%%%%%%%%%%%%%%%%%%%%%%%%%%%%
%%%%%%%%%%%%%%%%%%%%%%%%%%%%%%%%%%%%%%%%%%%%%%%%%
%% The function of the protocol in key deriving
%%%%%%%%%%%%%%%%%%%%%%%%%%%%%%%%%%%%%%%%%%%%%%%%%
%%%%%%%%%%%%%%%%%%%%%%%%%%%%%%%%%%%%%%%%%%%%%%%%%
\newcommand*{\getKey}{\ensuremath{\mathsf{getKey}}}
\newcommand*{\extractKey}{\ensuremath{\mathsf{extractKey}}}

%%%%%%%%%%%%%%%%%%%%%%%%%%%%%%%%%%%%%%%%%%%%%%%%%
%%%%%%%%%%%%%%%%%%%%%%%%%%%%%%%%%%%%%%%%%%%%%%%%%
%% The function of the protocol in wirelayer
%%%%%%%%%%%%%%%%%%%%%%%%%%%%%%%%%%%%%%%%%%%%%%%%%
%%%%%%%%%%%%%%%%%%%%%%%%%%%%%%%%%%%%%%%%%%%%%%%%%
\newcommand*{\getIV}{\ensuremath{\mathsf{getIV}}}
\newcommand*{\pak}{\ensuremath{\mathsf{pak}}}
\newcommand*{\processPacket}{\ensuremath{\mathsf{processPacket}}}

%%%%%%%%%%%%%%%%%%%%%%%%%%%%%%%%%%%%%%%%%%%%%%%%%
%%%%%%%%%%%%%%%%%%%%%%%%%%%%%%%%%%%%%%%%%%%%%%%%%
%% The operation in the protocol
%%%%%%%%%%%%%%%%%%%%%%%%%%%%%%%%%%%%%%%%%%%%%%%%%
%%%%%%%%%%%%%%%%%%%%%%%%%%%%%%%%%%%%%%%%%%%%%%%%%
\newcommand*{\SIG}{\ensuremath{\mathsf{SIG}}}
% \newcommand*{\Gen}{\ensuremath{\mathsf{Gen}}}   These protocol is defined in other macro.
% \newcommand*{\Sign}{\ensuremath{\mathsf{Sign}}}
\newcommand*{\Vfy}{\ensuremath{\mathsf{Vfy}}}
\newcommand*{\SE}{\ensuremath{\mathsf{SE}}}
% \newcommand*{\Enc}{\ensuremath{\mathsf{Enc}}}   These protocol is defined in other macro.
% \newcommand*{\Dec}{\ensuremath{\mathsf{Dec}}}
\newcommand*{\Init}{\ensuremath{\mathsf{Init}}}
% \newcommand*{\PRF}{\ensuremath{\mathsf{PRF}}}
\newcommand*{\abort}{\ensuremath{\mathsf{abort}}}
\newcommand*{\cert}{\ensuremath{\mathsf{cert}}}

%%%%%%%%%%%%%%%%%%%%%%%%%%%%%%%%%%%%%%%%%%%%%%%%%
%%%%%%%%%%%%%%%%%%%%%%%%%%%%%%%%%%%%%%%%%%%%%%%%%
%% The parameter in the protocol
%%%%%%%%%%%%%%%%%%%%%%%%%%%%%%%%%%%%%%%%%%%%%%%%%
%%%%%%%%%%%%%%%%%%%%%%%%%%%%%%%%%%%%%%%%%%%%%%%%%
\newcommand*{\SCFG}{\ensuremath{\mathsf{SCFG}}}
\newcommand*{\STK}{\ensuremath{\mathsf{STK}}}
\newcommand*{\NONC}{\ensuremath{\mathsf{NONC}}}
\newcommand*{\SCID}{\ensuremath{\mathsf{SCID}}}
\newcommand*{\cid}{\ensuremath{\mathsf{cid}}}
\newcommand*{\sqn}{\ensuremath{\mathsf{sqn}}}
\newcommand*{\ik}{\ensuremath{\mathsf{ik}}}
\newcommand*{\expy}{\ensuremath{\mathsf{expy}}}
\newcommand*{\iv}{\ensuremath{\mathsf{iv}}}
\newcommand*{\plaintext}{\ensuremath{\mathsf{plaintext}}}
\newcommand*{\doc}{\ensuremath{\mathsf{doc}}}
\newcommand*{\info}{\ensuremath{\mathsf{info}}}
\newcommand*{\duration}{\ensuremath{\mathsf{duration_t}}}
\newcommand*{\fin}{\ensuremath{\mathsf{fin}}}
\newcommand*{\mac}{\ensuremath{\mathsf{mac}}}
\newcommand*{\MAC}{\ensuremath{\mathsf{MAC}}}
\newcommand*{\return}{\ensuremath{\mathsf{return}}}
\newcommand*{\pInfo}{\ensuremath{\mathsf{pInfo}}}
\newcommand*{\init}{\ensuremath{\mathsf{init}}}
\newcommand*{\shareInfo}{\ensuremath{\mathsf{shareInfo}}}
\newcommand*{\strike}{\ensuremath{\mathsf{strike}}}
\newcommand*{\MsgCntC}[1]{\ensuremath{\mathsf{p_c^{#1}}}}
\newcommand*{\MsgCntS}[1]{\ensuremath{\mathsf{p_s^{#1}}}}

%%%%%%%%%%%%%%%%%%%%%%%%%%%%%%%%%%%%%%%%%%%%%%%%%
%%%%%%%%%%%%%%%%%%%%%%%%%%%%%%%%%%%%%%%%%%%%%%%%%
%% The number of oracle or party in the definition
%%%%%%%%%%%%%%%%%%%%%%%%%%%%%%%%%%%%%%%%%%%%%%%%%
%%%%%%%%%%%%%%%%%%%%%%%%%%%%%%%%%%%%%%%%%%%%%%%%%
\newcommand*{\nclient}{\ensuremath{n_c}}
\newcommand*{\nserver}{\ensuremath{n_s}}
\newcommand*{\noracle}{\ensuremath{n_o}}
\newcommand*{\nresumption}{\ensuremath{n_{\ell}}}

%%%%%%%%%%%%%%%%%%%%%%%%%%%%%%%%%%%%%%%%%%%%%%%%%
%%%%%%%%%%%%%%%%%%%%%%%%%%%%%%%%%%%%%%%%%%%%%%%%%
%% The oracle in the definition
%%%%%%%%%%%%%%%%%%%%%%%%%%%%%%%%%%%%%%%%%%%%%%%%%
%%%%%%%%%%%%%%%%%%%%%%%%%%%%%%%%%%%%%%%%%%%%%%%%%
\newcommand*{\pOracleAstFull}{\ensuremath{\pi^{p^{\ast}}_{i^{\ast},0}}}
\newcommand*{\qOracleAstFull}{\ensuremath{\pi^{q^{\ast}}_{j^{\ast},0}}}
\newcommand*{\cOracleAstFull}{\ensuremath{\pi^{c^{\ast}}_{i^{\ast},0}}}
\newcommand*{\sOracleAstFull}{\ensuremath{\pi^{s^{\ast}}_{j^{\ast},0}}}
\newcommand*{\pOracleAstOne}{\ensuremath{\pi^{p^{\ast}}_{i^{\ast},1}}}
\newcommand*{\qOracleAstOne}{\ensuremath{\pi^{q^{\ast}}_{j^{\ast},1}}}
\newcommand*{\cOracleAstOne}{\ensuremath{\pi^{c^{\ast}}_{i^{\ast},1}}}
\newcommand*{\sOracleAstOne}{\ensuremath{\pi^{s^{\ast}}_{j^{\ast},1}}}
\newcommand*{\pOracleAstRes}{\ensuremath{\pi^{p^{\ast}}_{i^{\ast},\ell^{\ast}}}}
\newcommand*{\qOracleAstRes}{\ensuremath{\pi^{q^{\ast}}_{j^{\ast},\ell^{\ast}}}}
\newcommand*{\cOracleAstRes}{\ensuremath{\pi^{c^{\ast}}_{i^{\ast},\ell^{\ast}}}}
\newcommand*{\sOracleAstRes}{\ensuremath{\pi^{s^{\ast}}_{j^{\ast},\ell^{\ast}}}}
\newcommand*{\pOracleAstEll}{\ensuremath{\pi^{p^{\ast}}_{i^{\ast},\ell}}}
\newcommand*{\qOracleAstEll}{\ensuremath{\pi^{q^{\ast}}_{j^{\ast},\ell}}}
\newcommand*{\sOracleAstEll}{\ensuremath{\pi^{s^{\ast}}_{i^{\ast},\ell}}}
\newcommand*{\cOracleAstEll}{\ensuremath{\pi^{c^{\ast}}_{j^{\ast},\ell}}}
\newcommand*{\pOracleAstResMinusOne}{\ensuremath{\pi^{p^{\ast}}_{i^{\ast},\ell^{\ast}-1}}}
\newcommand*{\qOracleAstResMinusOne}{\ensuremath{\pi^{q^{\ast}}_{j^{\ast},\ell^{\ast}-1}}}
\newcommand*{\sOracleAstResMinusOne}{\ensuremath{\pi^{s^{\ast}}_{i^{\ast},\ell^{\ast}-1}}}
\newcommand*{\cOracleAstResMinusOne}{\ensuremath{\pi^{c^{\ast}}_{j^{\ast},\ell^{\ast}-1}}}
\newcommand*{\pOracleAstEllPlusOne}{\ensuremath{\pi^{p^{\ast}}_{i^{\ast},\ell+1}}}
\newcommand*{\qOracleAstEllPlusOne}{\ensuremath{\pi^{q^{\ast}}_{j^{\ast},\ell+1}}}
\newcommand*{\sOracleAstEllPlusOne}{\ensuremath{\pi^{s^{\ast}}_{i^{\ast},\ell+1}}}
\newcommand*{\cOracleAstEllPlusOne}{\ensuremath{\pi^{c^{\ast}}_{j^{\ast},\ell+1}}}
\newcommand*{\pOracleAstX}{\ensuremath{\pi^{p^{\ast}}_{i^{\ast},x}}}
\newcommand*{\qOracleAstX}{\ensuremath{\pi^{q^{\ast}}_{j^{\ast},x}}}
\newcommand*{\sOracleAstX}{\ensuremath{\pi^{s^{\ast}}_{i^{\ast},x}}}
\newcommand*{\cOracleAstX}{\ensuremath{\pi^{c^{\ast}}_{j^{\ast},x}}}

%%%%%%%%%%%%%%%%%%%%%%%%%%%%%%%%%%%%%%%%%%%%%%%%%
%%%%%%%%%%%%%%%%%%%%%%%%%%%%%%%%%%%%%%%%%%%%%%%%%
%% The index in the definition
%%%%%%%%%%%%%%%%%%%%%%%%%%%%%%%%%%%%%%%%%%%%%%%%%
%%%%%%%%%%%%%%%%%%%%%%%%%%%%%%%%%%%%%%%%%%%%%%%%%
\newcommand*{\cIndexAstRes}{\ensuremath{(c^{\ast}, {i^{\ast}, \ell^{\ast})}}}
\newcommand*{\sIndexAstRes}{\ensuremath{(s^{\ast}, {j^{\ast}, \ell^{\ast})}}}
\newcommand*{\pIndexAstRes}{\ensuremath{(p^{\ast}, {i^{\ast}, \ell^{\ast})}}}
\newcommand*{\qIndexAstRes}{\ensuremath{(q^{\ast}, {j^{\ast}, \ell^{\ast})}}}
\newcommand*{\pIndexAstEll}{\ensuremath{(p^{\ast}, i^{\ast}, \ell)}}
\newcommand*{\qIndexAstEll}{\ensuremath{(q^{\ast}, i^{\ast}, \ell)}}
\newcommand*{\pindex}{\ensuremath{(p^{\ast}, i^{\ast}, \ell^{\ast})}}
\newcommand*{\qindex}{\ensuremath{(q^{\ast}, j^{\ast}, \ell^{\ast})}}
\newcommand*{\cindex}{\ensuremath{(c^{\ast}, i^{\ast}, \ell^{\ast})}}
\newcommand*{\sindex}{\ensuremath{(s^{\ast}, j^{\ast}, \ell^{\ast})}}
\newcommand*{\cindexell}{\ensuremath{(c^{\ast}, i^{\ast}, \ell)}}
\newcommand*{\sindexell}{\ensuremath{(s^{\ast}, j^{\ast}, \ell)}}
\newcommand*{\cindexellast}{\ensuremath{(c^{\ast}, i^{\ast}, \ell^{\ast})}}
\newcommand*{\sindexellast}{\ensuremath{(s^{\ast}, j^{\ast}, \ell^{\ast})}}
\newcommand*{\pindexell}{\ensuremath{(p^{\ast}, i^{\ast}, \ell)}}
\newcommand*{\qindexell}{\ensuremath{(q^{\ast}, j^{\ast}, \ell)}}
\newcommand*{\pindexellast}{\ensuremath{(p^{\ast}, i^{\ast}, \ell^{\ast})}}
\newcommand*{\qindexellast}{\ensuremath{(q^{\ast}, j^{\ast}, \ell^{\ast})}}
\newcommand*{\pindexX}{\ensuremath{(p^{\ast}, i^{\ast}, x)}}
\newcommand*{\qindexX}{\ensuremath{(q^{\ast}, j^{\ast}, x)}}
\newcommand*{\pindexzero}{\ensuremath{(p^{\ast}, i^{\ast}, 0)}}
\newcommand*{\qindexzero}{\ensuremath{(q^{\ast}, j^{\ast}, 0)}}
\newcommand*{\pindexout}{\ensuremath{(p^{\ast}, i^{\ast}, \ell^{\ast}, b^{\prime})}}

%%%%%%%%%%%%%%%%%%%%%%%%%%%%%%%%%%%%%%%%%%%%%%%%%
%%%%%%%%%%%%%%%%%%%%%%%%%%%%%%%%%%%%%%%%%%%%%%%%%
%% The assumption
%%%%%%%%%%%%%%%%%%%%%%%%%%%%%%%%%%%%%%%%%%%%%%%%%
%%%%%%%%%%%%%%%%%%%%%%%%%%%%%%%%%%%%%%%%%%%%%%%%%
% \newcommand*{\sig}{\ensuremath{\mathsf{sig}}}
\newcommand*{\ddh}{\ensuremath{\mathsf{ddh}}}
\newcommand*{\prfodh}{\ensuremath{\mathsf{prfodh}}}
\newcommand*{\sLHAE}{\ensuremath{\mathsf{LHAE}}}
\newcommand*{\LHAE}{\ensuremath{\mathsf{LHAE}}}

%%%%%%%%%%%%%%%%%%%%%%%%%%%%%%%%%%%%%%%%%%%%%%%%%
%%%%%%%%%%%%%%%%%%%%%%%%%%%%%%%%%%%%%%%%%%%%%%%%%
%% The property of RSACCE
%%%%%%%%%%%%%%%%%%%%%%%%%%%%%%%%%%%%%%%%%%%%%%%%%
%%%%%%%%%%%%%%%%%%%%%%%%%%%%%%%%%%%%%%%%%%%%%%%%%
\newcommand*{\rsaccesa}{\ensuremath{\mathsf{rsacce\shyph sa}}}
\newcommand*{\rsaccecb}{\ensuremath{\mathsf{rsacce\shyph cb}}}
\newcommand*{\rsaccecc}{\ensuremath{\mathsf{rsacce\shyph cc}}}

%%%%%%%%%%%%%%%%%%%%%%%%%%%%%%%%%%%%%%%%%%%%%%%%%
%%%%%%%%%%%%%%%%%%%%%%%%%%%%%%%%%%%%%%%%%%%%%%%%%
%% The operation of the adversary
%%%%%%%%%%%%%%%%%%%%%%%%%%%%%%%%%%%%%%%%%%%%%%%%%
%%%%%%%%%%%%%%%%%%%%%%%%%%%%%%%%%%%%%%%%%%%%%%%%%
\newcommand*{\Initiate}{\ensuremath{\mathsf{Initiate}}}
\newcommand*{\Resume}{\ensuremath{\mathsf{Resume}}}
\newcommand*{\Send}{\ensuremath{\mathsf{Send}}}
\newcommand*{\Reveal}{\ensuremath{\mathsf{Reveal}}}
\newcommand*{\Corrupt}{\ensuremath{\mathsf{Corrupt}}}
\newcommand*{\Encrypt}{\ensuremath{\mathsf{Encrypt}}}
\newcommand*{\Decrypt}{\ensuremath{\mathsf{Decrypt}}}

%%%%%%%%%%%%%%%%%%%%%%%%%%%%%%%%%%%%%%%%%%%%%%%%%
%%%%%%%%%%%%%%%%%%%%%%%%%%%%%%%%%%%%%%%%%%%%%%%%%
%% The client, server, and, party
%%%%%%%%%%%%%%%%%%%%%%%%%%%%%%%%%%%%%%%%%%%%%%%%%
%%%%%%%%%%%%%%%%%%%%%%%%%%%%%%%%%%%%%%%%%%%%%%%%%
\newcommand{\Client}{\mathcal{C}}
\newcommand{\Server}{\mathcal{S}}
\newcommand{\Party}{\mathcal{P}}
\newcommand{\crs}{\mathsf{crs}}
\renewcommand{\tag}{\mathsf{tag}}

\newcommand{\priv}{\mathsf{priv}}
\newcommand{\pub}{\mathsf{pub}}

\newcommand{\ch}{\mathsf{ch}}
\newcommand{\ans}{\mathsf{ans}}
\newcommand{\simSigma}{\mathsf{sim}\Sigma}
\newcommand{\hatGen}{\hat{\Gen}}
\newcommand{\hatSigma}{\hat{\Sigma}}
\newcommand{\hatsimSigma}{\mathsf{sim}\hat{\Sigma}}

\newcommand{\CS}{\algo{CS}}
\newcommand{\TC}{\algo{TC}}
\newcommand{\SSTC}{\algo{SSTC}}
\newcommand{\ssbind}{\mathsf{ss-bind}}

\newcommand{\Cc}{\algo{Cc}}
\newcommand{\Rc}{\algo{Rc}}
\newcommand{\Rv}{\algo{Rv}}
\newcommand{\FCRSGen}{\algo{FCRSGen}}
\newcommand{\fKGen}{\algo{fKGen}}
\newcommand{\fCc}{\algo{fCc}}
\newcommand{\fDc}{\algo{fDc}}

\newcommand{\Prover}{\algo{P}}
\newcommand{\Verifier}{\algo{V}}
\newcommand{\Simu}{\algo{S}}
\newcommand*{\CRSGen}{\algo{CRSGen}}

\newcommand{\dec}{\mathsf{dec}}

\newcommand{\mim}{\mathsf{mim}}
\newcommand{\simu}{\mathsf{sim}}

\newcommand{\ID}{\mathsf{ID}}
\newcommand{\cMiM}{\mathsf{cMiM}}

\newcommand{\PITM}[1]{{\langle}{#1}{\rangle}} %%%% usage $\PITM{g}$ 
\newcommand{\commit}{\mathsf{commit}}
\newcommand{\open}{\mathsf{open}}


%assumption
\newcommand*{\rsa}{\attack{rsa}}
\newcommand*{\dcr}{\attack{dcr}}
\newcommand*{\divisor}{\attack{divisor}}


%%%%%%%%%%%%%%%%
\newcommand{\mapright}[1]{\smash{\mathop{
 \hbox to 3cm{\rightarrowfill}}\limits^{#1 }}}
\newcommand{\mapleft}[1]{\smash{\mathop{
 \hbox to 3cm{\leftarrowfill}}\limits^{#1 }}}
\newcommand{\mapLeftRight}[1]{\smash{\mathop{
 \hbox to 3cm{\LeftRightarrowfill}}\limits^{#1 }}}
%%%%%%%%%%%%%%%%
\def\Leftarrowfill{$\m@th \mathord\Leftarrow \mkern-6mu
 \cleaders\hbox{$\mkern-2mu \mathord= \mkern-2mu$}\hfill
 \mkern-6mu \mathord=$}
%
\def\Rightarrowfill{$\m@th \mathord= \mkern-6mu
 \cleaders\hbox{$\mkern-2mu \mathord= \mkern-2mu$}\hfill
 \mkern-6mu \mathord\Rightarrow$}
%
\def\LeftRightarrowfill{$\m@th \mathord\Leftarrow \mkern-6mu
 \cleaders\hbox{$\mkern-2mu \mathord= \mkern-2mu$}\hfill
 \mkern-6mu \mathord\Rightarrow$}
%
\def\leftharpoondownfill{$\m@th \mathord\leftharpoondown \mkern-6mu
 \cleaders\hbox{$\mkern-2mu \mathord- \mkern-2mu$}\hfill
 \mkern-6mu \mathord-$}
%
\def\rightharpoonupfill{$\m@th \mathord- \mkern-6mu
 \cleaders\hbox{$\mkern-2mu \mathord- \mkern-2mu$}\hfill
 \mkern-6mu \mathord\rightharpoonup$}
%%%%%%%%%%%%%%%%%%%%




\begin{document}
\title{On the Security of QUIC}


% author names and affiliations
% use a multiple column layout for up to three different
% affiliations
\author{\IEEEauthorblockN{Masaya Iseki}
\IEEEauthorblockA{Tokyo Institute of Technology\\
Email: iseki.m.aa@m.titech.ac.jp}
\and
\IEEEauthorblockN{Eiichiro Fujisaki}
\IEEEauthorblockA{NTT Secure platform\\
Email: fujisaki.eiichiro@lab.ntt.co.jp}}
\maketitle

 %=====================================================
\begin{abstract}
%=====================================================
We study the security of Quick UDP Internet Connections
(QUIC for short) -- an experimental transport layer 
network protocol recently developed by Google.
Lychev et al.~\cite{LJBN15:QUIC} proposed a new security model, 
called Quick ACCE (QACCE), an extention of  
authenticated and channel confidentiality establishment (ACCE) model, and 
showed that QUIC is secure in the model. 
However, they also pointed out some vulnerability on QUIC, 
which may increase server loads easily and would possibly cause serious Distributed Denial of Service (DDoS) 
attacks. 
In this paper, 
we propose a new stronger security model, called  
resumable server-only ACCE (RSACCE).
Since the original QUIC does not meet our new security notion, 
we propose an enhanced QUIC protocol, without changing the spirit of QUIC, such as 
the two-key structure, server-only authentication, no server cash data, repeated usage of server-configuration (SCFG), 
and $1$-RTT full and $0$-RTT abbreviate handshakes.
The enhanced QUIC satisfies our strong security requirements under weaker assumptions  
and also improves the round complexity. 
Here the DDoS attacks mentioned above do not effectively work
against the enhanced QUIC protocol. 
\end{abstract}
 %=====================================================
\section{Introduction} \label{sec:intro}
%=====================================================
Quick UDP Internet Connections (QUIC for short) is a
new transport layer network protocol recently proposed
by Google \cite{QUIC,QUICDraft}, which is experimentally
implemented in Google Chrome.
The main purpose of developing QUIC is to provide an
alternative equivalence of TLS wrapping TCP, with much
reduced latency and better SPDY and HTTP/2 support.
Transport Layer Security (TLS) starts with a three-move
TCP handshake before initiating the TLS Handshake
Protocol.
In contrast, QUIC uses UDP and starts with its own
handshake, which reduces the total number of
interactions.
The cryptographic core of QUIC is specified in the QUIC
crypto protocol~\cite{QUIC:Crypto}, which consists of a
handshake protocol and a record layer protocol, as does
TLS.
Similarly to TLS, QUIC has two types of handshake
connections.
One is called a full handshake -- a handshake
``from scratch" between a client and a server.
The other is called an abbreviate handshake -- a
handshake which occurs when a client and a server have
once established a full handshake session and want to
establish a new session between them in an abbreviate way.
Unlike TLS, QUIC only supports the elliptic-curve
Diffie-Hellman key-exchange (ECDHE) cipher suites and
server authentication.
%
One of the good features of QUIC is that it can
establish an abbreviate session with $0$-RTT
connectivity overhead.
Namely, in the QUIC abbreviate handshake, a client can send
encrypted data to a server, concurrently with a new session.
We provide the abstract model of the full handshake and
abbreviate handshake protocols of QUIC in
Fig.~\ref{fig:quic_abst_1rtt}, ~\ref{fig:quic_abst_0rtt}.
By this property, an abbreviate handshake connection is also called
0-RTT connection and a full handshake connection is also called
1-RTT connection.

%=====================================================
\subsection{Prior Security Analyses and Some Security Concern} \label{sec:concern}
%=====================================================
To the best of our knowledge, there are only two
security analyses on QUIC~\cite{FG14:QUIC,LJBN15:QUIC}.
Both papers define new security models and show that
QUIC is secure in that model.
In~\cite{FG14:QUIC}, they formalized a secure
authenticated key-exchange as an extension of the
Bellare-Rogaway model~\cite{BR93:AKE} and analysed the
security of QUIC (with abbreviate handshakes).
However, the QUIC protocol analysed in \cite{FG14:QUIC}
is slightly different from the protocol given in the
source codes.
As described in Fig.~\ref{fig:quic_abst_1rtt},
~\ref{fig:quic_abst_0rtt}, the QUIC protocol makes a
server send a ciphertext (using authenticated
encryption) in the full handshake protocol, which cannot
preserve \textit{key-indistinguishability}.
Therefore, the authenticated and channel confidentiality
establishment (ACCE) model~\cite{JKSS12:ACCE} is more
suitable to analyse QUIC.
Another important security issue is that in~\cite{FG14:QUIC},
an adversary is allowed to send a ``test" query only to
a client oracle (to receive either a real session-key or
a random key from the client oracle), when a protocol
is server-only authenticated.
Apparently, the restriction is appropriate, because
an adversary can establish a session with a honest
server (due to the lack of client's certificate) to
share a session key.
However, if an abbreviate handshake is provided, we should consider
the attack that, after a honest client and a honest
server establish a full handshake session, an adversary
might hijack an abbreviate handshake session -- it might
impersonate the initial client and share a session key
with the server.
To protect the attack, we should allow an adversary to
send test queries to \textit{server} oracles in abbreviate
handshake sessions (including the full handshake session), as long
as the initial full handshake session is established
between a honest client and a honest server.
We can consider an attack: The adversary can
share a session key with the server and it can make the
server accept in an abbreviate handshake session. (Note that in a
full handshake session, it is a ``trivial" attack, because
an adversary can always do so.)
In~\cite{LJBN15:QUIC}, they formalized QACCE model which
is based on ACCE model and consider full handshakes and
abbreviate handshakes.
They also found that with replay attack on some
public parameters exchanged during the handshake, an
adversary could easily prevent QUIC from achieving
minimal latency advantages either by having it fall back
to TCP or by causing the client and server to have an
inconsistent view of their handshake leading to a failure
to complete the connection.
The adversary also can apply loads on the server using
these attacks.
Their security model QACCE does not consider these attacks
to prove the security of QUIC as it is.
These attacks are ruled out in the proposed security model.
On the other hand, in their security model QACCE, an adversary
is allowed to send a ``test" query to a server oracle.
However, they add a restriction that an adversary is allowed
to send it only the server oracle which has a matching
conversation with some client oracles.
An adversary needs to forge the query to make a server accept
and the server does not have a matching conversation with any
client oracle.
For this restriction, QACCE does not prevent the attack that
an adversary make a server accept.
In QUIC, a server does not make sure consistencies of a client
between 0-RTT connections and 1-RTT connections.
However, one of the attacks found in~\cite{LJBN15:QUIC} use this
property that a server does not make sure consistencies of a client
between 0-RTT connections and 1-RTT connections and the adversary
can establish the connection spoofing IP address.
This assist the adversary do Distributed Denial of Service
(DDoS) attack.
Our security model guarantees that only parties
that establish the initial full handshake session can
establish a new abbreviate handshake session.
This property mitigate one of the attacks found in
~\cite{LJBN15:QUIC}.
Our security model also prevent other attacks in
~\cite{LJBN15:QUIC}.
In QUIC, the client and server share two keys which are initial
key $\ik$ and forward secure key $\key$.
Other attacks~\cite{LJBN15:QUIC} make a client and server share
a different initial key $\ik$.
The server impersonation advantage in QACCE does not
cover this case because this advantage consider only forward
secure key $\key$.
Our security model also consider the security of an initial
key $\ik$ and this property protect the other attacks
~\cite{LJBN15:QUIC}.

%=====================================================
\subsection{Related Work} \label{sec:Related Work}
%=====================================================
There are a huge body of works on authenticated key
exchange protocols (See~\cite{CK01:AKE} for survey).
An important stream of research dates back to Bellare
and Rogaway~\cite{BR93:AKE}, followed by~\cite{DB96,
Blei98,JMDP00,JB02,EK09,KK05:TLS,KCRE08,SMOAJ08,KTT11,
Kraw01}.
However, as mentioned above, the QUIC full handshake
protocol does not satisfy key-indistinguishability as
in the Bellare-Rogaway like model, because a server
sends a ciphertext (using authenitcated encryption) in
the full handshake protocol, as does TLS.
TLS Handshake Protocol is recently analysed in various
security models, e.g., ~\cite{JKSS12:ACCE,KPW13:SACCE,
FS13:ACCE,GKS13:RACCE,BDKSS14:SSH,BFKPSB14:TLS}.
Still, the security model for analysing a server-only
authenticated connection of TLS, i.e., Server-Only
Authenticated and Confidential Channel Establishment
(SACCE)~\cite{KPW13:SACCE}, is not appropriate for QUIC.
There are two reasons. First, the abstract model of
handshake between QUIC and TLS is different. In QUIC,
the client and server share two keys which are initial
key $\ik$ and forward secure key $\key$ and secrets are not
reused in an abbreviate handshake. On the other hand,
in TLS, the client and server share one key and reuse
the secrets in an abbreviate handshake. Second, the
security model proposed in the previous study
~\cite{FG14:QUIC,LJBN15:QUIC} does not
consider the consistency of the client between a full
handshake and abbreviate handshakes.
The second reasons is important in order to prevent the
attacks \cite{LJBN15:QUIC}.

%=====================================================
\subsection{Our Results} \label{sec:proposal}
%=====================================================

Our contributions are:
\begin{itemize}
 \item{A security model which is appropriate for QUIC
 and more secure than QACCE.}

 \item{A new scheme which is more secure and efficient
 than original one.}
\end{itemize}

We introduce a new
security model, what we call \textit{Resumable} SACCE
(RSACCE) security, where we consider a server's message
confidentiality, as well as a client's message
confidentiality, where an adversary is allowed to send
an encryption query to a \textit{server} (to break a
server's message confidentiality) both in the full
handshake session and the abbreviate handshake sessions,
as far as the server establishes the initial full
handshake session with a \textit{honest} client.
Our security model also consider the consistency of the client between
0-RTT connections and 1-RTT connections, although a
server does not ensure the consistency in QUIC.
The consistency of the client between 0-RTT connections
and 1-RTT connections prevent one of the attacks.

We propose a more efficient and secure protocol with
the spirit of QUIC.
In~\cite{LJBN15:QUIC}, they find the five attacks for
QUIC.
Our proposal scheme can prevent the four attacks.
 %=====================================================
\section{Preliminaries}\label{sec:pre}
%=====================================================
We let ${\negl}(\spar)$ to denote an unspecified
function $f(\spar)$ such that
$f(\spar) ={\spar}^{-\omega(1)}$, saying that such a
function is negligible in $\spar$.
We use the standard notations of digital signature,
collision-resistant hash functions, and pseudo random
functions.
We also use a symmetric encryption as well as a few
known assumptions.

%=====================================================
\subsection{Digital Signature Schemes} \label{sec:signature}
%=====================================================
A digital signature scheme consists of the following
three algorithms $\SIG = (\Gen,\Sign,\Vfy)$, where
\begin{itemize}
 \item{$\Gen(1^{\spar})$ takes as input a security
 parameter $\kappa$ and generates a key pair $(pk, sk)$
 where $pk$ is a verification key and $sk$ is secret
 key.}
 \item{$\Sign(sk, m)$ takes as input a message $m$ and
 $sk$ and outputs a signature $\sigma$.}
 \item{$\Vfy(pk,\sigma,m)$ takes as input $pk$, $\sigma$
 and $m$ and outputs 1 if $\sigma$ is valid and 0
 otherwise.}
\end{itemize}
\begin{definition}[EUF-CMA]
We say that $\SIG$ is $(t, \epsilon_{\sig})$-secure
againt \textit{existentially unforgeable under
adaptive chosen-message attacks}, if for all PPT
adversaries $A$ that run in time $t$, the experiment
$\Expt^{\sf EUF}_{A}$ in Fig.~\ref{fig:EUF-CMA} returns
$1$ with at most probability $\epsilon_{\sf sig}$.

\begin{figure*}[!htb]
\begin{center}
\fbox{
\begin{minipage}[t]{0.38\textwidth}
\begin{tabular}[t]{l}
 $\Expt^{\sf EUF}_{A}\ \equiv$ \\
 $\quad (pk,sk) \leftarrow \SIG.\Gen(1^k)$ \\
 $\quad m^{\prime}, \sigma \leftarrow A^{\sf SIGN}(pk)$ \\
 \quad returns $1$ iff \\
 $(\quad m^{\prime} \not\in M) \, \wedge \, (\SIG.\Vfy(pk, m^{\prime}, \sigma) = 1)$ \\
\end{tabular}
\end{minipage}
\vline \quad
\begin{minipage}[t]{0.35\textwidth}
\begin{tabular}[t]{l}
 ${\sf Oracle\ SIGN(m)\ }\equiv$ \\
 $\quad M = M \cup m$ \\
 $\quad \text{returns }\SIG.\Sign(sk, m)$ \\
\end{tabular}
\end{minipage}
}
\caption{EUF-CMA security experiment}\label{fig:EUF-CMA}
\end{center}
\end{figure*}

\end{definition}

%=====================================================
\subsection{Collision-Resistant Hash Functions} \label{sec:CRH}
%=====================================================
Let $\HashH\sets \{H\}$ be a hash function family from
$\bits^*$ to $\bits^{\spar}$.
We say that a hash-function family $\HashH$ is
$(t,\epsilon_{H})$-collision-resistant (CR) if, for all
adversaries $A$ that run in time $t$, it holds that
\begin{equation}
 \Pr[H \gets {\HashH}; \, (x,y) \gets A(H): \, x \neq y
 \, \wedge \,(x) = H(y)] \leq \epsilon_{H}.
\end{equation}

%=====================================================
\subsection{Pseudo-Random Functions} \label{sec:prf}
%=====================================================
Let $\{X_n,Y_n\}_{n \in \N}$ be a sequence of domains
and let $\PRF = \{F_{s}\}_{s \in \Index}$ be a family
of functions from $X_n$ to $Y_n$, indexed by $s \in
\Index_{n}$ $(= \Index \cap \{0,1\}^{n})$, i.e.,
$F_{s} : X_n \to Y_n$, where for every $n$ and
$s \in \Index_n$, $\Index_n$ is efficiently samplable
and $F_s$ is efficiently computable.
We say that $\PRF$ is $(t,\epsilon_{\prf})$-pseudo-random
function (PRF) familiy if for all adversaries $A$ that
run in time $t\sets t(n)$,
\begin{eqnarray}
 |\Pr[s \gets {\Index}_{n}: \, A^{F_s(\cdot)}(1^n)=1] \, - \, \nonumber \\
 \Pr[s \gets {\Index}_{n}: \, A^{R_n}(1^n)=1] |
 \leq \epsilon_{\prf}(n),
\end{eqnarray}
where $R=\{R_n\}_{n\in \N}$ is the corresponding uniform
function ensemble (i.e., $\forall n$, $R_n$ is uniformly
distributed over the set of the functions from $X_n$ to
$Y_n$).

%=====================================================
\subsection{DDH and PRF-ODH Assumptions} \label{sec:assumption}
%=====================================================
Let $g$ be a generator of a finite cyclic group $G$ of
prime order $q$, i.e.,
$g \in G^{\times}(\sets G\backslash \{1\})$.

\begin{definition}[Decisional Diffie-Hellman Assumption]
 We say that the DDH problem is $(t,\epsilon_{\ddh})$-hard
 in $G$, if for all PPT adversaries $A$ that runs time
 in $t$, it holds that
 \begin{equation}
  |\Pr[A(p,g,g^a,g^b,g^{ab}) = 1] - \Pr[A(p,g,g^a,g^b,g^c)
  = 1]| \leq \epsilon_{\ddh}
 \end{equation}
 where $g \getsr G^{\times}$ and
 $a,b,c \getsr \mathbb{Z}_{q}$.
 We say that the DDH assumption holds in $G$ if for all
 polynomial (in $\spar$) $t(\spar)$, there exists
 $\epsilon_{\ddh}(\spar)=\negl(\spar)$ such that the
 DDH problem is $(t,\epsilon_{\ddh})$-hard in $G$.
\end{definition}

Let $f:G \times \bits^n \to \bits^{\mu}$ be a pseudo-random
function (PRF) with key-space $G$ that maps $\bits^n$ to
$\bits^{\mu}$ for some appropriate polynomials (in $\spar$)
$n(\spar)$ and $\mu(\spar)$.

\begin{definition}[PRF-ODH Assumption]
 We say that the PRF-ODH problem on $f$ is
 $(t, \epsilon_{\sf prfodh})$-hard, if for all PPT
 adversaries $A=(A_1,A_2)$ that run in time $t$,
 the experiment $\Expt^{\sf PRF-ODH}_{A}$ in
 Fig.~\ref{fig:PRF-ODH} returns $1$ with probability at
 most $\epsilon_{\sf prfodh}$.
 We say that the PRF-ODH assumption on $f$ holds if for
 all polynomial (in $\spar$) $t(\spar)$, there exists
 $\epsilon_{\sf prfodh}(\spar)=\negl(\spar)$ such that
 the PRF-ODH problem on $f$ is
 $(t,\epsilon_{\sf prfodh})$-hard.
\begin{figure*}[!htb]
\begin{center}
\fbox{
\begin{minipage}[t]{0.4\textwidth}
\begin{tabular}[t]{l}
 $\Expt^{\sf PRF-ODH}_{A}\ \equiv$ \\
 $\quad g \getsd G^{\star};$
 $\quad (st,\overline{m}) \gets A_1(g);$
 $\quad \overline{u},v \getsd \mathbb{Z}_q$ \\
 $\quad z_0 = f_{g^{\overline{u}v}}(m);$
 $\quad z_1 \xleftarrow{\$} \{0,1\}^{\mu}$ \\
 $\quad b \xleftarrow{\$} \{0,1\};$
 $\quad b^{\prime} \leftarrow A_2^{\sf ODH(\cdot,\cdot)}(st,g^{\overline{u}},z_b)$ \\
 \quad return $b^{\prime}$
\end{tabular}
\end{minipage}
\vline \quad
\begin{minipage}[t]{0.3\textwidth}
\begin{tabular}[t]{l}
 ${\sf Oracle\ ODH(h, m)\ }\equiv$ \\
 $\quad \text{If } (h, m) = (g^{\overline{u}}, \overline{m}) \vee h \not\in G$ \\
 $\quad \quad \text{return } \perp$ \\
 $\quad \text{return }f_{h^v}(m)$ \\
\end{tabular}
\end{minipage}
}
\caption{PRF-ODH security experiment}\label{fig:PRF-ODH}
\end{center}
\end{figure*}
\end{definition}

%=====================================================
\subsection{Length-Hiding Authenticated Encryption} \label{sec:SE}
%=====================================================

A symmetric encryption scheme consists of three algorithm
$SE = (\Init, \Enc, \Dec)$.
\begin{itemize}
 \item{$\Enc(k,\iv,H,m)$ takes as input a secret key
 $k \in \{1,0\}^{\kappa}$, a initial vector $\iv$, an
 output ciphertext length $len \in \mathbb{N}$, some
 header data $H \in \{0,1\}^{\ast}$, and a plaintext
 $m \in \{0,1\}^{\ast}$.
 It outputs a ciphertext $c \in \{0,1\}^{len}$.}
 \item{$\Dec(k,\iv,H,c)$ takes as input a secret key
 $k \in \{1,0\}^{\kappa}$, a initial vector $\iv$,
 some header data $H \in \{0,1\}^{\ast}$, and a
 ciphertext $c \in \{0,1\}^{len}$. It outputs a
 message $m$.}
\end{itemize}

\begin{definition}
 We say that a symmetric encryption scheme
 $\SE = (\Enc,\Dec)$ is $(t, \epsilon_{\LHAE})$-secure,
 if for all PPT adversaries $A$ that running in time
 $t$, it holds that
 \begin{eqnarray}
  \Pr[b\xleftarrow{\$}\{0,1\}; k\xleftarrow{\$}\{0,1\}^{\kappa}; \nonumber \\
   b^{\prime} \xleftarrow{\$} A^{\Encrypt, \Decrypt} : b = b^{\prime}] \leq \epsilon_{\LHAE},
 \end{eqnarray}
 where Encrypt and Decrypt oracles are described in
 Fig.~\ref{fig:LHAE_Enc_Dec}.
 We say that $\SE$ is secure if for all polynomial
 (in $\spar$) $t(\spar)$, there exists
 $\epsilon_{\LHAE}(\spar)=\negl(\spar)$ such that $\SE$
 is $(t, \epsilon_{\LHAE})$-secure.
\end{definition}

\begin{figure*}[!htb]
\begin{center}
\fbox{
\begin{minipage}[t]{0.42\textwidth}
\begin{tabular}[t]{l}
 $\Encrypt(m_0,m_1,\iv,H)$: \\
 $\quad C_0 \xleftarrow{\$} \SE.\Enc(k, \iv, H,m_0)$ \\
 $\quad C_1 \xleftarrow{\$} \SE.\Enc(k, \iv, H,m_1)$ \\
 $\quad \text{If } C_0 = \perp \text{ or } C_1 = \perp$ \\
 $\quad \quad \text{return } \perp$ \\
 $\quad \mathcal{C} = \mathcal{C} \cup C_b$ \\
 $\quad \text{return } C_b$ \\
\end{tabular}
\end{minipage}
\vline \quad
\begin{minipage}[t]{0.42\textwidth}
\begin{tabular}[t]{l}
 $\Decrypt(C,\iv,H)$: \\
 $\quad \text{If } b = 1 \wedge C \not\in \mathcal{C}$ \\
 $\quad \quad \text{return } \SE.\Dec(k,\iv,H,C)$ \\
 $\quad \text{return } \perp$ \\
\end{tabular}
\end{minipage}
}
\caption{Encrypt and Decrypt oracle in the LHAE security experiment}\label{fig:LHAE_Enc_Dec}
\end{center}
\end{figure*}
 %=====================================================
\section{``Resumable" Server-only Authenticated and Confidential Channel Establishment (RSACCE)} \label{sec:rsacce}
%=====================================================
\textcolor{blue}{
We consider the security of QUIC and its variants.
Our abstract model to capture the cryptographic core of them is a server-only
authenticated and confidential channel establishment (SACCE) protocol~\cite{KPW13:SACCE},
the server-only authentication version of an ACCE protocol~\cite{JKSS12:ACCE}.
In our SACCE protocol, a client initiates an instance of the protocol and
a server always sends the last message of the instance.
We then extend the security notion of SACCE~\cite{KPW13:SACCE}
to the notion of \textit{resumable} SACCE (RSACCE).
The main difference is that this model treats ``resumption".
As mentioned above, a resumption is an abbreviated handshake session
taking advantage of a prior established full handshake session
between the same client and server.
To treat resumptions in the server-only authenticated setting,
we should take into account the attack \textbf{where an adversarial client fools a server
and impersonates the initial client in a resumption session.}
Depending on the extent of the attack, we define two security requirements,
denoted \textit{channel confidentiality} and \textit{channel binding}.
The channel confidentiality implies that, in contrast to SACCE,
an adversary is allowed to submit
an encryption query \textit{not only to a client but to a server} in a resumption session.
We note that if an adversary can make a handshake with a server with the same session
key in some resumption session, then it can indeed break message confidentiality of
a ciphertext sent by the server.
We also consider channel binding in order to make a protocol robust,
in which a server should accept
only if a resumed session is started by the initial client
in the prior established full handshake session.
}



%--------------------------------------------------------
\subsection{Execution Environment} \label{sec:exec_env_party}
%--------------------------------------------------------
We basically borrow the notations from~\cite{JKSS12:ACCE,KPW13:SACCE}.
We denote by $\Client$ and $\Server$ the set of honest clients and servers, respectively.
We assume for simplicity that each client $\Client$ and server $\Server$ has a \textit{unique} identity number $c \in \Nset$~\footnote{Since a client has
no certified identity, the numbering is just conceptual.}
and $s \in \Nset$.
For the case we don't need to specify clients or servers, we also assume that each party $P \in \Server \cup \Client$ has a \textit{unique} identity number $p \in \Nset$.
%~\footnote{Since a client has
% no certified identity, the numbering is just conceptual.}.
We denoted by $P_p$ a party with identity number $p$.
In particular, $\Server_s$ has a unique key pair $(pk_s, sk_s)$, along with a certificate $cert_s=(s,pk_s)_{\text{CA}}$
signed by a certificate authority CA.
A party $P_p$ maintains a collection of oracles $\{\pi^p_{i,\ell }\}_{i,\ell}$
where oracle $\pi^p_{i, \ell}$ models party $P_p$ executing a single instance of a protocol in the $\ell$-th abbreviate handshake session
derived from  the $i$-th full handshake session.
When $\ell=0$, $\pi^p_{i,\ell}$ refers to the $i$-th full handshake session of $P_p$.
The oracle $\pi^p_{i, \ell}$ maintains as internal state the following variables:

\begin{itemize}
 \item{$\Lambda \in \{\accept, \preaccept, \reject, \emptyset\}$, the state of a handshake.}
 \item{$ik, k \in \mathcal{K}$, the session key where $\mathcal{K}$ is the key space of the protocol. $ik$ is a initial key and $k$ is a forward secure key.}
 \item{$\peer$, the intended partner. If $P_p$ is $\Client_c$, then $\pi^c_{i,\ell}$ maintains the identity of intended partener $\peer \in \Server$. Otherwise (if $P_p$ is $\Server_s$), it maintains $\peer:=SCID$ where $SCID$ is a random string chosen by $\pi^s_{i,\ell}$.}
 \item{$b$, the challenge bit chosen uniformly.}

% \item{$(b, C, u, v)$ are the variable to reply the adversary query. The detail about the adversary query is at (Sec.~\ref{sec:exec_env_adv}). $b \in \{0,1\}$ is a bit. $C$ is the array of ciphertexts. $u$ is the counter of Encrypt query, $v$ is the counter of Decrypt query.}
\end{itemize}
The inner state of each oracle is initialized to
 ( $\Lambda$ , ik, k, $\peer$) = ( $\emptyset$, $\emptyset$, $\emptyset$, $\emptyset$) ,
where variable $V=\emptyset$ denotes that variable $V$ is undefined.
On one hand, and
$b$ is chosen uniformly and fixed.

%--------------------------------------------------------
%\subsection{Execution Environment: Adversary} \label{sec:exec_env_adv}
%--------------------------------------------------------
The adversary issues the following queries to the oracles:
\begin{itemize}
 \item {$\Send(\pi^p_{i, \ell}, m)$:
    The adversary can use this query to send message $m$ to oracle $\pi^p_{i, \ell}$.
    The oracle will respond with an outgoing message according
    to the protocol specification and its internal state.
    The oracle $\pi^p_{i, \ell}$ replies with $\bot$, either
    (a) if $\ell \geq 1$ and $\pi^p_{i, {\ell-1}}$ has state $\Lambda\neq\accept$,
    or
    (b) if $\pi^p_{i,\ell}$ has reached state $\Lambda=\accept$.
    Otherwise, it does the following:
    If the adversary asks the first $\Send$-query to oracle $\pi^p_{i, \ell}$,
    then the oracle checks whether $m = \top$ consists of a special ``initiate client session'' symbol $\top$.
    If so, it responds with the first protocol message.
    If $\ell \geq 1$ and $\pi^p_{i,\ell-1}$ accepts, then $\pi^p_{i,\ell}$ \textit{inherits} variables
    $(\theta,\peer)$
    from $\pi^p_{i, \ell-1}$ (modeling a resumption session).
    The adversary can send this returned message to any oracle even if this oracle is not intended partner of the oracle $\pi^p_{i, \ell}$.}

 \item {$\Reveal(\pi^p_{i,\ell})$ :
    The oracle $\pi^p_{i,\ell}$ returns key $k$ and cache data $\theta$
    to respond this query. }

 \item {$\Corrupt(P_p)$:
    If $P_p$ is $\Client_c$, then returns $\bot$. Otherwise, $P_p$ is $\Server_s$ and returns
    long-term secret key $sk_s$.
    If $P_p$ is $\Server_s$ and $\Corrupt(P_p)$ is the $\tau$-th query issued by the adversary,
    $P_p$ is said $\tau$-\textit{corrupted}.
    For parties that are not corrupted we define $\tau = \infty$.}

 \item {$\Encrypt(\pi^p_{i,\ell}, m_0, m_1, len, H)$:
    If $\pi^p_{i,\ell}$ has state
    $\Lambda$ $\not\in$ \\ $\{\preaccept, \accept\}$, it returns $\bot$.
    Otherwise, it makes a challenge ciphertext according to the procedure in Fig.~\ref{fig:LHAE_rsacce}.}

 \item {$\Decrypt(\pi^p_{i, \ell}, c, H)$:
    The oracle $\pi^p_{i, \ell}$ replies according to the procedure in Fig.~\ref{fig:LHAE_rsacce}.}
\end{itemize}

\begin{figure*}[!htb]
\begin{center}
\fbox{
\begin{minipage}[t]{0.42\textwidth}
\begin{tabular}[t]{l}
 $\Encrypt(\pi^p_{i,\ell},m_0,m_1,len,H)$: \\
 $\quad C_0 \xleftarrow{\$} \SE.\Enc(k, len, H,m_0)$ \\
 $\quad C_1 \xleftarrow{\$} \SE.\Enc(k, len, H,m_1)$ \\
 $\quad \text{If } C_0 = \perp \text{ or } C_1 = \perp$ \\
 $\quad \quad \text{return } \perp$ \\
 $\quad \mathcal{C} = \mathcal{C} \cup C_b$ \\
 $\quad \text{return } C_b$ \\
\end{tabular}
\end{minipage}
\vline \quad
\begin{minipage}[t]{0.42\textwidth}
\begin{tabular}[t]{l}
 $\Decrypt(\pi^p_{i,\ell},C,H)$: \\
 $\quad \text{If } b = 1 \wedge C \not\in \mathcal{C}$ \\
 $\quad \quad \text{return } \SE.\Dec(k,H,C)$ \\
 $\quad \text{return } \perp$ \\
\end{tabular}
\end{minipage}
}
\caption{Encrypt and Decrypt oracle in the RSACCE security experiment}
 \label{fig:LHAE_rsacce}
\end{center}
\end{figure*}


 In this model, $\Reveal$ query exposes not only session key $k$ but also secret cache data $\theta$.

%--------------------------------------------------------
\subsection{Security Definition} \label{sec:sec_def}
%--------------------------------------------------------

We define the security model of resumable server-only authenticated confidential channel establishment
(RSACCE).


\subsubsection{Matching Conversations}
In our SACCE protocol, a client always initiates an instance of the protocol and
a server always sends the last message of the instance.
We define matching conversations as in~\cite{JKSS12:ACCE}.
%Let $P_i$ be a client and $P_j$ be a server.
We denote by $T^p_{i,\ell}$ the transcript of $\pi^p_{i,\ell}$, i.e., the history of
all messages sent and received by $\pi^p_{i,\ell}$ in chronological order (not including the initialization-symbol $\top$).
For two transcripts, $T^p_{i,\ell}$ and $T^{p^{\prime}}_{j,\ell'}$
We say that $T^p_{i,\ell}$ is a \textit{prefix} of $T^{p^{\prime}}_{j,\ell'}$ if
$T^p_{i,\ell}$ contains at least one message, and
$T^p_{i,\ell}$ is identical to $T^{p^{\prime}}_{j,\ell'}$ except the last message sent
by $\pi^{p^{\prime}}_{j,\ell'}$.

\begin{definition}[Matching conversations]
 We say that $\pi^c_{i,\ell}$ and $\pi^s_{j,\ell'}$ have a matching conversation with each other if  in addition to $\ell=\ell'$,
 \begin{itemize}
  \item{Both oracles, $\pi^c_{i, \ell}$ and $\pi^s_{j,\ell'}$, accept and
  $T^c_{i,\ell} = T^s_{j,\ell'}$; or }
  \item{The server oracle $\pi^s_{j, \ell'}$ accepts,
  and $T^c_{i,\ell}$ is a prefix of $T^s_{j,\ell'}$.}
 \end{itemize}
\end{definition}
\begin{remark}
 The second condition is necessary because in our SACCE protocol
 a server accepts a session before sending the last message to a client
 and cannot know whether a client indeed receives it.
\end{remark}

\subsubsection{RSACCE Game}
We define the RSACCE game between an adversary $A$ and a challenger.
In this game, the challenger firstly instantiates the collection of oracles
$\{\pi^p_{i,\ell}\}$.
Then the challenger generates the certificate's signing/verification keys;
generates long-term keys $(pk_s, sk_s)$ for all servers;
and issues the certificates for all public keys.
The adversary receives all public keys $pk_s$ with identity $s$ as input.
Now the adversary may start by issuing $\Send$, $\Reveal$, $\Corrupt$, $\Encrypt$ and $\Decrypt$ queries.
Finally, the adversary outputs $(p, i, \ell, b^{\prime})$ and terminates.

\begin{definition}[Correctness]
 Assume a ``benign" adversary $A$, which picks two arbitrary oracles, $\pi^c_{i, \ell}$ and $\pi^s_{j, \ell^{\prime}}$,
 and performs a sequence of $\Send$-queries by faithfully forwarding all messages
 between $\pi^c_{i, \ell}$ and $\pi^s_{j, \ell^{\prime}}$.
 Let $k^c_{i, \ell}$ denote the key computed by $\pi^c_{i, \ell}$
 and let $k^s_{j, \ell^{\prime}}$ denote the key computed by $\pi^s_{j, \ell^{\prime}}$.
 We say that RSACCE protocol is \textit{correct}, if two arbitrary server and client oracles, $\pi^c_{i, \ell}$ and $\pi^s_{j, \ell^{\prime}}$, always hold that:
 \begin{itemize}
  \item{Both oracles, $\pi^c_{i, \ell}$ and $\pi^s_{j, \ell^{\prime}}$, have \textit{a matching conversation} with each other; and}
  \item{Both session keys, $k^c_{i, \ell}$ and $k^s_{j, \ell^{\prime}}$, are the same ($k^c_{i, \ell} = k^s_{j, \ell^{\prime}}$).}
 \end{itemize}
\end{definition}

\subsubsection{Security Requirements}
We now define the following advantage measures.

\begin{definition}[Server Authentication] \label{def:rsacce-sa}
 $\Adv^{\rsaccesa}_{\Pi}$ (A) is the probability that when $A$ terminates, there is
 a client oracle $\pi^c_{i, \ell}$ such that the following conditions hold:
 \begin{itemize}
  \item{$\pi^c_{i, \ell}$ accepts when $A$ issues its $\tau_0$-th query with intended partner $\peer=s$, }
  \item{$\Server_s$ is $\tau_{s}$-corrupted with $\tau_0 < \tau_{s}$ and}
  \item{There is no server oracle $\pi^s_{j, \ell}$ such that $\pi^c_{i,\ell}$ has a matching conversation
  with $\pi^s_{j,\ell}$ or there exist plural oracles that have a matching conversation with $\pi^c_{i,\ell}$.}
 \end{itemize}
 We say that protocol $\Pi$ has \textit{server authentication}, if
 $\Adv^{\rsaccesa}_{\Pi}(A)$ is negligible in $\kappa$.
\end{definition}

\begin{remark}
 The \textit{server authentication}  is a natural extension of the counter part of~\cite{KPW13:SACCE},
 which implies that if a client accepts, protocol $\Pi$ guarantees that the client has a matching conversation with
 the intended parter (server). In contrast to the original definition~\cite{KPW13:SACCE},
 we allow an adversary to submit $\Corrupt$ queries.
 \end{remark}

\begin{definition}[Channel Confidentiality] \label{def:rsacce-cc}
 $\Adv^{\rsaccecc}_{\Pi}$ (A) is defined to be $x$ - $\frac{1}{2}$ where $x$ is the probability that the adversary $A$ outputs $(p, i, \ell, b^{\prime})$, with $0\leq \ell$,
 such that $b = b^{\prime}$ where $b^{\prime} \in \bits$ is set during the $\Encrypt(\pi^p_{i,\ell},\dots)$ query and we define $b=\bot$ unless the following conditions hold:
 \begin{itemize}
%  \item{$\pi_{s,0}^i$ reaches state $\Lambda=\accept$.}
  \item{$\pi^p_{i,\ell}$ reaches state $\Lambda=\preaccept$ when $A$ issues
  its $\tau_0$-th query. (It implies by definition that for all $0\leq k < \ell$,
  $\pi^p_{i,k}$ has already accepted.)}

  \item{If $\pi^p_{i,\ell}$ is a client oracle,
  the intended pater $\Server_s$ is $\tau_s$-corrupted with $\tau_0 < \tau_s$. }

  \item{If $P_p$ is a server $\Server_s$, then
  there is a client $\Client_c$ maintaining oracle $\pi^c_{j,0}$ that has a matching
  conversation with $\pi^s_{i,0}$.}

  \item{For all $0\leq k \leq \ell$, the adversary does not issue a $\Reveal$ query to
  $\pi^p_{i,k}$, nor to $\pi^{p^{\prime}}_{j,k}$ such that $\pi^p_{i,k}$ has a matching conversation
  with $\pi^{p^{\prime}}_{j,k}$.}

 \end{itemize}
 We say that protocol $\Pi$ has \textit{channel confidentiality}
 if $\Adv^{\rsaccecc}_{\Pi}(A)$ is negligible in $\kappa$.
\end{definition}

\begin{remark}
We extend the security requirement of \textit{channel confidentiality} of SACCE~\cite{KPW13:SACCE}.
The original security notion takes care of message confidentiality of ciphertexts
sent by a (honest) client, only.  It is because an adversary can trivially play a role of an honest client and
have a matching conversation with a (honest) server with the same session key.
In our definition, the third condition indicates that an adversary may send
the $\Encrypt$ query to even \textit{server} $\Server_s$
(to break message confidentiality of the challenge ciphertext sent by \textit{server oracle} $\pi^s_{i,\ell}$)
as long as the initial full handshake is established between
$\Server_s$ and an \textit{honest} client $\Client_c$ (not the adversary).
Later, if the adversary can hijack some resumption session and
share a new session key with server $\Server_s$,
then it can break our channel confidentiality.
The forth condition specifies forward-secrecy in the sense that message confidentiality of $\pi^p_{i,\ell}$
is preserved, as long as the prior related oracles (including itself), $\pi^p_{i,k}$,
with $0\leq k\leq \ell$, or their peer oracles do not reveal their cached data and session keys.
\end{remark}

We now present an additional advantage measure.

\begin{definition}[Channel Binding] \label{def:rsacce-cb}
 $\Adv^{\rsaccecb}_{\Pi}$ (A) is the probability that when $A$ terminates, there exists
 a server oracle $\pi^s_{j, \ell}$, with $0< \ell$, such that the following conditions hold:

 \begin{itemize}
  \item{$\pi^s_{j,\ell}$ reaches state $\Lambda=\accept$ when $A$ issues
  its $\tau_0$-th query. (It implies by definition that for all $0\leq k < \ell$,
  $\pi^s_{j,k}$ has already accepted.)}

  \item{There is a client $\Client_c$ maintaining an oracle $\pi^c_{i,0}$ with $\peer=s$
  that has a matching conversation with $\pi^s_{j,0}$.}

  \item{For all $0\leq k \leq \ell$, the adversary does not issue a $\Reveal$ query either to
  $\pi^c_{i,k}$ nor to $\pi^s_{j,k}$,
  such that $\pi^c_{i,k}$ has a matching conversation with $\pi^s_{j,k}$.}

  \item{Client oracle $\pi^c_{i, \ell}$  does not have a matching conversation
  with $\pi^s_{j,\ell}$.}
%  \item{Client oracle $\pi^j_{t, \ell}$ has state $\peer \neq i$.}

 \end{itemize}
 We say that protocol $\Pi$ has \textit{channel confidentiality}
 if $\Adv^{\rsaccecc}_{\Pi}$ (A) is negligible in $\kappa$.
\end{definition}

\begin{remark}
Channel binding is a security property to protect something like client authentication.
While clients do not have certificates, one would need a guarantee that if a server accepts
in a resumption session, the server should have a matching conversation only with the client
in the initially established full handshake session.
We note that to break channel binding, it is unnecessary for an adversary to share a session key with a server
(if so, it breaks channel confidentiality).
If an adversary breaks channel binding, then the client and server could not resume a new resumption session.
\end{remark}

\subsubsection{RSACCE security}
We define the two levels of RSACCE security.

\begin{definition}[RSACCE secure]
 We say that the protocol $\Pi$ is RSACCE secure if $\Pi$ satisfies \textbf{correctness},
 \textbf{server authentication}, and \textbf{channel confidentiality}.
\end{definition}

\begin{definition}[Strong RSACCE secure]
 We say that the protocol $P$ is \textbf{strong} RSACCE secure if $\Pi$ is RSACCE secure
 and also satisfies \textbf{channel binding}.
\end{definition}
 %=====================================================
\section{Quick UDP Internet Connections} \label{sec:quic}
%=====================================================

Quick UDP Internet Connections (QUIC) is a protocol
developed by Google and this protocol is still under
development.
The concepts of QUIC are (1) to reduce connectivity
overheads before a client sends encrypted data and
(2) to obtain better security guarantee than TLS.
To realize concept (1), QUIC is not defined over
TCP but UDP, because TCP requires three-move handshake
before initiating a cryptographic handshake, but UDP
does not have. In addition, a client can send encrypted
data concurrently with CHLO which is second query of the
client.
To realize concept (2), QUIC supports only secure cipher
suites. Especially, a support algorithm of key exchange
is only ephemeral elliptic curve Diffie-Hellman.

%=====================================================
\subsection{1-RTT Connection Establishment} \label{sec:quic_1rtt}
%=====================================================

We provide the abstract model of 1-RTT connection establishment
for the initial key in Fig.\ref{fig:quic_abst_1rtt_init} and
the final key in Fig.\ref{fig:quic_abst_1rtt_last}.
%
\begin{figure*}[htb]
\begin{center}

\fbox{
\begin{minipage}[t]{0.39\textwidth}
\begin{tabular}[c]{l}
 $\quad Client$ \\
 $ $ \\
 \setcounter{nombre}{0}%
 $\prob.\quad \cid \xleftarrow{\$} \{0,1\}^{64} $ \\
 $\prob.\quad m_1 = \cid$ \\
 $\prob.\quad send\ m_1$ \\
 $ $\\
 \setcounter{nombre}{0}%
 $\prob.\quad receive\ m_2$ \\
 $\prob.\quad (\SCID, T_s, \sigma_s) = \SCFG_{pub}$ \\
 $\prob.\quad \doc = \SCID \| T_s$ \\
 $\prob.\quad \text{If } \SIG.\Vfy(pk_s, \sigma_s, \doc)$ \\
 $\prob.\quad \quad \Lambda = \text{'reject' and abort}$ \\
 $\prob.\quad t_c \xleftarrow{\$} \Zset_{q}^{\ast} $ \\
 $\prob.\quad T_c = g^{t_c} $ \\
 $\prob.\quad \NONC \xleftarrow{\$} \{0,1\}^{160} $ \\
 $\prob.\quad m_3 = (T_c, \NONC, \STK, \SCID, \cid)$ \\
 $\prob.\quad send\ m_3$ \\
 $\prob.\quad pms = T_s^{t_c}$ \\
 $\prob.\quad ms = \PRF(pms, NONC)$ \\
 $\prob.\quad \ik = \PRF(ms, m_1 \| m_2 \| m_3)$ \\
 $\prob.\quad \Lambda = \preaccept$ \\
\end{tabular}
\end{minipage}%
}
% middle
 \begin{minipage}[t]{0.13\textwidth}
  \centering
  \begin{tabular}{c}
   $ $ \\
   $ $ \\
   $ $ \\
   $\xrightarrow{m_1}$ \\
   $ $ \\
   $\xleftarrow{m_2}$ \\
   $ $ \\
   $ $ \\
   $ $ \\
   $ $ \\
   $ $ \\
   $ $ \\
   $\xrightarrow{m_3}$ \\
   $ $ \\
  \end{tabular}
 \end{minipage}%
\fbox{
\begin{minipage}[t]{0.39\textwidth}
\begin{tabular}[c]{l}
 $\quad Server$ \\
 $ $ \\
 $ $ \\
 $ $ \\
 \setcounter{nombre}{0}%
 $\prob.\quad receive\ m_1$ \\
 $\prob.\quad \text{choose } \SCFG = (\SCFG_{pub}, t_s) $\\
 $\prob.\quad (\SCID, T_s, \sigma_s) = \SCFG_{pub}$ \\
 $\prob.\quad \makeSTKQUIC$ \\
 $\prob.\quad m_2 = (\SCFG_{pub}, \STK, \cid)$ \\
 $\prob.\quad send\ m_2$ \\
 $ $ \\
 $ $ \\
 $ $ \\
 $ $ \\
 $ $ \\
 \setcounter{nombre}{0}%
 $\prob.\quad receive\ m_3$ \\
 $\prob.\quad \text{check $\STK$ using $\key_{\STK}$}$ \\
 $\prob.\quad \text{search $\SCFG$ with $\SCID$}$ \\
 $\prob.\quad pms = T_c^{t_s}$ \\
 $\prob.\quad ms = \PRF(pms, \NONC)$ \\
 $\prob.\quad \ik = \PRF(ms, m_1 \| m_2 \| m_3)$ \\
 $\prob.\quad \Lambda = \preaccept$ \\
\end{tabular}
\end{minipage}%
} \vspace{10pt}

% 1-RTT connection establishment for final key
% \vspace{10pt}\\

% \ONERTTtrue
% \ORIGINALtrue
% \fbox{
\begin{minipage}[t]{0.39\textwidth}
\begin{tabular}[c]{l}
 $ $ \\
 $ $ \\
 $ $ \\
 $ $ \\
 \setcounter{nombre}{0}%
\ifONERTT
 $\prob.\quad receive\ m_4$ \\
\else
 $\prob.\quad receive\ m_2$ \\
\fi
 $\prob.\quad T_s^{\prime} \| \STK = \SE.\Dec(\ik, c)$ \\
 $\prob.\quad pms^{\prime} = T_s^{\prime t_c} $ \\
 $\prob.\quad ms^{\prime} = \PRF(pms^{\prime}, \NONC) $ \\
\ifONERTT
 $\prob.\quad \key = \PRF(ms^{\prime}, m_1 \| m_2 \| m_3 \| m_4)$ \\
\else
 $\prob.\quad \key = \PRF(ms^{\prime}, m_1 \| m_2)$ \\
\fi
 $\prob.\quad \Lambda = \accept$ \\
 $\prob.\quad \theta = (\SCFG_{pub}, \STK)$ \\
\end{tabular}
\end{minipage}%
}
\begin{minipage}[t]{0.13\textwidth}
\centering
\begin{tabular}[c]{l}
 $ $\\
 $ $\\
 $ $\\
\ifONERTT
 $\xleftarrow{m_4}$\\
\else
 $\xleftarrow{m_2}$\\
\fi
 $ $\\
 $ $\\
 $ $\\
 $ $\\
 $ $\\
\end{tabular}
\end{minipage}%
\fbox{
\begin{minipage}[t]{0.39\textwidth}
\begin{tabular}[c]{l}
 \setcounter{nombre}{0}%
 $\prob.\quad t_s^{\prime} \xleftarrow{\$} \Zset_{q}^{\ast}$ \\
 $\prob.\quad T_s^{\prime} = g^{t_s^{\prime}}$ \\
 $\prob.\quad \makeSTKQUIC$ \\
 $\prob.\quad \plaintext = T_s^{\prime} \| \STK$ \\
 $\prob.\quad c = (\SE.\Enc(\ik, \plaintext)$ \\
\ifONERTT
 $\prob.\quad m_4 = (c, \cid)$ \\
 $\prob.\quad send\ m_4$ \\
\else
 $\prob.\quad m_2 = (c, \cid)$ \\
 $\prob.\quad send\ m_2$ \\
\fi
 $\prob.\quad pms^{\prime} = T_c{t_s^{\prime}} $ \\
 $\prob.\quad ms^{\prime} = \PRF(pms^{\prime}, \NONC) $ \\
\ifONERTT
 $\prob.\quad \key = \PRF(ms^{\prime}, m_1 \| m_2 \| m_3 \| m_4)$ \\
\else
 $\prob.\quad \key = \PRF(ms^{\prime}, m_1 \| m_2)$ \\
\fi
 $\prob.\quad \Lambda = \accept$ \\
\end{tabular}
\end{minipage}%
} \vspace{10pt}

\caption{Abstract model of 1-RTT connection establishment for an initial key in QUIC handshake}\label{fig:quic_abst_1rtt_init}
\end{center}
\end{figure*}
\ONERTTtrue
\ORIGINALtrue
\begin{figure*}[htb]
\begin{center}

\fbox{
\begin{minipage}[t]{0.39\textwidth}
\begin{tabular}[c]{l}
 $ $ \\
 $ $ \\
 $ $ \\
 $ $ \\
 \setcounter{nombre}{0}%
\ifONERTT
 $\prob.\quad receive\ m_4$ \\
\else
 $\prob.\quad receive\ m_2$ \\
\fi
 $\prob.\quad T_s^{\prime} \| \STK = \SE.\Dec(\ik, c)$ \\
 $\prob.\quad pms^{\prime} = T_s^{\prime t_c} $ \\
 $\prob.\quad ms^{\prime} = \PRF(pms^{\prime}, \NONC) $ \\
\ifONERTT
 $\prob.\quad \key = \PRF(ms^{\prime}, m_1 \| m_2 \| m_3 \| m_4)$ \\
\else
 $\prob.\quad \key = \PRF(ms^{\prime}, m_1 \| m_2)$ \\
\fi
 $\prob.\quad \Lambda = \accept$ \\
 $\prob.\quad \theta = (\SCFG_{pub}, \STK)$ \\
\end{tabular}
\end{minipage}%
}
\begin{minipage}[t]{0.13\textwidth}
\centering
\begin{tabular}[c]{l}
 $ $\\
 $ $\\
 $ $\\
\ifONERTT
 $\xleftarrow{m_4}$\\
\else
 $\xleftarrow{m_2}$\\
\fi
 $ $\\
 $ $\\
 $ $\\
 $ $\\
 $ $\\
\end{tabular}
\end{minipage}%
\fbox{
\begin{minipage}[t]{0.39\textwidth}
\begin{tabular}[c]{l}
 \setcounter{nombre}{0}%
 $\prob.\quad t_s^{\prime} \xleftarrow{\$} \Zset_{q}^{\ast}$ \\
 $\prob.\quad T_s^{\prime} = g^{t_s^{\prime}}$ \\
 $\prob.\quad \makeSTKQUIC$ \\
 $\prob.\quad \plaintext = T_s^{\prime} \| \STK$ \\
 $\prob.\quad c = (\SE.\Enc(\ik, \plaintext)$ \\
\ifONERTT
 $\prob.\quad m_4 = (c, \cid)$ \\
 $\prob.\quad send\ m_4$ \\
\else
 $\prob.\quad m_2 = (c, \cid)$ \\
 $\prob.\quad send\ m_2$ \\
\fi
 $\prob.\quad pms^{\prime} = T_c{t_s^{\prime}} $ \\
 $\prob.\quad ms^{\prime} = \PRF(pms^{\prime}, \NONC) $ \\
\ifONERTT
 $\prob.\quad \key = \PRF(ms^{\prime}, m_1 \| m_2 \| m_3 \| m_4)$ \\
\else
 $\prob.\quad \key = \PRF(ms^{\prime}, m_1 \| m_2)$ \\
\fi
 $\prob.\quad \Lambda = \accept$ \\
\end{tabular}
\end{minipage}%
}

\caption{Abstract model of 1-RTT connection establishment for final key in QUIC handshake}\label{fig:quic_abst_1rtt_last}
\end{center}
\end{figure*}
%
A server generate server config $\SCFG$ before a request from a client.
We describe the detail of generation of $\SCFG$ in section \ref{sec:proposed_scheme}.
SCFG is composed of seven parameters AEAD, SCID, PDMD,
PUBS, KEXS, and OBIT, and EXPY. The important parameters
are SCID which is an opaque 16 byte identifier for
this server config, PUBS which is server's
Diffie-Hellman public value, and EXPY which is expiry time
for this server config. The details of other parameters
are described in~\cite{QUIC:Crypto}.
Our definition considers only important parameters.
Firstly, the client sends an inchoate client
hello ($m_1$) to start the connection.
The client needs $\STK$ to make a server pre-accept and share
a session key since the server require the $\STK$ to the client
when the server calculates initial key $\ik$ or final key $\key$.
After the server receives inchoate client hello, it
sends a rejection ($m_2$). The rejection ($m_2$) contains
source address token (STK), server config ($\SCFG$) which
include a signature $\sigma_s$ of the server config generated
by the server long term secret key $sk_s$. The client uses
STK in future queries to demonstrate ownership of their
source IP address.
The client can calculates the initial key after receiving
a rejection ($m_2$). Before this calculation, the client
checks the server config $\SCFG$ and generate $\NONC$ which consists
of a random value, and ephemeral
Diffie-Hellman values $t_c$.
The client sends a client hello ($m_3$) to the server.
After the server receives a client hello ($m_3$), the
server check this query. $\STK$ is an opaque byte string
from the client's point of view. From the server's point
of view it's an authenticated-encryption block that
contains, at least, the client's IP address and a time
stamp by the server. The server decrypt $\STK$ and
validate time and source IP address.
The client and server
exchange data encrypted and authenticated using authenticated
symmetric encryption $\SE$ with the initial key $\ik$ after
sharing the initial key $\ik$.
The server send a server hello ($m_4$) which contains
a ciphertext which consists of a new $\STK$ and ephemeral server's
Diffie-Hellman public value after exchanging data and
calculate the final key $\key$.
The client validate a server hello ($m_4$) and calculate
the final key $\key$ and cache the data.

%=====================================================
\subsection{0-RTT Connection Establishment} \label{sec:quic_0rtt}
%=====================================================

We provide the abstract model of QUIC for 0-RTT in
Fig.\ref{fig:quic_abst_0rtt}.

%
\begin{figure*}[htb]
\begin{center}

% 0-RTT connection establishment for initial key
% \vspace{10pt}\\

\fbox{
\begin{minipage}[t]{0.39\textwidth}
\begin{tabular}[c]{l}
 \setcounter{nombre}{0}%
 $\quad Client$ \\
 $ $ \\
 $\prob.\quad \theta = (\SCFG_{pub}, \STK)$ \\
 $\prob.\quad \cid \xleftarrow{\$} \{0,1\}^{64} $ \\
 $\prob.\quad t_c \xleftarrow{\$} \Zset_{q}^{\ast} $ \\
 $\prob.\quad T_c = g^{t_c} $ \\
 $\prob.\quad \NONC \xleftarrow{\$} \{0,1\}^{160} $ \\
 $\prob.\quad m_1 = (T_c, \NONC, \cid, \STK, \SCID)$ \\
 $\prob.\quad send\ m_1$ \\
 $\prob.\quad pms = T_s^{t_c}$ \\
 $\prob.\quad ms = \PRF(pms, \NONC)$ \\
 $\prob.\quad \ik = \PRF(ms, m_1)$ \\
 $\prob.\quad \Lambda = \preaccept$ \\
\end{tabular}
\end{minipage}%
}
% middle
 \begin{minipage}[t]{0.13\textwidth}
  \centering
  \begin{tabular}{c}
   $ $ \\
   $ $ \\
   $ $ \\
   $ $ \\
   $ $ \\
   $\xrightarrow{m_1}$ \\
   $ $ \\
   $ $ \\
   $ $ \\
   $ $ \\
  \end{tabular}
 \end{minipage}%
\fbox{
\begin{minipage}[t]{0.39\textwidth}
\begin{tabular}[c]{l}
 $\quad Server$ \\
 $ $ \\
 $ $\\
 $ $\\
 $ $\\
 $ $\\
 $ $\\
 $ $\\
 \setcounter{nombre}{0}%
 $\prob.\quad receive\ m_1$ \\
 $\prob.\quad \text{check $\STK$ using $\key_{\STK}$}$ \\
 $\prob.\quad \text{search $\SCFG$ with $\SCID$}$ \\
 $\prob.\quad pms = T_c^{t_s}$ \\
 $\prob.\quad ms = \PRF(pms, \NONC)$ \\
 $\prob.\quad \ik = \PRF(ms, m_1)$ \\
 $\prob.\quad \Lambda = \preaccept$ \\
\end{tabular}
\end{minipage}%
} \vspace{10pt}

% 0-RTT connection establishment for final key
% \vspace{10pt}\\

% \ONERTTfalse
% \ORIGINALtrue
% \fbox{
\begin{minipage}[t]{0.39\textwidth}
\begin{tabular}[c]{l}
 $ $ \\
 $ $ \\
 $ $ \\
 $ $ \\
 \setcounter{nombre}{0}%
\ifONERTT
 $\prob.\quad receive\ m_4$ \\
\else
 $\prob.\quad receive\ m_2$ \\
\fi
 $\prob.\quad T_s^{\prime} \| \STK = \SE.\Dec(\ik, c)$ \\
 $\prob.\quad pms^{\prime} = T_s^{\prime t_c} $ \\
 $\prob.\quad ms^{\prime} = \PRF(pms^{\prime}, \NONC) $ \\
\ifONERTT
 $\prob.\quad \key = \PRF(ms^{\prime}, m_1 \| m_2 \| m_3 \| m_4)$ \\
\else
 $\prob.\quad \key = \PRF(ms^{\prime}, m_1 \| m_2)$ \\
\fi
 $\prob.\quad \Lambda = \accept$ \\
 $\prob.\quad \theta = (\SCFG_{pub}, \STK)$ \\
\end{tabular}
\end{minipage}%
}
\begin{minipage}[t]{0.13\textwidth}
\centering
\begin{tabular}[c]{l}
 $ $\\
 $ $\\
 $ $\\
\ifONERTT
 $\xleftarrow{m_4}$\\
\else
 $\xleftarrow{m_2}$\\
\fi
 $ $\\
 $ $\\
 $ $\\
 $ $\\
 $ $\\
\end{tabular}
\end{minipage}%
\fbox{
\begin{minipage}[t]{0.39\textwidth}
\begin{tabular}[c]{l}
 \setcounter{nombre}{0}%
 $\prob.\quad t_s^{\prime} \xleftarrow{\$} \Zset_{q}^{\ast}$ \\
 $\prob.\quad T_s^{\prime} = g^{t_s^{\prime}}$ \\
 $\prob.\quad \makeSTKQUIC$ \\
 $\prob.\quad \plaintext = T_s^{\prime} \| \STK$ \\
 $\prob.\quad c = (\SE.\Enc(\ik, \plaintext)$ \\
\ifONERTT
 $\prob.\quad m_4 = (c, \cid)$ \\
 $\prob.\quad send\ m_4$ \\
\else
 $\prob.\quad m_2 = (c, \cid)$ \\
 $\prob.\quad send\ m_2$ \\
\fi
 $\prob.\quad pms^{\prime} = T_c{t_s^{\prime}} $ \\
 $\prob.\quad ms^{\prime} = \PRF(pms^{\prime}, \NONC) $ \\
\ifONERTT
 $\prob.\quad \key = \PRF(ms^{\prime}, m_1 \| m_2 \| m_3 \| m_4)$ \\
\else
 $\prob.\quad \key = \PRF(ms^{\prime}, m_1 \| m_2)$ \\
\fi
 $\prob.\quad \Lambda = \accept$ \\
\end{tabular}
\end{minipage}%
} \vspace{10pt}

\caption{Abstract model of 0-RTT connection establishment for an initial key in QUIC handshake}\label{fig:quic_abst_0rtt}
\end{center}
\end{figure*}
%
Firstly, the client sends a client hello using cache data.
The server searches $\SCFG$ with $\SCID$ and validate a
client hello.
After the client sends a client hello or the server validate
a client hello, they calculate the initial key $\ik$.
The flow for the final key $\key$ is the same as
1-RTT case.


%=====================================================
\subsection{Security of QUIC} \label{sec:quic_detail}
%=====================================================

In~\cite{LJBN15:QUIC}, they found five attacks:
(1) Server Config Replay Attack,
(2) Source-Address Token Replay Attack,
(3) Connection ID Manipulation Attack,
(4) Source-Address Token Manipulation Attack,
(5) Crypto Stream Offset Attack.

These attacks assist the adversary to do Distributed Denial of Service
(DDoS) attack.

The adversary make a client and server share a different
initial key $\ik$ using (1), (3), and (4).
These attacks break server authentication because the client
reaches accept state and there are no server oracle which has
matching conversations for an initial key with the client oracle in 1-RTT connection.
The adversary make a server accept in 0-RTT connection
using (2).
This attack breaks channel confidentiality because the adversary
can share the session key with the target server.
We do not consider the attack (5) because this attack is not
about handshakes.

For more details about (1), (3) and (4), the adversary can forge
$\STK$ and $\cid$ because there are no authentication mechanism
for these parameters.
The server generate signature for $\SCFG$, however, the generation
of $\SCFG$ is independent from client's first query and these parameters
which generated at client's first query
are not certificated and even authenticated.
If the adversary forge these parameters, the client and server cannot
detect it.

For more details about (2), although QUIC has a mechanism to prevent forgery,
whose mechanisms is called \textit{source-address token}
(see below), the adversary can forge an abbreviate handshake
query as follows: An adversary can obtain $\SCID$ and $\STK$ from $\REJ$
response, which is the first response from a server oracle.
Then, the adversary make $(\overline{T}_c^{\prime},
\overline{\NONC}^{\prime}, \overline{\cid})$ and send $(\overline{T}_c^{\prime},
\SCID, \overline{\NONC}^{\prime}, \STK, \overline{\cid})$ to the server after a
full handshake between the server and its intended partner
is established.
The server cannot distinguish whether the query in an abbreviate handshake
comes from the intended partner or not. Because there is no
authentication mechanism for this query. The server accepts
this query and calculates the session keys with the value
of the adversary.

If the protocol satisfies RSACCE secure, the protocol prevents these attacks (1)~(4).

\subsubsection{Source Address Token} \label{sec:source_address_token}
Source-address token ($\STK$) is introduced to QUIC in order to
prevent IP address spoofing.
A server generates and sends a new STK every time he sends a
message to a client.
The client updates STK when the client receives a new one from
the server and sends it back along with his message.
$\STK$ is an opaque byte string from the client's point of view.
From the server's point of view it's an authenticated-encryption
block that contains, at least, the client's IP address and a time
stamp by the server.
$\STK$ is encrypted except for the first server's query ($\REJ$).
However, in our model the adversary has full control over the
communication network and hence it can obtain $\STK$ and use it to
forge a future query.
 %=====================================================
\section{Our proposed scheme} \label{sec:proposed_scheme}
%=====================================================

\textcolor{red} {
	DDoSを防ぐ重要性とQUICがTLS1.3を今後適用することを鑑みて
	新しいプロトコルを作ったことを説明する。
}
We note that QUIC does not meet RSACCE secure.
We propose a new scheme which is more secure and efficient
than original QUIC and it satisfy RSACCE secure.


%=====================================================
\subsection{1-RTT Connection Establishment} \label{sec:quic_prop_1rtt}
%=====================================================

The abstract model of our proposed scheme is in
Fig.~\ref{fig:quic_prop_1rtt}.

\begin{figure*}[!htp]
 \begin{center}

\begin{enumerate}
 \item{Initiate} \\
% client side
 \fbox{
  \begin{minipage}[t]{0.38\textwidth}
  \centering
   \begin{tabular}{c}
    $ $ \\
    $ $ \\
    $ $ \\
   \end{tabular}
  \end{minipage}%
 }
% middle
 \begin{minipage}[t]{0.13\textwidth}
  \centering
  \begin{tabular}{c}
   $ $ \\
   $ $ \\
   $ $ \\
  \end{tabular}
 \end{minipage}%
% server side
 \fbox{
  \begin{minipage}[t]{0.38\textwidth}
   \centering
   \begin{tabular}{c}
    $(pk_s, sk_s) = \SIG.\Gen()$ \\
    $(\SCFG_{pub}, t_s) = \scfgGen(sk_s)$ \\
    $k_{\STK} \xleftarrow{\$} \{0,1\}^{\lambda}$ \\
   \end{tabular}
  \end{minipage}%
 }
 \item{Key Agreement} \\
% client side
 \fbox{
  \begin{minipage}[t]{0.38\textwidth}
  \centering
   \begin{tabular}{c}
    $m_1 = \initialCHLO()$ \\
    $ $ \\
    $\checkSCFG(\SCFG_{pub}) $ \\
    $\shareInfo_c = (\NONC, \cid, T_s, t_c) $ \\
    $\ik = \getKey_c(\shareInfo_c, m_1, 1) $ \\
    $T_s^{\prime} = \receiveSHLO(m_2, \ik)$ \\
    $\shareInfo_c^{\prime} = (\NONC, \cid, T_s^{\prime}, t_c)$ \\
    $m = m_1 \| m_2$ \\
    $\peer = S$ \\
    $k = \getKey_c(\shareInfo_c^{\prime}, m, 0)$ \\
    $\Lambda = \accept$ \\
   \end{tabular}
  \end{minipage}%
 }
% middle
 \begin{minipage}[t]{0.13\textwidth}
  \centering
  \begin{tabular}{c}
   $\xrightarrow{m_1}$ \\
   $ $ \\
   $\xleftarrow{m_2}$ \\
   $ $ \\
   $ $ \\
   $ $ \\
   $ $ \\
  \end{tabular}
 \end{minipage}%
% server side
 \fbox{
  \begin{minipage}[t]{0.38\textwidth}
   \centering
   \begin{tabular}{c}
    $ $ \\
    $ $ \\
    $\shareInfo_s = (\NONC, \cid, T_c, t_s)$ \\
    $\ik = \getKey_s(\shareInfo_s, m_1, 1)$ \\
    $ret = \SHLO(m_1, \ik, 0)$ \\
    $m_2 = ret \| \SCFG_{pub} $ \\
    $\shareInfo_s^{\prime} = (\NONC, \cid, T_c, t_s^{\prime}) $ \\
    $m = m_1 \| m_2$ \\
    $\peer = C$ \\
    $k=\getKey_s(\shareInfo_s^{\prime}, m, 0)$ \\
    $\Lambda = \accept$ \\
   \end{tabular}
  \end{minipage}%
 }
 \item{Data Exchange} \\
% client side
 \fbox{
  \begin{minipage}[t]{0.38\textwidth}
  \centering
   \begin{tabular}{c}
    $ $ \\
    $ $ \\
    $ $ \\
    $\text{for each } \alpha \in {0,...,\MsgCntC{0}}$ \\
    $\sqn_c = \alpha + 1$ \\
    $m_3^{\alpha} = \pak(k, sqn_c, M_c^{\alpha})$ \\
    $ $ \\
    $m_3 = (m_3^{0},...,m_3^{\MsgCntC{0}})$ \\
    $\processPacket(k, m_4)$ \\
   \end{tabular}
  \end{minipage}%
 }
% middle
 \begin{minipage}[t]{0.13\textwidth}
  \centering
  \begin{tabular}{c}
   $ $ \\
   $ $ \\
   $ $ \\
   $ $ \\
   $\xrightarrow{m_3}$ \\
   $\xleftarrow{m_4}$ \\
  \end{tabular}
 \end{minipage}%
% server side
 \fbox{
  \begin{minipage}[t]{0.38\textwidth}
   \centering
   \begin{tabular}{c}
    $\text{If the first message in $m_3$ does not come} $ \\
    $\text{in $\duration$ QUIC regards the first query as}$ \\
    $\text{DoS attack and this connection is closed.}$ \\
    $\text{for each } \beta \in {0,...,\MsgCntS{0}}$ \\
    $\sqn_s = \beta + 1$ \\
    $m_4^{\beta} = \pak(k, \sqn_s, M_s^{\beta})$ \\
    $ $ \\
    $m_4 = (m_4^{\MsgCntS{0}+1},...,m_4^{\MsgCntS{1}})$ \\
    $\processPacket(ik, m_3)$ \\
   \end{tabular}
  \end{minipage}%
 }
\end{enumerate}

 \caption{Abstract model of 1-RTT our proposed scheme}\label{fig:quic_prop_1rtt}
 \end{center}
\end{figure*}

We define three phases of our proposed scheme handshake in 1-RTT:
(1) \textbf{Initiate},
(2) \textbf{Key Agreement},
(3) \textbf{Data Exchange}.
The flow of (1), (3) are the same as the same as the original QUIC
handshake in 1-RTT.

%=====================================================
\subsubsection{Key Agreement}
%=====================================================
In this phase, the client sends an initial client
hello (initialCHLO) which contains connection id,
a client's Diffie-Hellman public value $T_c$, a client
nonce $\NONC$, and some information such as server name,
protocol version, and user agent id. In our definition,
the some informations are omitted.
\\
\noindent
\underline{$\initialCHLO()$:} \\
 \setcounter{nombre}{0}%
 $\prob.\quad \cid \xleftarrow{\$} \{0,1\}^{64} $ \\
 $\prob.\quad \pInfo = (IP_c, IP_s, port_c, port_s)$ \\
 $\prob.\quad t_c \xleftarrow{\$} \Zset_{q}^{\ast}$ \\
 $\prob.\quad T_c = g^{t_c}$ \\
 $\prob.\quad r \xleftarrow{\$} \{0,1\}^{160}$ \\
 $\prob.\quad \NONC = currentTime \| r$ \\
 $\prob.\quad \return\ (\pInfo, \cid, \NONC, T_c)$ \\
%
After the server receives initial client hello, it
sends a server hello (SHLO). The server hello contains
source address token (STK), server config (SCFG),
a certificate, a signature of server config generated
by the server long term secret key, ephemeral server's
Diffie-Hellman public value $T_s^{\prime}$. The client use
STK in future queries to demonstrate ownership of their
source IP address.
\\
\noindent
\underline{$\SHLO(m, \ik, \sqn)$:} \\
 \setcounter{nombre}{0}%
 $\prob.\quad (\pInfo, \cid, \NONC, T_c) = m$ \\
 $\prob.\quad \STK = \makeSTK()$ \\
 $\prob.\quad \pInfo = (IP_s, IP_c, port_s, port_c)$ \\
 $\prob.\quad (\ik_c, \ik_s, \iv_c, \iv_s) = \ik$ \\
 $\prob.\quad t_s^{\prime} \xleftarrow{\$} \Zset_{q}^{\ast}$ \\
 $\prob.\quad T_s^{\prime} = g^{t_s^{\prime}}$ \\
 $\prob.\quad \plaintext = T_{s}^{\prime} \| \STK \| k_{MAC} \| \SCID$\\
 $\prob.\quad H = (\cid, \sqn)$ \\
 $\prob.\quad c = \SE.\Enc(\ik_c, \iv_c \| \sqn, H, \plaintext)$ \\
 $\prob.\quad \return\ (\pInfo, \cid, \SCFG_{pub}, H, c)$ \\
\\
\underline{$\makeSTK()$:} \\
 \setcounter{nombre}{0}%
 $\prob.\quad \iv_{\STK} \xleftarrow{\$} \{0,1\}^{96}$ \\
 $\prob.\quad k_{MAC} \xleftarrow{\$} \{0,1\}^{\mu}$ \\
 $\prob.\quad \plaintext = IP_c \| currentTime \| k_{\MAC}$ \\
 $\prob.\quad \STK \leftarrow \iv_{\STK} \| \SE.\Enc(k_{\STK}, len, \iv_{\STK}, \plaintext)$ \\
 $\prob.\quad \return\ \STK$ \\
%
After the client receives a server hello, the client
checks server config and calculate a forward secure key.
\\
\noindent
\underline{$\receiveSHLO(m, \ik)$:} \\
 \setcounter{nombre}{0}%
 $\prob.\quad (\pInfo, \cid, \SCFG_{pub}, H, c) = m$ \\
 $\prob.\quad (\ik_c, \ik_s, \iv_c, \iv_s) = \ik$ \\
 $\prob.\quad (\cid, \sqn) = H$ \\
 $\prob.\quad \plaintext = \SE.\Dec(\ik_c, \iv_c \| \sqn, H, c)$ \\
 $\prob.\quad \text{If }\plaintext = \perp$ \\
 $\prob.\quad \quad \Lambda = \text{'reject' and abort}$ \\
 $\prob.\quad T_{s}^{\prime} \| \STK \| \key_{MAC} \| \SCID^{\prime} = \plaintext $ \\
 $\prob.\quad \text{If }\SCID^{\prime} \neq \SCID$ \\
 $\prob.\quad \quad \Lambda = \text{'reject' and abort}$ \\
 $\prob.\quad \return\ (T_s^{\prime}, \STK, \key_{MAC})$ \\
\\
\underline{$\checkSCFG(\SCFG_{pub})$:} \\
 \setcounter{nombre}{0}%
 $\prob.\quad (\SCID, T_s, \expy, \sigma_s, \cert_s) = \SCFG_{pub}$ \\
 $\prob.\quad \text{If } \expy \leq currentTime$ \\
 $\prob.\quad \quad \Lambda = \text{'reject' and abort}$ \\
 $\prob.\quad pk_s = \getPK(cert_s)$ \\
 $\prob.\quad str = \text{ QUIC server config signature }$ \\
 $\prob.\quad \doc = str \| 0x00 \| \SCID \| T_s \| \expy$ \\
 $\prob.\quad \text{If } \SIG.\Vfy(pk_s, \sigma_s, \doc) = \perp$ \\
 $\prob.\quad \quad \Lambda = \text{'reject' and abort}$ \\
%
After the server sends a server hello or the
client validates a server hello, they calculate forward secure
key $\key$.
\noindent
\underline{$\getKey_c(\shareInfo, m, \init)$:} \\
 \setcounter{nombre}{0}%
 $\prob.\quad (\NONC, \cid ,T_s, t_c) = \shareInfo$ \\
 $\prob.\quad pms = T_s^{t_c}$ \\
 $\prob.\quad \return\ \extractKey(pms, \NONC, \cid, m, 40, \init)$ \\
\underline{$\getKey_s(\shareInfo, m, \init)$:} \\
 \setcounter{nombre}{0}%
 $\prob.\quad (\NONC, \cid ,T_c, t_s) = \shareInfo$ \\
 $\prob.\quad pms = T_c^{t_s}$ \\
 $\prob.\quad \return\ \extractKey(pms, \NONC, \cid, m, 40, \init)$ \\
\underline{$\extractKey(pms, \NONC, \cid, m, \ell, \init)$:}\\
 \setcounter{nombre}{0}%
 $\prob.\quad ms = \PRF(pms, \NONC)$ \\
 $\prob.\quad \text{If } \init = 1$ \\
 $\prob.\quad \quad str = \text{ QUIC key expansion }$ \\
 $\prob.\quad \text{Else }$ \\
 $\prob.\quad \quad str = \text{ QUIC forward secure expansion }$ \\
 $\prob.\quad \info = str \| 0x00 \| \cid \| m \| \SCFG_{pub}$ \\
 $\prob.\quad \return\ \text{the first $\ell$ octets (i.e. bytes) of T = }$ \\
 $\quad \quad \text{(T(1),T(2), ...), where for all $i \in \Nset$, $T(i) = $} $\\
 $\quad \quad \text{$\PRF(ms, T(i-1) \| \info \| 0x0i)$ and $T(0) = \epsilon$} $\\

%=====================================================
\subsection{0-RTT Connection Establishment} \label{sec:quic_prop_0rtt}
%=====================================================

The abstract model of our proposed scheme is in
Fig.~\ref{fig:quic_prop_0rtt}.

\begin{figure*}[!htp]
 \begin{center}

\begin{enumerate}
 \item{Initial Key Agreement} \\
% client side
 \fbox{
  \begin{minipage}[t]{0.38\textwidth}
  \centering
   \begin{tabular}{c}
    $\quad Client$ \\
    $ $ \\
    $(\SCFG_{pub}, \STK, \key_{MAC}) = \theta$ \\
    $m_1 = \CHLO(\STK, \SCFG_{pub}, \key_{MAC})$ \\
    $\shareInfo_c = (\NONC, \cid, T_s, t_c^{\ast})$ \\
    $\ik = \getKey_c(\shareInfo_c, m_1, 1)$ \\
    $\Lambda = \preaccept$ \\
   \end{tabular}
  \end{minipage}%
 }
% middle
 \begin{minipage}[t]{0.13\textwidth}
  \centering
  \begin{tabular}{c}
   $\xrightarrow{m_1}$ \\
   $ $ \\
  \end{tabular}
 \end{minipage}%
% server side
 \fbox{
  \begin{minipage}[t]{0.38\textwidth}
   \centering
   \begin{tabular}{c}
    $\quad Server$ \\
    $ $ \\
    $\checkQuery(m_1, k_{\STK}, IP_c)$ \\
    $\shareInfo_s = (\NONC, \cid, T_c^{\ast}, t_s)$ \\
    $\ik = \getKey_s(\shareInfo_s, m_1, 1)$ \\
    $\Lambda = \preaccept$ \\
   \end{tabular}
  \end{minipage}%
 }
 \item{Initial Data Exchange} \\
% client side
 \fbox{
  \begin{minipage}[t]{0.38\textwidth}
  \centering
   \begin{tabular}{c}
    $\text{for each } \alpha \in {\MsgCntC{0}+1,...,\MsgCntC{1}}$ \\
    $\sqn_c = \alpha + 2$ \\
    $m_{2}^{\alpha} = \pak(ik, sqn_c, M_c^{\alpha})$ \\
    $ $ \\
    $m_{2} = (m_{2}^{\MsgCntC{0}+1},...,m_{2}^{\MsgCntC{1}})$ \\
    $\processPacket(ik, m_{3})$ \\
   \end{tabular}
  \end{minipage}%
 }
% middle
 \begin{minipage}[t]{0.13\textwidth}
  \centering
  \begin{tabular}{c}
   $ $ \\
   $ $ \\
   $ $ \\
   $ $ \\
   $\xrightarrow{m_{2}}$ \\
   $\xleftarrow{m_{3}}$ \\
  \end{tabular}
 \end{minipage}%
% server side
 \fbox{
  \begin{minipage}[t]{0.38\textwidth}
   \centering
   \begin{tabular}{c}
    $\text{for each } \beta \in {\MsgCntS{0}+1,...,\MsgCntS{1}}$ \\
    $\sqn_s = \beta + 2$ \\
    $m_{3}^{\beta} = \pak(ik, \sqn_s, M_s^{\beta})$ \\
    $ $ \\
    $m_{3} = (m_{3}^{\MsgCntS{0}+1},...,m_{3}^{\MsgCntS{1}})$ \\
    $\processPacket(ik, m_{2})$ \\
   \end{tabular}
  \end{minipage}%
 }
 \item{Key Agreement} \\
% client side
 \fbox{
  \begin{minipage}[t]{0.38\textwidth}
  \centering
   \begin{tabular}{c}
    $(T_s^{\prime}, \STK, \key_{MAC}) = \receiveSHLO(m_4)$ \\
    $\shareInfo_c = (\NONC, \cid, T_s^{\prime}, t_c^{\ast})$ \\
    $m = m_1 \| m_4$ \\
    $k = \getKey_c(\shareInfo_c, m, 0)$ \\
    $\theta = (\SCFG_{pub}, \STK, \key_{MAC})$ \\
    $\Lambda = \accept$ \\
   \end{tabular}
  \end{minipage}%
 }
% middle
 \begin{minipage}[t]{0.13\textwidth}
  \centering
  \begin{tabular}{c}
   $ $ \\
   $\xleftarrow{m_{4}}$ \\
   $ $ \\
  \end{tabular}
 \end{minipage}%
% server side
 \fbox{
  \begin{minipage}[t]{0.38\textwidth}
   \centering
   \begin{tabular}{c}
    $ \sqn_s = \MsgCntS{1} + 3$ \\
    $m_{4} = \SHLO(m_1, \ik, \sqn_s)$ \\
    $\shareInfo_s = (\NONC, \cid, T_c^{\ast}, t_s^{\prime})$ \\
    $m = m_1 \| m_4$ \\
    $k = \getKey_s(\shareInfo_s, m, 0)$ \\
    $\Lambda = \accept$ \\
   \end{tabular}
  \end{minipage}%
 }
 \item{Data Exchange} \\
% client side
 \fbox{
  \begin{minipage}[t]{0.38\textwidth}
  \centering
   \begin{tabular}{c}
    $\text{for each } \alpha \in {\MsgCntC{1}+1,...,\MsgCntC{2}}$ \\
    $\sqn_c = \alpha + 3$ \\
    $m_{5}^{\alpha} = \pak(k, sqn_c, M_c^{\alpha})$ \\
    $ $ \\
    $m_{5} = (m_{5}^{\MsgCntC{1}+1},...,m_{5}^{\MsgCntC{2}})$ \\
    $\processPacket(k, m_{6})$ \\
   \end{tabular}
  \end{minipage}%
 }
% middle
 \begin{minipage}[t]{0.13\textwidth}
  \centering
  \begin{tabular}{c}
   $ $ \\
   $ $ \\
   $ $ \\
   $ $ \\
   $\xrightarrow{m_{5}}$ \\
   $\xleftarrow{m_{6}}$ \\
  \end{tabular}
 \end{minipage}%
% server side
 \fbox{
  \begin{minipage}[t]{0.38\textwidth}
   \centering
   \begin{tabular}{c}
    $\text{for each } \beta \in {\MsgCntS{1}+1,...,\MsgCntS{2}}$ \\
    $\sqn_s = \beta + 3$ \\
    $m_{6}^{\beta} = \pak(k, \sqn_s, M_s^{\beta})$ \\
    $ $ \\
    $m_{6} = (m_{6}^{\MsgCntS{1}+1},...,m_{6}^{\MsgCntS{2}})$ \\
    $\processPacket(ik, m_{5})$ \\
   \end{tabular}
  \end{minipage}%
 }
\end{enumerate}
 \caption{Abstract model of 0-RTT in our proposed scheme}\label{fig:quic_prop_0rtt}
 \end{center}
\end{figure*}

We define four phases of our proposed scheme handshake in 0-RTT:
(1) \textbf{Initial Key Agreement},
(2) \textbf{Initial Data Exchange},
(3) \textbf{Key Agreement}, and
(4) \textbf{Data Exchange}.
The flow of (2), (4) are the same as the 1-RTT
handshake.

%=====================================================
\subsubsection{Initial Key Agreement}
%=====================================================
In this phase, the client sends an client with source
address token $\STK$.
\\
\noindent
\underline{$\CHLO(\STK, \SCFG_{pub}, k_{MAC})$:} \\
 \setcounter{nombre}{0}%
 $\prob.\quad \cid \xleftarrow{\$} \{0,1\}^{64}$ \\
 $\prob.\quad r \xleftarrow{\$} \{0,1\}^{160}$ \\
 $\prob.\quad \NONC = currentTime \| r$ \\
 $\prob.\quad t_c^{\ast} \xleftarrow{\$} \Zset_{q}^{\ast}$ \\
 $\prob.\quad T_c^{\ast} = g^{t_c^{\ast}}$ \\
 $\prob.\quad \pInfo = (IP_c, IP_s, port_c, port_s)$ \\
 $\prob.\quad \doc = T_c^{\ast} \| \NONC \| \cid \| \STK \| \SCID$ \\
 $\prob.\quad \mac = \PRF(k_{MAC}, \doc) $ \\
 $\prob.\quad \return\ (\pInfo, \cid, \STK, \SCID, \NONC, T_c^{\ast}, \mac)$ \\
\\
\underline{$\checkQuery(m, k_{\STK}, IP_c)$:} \\
 \setcounter{nombre}{0}%
 $\prob.\quad (\pInfo, \cid, \STK, \SCID, \NONC, T_c^{\ast}, \mac) = m$ \\
 $\prob.\quad (\iv_{\STK}, c) = \STK$ \\
 $\prob.\quad IP_c^{\prime} \| currentTime \| k_{\MAC} = \SE.\Dec(k_{\STK}, \iv_{\STK}, c)$ \\
 $\prob.\quad \doc = T_c^{\ast} \| \NONC \| \cid \| \STK \| \SCID$ \\
 $\prob.\quad \text{If } \PRF(k_{MAC}, \doc) \neq \mac$ \\
 $\prob.\quad \quad \Lambda = \text{'reject' and abort}$ \\
 $\prob.\quad (time_{\NONC}, r) = \NONC$ \\
 $\prob.\quad \text{If } (IP_c^{\prime}, currentTime) = \perp$, or \\
 $\prob.\quad \quad IP_c^{\prime} \neq IP_c$, or $time_{\STK} \leq time_{allowed}$\\
 $\prob.\quad \quad r \in \strike$, or $time_{\NONC} \not\in \strike_{rng}$ \\
 $\prob.\quad \quad \quad \Lambda = \text{'reject' and abort}$ \\
\\
%=====================================================
\subsubsection{Key Agreement}
%=====================================================
In this phase, the server run $\SHLO$ to share forward
secure key $\key$ and the client validate the query running
$\receiveSHLO$.
\\
\noindent
\underline{$\SHLO(m, \ik, \sqn)$:} \\
 \setcounter{nombre}{0}%
 $\prob.\quad (\pInfo, \cid, \STK, \SCID, \NONC, T_c) = m$ \\
 $\prob.\quad (\ik_c, \ik_s, \iv_c, \iv_s) = \ik$ \\
 $\prob.\quad t_s^{\prime} \xleftarrow{\$} \Zset_{q}^{\ast}$ \\
 $\prob.\quad T_s^{\prime} = g^{t_s^{\prime}}$ \\
 $\prob.\quad \STK = \makeSTK()$ \\
 $\prob.\quad H = (\cid, \sqn)$ \\
 $\prob.\quad \plaintext = \SCFG_{pub} \| T_s^{\prime} \| \STK $\\
 $\prob.\quad c = \SE.\Enc(\ik_c, \iv_c \| \sqn, H, \plaintext)$ \\
 $\prob.\quad \return\ (\pInfo, H, c)$ \\
\underline{$\receiveSHLO(m, \ik)$:} \\
 \setcounter{nombre}{0}%
 $\prob.\quad (\pInfo, H, c) = m$ \\
 $\prob.\quad (\ik_c, \ik_s, \iv_c, \iv_s) = \ik$ \\
 $\prob.\quad (\cid, \sqn) = H$ \\
 $\prob.\quad \plaintext = \SE.\Dec(\ik_c, \iv_c \| \sqn, H, c)$ \\
 $\prob.\quad \text{If }\plaintext = \perp$ \\
 $\prob.\quad \quad \Lambda = \text{'reject' and abort}$ \\
 $\prob.\quad \SCFG_{pub} \| T_s^{\prime} \| \STK  = \plaintext $ \\
 $\prob.\quad \return\ T_s^{\prime}$ \\
%
After the server sends a server hello or the
client validates a server hello, they calculate forward secure
key $\key$.

%=====================================================
\subsection{Security of our proposed scheme} \label{sec:quic_proof}
%=====================================================



\begin{theorem} \label{theorem:proposed_scheme}
 Let $\mu$ be the output length of $\PRF$, let $\lambda$ be
 the length of $\SCID$, let $\nclient$ be the number of
 clients, let $\nserver$ be the number of servers, let
 $\noracle$ be the number of oracles of each parties, and
 let $n_{\ell}$ be the maximum number of resumptions. Assume
 that the $\PRF$ is $(t, \epsilon_{\prf})$-pseudo-random
 function family, the signature shceme
 $\SIG$ is $(t, \epsilon_{\sig})$-secure against existentially
 unforgeable under adaptive chosen-message attacks, the DDH
 problem on $G$ is $(t, \epsilon_{\ddh})$-hard, the hash
 function family $\mathcal{H}$ is
 $(t,\epsilon_{H})$-collision-resistant (CR), the symmetric
 authenticated encryption scheme $\SE$ is
 $(t, \epsilon_{\sLHAE})$-secure.
 Then for all PPT adversaries, our proposed scheme is RSACCE secure.
\end{theorem}

Our proposed scheme can prevent the five attacks in~\cite{LJBN15:QUIC}.

\begin{itemize}
 \item{Server Config Replay Attack}
  In this attack, the adversary collect
 \item{Source-Address Token Replay Attack}
 \item{Connection ID Manipulation Attack}
 \item{Source-Address Token Manipulation Attack}
 \item{Crypto Stream Offset Attack}
\end{itemize}

We prove Theorem~\ref{theorem:proposed_scheme} by proving two lemmas.

\begin{lemma} \label{lemma:proposed_scheme_rsacce-sa}
 $\Adv^{\rsaccesa}_{P}(A)$ is at most
 \begin{equation}
  \Adv^{\rsaccesa}_{P}(A) \leq \frac{n_s n_c}{2^{\lambda}} m_s \epsilon_{\sig}
 \end{equation}
\end{lemma}
%
\begin{proof}
 The proof proceeds in a sequence of games. \vspace{10pt}\\
 {\bfseries Game 0.} This game equals the \textit{server authentication} experiment in Def.~\ref{def:rsacce-sa}.\\
 \begin{equation}
  \Adv_0 = \Adv^{\rsaccesa}_{P}(A)
 \end{equation}%
%
%
 \textbf{Game 1.} We try to guess which client oracle will be the first oracle to break \textit{server authentication}. If our guess is wrong, i.e. if there is another client oracle that breaks \textit{server authentication} before, then we abort the game.

 Technically, the game is identical to Game 0, except for the following. The challenger guesses three random indexes $(c^{\ast}, i^{\ast}, \ell^{\ast}) \xleftarrow{\$} [\nclient] \times [\noracle] \times [n_{\ell}]$. If there exists a client oracle $\pi^c_{i,\ell}$ that breaks server authentication, and $(c, i, \ell) \neq c^{\ast}, i^{\ast}, \ell^{\ast})$, then the challenger aborts the game. Note that if the first oracle $\pi^c_{i,\ell}$ that breaks server authentication, then with probability $1/(\nclient \noracle n_{\ell})$ we have $(c,i,\ell) = (c^{\ast}, i^{\ast}, \ell^{\ast})$, and thus
 \begin{equation}
  \Adv_0 \leq \nclient \noracle n_{\ell} \Adv_1.
 \end{equation}%
 Note that in this game the attacker can only break the security of the protocol, if oracle $\pi^{c^{\ast}}_{i^{\ast},\ell^{\ast}}$ is the first oracle that breaks server authentication, as otherwise the game is aborted.
\vspace{10pt}\\%
%
%
 \textbf{Game 2.} Again the challenger proceeds as before, but we add an abort rule. We want to make sure that $\pi^{c^\ast}_{i^{\ast},0}$ receives as input exactly the Diffie-Hellman value $T_s$ in $\SCFG_{pub}$ that was selected by some other uncorrupted server oracle.

 Technically, we abort and raise event $\abort_\SIG$, if oracle $\pi^{c^{\ast}}_{i^{\ast},0}$ ever receives as input a message $\cert_s$ indicating intended partner $\peer = s$ and server config $\SCFG_{pub} = (\SCID, T_s, \expy, \sigma_s, \cert_s)$ such taht $\sigma_s$ is a valid signature over $\SCID \| T_s \| \expy \| \cert_s$, however there exists no oracle $\pi^s_{j,0}$ which has previously output $\sigma_s$. Clearly we have
 \begin{equation}
  |\Adv_2 - \Adv_1| = \Pr[\abort_{\SIG}].
 \end{equation}%

 Note that the experiment is aborted, if $\pi^{c^{\ast}}_{i^{\ast},0}$ satisfies server authentication, due to Game 1. This means that server $\Server_s$ must be $\tau_s$-corrupted with $\tau_s = \infty$ (i.e. not corrupted) when $\pi^{c^{\ast}}_{i^{\ast},0}$ accepts (as otherwise $\pi^{c^{\ast}}_{i^{\ast},0}$ satisfies server authentication). To show that $\Pr[\abort_{\SIG}] \leq \ell \epsilon_{\SIG}$, we construct a signature forger as follows. The forger receives as input a public key $pk^{\ast}$ and simulates the challenger for $\mathcal{A}$. It guesses an index $\phi \xleftarrow{\$}[\nserver]$, sets $pk_{\phi} = pk^{\ast}$, and generates all long-term public/secret keys as before. Then it proceeds as the challenger in Game 3, except that it uses its chosen message oracle to generate a signature under $pk_{\phi}$ when necessary.

 If $\phi = s$, which happens with probability $1/\nserver$, then the forger can use the signature received by $\pi^{c^{\ast}}_{i^{\ast},\ell^{\ast}}$ to break the EUF-CMA security of the signature scheme with success probability $\epsilon_{\SIG}$, so $\Pr[\abort_{\SIG}]/\ell \leq \epsilon_{\SIG}$. Therefore if $\Pr[\abort_{\SIG}]$ is not negligible, then $\epsilon_{\SIG}$ is not negligible as well and we have
 \begin{equation}
  |\Adv_2 - \Adv_1| = \nserver \epsilon_{\SIG}.
 \end{equation}%

 Note that in Game 2 oracle $\pi^{c^{\ast}}_{i^{\ast},0}$ receives as input a Diffie-Hellman value $T_s$ such that $T_s$ was chosen by another oracle, but not by the attacker. Note also that there is unique oracle that issued a signature $\sigma_s$ containing SCID.
\vspace{10pt}\\%
%
%
 \textbf{Game 3.} In this game we want to make sure that we know which oracle $\pi^s_{j,0}$ will issue the signature $\sigma_s$ that $\cOracleAstFull$ receives. The challenger in this game proceeds as before, however additionally guesses two indices $(s^{\ast}, j^{\ast}) \xleftarrow{\$} [\nserver] \times [\noracle]$.

 We know that there must exists at least one oracle that outputs $\sigma_s$ such that $\sigma_s$ is forwarded to $\cOracleAstFull$, due to Game 2. Thus we have
 \begin{equation}
  \Adv_3 \leq \nserver \noracle \Adv_4
 \end{equation}%
 Note that in this game we know exactly that oracle $\sOracleAstFull$ chooses the Diffie-Hellman share $T_s$ that $\cOracleAstFull$ uses to compute its premaster secret.
 \vspace{10pt}\\
%
%
 \textbf{Game 4.} Let $T_{\cIndexAstRes} = g^u$ denote the Diffie-Hellman share chosen by $\cOracleAstRes$, let $T_{\sIndexAstRes} = g^v$ denote the share chosen by its partner $\sOracleAstRes$, and let $ik_{\cIndexAstRes}$ and $k_{\cIndexAstRes}$ are the key computed by $\cOracleAstRes$, let $pms_{\ik}$ and $ms_{\ik}$ denote a premaster secret and master secret for initial key, let $pms_{k}$ and $ms_{k}$ denote a premaster secret and master secret for session key. Thus, both oracles compute the premaster secret as $pms = g^{uv}$.

 The challenger in this game proceeds as before, however replaces the premaster secret $pms_{\ik}$ of $\cOracleAstRes$ and $\sOracleAstRes$ with a random group element $\widetilde{pms_{\ik}} = g^w$, $w \xleftarrow{\$} \Zset_p$. Note that both $g^u$ and $g^v$ are chosen by oracles $\cOracleAstRes$ and $\sOracleAstRes$, respectively, as otherwise $\cOracleAstRes$ would not have a matching conversation to $\sOracleAstRes$ and the game would be aborted.

 Distinguish Game 4 from Game 3 implies an algorithm solving the decisional Diffie-Hellman problem, thus
 \begin{equation}
  |\Adv_{4} - \Adv_{3}| \leq \epsilon_{\ddh}
 \end{equation}%
%
%
 \textbf{Game 5.} In this game we replace the value $ms_{\ik} = \PRF(\widetilde{pms_{\ik}}, \NONC)$ with a random value $\widetilde{ms_{\ik}}$.

 Distinguishing Game 5 from Game 4 implies an algorithm breaking the security of the pseudo random function $\PRF$, thus
 \begin{equation}
  |\Adv_{5} - \Adv_{4}| \leq \epsilon_{\prf}
 \end{equation}%
%
%
 \textbf{Game 6.} In this game we replace the function $\PRF(\widetilde{ms_{\ik}},\cdot)$ with a random function. If $\sOracleAstRes$ uses the same master secret $\widetilde{ms_{\ik}}$ as $\cOracleAstRes$, then the function $\PRF(\widetilde{ms_{\ik}},\cdot)$ used by $\sOracleAstRes$ is replaced as well. Of course the same random function is used for both oracles sharing the same $\widetilde{ms_{\ik}}$.

 Distinguishing Game 6 from Game 5 implies an algorithm breaking the security of the pseudo random function $\PRF$, thus
 \begin{equation}
  |\Adv_6 - \Adv_5| \leq \epsilon_{\prf}.
 \end{equation}%
%
%
 \textbf{Game 7.} Now we use that the key $\ik$ used by $\cOracleAstRes$ and $\sOracleAstRes$ in the symmetric encryption scheme uniformly at random and independent of all QUIC handshake messages.

 The adversary have to make ciphertext $c$ such that $\SE$.$\Dec$ ( $\ik$, $\iv$, H, $c$) $\neq \perp$ without knowing the key $k$. It implies an algorithm breaking the LHAE security of the symmetric encryption scheme, we have
 \begin{equation}
  \Adv_7 \leq 1/2 + \epsilon_{\LHAE}.
 \end{equation}%
\end{proof}

\begin{lemma} \label{lemma:proposed_scheme_rsacce-cc}
 $\Adv^{\rsaccecc}_{P}(A)$ is at most
 \begin{equation}
  \Adv^{\rsaccecc}_{P}(A) \leq \Adv^{\rsaccesa}_{P}(A) + n_sn_c\ell(\epsilon_{\ddh} + 2\epsilon_{\prf} + 2\epsilon_{\sLHAE})
 \end{equation}
\end{lemma}
%
\begin{proof}
 The proof proceeds in a sequence of games. \vspace{10pt}\\
 \textbf{Game 0.} This game equals the \textit{channel confidetility} security experiment.
 \begin{equation}
  \Adv_0 = \Adv^{\rsaccecc}_{P}(A)
 \end{equation}%
%
%
 \textbf{Game 1.} The challenger in this game proceeds as before, however it aborts and chooses $b^{\prime}$ uniformly random, if there exists any oracle that breaks server authentication. Thus we have
 \begin{equation}
  |\Adv_1 - \Adv_0| = \Adv^{\rsaccesa}_{P}(A).
 \end{equation}%
 Note that if there exists the oracle which breaks \textit{server authentication}, the adversary easily breaks \textit{channel confidentiality}. If the target oracle breaks \textit{server authentication}, the adversary or unrelated oracle (i.e. it is not an intended partner of the oracle) can establish the session with the oracle. The adversary can issue $\Reveal$-query to unrelated oracle which is out of the restriction of \textit{channel confidentiality}. Then the adversary can know the session key of the target oracle.
\vspace{10pt}\\%
%
%
 \textbf{Game 2.} The challenger in this game proceeds as before, however in addition guesses indices $(p^{\ast}, i^{\ast}, \ell^{\ast}) \xleftarrow{\$} [n_s + n_c] \times [n_o] \times [n_{\ell}]$. It aborts and chooses $b^{\prime}$ at random, if the attacker issues a $\Encrypt$-query with $(p,i,\ell) \neq (p^{\ast}, i^{\ast}, \ell^{\ast})$. With probability $1/((n_s+n_c)n_o n_{\ell})$ we have $(p,i,\ell) = (p^{\ast}, i^{\ast}, \ell^{\ast})$, and thus
 \begin{equation}
  \Adv_1 = (n_s + n_c) n_o n_{\ell}\Adv_2.
 \end{equation}%
 Note that in Game 2 we know that $\mathcal{A}$ will issue a $\Encrypt$-query to oracle $\pOracleAstRes$. Note also that $\pOracleAstRes$ has a unique partner due to Game 1. In the sequel we denote with $\qOracleAstRes$ the unique oracle such that $\pOracleAstRes$ has a matching conversations for an initial key or a final key with $\qOracleAstRes$, and say that $\qOracleAstRes$ is the intended partner of $\pOracleAstRes$.
\vspace{10pt}\\%
%
%
 \textbf{Game 3 + $4\ell$.} We repeat the game until randomizing the session between $\pOracleAstRes$ and $\qOracleAstRes$. Initially $\ell = 0$. The reason of repeating is that the adversary always return correct $b^{\prime}$ if the adversary can obtain $k_{mac}$ of $\pOracleAstEll$ or $\qOracleAstEll$. We need to prevent the adversary obtaining all $k_{mac}$ in $ 0 \leq \ell \leq \ell^{\ast}$.
 Let $T_{\pindexell} = g^u$ denote the DH public key chosen by $\pOracleAstEll$, let $T_{\qindexell} = g^v$ denote the share chosen by its partner $\qOracleAstEll$, and let $\ik_{\pindexell}$ and $k_{\pindexell}$ are the key computed by $\pOracleAstEll$, let $pms_{\ik}$ and $ms_{\ik}$ denote a premaster secret and master secret for an initial key, let $pms_{k}$ and $ms_{k}$ denote a premaster secret and master secret for session key. Thus, both oracles compute the premaster secret as $pms = g^{uv}$.

 The challenger in this game proceeds as before, however replaces the premaster secret $pms_{\ik}$ of $\pOracleAstEll$ and $\qOracleAstEll$ with a random group element $\widetilde{pms_{\ik}} = g^w$, $w \xleftarrow{\$} \Zset_p$. Note that both $g^u$ and $g^v$ are chosen by oracles $\pOracleAstEll$ and $\qOracleAstEll$, respectively, as otherwise $\pOracleAstEll$ would not have a matching conversations for an initial key with $\qOracleAstEll$ and the game would be aborted.

 Suppose that there exists an algorithm $\mathcal{A}$ distinguishes Game 3 + $4\ell$ from Game 2 + $4\ell$. Then we can construct an algorithm $\mathcal{B}$ solving the DDH problem as follows. $\mathcal{B}$ receives as input $(g,g^u,g^v,g^w)$. The challenger defines $T_{\pindexell} := g^u$ and $T_{\qindexell} := g^v$, and the premaster secret of $\pOracleAstEll$ and $\qOracleAstEll$ equal to $pms_{\ik} := g^w$. Note that $\mathcal{B}$ can simulate all messages exchanged between $\pOracleAstEll$ and $\qOracleAstEll$ properly. Since all other oracles are not modified, $\mathcal{B}$ can simulate these oracles properly as well.

 If $w=uv$, then the view of $\mathcal{A}$ when interacting with $\mathcal{B}$ is identical to Game 2 + $4\ell$, while if $w \xleftarrow{\$}\Zset_p$ then it is identical to Game 3 + $4\ell$. Thus, the DDH assumption implies that
 \begin{equation}
  |\Adv_{3 + 4\ell} - \Adv_{2 + 4\ell}| \leq \epsilon_{\ddh}
 \end{equation}%
%
%
 \textbf{Game 4 + $4\ell$.} In Game 4 + 4$\ell$ we make use of the fact that the premaster secret $\widetilde{pms_{\ik}}$ of $\pOracleAstEll$ and $\qOracleAstEll$ is chosen uniformly random. We thus replace the value $ms_{\ik} = \PRF(\widetilde{pms_{\ik}}, \NONC)$ with a random value $\widetilde{ms_{\ik}}$.

 Distinguish Game 4 + 4$\ell$ from Game 3 + 4$\ell$ implies an algorithm breaking the security of the pseudo random function $\PRF$, thus
 \begin{equation}
  |\Adv_{4 + 4\ell} - \Adv_{3 + 4\ell}| \leq \epsilon_{\prf}
 \end{equation}%
%
%
 \textbf{Game 5 + $4\ell$.} In this game we replace the function $\PRF(\widetilde{ms_{\ik}}, \cdot)$ used by $\pOracleAstEll$ and $\qOracleAstEll$ with a random function $F_{\widetilde{ms_{\ik}}}$. Of course the same random function is used for both oracles $\pOracleAstEll$ and $\qOracleAstEll$. Distinguishing Game 5 + $4\ell$ from Game 4 + $4\ell$ again implies an algorithm breaking the security of the pseudo random function $\PRF$.
 \begin{equation}
  |\Adv_{5 + 4\ell} - \Adv_{4 + 4\ell}| \leq \epsilon_{\prf}
 \end{equation}%

 Note that the adversary cannot obtain the DH public key and $k_{mac}$ because the adversary obtain nothing from the transcription due to randomization of the initial key $\ik$. These changes prevent trivially attack. If the adversary obtain the MAC key $k_{mac}$, the adversary can hijack the session following way: the adversary generate $T_c$, $\NONC$ by himself and calculate MAC generated by MAC key $k_{mac}$ and send these value with $\SCID$. The server cannot reject this query made by the adversary because MAC is valid. The adversary can share the secret key with the server and always return correct $b^{\prime}$.
\vspace{10pt}\\%
%
%
 \textbf{Game 6 + $4\ell$.} Now we use that the key $\ik_{\pindexell}$ and $\ik_{\qindexell}$ in the symmetric encryption scheme uniformly at random and independent of all QUIC handshake messages. In this we replace the value $k_{mac}$ with another random value $\widetilde{k_{mac}}$.

 Suppose that there exists an algorithm $\mathcal{A}$ distinguishes Game 6 + 4 $\ell$ from Game 5 + 4 $\ell$. Then we can construct an algorithm $\mathcal{B}$ breaking $\LHAE$ secure. By assumption, the simulator $\mathcal{B}$ is given access to an encryption oracle $\Encrypt$ and a decryption oracle $\Decrypt$.

 Since by assumption any attacker has advantage at most $\epsilon_{\LHAE}$ in breaking the $\LHAE$ security of the symmetric encryption scheme, we have
 \begin{equation}
  |\Adv_{6 + 4\ell} - \Adv_{5 + 4\ell}| \leq \epsilon_{\LHAE}.
 \end{equation}%
 After this game, we add $\ell = \ell + 1$. If $\ell \leq \ell^{\ast}$ we repeat the game.
\vspace{10pt}\\%
%
%
 For next step, there are two cases. The first case is that the adversary issue $\Encrypt$-query $\pOracleAstRes$ or $\qOracleAstRes$ whose state $\Lambda = \preaccept$. The second case is that the adversary issue $\Encrypt$-query $\pOracleAstRes$ or $\qOracleAstRes$ whose state $\Lambda = \accept$.

 We define the first case as Game$_{a}$ and the second case as Game$_{b}$
\vspace{10pt}\\%
%
%
 \textbf{Game$_a$ 7 + 4$\ell^{\ast}$.} Now we use that the key $\ik_{\pindexell}$ and $\ik_{\qindexell}$ which is independent of all QUIC handshake messages.

 In this game we construct a simulator $\mathcal{B}$ that uses a RSACCE attacker $\mathcal{A}$ to break the security of the underlying $\LHAE$ secure symmetric encryption scheme. By assumption, the simulator $\mathcal{B}$ is given access to an encryption oracle $\Encrypt$ and a decryption oracle $\Decrypt$. $\mathcal{B}$ embeds that $\LHAE$ experiment by simply forwarding all $\Encrypt(\pOracleAstRes,\cdot)$ queries to $\Encrypt$, and all $\Decrypt(\qOracleAstRes,\cdot)$ queries to $\Decrypt$. Otherwise it proceeds as the challenger in Game 6.

 Observe that the values generated in this game are exactly distributed as in the previous game. We thus have
 \begin{equation}
  \Adv_{7 + 4\ell^{\ast}} = \Adv_{6 + 4\ell^{\ast}}
 \end{equation}%
 If $\mathcal{A}$ outputs a quad $\pindexout$, then $\mathcal{B}$ forwards $b^{\prime}$ to the $\LHAE$ challenger. Otherwise it outputs a random bit. Since the simulator essentially relays all messages it is easy to see that an attacker $\mathcal{A}$ having advantage $\epsilon^{\prime}$ yields an attacker $\mathcal{B}$ against the $\LHAE$ security of the encryption scheme with success probability at least $1/2 + \epsilon^{\prime}$.

 Since by assumption any attacker has advantage at most $\epsilon_{\LHAE}$ in breaking the $\LHAE$ security of the symmetric encryption scheme, we have
 \begin{equation}
  \Adv_{7 + 4\ell^{\ast}} \leq \frac{1}{2} + \epsilon_{\LHAE}.
 \end{equation}%
%
%
 \textbf{Game$_b$ 7 + 4$\ell^{\ast}$.} The challenger in this game proceeds as before, however replaces the premaster secret $pms_{k}$ of $\pOracleAstRes$ and $\qOracleAstRes$ with a random group element $\widetilde{pms_{k}} = g^w$, $w \xleftarrow{\$} \Zset_p$. Note that both $g^u$ and $g^v$ are chosen by oracles $\pOracleAstRes$ and $\qOracleAstRes$, respectively, as otherwise $\pOracleAstRes$ would not have a matching conversations for a final key with $\qOracleAstRes$ and the game would be aborted.

 Distinguish Game$_b$ 7 + 4$\ell^{\ast}$ from Game 6 + 4$\ell^{\ast}$ implies an algorithm breaking the security of the decisional Diffie-Hellman problem, thus
 \begin{equation}
  |\Adv_{7 + 4\ell^{\ast}} - \Adv_{6 + 4\ell^{\ast}}| \leq \epsilon_{\ddh}
 \end{equation}%
%
%
 \textbf{Game$_b$ 8 + 4$\ell^{\ast}$.} In this game we make use of the fact that the premaster secret $\widetilde{pms_{k}}$ of $\pOracleAstRes$ and $\qOracleAstRes$ is chosen uniformly random. We thus replace the value $ms_{k} = \PRF(\widetilde{pms_{k}}, \NONC)$ with a random value $\widetilde{ms_{k}}$.

 Distinguish Game$_b$ 8 + 4$\ell^{\ast}$ from Game$_b$ 7 + 4$\ell^{\ast}$ implies an algorithm breaking the security of the pseudo random function $\PRF$, thus
 \begin{equation}
  |\Adv_{8 + 4\ell^{\ast}} - \Adv_{7 + 4\ell^{\ast}}| \leq \epsilon_{\prf}
 \end{equation}%
%
%
 \textbf{Game$_b$ 9 + 4$\ell^{\ast}$} In this game we replace the all function $\PRF(\widetilde{ms_{k}}, \cdot)$ used by $\pOracleAstRes$ and $\qOracleAstRes$ with a random function $F_{\widetilde{ms_{k}}}$. Of course the same random function is used for both oracles $\pOracleAstRes$ and $\qOracleAstRes$. Distinguishing Game$_b$ 9 + 4$\ell^{\ast}$ from Game$_b$ 8 + 4$\ell^{\ast}$ again implies an algorithm breaking the security of the pseudo random function $\PRF$.
 \begin{equation}
  |\Adv_{9 + 4\ell^{\ast}} - \Adv_{8 + 4\ell^{\ast}}| \leq \epsilon_{\prf}
 \end{equation}%
%
%
 \textbf{Game$_b$ 10 + 4$\ell^{\ast}$.} Now we use that the key $k_{\pindexell}$ and $k_{\qindexell}$ which is independent of all QUIC handshake messages.

 In this game we construct a simulator $\mathcal{B}$ that uses a RSACCE attacker $\mathcal{A}$ to break the security of the underlying $\sLHAE$ secure symmetric encryption scheme. By assumption, the simulator $\mathcal{B}$ is given access to an encryption oracle $\Encrypt$ and a decryption oracle $\Decrypt$. $\mathcal{B}$ embeds that $\sLHAE$ experiment by simply forwarding all $\Encrypt(\pOracleAstRes,\cdot)$ queries to $\Encrypt$, and all $\Decrypt(\qOracleAstRes,\cdot)$ queries to $\Decrypt$.

 Observe that the values generated in this game are exactly distributed as in the previous game. We thus have
 \begin{equation}
  \Adv_{10 + 4\ell^{\ast}} = \Adv_{9 + 4\ell^{\ast}}
 \end{equation}%
 If $\mathcal{A}$ outputs a quad $\pindexout$, then $\mathcal{B}$ forwards $b^{\prime}$ to the $\sLHAE$ challenger. Otherwise it outputs a random bit. Since the simulator essentially relays all messages it is easy to see that an attacker $\mathcal{A}$ having advantage $\epsilon^{\prime}$ yields an attacker $\mathcal{B}$ against the $\LHAE$ security of the encryption scheme with success probability at least $1/2 + \epsilon^{\prime}$.

 Since by assumption any attacker has advantage at most $\epsilon_{\LHAE}$ in breaking the $\LHAE$ security of the symmetric encryption scheme, we have
 \begin{equation}
  \Adv_{10 + 4\ell^{\ast}} \leq \frac{1}{2} + \epsilon_{\LHAE}.
 \end{equation}%
\end{proof}
 %=====================================================
\section{Conclusion} \label{sec:conclusion}
%=====================================================

We propose a new security model, \textit{Resumable} SACCE (RSACCE) security
to cover the attacks~\cite{LJBN15:QUIC}.
This model consider the security of the initial key $\ik$ in 1-RTT connection
and ensure the consistency of the client between a 1-RTT connection and 0-RTT
connections.
We also propose a new scheme which is more secure and efficient than original QUIC.
In our proposed scheme, a client sends its Diffie-Hellman public value in a
first query.

Our proposed scheme is similar to TLS 1.3 and the handshake of QUIC will be replaced
TLS 1.3. However, there are another security concern which add a server loads in our
proposed scheme and QUIC + TLS 1.3.
We also suggest the scheme to solve above issue.

% \newif\ifORIGINAL
%  \ORIGINALtrue

\begin{figure*}[htb]
\begin{center}

1-RTT connection establishment \vspace{10pt}\\

\fbox{
\begin{minipage}[t]{0.39\textwidth}
\begin{tabular}[c]{l}
 $\quad Client$ \\
 $ $ \\
 \setcounter{nombre}{0}%
 $\prob.\quad \cid \xleftarrow{\$} \{0,1\}^{64} $ \\
 $\prob.\quad m_1 = \cid$ \\
 $ $\\
 \setcounter{nombre}{0}%
 $\prob.\quad (\SCID, T_s, \sigma_s) = \SCFG_{pub}$ \\
 $\prob.\quad \doc = \SCID \| T_s$ \\
 $\prob.\quad \text{If } \SIG.\Vfy(pk_s, \sigma_s, \doc)$ \\
 $\prob.\quad \quad \Lambda = \text{'reject' and abort}$ \\
 $\prob.\quad t_c \xleftarrow{\$} \Zset_{q}^{\ast} $ \\
 $\prob.\quad T_c = g^{t_c} $ \\
 $\prob.\quad \NONC \xleftarrow{\$} \{0,1\}^{160} $ \\
 $\prob.\quad m_3 = (T_c, \NONC, \STK, \SCID, \cid)$ \\
 $\prob.\quad pms = T_s^{t_c}$ \\
 $\prob.\quad ms = \PRF(pms, NONC)$ \\
 $\prob.\quad \ik = \PRF(ms, m_1 \| m_2 \| m_3)$ \\
 $\prob.\quad \Lambda = \preaccept$ \\
\end{tabular}
\end{minipage}%
}
% middle
 \begin{minipage}[t]{0.13\textwidth}
  \centering
  \begin{tabular}{c}
   $ $ \\
   $ $ \\
   $ $ \\
   $\xrightarrow{m_1}$ \\
   $ $ \\
   $\xleftarrow{m_2}$ \\
   $ $ \\
   $ $ \\
   $\xrightarrow{m_3}$ \\
   $ $ \\
   $ $ \\
   $ $ \\
   $ $ \\
   $ $ \\
  \end{tabular}
 \end{minipage}%
\fbox{
\begin{minipage}[t]{0.39\textwidth}
\begin{tabular}[c]{l}
 $\quad Server$ \\
 $ $ \\
 $ $ \\
 \setcounter{nombre}{0}%
 $\prob.\quad \text{choose } \SCFG = (\SCFG_{pub}, t_s) $\\
 $\prob.\quad (\SCID, T_s, \sigma_s) = \SCFG_{pub}$ \\
 $\prob.\quad \makeSTKQUIC$ \\
 $\prob.\quad m_2 = (\SCFG_{pub}, \STK, \cid)$ \\
 $ $ \\
 $ $ \\
 $ $ \\
 \setcounter{nombre}{0}%
 $\prob.\quad \text{search $\SCFG$ with $\SCID$}$ \\
 $\prob.\quad pms = T_c^{t_s}$ \\
 $\prob.\quad ms = \PRF(pms, \NONC)$ \\
 $\prob.\quad \ik = \PRF(ms, m_1 \| m_2 \| m_3)$ \\
 $\prob.\quad \Lambda = \preaccept$ \\
\end{tabular}
\end{minipage}%
} \vspace{10pt}

1-RTT connection establishment for forward secure key
\vspace{10pt}\\

\fbox{
\begin{minipage}[t]{0.39\textwidth}
\begin{tabular}[c]{l}
 $ $ \\
 $ $ \\
 $ $ \\
 $ $ \\
 \setcounter{nombre}{0}%
\ifONERTT
 $\prob.\quad receive\ m_4$ \\
\else
 $\prob.\quad receive\ m_2$ \\
\fi
 $\prob.\quad T_s^{\prime} \| \STK = \SE.\Dec(\ik, c)$ \\
 $\prob.\quad pms^{\prime} = T_s^{\prime t_c} $ \\
 $\prob.\quad ms^{\prime} = \PRF(pms^{\prime}, \NONC) $ \\
\ifONERTT
 $\prob.\quad \key = \PRF(ms^{\prime}, m_1 \| m_2 \| m_3 \| m_4)$ \\
\else
 $\prob.\quad \key = \PRF(ms^{\prime}, m_1 \| m_2)$ \\
\fi
 $\prob.\quad \Lambda = \accept$ \\
 $\prob.\quad \theta = (\SCFG_{pub}, \STK)$ \\
\end{tabular}
\end{minipage}%
}
\begin{minipage}[t]{0.13\textwidth}
\centering
\begin{tabular}[c]{l}
 $ $\\
 $ $\\
 $ $\\
\ifONERTT
 $\xleftarrow{m_4}$\\
\else
 $\xleftarrow{m_2}$\\
\fi
 $ $\\
 $ $\\
 $ $\\
 $ $\\
 $ $\\
\end{tabular}
\end{minipage}%
\fbox{
\begin{minipage}[t]{0.39\textwidth}
\begin{tabular}[c]{l}
 \setcounter{nombre}{0}%
 $\prob.\quad t_s^{\prime} \xleftarrow{\$} \Zset_{q}^{\ast}$ \\
 $\prob.\quad T_s^{\prime} = g^{t_s^{\prime}}$ \\
 $\prob.\quad \makeSTKQUIC$ \\
 $\prob.\quad \plaintext = T_s^{\prime} \| \STK$ \\
 $\prob.\quad c = (\SE.\Enc(\ik, \plaintext)$ \\
\ifONERTT
 $\prob.\quad m_4 = (c, \cid)$ \\
 $\prob.\quad send\ m_4$ \\
\else
 $\prob.\quad m_2 = (c, \cid)$ \\
 $\prob.\quad send\ m_2$ \\
\fi
 $\prob.\quad pms^{\prime} = T_c{t_s^{\prime}} $ \\
 $\prob.\quad ms^{\prime} = \PRF(pms^{\prime}, \NONC) $ \\
\ifONERTT
 $\prob.\quad \key = \PRF(ms^{\prime}, m_1 \| m_2 \| m_3 \| m_4)$ \\
\else
 $\prob.\quad \key = \PRF(ms^{\prime}, m_1 \| m_2)$ \\
\fi
 $\prob.\quad \Lambda = \accept$ \\
\end{tabular}
\end{minipage}%
} \vspace{10pt}

0-RTT connection establishment for initial key
\vspace{10pt}\\

\fbox{
\begin{minipage}[t]{0.39\textwidth}
\begin{tabular}[c]{l}
 \setcounter{nombre}{0}%
 $\prob.\quad \theta = (\SCFG_{pub}, \STK)$ \\
 $\prob.\quad \cid \xleftarrow{\$} \{0,1\}^{64} $ \\
 $\prob.\quad t_c \xleftarrow{\$} \Zset_{q}^{\ast} $ \\
 $\prob.\quad T_c = g^{t_c} $ \\
 $\prob.\quad \NONC \xleftarrow{\$} \{0,1\}^{160} $ \\
 $\prob.\quad m_5 = (T_c, \NONC, \cid, \STK, \SCID)$ \\
 $\prob.\quad pms = T_s^{t_c}$ \\
 $\prob.\quad ms = \PRF(pms, \NONC)$ \\
 $\prob.\quad \ik = \PRF(ms, m_5)$ \\
 $\prob.\quad \Lambda = \preaccept$ \\
\end{tabular}
\end{minipage}%
}
% middle
 \begin{minipage}[t]{0.13\textwidth}
  \centering
  \begin{tabular}{c}
   $ $ \\
   $ $ \\
   $ $ \\
   $ $ \\
   $ $ \\
   $\xrightarrow{m_5}$ \\
   $ $ \\
   $ $ \\
   $ $ \\
   $ $ \\
  \end{tabular}
 \end{minipage}%
\fbox{
\begin{minipage}[t]{0.39\textwidth}
\begin{tabular}[c]{l}
 $ $\\
 $ $\\
 $ $\\
 $ $\\
 $ $\\
 \setcounter{nombre}{0}%
 $\prob.\quad \text{search $\SCFG$ with $\SCID$}$ \\
 $\prob.\quad pms = T_c^{t_s}$ \\
 $\prob.\quad ms = \PRF(pms, \NONC)$ \\
 $\prob.\quad \ik = \PRF(ms, m_5)$ \\
 $\prob.\quad \Lambda = \preaccept$ \\
\end{tabular}
\end{minipage}%
} \vspace{10pt}

0-RTT connection establishment for forward secure key
\vspace{10pt}\\

\fbox{
\begin{minipage}[t]{0.39\textwidth}
\begin{tabular}[c]{l}
 $ $ \\
 $ $ \\
 $ $ \\
 $ $ \\
 \setcounter{nombre}{0}%
\ifONERTT
 $\prob.\quad receive\ m_4$ \\
\else
 $\prob.\quad receive\ m_2$ \\
\fi
 $\prob.\quad T_s^{\prime} \| \STK = \SE.\Dec(\ik, c)$ \\
 $\prob.\quad pms^{\prime} = T_s^{\prime t_c} $ \\
 $\prob.\quad ms^{\prime} = \PRF(pms^{\prime}, \NONC) $ \\
\ifONERTT
 $\prob.\quad \key = \PRF(ms^{\prime}, m_1 \| m_2 \| m_3 \| m_4)$ \\
\else
 $\prob.\quad \key = \PRF(ms^{\prime}, m_1 \| m_2)$ \\
\fi
 $\prob.\quad \Lambda = \accept$ \\
 $\prob.\quad \theta = (\SCFG_{pub}, \STK)$ \\
\end{tabular}
\end{minipage}%
}
\begin{minipage}[t]{0.13\textwidth}
\centering
\begin{tabular}[c]{l}
 $ $\\
 $ $\\
 $ $\\
\ifONERTT
 $\xleftarrow{m_4}$\\
\else
 $\xleftarrow{m_2}$\\
\fi
 $ $\\
 $ $\\
 $ $\\
 $ $\\
 $ $\\
\end{tabular}
\end{minipage}%
\fbox{
\begin{minipage}[t]{0.39\textwidth}
\begin{tabular}[c]{l}
 \setcounter{nombre}{0}%
 $\prob.\quad t_s^{\prime} \xleftarrow{\$} \Zset_{q}^{\ast}$ \\
 $\prob.\quad T_s^{\prime} = g^{t_s^{\prime}}$ \\
 $\prob.\quad \makeSTKQUIC$ \\
 $\prob.\quad \plaintext = T_s^{\prime} \| \STK$ \\
 $\prob.\quad c = (\SE.\Enc(\ik, \plaintext)$ \\
\ifONERTT
 $\prob.\quad m_4 = (c, \cid)$ \\
 $\prob.\quad send\ m_4$ \\
\else
 $\prob.\quad m_2 = (c, \cid)$ \\
 $\prob.\quad send\ m_2$ \\
\fi
 $\prob.\quad pms^{\prime} = T_c{t_s^{\prime}} $ \\
 $\prob.\quad ms^{\prime} = \PRF(pms^{\prime}, \NONC) $ \\
\ifONERTT
 $\prob.\quad \key = \PRF(ms^{\prime}, m_1 \| m_2 \| m_3 \| m_4)$ \\
\else
 $\prob.\quad \key = \PRF(ms^{\prime}, m_1 \| m_2)$ \\
\fi
 $\prob.\quad \Lambda = \accept$ \\
\end{tabular}
\end{minipage}%
} \vspace{10pt}

\caption{Abstract model of QUIC}\label{fig:quic_tls}
\end{center}
\end{figure*}
%  \begin{figure*}[htb]
\begin{center}

1-RTT connection establishment \vspace{10pt}\\

\fbox{
\begin{minipage}[t]{0.39\textwidth}
\begin{tabular}[c]{l}
 $\quad Client$ \\
 $ $ \\
 $1.\ \ \cid \xleftarrow{\$} \{0,1\}^{64} $ \\
 $2.\ \ t_c \xleftarrow{\$} \Zset_{q}^{\ast} $ \\
 $3.\ \ T_c = g^{t_c} $ \\
 $4.\ \ \NONC \xleftarrow{\$} \{0,1\}^{160} $ \\
 $5.\ \ m_1 = (\NONC, \cid, T_c)$ \\
 $ $ \\
 $1.\ \ \shareInfo = (\NONC, \cid, \SCFG.T_s, t_c)$ \\
 $2.\ \ m = m_1$ \\
 $3.\ \ \ik = \getKey_c(\shareInfo, m, 1)$ \\
 $4.\ \ \plaintext = \SE.\Dec(\ik, c)$ \\
 $ $ \\
 $1.\ \ \shareInfo = (\NONC, \cid, T_s^{\prime}, t_c)$ \\
 $2.\ \ m = m_1 \| m_2$ \\
 $3.\ \ \key = \getKey_c(\shareInfo, m, 0)$ \\
\end{tabular}
\end{minipage}%
}
% middle
 \begin{minipage}[t]{0.13\textwidth}
  \centering
  \begin{tabular}{c}
   $ $ \\
   $ $ \\
   $ $ \\
   $\xrightarrow{m_1}$ \\
   $ $ \\
   $ $ \\
   $ $ \\
   $ $ \\
   $ $ \\
   $\xleftarrow{m_2}$ \\
   $ $ \\
   $ $ \\
   $ $ \\
   $ $ \\
   $ $ \\
   $ $ \\
  \end{tabular}
 \end{minipage}%
\fbox{
\begin{minipage}[t]{0.39\textwidth}
\begin{tabular}[c]{l}
 $\quad Server$ \\
 $ $ \\
 $1.\ \ \text{choose }\SCFG$ \\
 $2.\ \ \shareInfo = (\NONC, \cid, \SCFG.T_c, t_s)$ \\
 $3.\ \ m = m_1$ \\
 $4.\ \ \ik = \getKey_s(\shareInfo, m, 1)$ \\
 $5.\ \ \key_{\mac} \xleftarrow{\$} \{0,1\}^{\lambda}$ \\
 $6.\ \ \STK = \SE.\Enc(\key_{\STK}, time\|IP_c\|\key_{mac})$ \\
 $7.\ \ t_s^{\prime} \xleftarrow{\$} \Zset_{q}^{\ast}$ \\
 $8.\ \ T_s^{\prime} = g^{t_s^{\prime}}$ \\
 $9.\ \ \plaintext = T_s^{\prime} \| \STK \| \key_{\mac} \| \SCFG.\SCID$ \\
 $10.\  c = \SE.\Enc(\ik, \plaintext)$ \\
 $11.\  m_2 = (\SCFG_{pub}, c)$ \\
 $ $ \\
 $1.\ \ \shareInfo = (\NONC, \cid, T_c, t_s^{\prime})$ \\
 $2.\ \ m = m_1 \| m_2$ \\
 $3.\ \ \key = \getKey_s(\shareInfo, m, 0)$ \\
\end{tabular}
\end{minipage}%
} \vspace{10pt}

0-RTT connection establishment
\vspace{10pt}\\

\fbox{
\begin{minipage}[t]{0.39\textwidth}
\begin{tabular}[c]{l}
 $1.\ \ \cid \xleftarrow{\$} \{0,1\}^{64} $ \\
 $2.\ \ t_c \xleftarrow{\$} \Zset_{q}^{\ast} $ \\
 $3.\ \ T_c = g^{t_c} $ \\
 $4.\ \ \NONC \xleftarrow{\$} \{0,1\}^{160} $ \\
 $5.\ \ \doc = T_c \| \NONC \| \cid \| \STK \| \SCFG.\SCID$ \\
 $6.\ \ \mac = \PRF(\key_{mac}, \doc)$ \\
 $7.\ \ m_3 = (\doc, \mac)$ \\
 $ $ \\
 $1.\ \ \shareInfo = (\NONC, \cid, T_s, t_c)$ \\
 $2.\ \ m = m_3$ \\
 $3.\ \ \ik = \getKey_c(\shareInfo, m, 1)$ \\
\end{tabular}
\end{minipage}%
}
% middle
 \begin{minipage}[t]{0.13\textwidth}
  \centering
  \begin{tabular}{c}
   $ $ \\
   $ $ \\
   $ $ \\
   $\xrightarrow{m_3}$ \\
   $ $ \\
   $ $ \\
   $ $ \\
   $ $ \\
   $ $ \\
   $ $ \\
   $ $ \\
  \end{tabular}
 \end{minipage}%
\fbox{
\begin{minipage}[t]{0.39\textwidth}
\begin{tabular}[c]{l}
 $ $\\
 $ $\\
 $1.\ \ \checkQuery(\STK, k_{\STK}, \NONC, IP_c)$ \\
 $2.\ \ \key_{\mac} = \SE.\Dec(\key_{\STK}, \STK) $\\
 $3.\ \ \doc = T_c \| \NONC \| \cid \| \STK \| \SCFG.\SCID$ \\
 $4.\ \ \text{If }\mac \neq \PRF(\key_{\mac}, \doc)$ \\
 $5.\ \ \quad \Lambda = \text{'reject' and abort}$ \\
 $ $\\
 $1.\ \ \shareInfo = (\NONC, \cid, T_c, \SCFG.t_s)$ \\
 $2.\ \ m = m_3$ \\
 $3.\ \ \ik = \getKey_s(\shareInfo, m, 1)$ \\
\end{tabular}
\end{minipage}%
}

\caption{Abstract model of the propesed QUIC}\label{fig:quic_tls}
\end{center}
\end{figure*}
%  \begin{figure*}[htb]
\begin{center}

1-RTT connection establishment for initial key \vspace{10pt}\\

\fbox{
\begin{minipage}[t]{0.39\textwidth}
\begin{tabular}[c]{l}
 $\quad Client$ \\
 $ $ \\
 $1.\ \ \cid \xleftarrow{\$} \{0,1\}^{64} $ \\
 $2.\ \ m_1 = \cid$ \\
 $ $\\
 $1.\ \ \checkSCFG(\SCFG_{pub})$ \\
 $2.\ \ t_c \xleftarrow{\$} \Zset_{q}^{\ast} $ \\
 $3.\ \ T_c = g^{t_c} $ \\
 $4.\ \ \NONC \xleftarrow{\$} \{0,1\}^{160} $ \\
 $5.\ \ \shareInfo = (\NONC, \cid, \SCFG.T_s, t_c)$ \\
 $6.\ \ m = m_1 \| m_2 \| m_3$ \\
 $7.\ \ \ik = \getKey_c(\shareInfo, m, 1)$ \\
 $8.\ \ \doc = T_c, \NONC, \STK, \SCFG.\SCID$ \\
 $9.\ \ \mac = \PRF(\ik, \doc)$ \\
 $10.\  m_3 = (\doc, \mac)$ \\
\end{tabular}
\end{minipage}%
}
% middle
 \begin{minipage}[t]{0.13\textwidth}
  \centering
  \begin{tabular}{c}
   $ $ \\
   $ $ \\
   $ $ \\
   $\xrightarrow{m_1}$ \\
   $ $ \\
   $\xleftarrow{m_2}$ \\
   $ $ \\
   $ $ \\
   $\xrightarrow{m_3}$ \\
   $ $ \\
   $ $ \\
   $ $ \\
   $ $ \\
   $ $ \\
  \end{tabular}
 \end{minipage}%
\fbox{
\begin{minipage}[t]{0.39\textwidth}
\begin{tabular}[c]{l}
 $\quad Server$ \\
 $ $ \\
 $ $ \\
 $1.\ \ \text{choose } \SCFG $\\
 $2.\ \ \STK = \SE.\Enc(\key_{\STK}, time\|IP_c)$ \\
 $3.\ \ m_2 = (\SCFG_{pub}, \STK)$ \\
 $ $ \\
 $ $ \\
 $ $ \\
 $1.\ \ \checkQuery(\STK, k_{\STK}, \NONC, IP_c)$ \\
 $2.\ \ \shareInfo = (\NONC, \cid, T_c, \SCFG.t_s)$ \\
 $3.\ \ m = m_1 \| m_2 \| m_3$ \\
 $4.\ \ \ik = \getKey_s(\shareInfo, m, 1)$ \\
 $5.\ \ \doc = T_c, \NONC, \STK, \SCFG.\SCID$ \\
 $6.\ \ \text{If }\mac \neq \PRF(\key_{\mac}, \doc)$ \\
 $7.\ \ \quad \Lambda = \text{'reject' and abort}$ \\
\end{tabular}
\end{minipage}%
} \vspace{10pt}

1-RTT connection establishment for last key
\vspace{10pt}\\

\fbox{
\begin{minipage}[t]{0.39\textwidth}
\begin{tabular}[c]{l}
 $ $ \\
 $ $ \\
 $ $ \\
 $ $ \\
 $1.\ \ T_s^{\prime} \| \STK = \SE.\Dec(\ik, m_4)$ \\
 $ $ \\
 $ $ \\
 $1.\ \ \shareInfo = (\NONC, \cid, T_s^{\prime}, t_c)$ \\
 $2.\ \ m = m_1 \| m_2 \| m_3 \| m_4 $ \\
 $3.\ \ \key = \getKey_c(\shareInfo, m, 0)$ \\
\end{tabular}
\end{minipage}%
}
\begin{minipage}[t]{0.13\textwidth}
\centering
\begin{tabular}[c]{l}
 $ $\\
 $ $\\
 $ $\\
 $\xleftarrow{m_4}$\\
 $ $\\
 $ $\\
 $ $\\
 $ $\\
 $ $\\
\end{tabular}
\end{minipage}%
\fbox{
\begin{minipage}[t]{0.39\textwidth}
\begin{tabular}[c]{l}
 $1.\ \ t_s^{\prime} \xleftarrow{\$} \Zset_{q}^{\ast}$ \\
 $2.\ \ T_s^{\prime} = g^{t_s^{\prime}}$ \\
 $3.\ \ \key_{\mac} \xleftarrow{\$} \{0,1\}^{\lambda}$ \\
 $4.\ \ \STK = \SE.\Enc(\key_{\STK}, time\|IP_c\|\key_{\mac})$ \\
 $5.\ \ \plaintext = T_s^{\prime} \| \STK \| \key_{\mac}$ \\
 $6.\ \ m_4 = \SE.\Enc(\ik, \plaintext)$ \\
 $ $ \\
 $1.\ \ \shareInfo = (\NONC, \cid, T_c, t_s^{\prime})$ \\
 $2.\ \ m = m_1 \| m_2 \| m_3 \| m_4 $ \\
 $3.\ \ \key = \getKey_s(\shareInfo, m, 0)$ \\
\end{tabular}
\end{minipage}%
} \vspace{10pt}

0-RTT connection establishment
\vspace{10pt}\\

\fbox{
\begin{minipage}[t]{0.39\textwidth}
\begin{tabular}[c]{l}
 $1.\ \ \cid \xleftarrow{\$} \{0,1\}^{64} $ \\
 $2.\ \ t_c \xleftarrow{\$} \Zset_{q}^{\ast} $ \\
 $3.\ \ T_c = g^{t_c} $ \\
 $4.\ \ \NONC \xleftarrow{\$} \{0,1\}^{160} $ \\
 $5.\ \ \doc = T_c \| \NONC \| \cid \| \STK \| \SCFG.\SCID$ \\
 $6.\ \ \mac = \PRF(\key_{mac}, \doc)$ \\
 $7.\ \ m_5 = (\doc, \mac)$ \\
 $ $ \\
 $1.\ \ \shareInfo = (\NONC, \cid, \SCFG.T_s, t_c)$ \\
 $2.\ \ m = m_5$ \\
 $3.\ \ \ik = \getKey_c(\shareInfo, m, 1)$ \\
\end{tabular}
\end{minipage}%
}
% middle
 \begin{minipage}[t]{0.13\textwidth}
  \centering
  \begin{tabular}{c}
   $ $ \\
   $ $ \\
   $ $ \\
   $\xrightarrow{m_5}$ \\
   $ $ \\
   $ $ \\
   $ $ \\
   $ $ \\
   $ $ \\
   $ $ \\
  \end{tabular}
 \end{minipage}%
\fbox{
\begin{minipage}[t]{0.39\textwidth}
\begin{tabular}[c]{l}
 $ $\\
 $ $\\
 $ $\\
 $1.\ \ \checkQuery(\STK, k_{\STK}, \NONC, IP_c)$ \\
 $2.\ \ \key_{\mac} = \SE.\Dec(\key_{\STK}, \STK) $\\
 $3.\ \ \doc = T_c \| \NONC \| \cid \| \STK \| \SCFG.\SCID$ \\
 $4.\ \ \text{If }\mac \neq \PRF(\key_{\mac}, \doc)$ \\
 $5.\ \ \quad \Lambda = \text{'reject' and abort}$ \\
 $ $\\
 $1.\ \ \shareInfo = (\NONC, \cid, T_c, \SCFG.t_s)$ \\
 $2.\ \ m = m_1 \| m_2 \| m_3$ \\
 $3.\ \ \ik = \getKey_s(\shareInfo, m, 1)$ \\
\end{tabular}
\end{minipage}%
}

\caption{Abstract model of the proposed QUIC}\label{fig:quic_tls}
\end{center}
\end{figure*}

\bibliographystyle{abbrv}
\bibliography{mybib,confCryp,confComp}

\end{document}
