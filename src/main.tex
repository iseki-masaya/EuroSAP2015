\documentclass[conference]{IEEEtran}
\usepackage[cmex10]{amsmath}
\usepackage{url}

%% for debug
\usepackage{color}

\newtheorem{definition}{Definition}
\newtheorem{lemma}{Lemma}
\newtheorem{note}{Note}
\newtheorem{remark}{Remark}
\newtheorem{theorem}{Theorem}
\newtheorem{proof}{Proof}

\newcounter{nombre}
\renewcommand{\thenombre}{\arabic{nombre}}
\newcommand{\prob}[1][]{\refstepcounter{nombre}\thenombre}

% \input{macros_lncs}
\input{mynotation}
\input{addnotation}
\input{tmpstyle}

\begin{document}
\title{On the Security of QUIC}


% author names and affiliations
% use a multiple column layout for up to three different
% affiliations
\author{\IEEEauthorblockN{Masaya Iseki}
\IEEEauthorblockA{Tokyo Institute of Technology\\
Email: iseki.m.aa@m.titech.ac.jp}
\and
\IEEEauthorblockN{Eiichiro Fujisaki}
\IEEEauthorblockA{NTT Secure platform\\
Email: fujisaki.eiichiro@lab.ntt.co.jp}}
\maketitle

 %=====================================================
\begin{abstract}
%=====================================================
We study the security of Quick UDP Internet Connections
(QUIC for short) -- an experimental transport layer
network protocol recently developed by Google -- and
show some security concern, when abbreviated handshakes,
aka ``resumptions", are established.
To explain our concern, we propose a new security model,
extending server-only authenticated and channel
confidentiality establishment (SACCE), so that
authentication and channel confidentiality can be
evaluated including abbreviated handshake sessions.
We then show that QUIC meets the weaker version of our
security notion, but not the stronger one.
On one hand, QUIC with an optional client encrypted tag
value (CETV) mechanism, satisfies the stronger one.
We finally present a more efficient protocol with the
spirit of QUIC, because QUIC with CETV increases
communication and computational costs.
\end{abstract}
 %=====================================================
\section{Introduction} \label{sec:intro}
%=====================================================
Quick UDP Internet Connections (QUIC for short) is a
new transport layer network protocol recently proposed
by Google \cite{QUIC,QUICDraft}, which is experimentally
implemented in Google Chrome.
The main purpose of developing QUIC is to provide an
alternative equivalence of TLS wrapping TCP, with much
reduced latency and better SPDY and HTTP/2 support.
Transport Layer Security (TLS) starts with a three-move
TCP handshake before initiating the TLS Handshake
Protocol.
In contrast, QUIC uses UDP and starts with its own
handshake, which reduces the total number of
interactions.
The cryptographic core of QUIC is specified in the QUIC
crypto protocol~\cite{QUIC:Crypto}, which consists of a
handshake protocol and a record layer protocol, as does
TLS.
Similarly to TLS, QUIC has two types of handshake
connections.
One is called a full handshake -- a handshake
``from scratch" between a client and a server.
The other is called an abbreviate handshake -- a
handshake which occurs when a client and a server have
once established a full handshake session and want to
establish a new session between them in an abbreviate way.
Unlike TLS, QUIC only supports the elliptic-curve
Diffie-Hellman key-exchange (ECDHE) cipher suites and
server authentication.
%
One of the good features of QUIC is that it can
establish an abbreviate session with $0$-RTT
connectivity overhead.
Namely, in the QUIC abbreviate handshake, a client can send
encrypted data to a server, concurrently with a new session.
We provide the abstract model of the full handshake and
abbreviate handshake protocols of QUIC in
Fig.~\ref{fig:quic_abst_1rtt}, ~\ref{fig:quic_abst_0rtt}.
By this property, an abbreviate handshake connection is also called
0-RTT connection and a full handshake connection is also called
1-RTT connection.

%=====================================================
\subsection{Prior Security Analyses and Some Security Concern} \label{sec:concern}
%=====================================================
To the best of our knowledge, there are only two
security analyzes on QUIC~\cite{FG14:QUIC,LJBN15:QUIC}.
Both papers define the new security models and show that
QUIC is secure in that model.
In~\cite{FG14:QUIC}, they formalized a secure
authenticated key-exchange as an extension of the
Bellare-Rogaway model~\cite{BR93:AKE} and analyzed the
security of QUIC (with abbreviate handshakes).
However, the QUIC protocol analyzed in \cite{FG14:QUIC}
is slightly different from the protocol given in the
source codes.
As described in Fig.~\ref{fig:quic_abst_1rtt},
~\ref{fig:quic_abst_0rtt}, the QUIC protocol makes a
server send a ciphertext (using authenticated
encryption) in the full handshake protocol, which cannot
preserve \textit{key-indistinguishability}.
Therefore, the authenticated and channel confidentiality
establishment (ACCE) model~\cite{JKSS12:ACCE} is more
suitable to analyze QUIC.
Another important security issue is that in~\cite{FG14:QUIC},
an adversary is allowed to send a ``test" query only to
a client oracle (to receive either a real session-key or
a random key from the client oracle), when a protocol
is server-only authenticated.
Apparently, the restriction is appropriate, because
an adversary can establish a session with a honest
server (due to the lack of client's certificate) to
share a session key.
However, if an abbreviate handshake is provided, we should consider
the attack that, after a honest client and a honest
server establish a full handshake session, an adversary
might hijack an abbreviate handshake session -- it might
impersonate the initial client and share a session key
with the server.
To protect the attack, we should allow an adversary to
send test queries to \textit{server} oracles in abbreviate
handshake sessions (including the full handshake session), as long
as the initial full handshake session is established
between a honest client and a honest server.
We can consider an attack: The adversary can
share a session key with the server and it can make the
server accept in an abbreviate handshake session. (Note that in a
full handshake session, it is a ``trivial" attack, because
an adversary can always do so.)
In~\cite{LJBN15:QUIC}, they formalized QACCE model which
is based on ACCE model and consider full handshakes and
abbreviate handshakes.
They also found that with replay attack on some
public parameters exchanged during the handshake, an
adversary could easily prevent QUIC from achieving
minimal latency advantages either by having it fall back
to TCP or by causing the client and server to have an
inconsistent view of their handshake leading to a failure
to complete the connection.
The adversary also can apply loads on the server using
these attacks.
Their security model QACCE does not consider these attacks
to prove the security of QUIC as it is.
These attacks are ruled out in the proposed security model.
On the other hand, in their security model QACCE, an adversary
is allowed to send a ``test" query to a server oracle.
However, they add a restriction that an adversary is allowed
to send it only the server oracle which has a matching
conversation with some client oracles.
An adversary needs to forge the query to make a server accept
and the server does not have a matching conversation with any
client oracle.
For this restriction, QACCE does not prevent the attack that
an adversary make a server accept.
In QUIC, a server does not make sure consistencies of a client
between 0-RTT connections and 1-RTT connections.
However, one of the attacks found in~\cite{LJBN15:QUIC} use this
property that a server does not make sure consistencies of a client
between 0-RTT connections and 1-RTT connections and the adversary
can establish the connection spoofing IP address.
This enable the adversary to do Distributed Denial of Service
(DDoS) attack.
Our security model guarantees that only parties
that establish the initial full handshake session can
establish a new abbreviate handshake session.
This property mitigate one of the attacks found in
~\cite{LJBN15:QUIC}.
Our security model also prevent other attacks.
In QUIC, the client and server share two keys which are initial
key $\ik$ and last key $\key$.
Other attacks~\cite{LJBN15:QUIC} make a client and server share
a different initial key $\ik$.
The server impersonation advantage in QACCE does not
cover this case because this advantage consider only forward
secure key $\key$.
Our security model also consider the security of an initial
key $\ik$ and this property protect the other attacks
~\cite{LJBN15:QUIC}.

%=====================================================
\subsection{Related Work} \label{sec:Related Work}
%=====================================================
There are a huge body of works on authenticated key
exchange protocols (See~\cite{CK01:AKE} for survey).
An important stream of research dates back to Bellare
and Rogaway~\cite{BR93:AKE}, followed by~\cite{DB96,
Blei98,JMDP00,JB02,EK09,KK05:TLS,KCRE08,SMOAJ08,KTT11,
Kraw01}.
However, as mentioned above, the QUIC full handshake
protocol does not satisfy key-indistinguishability as
in the Bellare-Rogaway like model, because a server
sends a ciphertext (using authenticated encryption) in
the full handshake protocol, as does TLS.
TLS Handshake Protocol is recently analyzed in various
security models, e.g., ~\cite{JKSS12:ACCE,KPW13:SACCE,
FS13:ACCE,GKS13:RACCE,BDKSS14:SSH,BFKPSB14:TLS}.
Still, the security model for analyzing a server-only
authenticated connection of TLS, i.e., Server-Only
Authenticated and Confidential Channel Establishment
(SACCE)~\cite{KPW13:SACCE}, is not appropriate for QUIC.
There are two reasons. First, the abstract model of
handshake between QUIC and TLS is different. In QUIC,
the client and server share two keys which are initial
key $\ik$ and last key $\key$ and secrets are not
reused in an abbreviate handshake. On the other hand,
in TLS, the client and server share one key and reuse
the secrets in an abbreviate handshake. Second, the
security model proposed in the previous study
~\cite{FG14:QUIC,LJBN15:QUIC} does not
consider the consistency of the client between a full
handshake and abbreviate handshakes.
The second reasons is important in order to prevent the
attacks \cite{LJBN15:QUIC}.

%=====================================================
\subsection{Our Results} \label{sec:proposal}
%=====================================================

Our contributions are:
\begin{itemize}
 \item{A security model which is appropriate for QUIC
 and more secure than QACCE.}

 \item{A new scheme which is more secure and efficient
 than original one.}
\end{itemize}

We introduce a new
security model, what we call \textit{Resumable} SACCE
(RSACCE) security, where we consider a server's message
confidentiality, as well as a client's message
confidentiality, where an adversary is allowed to send
an encryption query to a \textit{server} (to break a
server's message confidentiality) both in the full
handshake session and the abbreviate handshake sessions,
as far as the server establishes the initial full
handshake session with a \textit{honest} client.
Our security model also consider the consistency of the client between
0-RTT connections and 1-RTT connections.
The consistency of the client between 0-RTT connections
and 1-RTT connections prevent one of the attacks.

We also propose a more secure and efficient protocol with
the spirit of QUIC.
In~\cite{LJBN15:QUIC}, they found the five attacks for
QUIC and our proposed scheme can prevent the four attacks.
 %=====================================================
\section{Preliminaries}\label{sec:pre}
%=====================================================
We let ${\negl}(\spar)$ to denote an unspecified function $f(\spar)$ such that
$f(\spar) ={\spar}^{-\omega(1)}$, saying that such a function is negligible in $\spar$.
We use the standard notations of digital signature, collision-resistant hash functions, and pseudo random functions.
We also use a stateful symmetric encryption as well as a few known assumptions.
For completeness, we provide their notions given in Appendix~\ref{app:def}.
 %=====================================================
\section{``Resumable" Server-only Authenticated and Confidential Channel Establishment (RSACCE)} \label{sec:rsacce}
%=====================================================
We consider the security of QUIC and its variants.
Our abstract model to capture the cryptographic core
of them is a server-only authenticated and confidential
channel establishment (SACCE) protocol~\cite{KPW13:SACCE},
the server-only authentication version of an ACCE
protocol~\cite{JKSS12:ACCE}.
In our SACCE protocol, a client initiates an instance
of the protocol and a server always sends the last
message of the instance.
We then extend the security notion of
SACCE~\cite{KPW13:SACCE} to the notion of
\textit{resumable} SACCE (RSACCE).
The main difference between SACCE and RSACCE is that this model treats
abbreviate handshakes.
As mentioned above, an abbreviate handshake takes
advantage of a prior established full handshake session
between the same client and server.
To treat abbreviate handshakes in the server-only
authenticated setting, we should take into account the
attack
\textbf{where an adversarial client fools a
server and impersonates the initial client in an
abbreviate handshake session.}
Depending on the extent of the attack, we define two
security requirements, denoted
\textit{server authentications} and
\textit{channel confidentiality}.
The channel confidentiality implies that, in contrast
to SACCE, an adversary is allowed to submit an
encryption query
\textit{not only to a client but to a server} in an
abbreviate handshake session.
We note that if an adversary can make a handshake with
a server with the same session key in some abbreviate
handshake session, then it can indeed break message
confidentiality of a ciphertext sent by the server.
QACCE also treats abbreviate handshakes.
However, QACCE does not cover the attack because the
adversary is allowed to send encryption query only to
oracles which has matching conversation with intended
partner.
The main differences between QACCE and RSACCE are that
this model (1) ensure the consistency of a client between
1-RTT connections and 0-RTT connections and (2) cover the
security of initial key.
For more details about (2), in this model, the adversary
cannot make a client and server share a different key.
However, in QACCE, the adversary can make a client and
server share a different key because QACCE only consider
the security of forward secure key.

%=====================================================
\subsection{Execution Environment} \label{sec:exec_env_party}
%=====================================================
We basically borrow the notations
from~\cite{JKSS12:ACCE,KPW13:SACCE}.
We denote by $\Client$ and $\Server$ the set of honest
clients and servers, respectively.
We assume for simplicity that each client $\Client$
and server $\Server$ has a \textit{unique} identity
number $c \in \Nset$
\footnote{Since a client has no certified identity,
the numbering is just conceptual.}
and $s \in \Nset$.
For the case we don't need to specify clients or
servers, we also assume that each party
$P \in \Server \cup \Client$ has a \textit{unique}
identity number $p \in \Nset$.
In particular, $\Server$ with identity number $s$ has a unique key pair
$(\pk_s, \sk_s)$, along with a certificate
$\cert_s=(s,\pk_s)_{\text{CA}}$ signed by a certificate
authority CA.
A party $P$ with identity number $p$ maintains a collection of oracles
$\{\pi^p_{i,\ell }\}_{i,\ell}$ where oracle
$\pi^p_{i, \ell}$ models party $P$ executing a single
instance of a protocol in the $\ell$-th abbreviate
handshake session derived from the $i$-th full
handshake session.
When $\ell=0$, $\pi^p_{i,\ell}$ refers to the $i$-th
full handshake session of $P$ with identity number $p$.
The oracle $\pi^p_{i, \ell}$ maintains as internal
state the following variables:

\begin{itemize}
 \item{$\Lambda \in \{\accept, \preaccept, \reject,
 \emptyset\}$, the state of a handshake.}

 \item{$\ik, \key \in \mathcal{K}$, the session key
 where $\mathcal{K}$ is the key space of the protocol.
 $\ik$ is called initial key which calculated with
 ephemeral client's Diffie-Hellman value and static
 server's Diffie-Hellman value, and $\key$ is called
 forward-secure key which calculated with ephemeral
 client's Diffie-Hellman value and ephemeral server's
 Diffie-Hellman value.}

 \item{$\peer$, the intended partner. If $P$ is
 $\Client$ with identity number $c$, then $\pi^c_{i,\ell}$ maintains the
 identity of intended partner $\peer \in \Server$.
 Otherwise (if $P$ is $\Server$), it maintains
 the identity of intended partner $\peer \in \Client$.}

 \item{$b$, the challenge bit chosen uniformly.}

 \item{$\theta$, the cache data. The client and server
 cache the information for the future sessions.}
\end{itemize}
The inner state of each oracle is initialized to
($\Lambda$, $\ik$, $\key$, $\peer$, $\theta$) = ($\emptyset$,
$\emptyset$, $\emptyset$, $\emptyset$, $\emptyset$), where
variable $V=\emptyset$ denotes that variable $V$
is undefined.
On one hand, and $b$ is chosen uniformly and fixed.

The adversary issues the following queries to the
oracles:
\begin{itemize}
 \item {$\Send(\pi^p_{i, \ell}, m)$:
 The adversary can use this query to send message
 $m$ to the oracle $\pi^p_{i, \ell}$.
 The oracle will respond with an outgoing message
 according to the protocol specification and its
 internal state.
 The oracle $\pi^p_{i, \ell}$ replies with $\bot$,
 either (a) if $\ell \geq 1$ and
 $(\theta,\peer) = (\emptyset, \emptyset)$, or (b)
 if $\pi^p_{i,\ell}$ has reached state
 $\Lambda = \accept$.
 Otherwise, it does the following: If the adversary
 asks the first $\Send$-query to oracle
 $\pi^p_{i, \ell}$, then the oracle checks whether
 $m = \top$ consists of a special
 ``initiate client session'' symbol $\top$.
 If so, it responds with the first protocol message.
 If $\ell \geq 1$,
 then $\pi^p_{i,\ell}$ \textit{inherits} variables $(\theta,\peer)$ from
 $\pi^p_{i, k}$ ($k < \ell$) which reaches accept state latest
 (modeling an abbreviate handshake session).
 The adversary can send this returned message to any
 oracle even if this oracle is not intended partner of
 the oracle $\pi^p_{i, \ell}$.}

 \item {$\Reveal(\pi^p_{i,\ell})$:
 The oracle $\pi^p_{i,\ell}$ returns session keys
 $\ik$, $\key$, and cache data $\theta$ to respond this query.
 If $\Reveal(\pi^p_{i,\ell})$ is $\tau$-th query issued by the adversary,
 $\pi^p_{i,\ell}$ is said $\tau$-\textit{revealed}.
 For oracles that are not revealed we define
 $\tau = \infty$.}

 \item {$\Corrupt(P)$:
 If $P$ is $\Client$, then returns $\bot$.
 Otherwise, $P$ is $\Server$ with identity number $s$ and returns long-term
 secret key $\sk_s$ and other secrets which are kept
 for a long time such as static Diffie-Hellman secret
 value.
 If $P$ is $\Server$ and $\Corrupt(P)$ is the
 $\tau$-th query issued by the adversary, $P$ is said
 $\tau$-\textit{corrupted}.
 For parties that are not corrupted we define
 $\tau = \infty$.}

\item {$\Corrupt(\pi^p_{i,\ell})$:
 If $\pi^p_{i,\ell}$ is a client oracle, then returns $\bot$.
 Otherwise, $\pi^p_{i,\ell}$ is a server oracle and returns
 Diffie-Hellman secret value $t_s$ in $\SCFG$.
 If $\pi^p_{i,\ell}$ is a server oracle and $\Corrupt(\pi^p_{i,\ell})$ is the
 $\tau$-th query issued by the adversary, $\pi^p_{i,\ell}$ is said
 $\tau$-\textit{corrupted}.
 For oracles that are not corrupted we define
 $\tau = \infty$.}

 \item {$\Encrypt(\pi^p_{i,\ell}, m_0, m_1, \iv, H)$:
 If $\pi^p_{i,\ell}$ has state
 $\Lambda$ $\not\in$ \\ $\{\preaccept, \accept\}$,
 it returns $\bot$.
 Otherwise, it makes a challenge ciphertext according to
 the procedure in Fig.~\ref{fig:LHAE_rsacce}.}

 \item {$\Decrypt(\pi^p_{i, \ell}, c, \iv, H)$:
 The oracle $\pi^p_{i, \ell}$ replies according to the
 procedure in Fig.~\ref{fig:LHAE_rsacce}.}
\end{itemize}

\begin{figure*}[!htb]
\begin{center}
\fbox{
\begin{minipage}[t]{0.42\textwidth}
\begin{tabular}[t]{l}
 $\Encrypt(\pi^p_{i,\ell},m_0,m_1,\iv,H)$: \\
 $\quad C_0 \xleftarrow{\$} \SE.\Enc(k, \iv, H,m_0)$ \\
 $\quad C_1 \xleftarrow{\$} \SE.\Enc(k, \iv, H,m_1)$ \\
 $\quad \text{If } C_0 = \perp \text{ or } C_1 = \perp$ \\
 $\quad \quad \text{return } \perp$ \\
 $\quad \mathcal{C} = \mathcal{C} \cup C_b$ \\
 $\quad \text{return } C_b$ \\
\end{tabular}
\end{minipage}
\vline \quad
\begin{minipage}[t]{0.42\textwidth}
\begin{tabular}[t]{l}
 $\Decrypt(\pi^p_{i,\ell},C,\iv,H)$: \\
 $\quad \text{If } b = 1 \wedge C \not\in \mathcal{C}$ \\
 $\quad \quad \text{return } \SE.\Dec(k,\iv,H,C)$ \\
 $\quad \text{return } \perp$ \\
\end{tabular}
\end{minipage}
}
\caption{Encrypt and Decrypt oracle in the RSACCE security experiment}
 \label{fig:LHAE_rsacce}
\end{center}
\end{figure*}

%=====================================================
\subsection{Security Definition} \label{sec:sec_def}
%=====================================================

We define the security model of resumable server-only
authenticated confidential channel establishment (RSACCE).

\subsubsection{Matching Conversations}
In our SACCE protocol, a client always initiates an
instance of the protocol and a server always sends the
last message of the instance in a whole handshake.
We define matching conversations as in~\cite{JKSS12:ACCE}.
We denote by $T^p_{i,\ell}$ the transcript of
$\pi^p_{i,\ell}$, i.e., the history of all messages sent
and received by $\pi^p_{i,\ell}$ in chronological order
(not including the initialization-symbol $\top$).
For two transcripts, $T^p_{i,\ell}$ and
$T^{p^{\prime}}_{j,\ell'}$ we say that $T^p_{i,\ell}$
is a \textit{prefix} of $T^{p^{\prime}}_{j,\ell'}$ if
$T^p_{i,\ell}$ contains at least one message, and
$T^p_{i,\ell}$ is identical to
$T^{p^{\prime}}_{j,\ell'}$ except the last message sent
by $\pi^{p^{\prime}}_{j,\ell'}$.
In QUIC, there are two type keys initial key $\ik$ and
forward secure key $\key$.
We define two type Matching conversations according to
a state of the instance.

\begin{definition}[Matching conversations for initial key]
 We say that $\pi^c_{i,\ell}$ and $\pi^s_{j,\ell'}$ have
 a matching conversation for initial key with each other if
 in addition to $\ell=\ell'$,
 \begin{itemize}
  \item{Both oracles, $\pi^c_{i, \ell}$ and
  $\pi^s_{j,\ell'}$, pre-accept ($\Lambda = \preaccept$) and
  $T^c_{i,\ell} = T^s_{j,\ell'}$}
 \end{itemize}
\end{definition}
\begin{remark}
 The client reach a state $\preaccept$ before the
 server receives the message. However, the client can know whether
 a server indeed receives it since the server will send the
 last message to a client.
\end{remark}
\begin{definition}[Matching conversations for forward secure key]
 We say that $\pi^c_{i,\ell}$ and $\pi^s_{j,\ell'}$ have
 a matching conversation with each other if in addition
 to $\ell=\ell'$,
 \begin{itemize}
  \item{Both oracles, $\pi^c_{i, \ell}$ and
  $\pi^s_{j,\ell'}$, accept ($\Lambda = \accept$) and
  $T^c_{i,\ell} = T^s_{j,\ell'}$; or}

  \item{The server oracle $\pi^s_{j, \ell'}$ accepts
  ($\Lambda = \accept$),
  and $T^c_{i,\ell}$ is a prefix of $T^s_{j,\ell'}$.}
 \end{itemize}
\end{definition}
\begin{remark}
 The second condition is necessary because in our SACCE
 protocol a server accepts a session before sending the
 last message to a client and cannot know whether a
 client indeed receives it.
\end{remark}

\subsubsection{RSACCE Game}
We define the RSACCE game between an adversary $A$ and
a challenger.
In this game, the challenger firstly instantiates the
collection of oracles $\{\pi^p_{i,\ell}\}$.
Then the challenger generates the certificate's
signing/verification keys; generates long-term keys
$(\pk_s, \sk_s)$ for all servers; and issues the
certificates for all public keys.
The adversary receives all public keys $\pk_s$ with
identity $s$ as input.
Now the adversary may start by issuing $\Send$,
$\Reveal$, $\Corrupt$, $\Encrypt$ and $\Decrypt$ queries.
Finally, the adversary outputs
$(p, i, \ell, b^{\prime})$ and terminates.

\begin{definition}[Correctness]
 Assume a ``benign" adversary $A$, which picks two
 arbitrary oracles, $\pi^c_{i, \ell}$ and
 $\pi^s_{j, \ell^{\prime}}$, and performs a sequence of
 $\Send$-queries by faithfully forwarding all messages
 between $\pi^c_{i, \ell}$ and $\pi^s_{j, \ell^{\prime}}$.
 Let $\ik^c_{i, \ell}$ and $\key^c_{i, \ell}$ denote the
 key computed by $\pi^c_{i, \ell}$ and let
 $\ik^s_{j, \ell^{\prime}}$ and $\key^s_{j, \ell^{\prime}}$
 denote the key computed by $\pi^s_{j, \ell^{\prime}}$.
 We say that RSACCE protocol is \textit{correct}, if two
 arbitrary server and client oracles, $\pi^c_{i, \ell}$
 and $\pi^s_{j, \ell^{\prime}}$, always hold that:
 \begin{itemize}
  \item{Both oracles, $\pi^c_{i, \ell}$ and
  $\pi^s_{j, \ell^{\prime}}$, have
  \textit{a matching conversation} for initial key
  and forward secure key with each other; and}

  \item{Both initial keys, $\ik^c_{i, \ell}$ and
  $\ik^s_{j, \ell^{\prime}}$, are the same
  ($\ik^c_{i, \ell} = \ik^s_{j, \ell^{\prime}}$).}

  \item{Both forward-secure keys, $\key^c_{i, \ell}$ and
  $\key^s_{j, \ell^{\prime}}$, are the same
  ($\key^c_{i, \ell} = \key^s_{j, \ell^{\prime}}$).}
 \end{itemize}
\end{definition}

\subsubsection{Security Requirements}
We now define the following advantage measures.

\begin{definition}[Server Authentication] \label{def:rsacce-sa}
 $\Adv^{\rsaccesa}_{\Pi}$ (A) is the probability that when
 $A$ terminates, there is a client oracle $\pi^c_{i, \ell}$
 such that the following conditions hold:
 \begin{itemize}
  \item{If $\ell = 0$, $\pi^c_{i, 0}$ pre-accepts
  ($\Lambda = \preaccept$) when $A$ issues its
  $\tau_0$-th query with intended partner $\peer=s$, }

  \item{If $\ell = 0$, there is no server oracle
  $\pi^s_{j, 0}$ such
  that $\pi^c_{i,0}$ has a matching conversation for initial
  key with
  $\pi^s_{j,0}$ or there exist plural oracles that have
  a matching conversation for initial key with $\pi^c_{i,0}$,}

  \item{If $\ell \neq 0$, $\pi^c_{i, \ell}$
  accepts ($\Lambda = \accept$) when $A$ issues its
  $\tau_0$-th query with intended partner $\peer=s$, }

  \item{If $\ell \neq 0$, there is no server oracle
  $\pi^s_{j, \ell}$ such
  that $\pi^c_{i,\ell}$ has a matching conversation for forward
  secure key with
  $\pi^s_{j,\ell}$ or there exist plural oracles that have
  a matching conversation with $\pi^c_{i,\ell}$, and}

  \item{$\Server$ with identity number $s$ is $\tau_{s}$-corrupted with
  $\tau_0 < \tau_{s}$ and the server oracle $\pi^s_{j, \ell}$ is
  $\tau_{so}$-corrupted with $\tau_0 < \tau_{so}$. }

 \end{itemize}
 We say that protocol $\Pi$ has \textit{server authentication},
 if $\Adv^{\rsaccesa}_{\Pi}(A)$ is negligible in $\kappa$.
\end{definition}

\begin{remark}
 The \textit{server authentication} is a natural extension
 of the counter part of~\cite{KPW13:SACCE}, which implies
 that if a client accepts, protocol $\Pi$ guarantees that
 the client has a matching conversation with the intended
 parter (server). In contrast to the original
 definition~\cite{KPW13:SACCE}, we allow an adversary to
 submit $\Corrupt$ queries.
 In our definition, the fifth conditions ensure the
 consistency of the server handshake sessions.
 If the adversary can issue $\Corrupt$ query to the server
 $\Server$ with identity number $s$ or the server oracle $\pi^s_{j, \ell}$
 before $\pi^s_{j, \ell}$ reaches accept, she can
 impersonate the server because she can make valid response
 (SHLO) using long-term secret key $sk_s$ or server's
 Diffie-Hellman secret value $t_s$.
 In 0-RTT connection, the client immediately reaches pre-accept state
 after sending the first message. If the adversary drop this message, the adversary
 easily break our server authentication since there is no server oracle
 such that the client oracle has a matching conversation for initial key.

 \end{remark}

\begin{definition}[Channel Confidentiality] \label{def:rsacce-cc}
 $\Adv^{\rsaccecc}_{\Pi}$ (A) is defined to be
 $x$ - $\frac{1}{2}$ where $x$ is the probability that
 the adversary $A$ outputs $(p, i, \ell, b^{\prime})$,
 with $0\leq \ell$, such that $b = b^{\prime}$ where
 $b^{\prime} \in \bits$ is set during the
 $\Encrypt(\pi^p_{i,\ell},\dots)$ query and we define
 $b=\bot$ unless the following conditions hold:
 \begin{itemize}
  \item{$\pi^p_{i,\ell}$ pre-accepts
  ($\Lambda = \preaccept$) when $A$
  issues its $\tau_0$-th query.}

  \item{If $\pi^p_{i,\ell}$ is a client oracle, the
  intended pater $\Server$ with identity number $s$ is $\tau_s$-corrupted
  with $\tau_0 < \tau_s$ and the server oracle $\pi^s_{j,\ell}$ is
  $\tau_{so}$-corrupted with $\tau_0 < \tau_{so}$.}

  \item{If $P$ is a server $\Server$ with identity number $s$, then there
  is a client $\Client$ with identity number $c$ maintaining oracle
  $\pi^c_{j,0}$ that has a matching conversation for initial key and
  forward secure key with $\pi^s_{i,0}$.}

  \item{The adversary does not issue a $\Reveal$ query
  to either $\pi^p_{i,\ell}$, nor to
  $\pi^{p^{\prime}}_{j,\ell}$ such that $\pi^p_{i,\ell}$
  has a matching conversation with
  $\pi^{p^{\prime}}_{j,\ell}$, and client oracles $\pi^c_{i,k}$
  ($k < \ell$) is $\tau_{co}$-revealed with $\tau_0 < \tau_{co}$.}
 \end{itemize}
 We say that protocol $\Pi$ has
 \textit{channel confidentiality} if
 $\Adv^{\rsaccecc}_{\Pi}(A)$ is negligible in $\kappa$.
\end{definition}

\begin{remark}
We extend the security requirement of
\textit{channel confidentiality} of SACCE~\cite{KPW13:SACCE}.
The original security notion takes care of message
confidentiality of ciphertexts sent by a (honest)
client, only. It is because an adversary can trivially
play a role of an honest client and have a matching
conversation with a (honest) server with the same
session key. In our definition, the third condition
indicates that an adversary may send the $\Encrypt$
query to even \textit{server} $\Server$ (to break
message confidentiality of the challenge ciphertext
sent by \textit{server oracle} $\pi^s_{i,\ell}$) as
long as the initial full handshake is established
between $\Server$ with identity number $s$ and an \textit{honest} client
$\Client$ with identity number $c$ (not the adversary). Later, if the
adversary can hijack some abbreviate handshake session
and share a new session key with server $\Server$,
then it can break our channel confidentiality.
The second condition specifies forward-secrecy among
the other sessions in the
sense that message confidentiality of $\pi^p_{i,\ell}$
is preserved, as long as the adversary does not
corrupt the server's long-term secret key $sk_s$ and Diffie-Hellman
secret value $t_s$ in $\SCFG$.
The fourth condition specifies forward-secrecy among
a full handshake and abbreviate handshakes in the sense
that message confidentiality of $\pi^p_{i,\ell}$ is preserved,
as long as the adversary does not reveal the oracles.
\end{remark}

\subsubsection{RSACCE security}
We define the RSACCE security.

\begin{definition}[RSACCE secure]
 We say that the protocol $\Pi$ is RSACCE secure
 if $\Pi$ satisfies \textbf{correctness},
 \textbf{server authentication},
 \textbf{channel confidentiality}.
\end{definition}
 %=====================================================
\section{Quick UDP Internet Connections} \label{sec:quic}
%=====================================================

\begin{figure*}[!htp]
\begin{center}

Fullhandshake between a client and a server \vspace{10pt}\\

\fbox{
\begin{minipage}[t]{0.35\textwidth}
\begin{tabular}[t]{l}
 $\quad Client$ \\
 $$ \\
 $1.\ \ \peer = S, \text{$S$ is determined from $cert_s$}$ \\
 $2.\ \ pk_s = \getPK(cert_s)$ \\
 $3.\ \ \text{If } \SIG.\Vfy(pk_s, \sigma_s, T_s \| \SCID) = \perp$ \\
 $4.\ \ \quad \Lambda = \text{'reject' and abort}$ \\
 $5.\ \ t_c \xleftarrow{\$} \mathbb{Z}_q$ \\
 $6.\ \ T_c=g^{t_c}$ \\
 $7.\ \ pms = T_s^{t_c}$ \\
 $8.\ \ \NONC \leftarrow \$$ \\
 $9.\ \ ms = \PRF(pms, \NONC)$ \\
 $10.\  trans = T_s \| T_c \| \SCID \| \NONC$ \\
 $11.\  k=\PRF(ms, trans)$ \\
 $12.\  \Lambda=\preaccept$ \\
 $13.\  m_2 = (T_c, \SCID, \NONC)$ \\
 $$\\
 $1.\ \ \text{If } \SE.\Dec(k,c_s,H,st_d) = \perp$ \\
 $2.\ \ \quad \Lambda=\text{'reject' and abort}$ \\
 $3.\ \ T_s^{\ast} = \SE.\Dec(k,c_s,H,st_d)$ \\
 $4.\ \ pms^{\ast} = T_s^{\ast t_c}$ \\
 $5.\ \ ms^{\ast} = \PRF(pms^{\ast}, \NONC)$ \\
 $6.\ \ trans^{\ast} = T_s^{\ast} \| T_c \| \SCID \| \NONC$ \\
 $7.\ \ k=\PRF(ms^{\ast}, trans^{\ast})$ \\
 $8.\ \ \Lambda=\accept$ \\
 $9.\ \ \theta \leftarrow (T_s, \SCID)$ \\
\end{tabular}
\end{minipage}%
}
\begin{minipage}[t]{0.13\textwidth}
\begin{tabular}[t]{l}
 $\xrightarrow{\text{inchoateCHLO}}$\\
 $$\\
 $$\\
 $$\\
 $\xleftarrow{\hspace{9pt}\text{REJ}(m_1)\hspace{9pt}}$\\
 $$\\
 $$\\
 $$\\
 $$\\
 $$\\
 $$\\
 $\xrightarrow{\hspace{5pt}\text{CHLO}(m_2)\hspace{5pt}}$\\
 $$\\
 $$\\
 $$\\
 $$\\
 $\xleftarrow{\hspace{5pt}\text{SHLO}(m_3)\hspace{5pt}}$\\
\end{tabular}
\end{minipage}%
\fbox{
\begin{minipage}[t]{0.35\textwidth}
\begin{tabular}[t]{l}
 $\quad Server$ \\
 $1.\ \ \SCID \leftarrow \{0,1\}^{\lambda} $ \\
 $2.\ \ t_s \xleftarrow{\$} \mathbb{Z}_q$ \\
 $3.\ \ T_s = g^{t_s}$ \\
 $4.\ \ \sigma_s = \SIG.\Sign(sk_s,T_s \| \SCID)$ \\
 $5.\ \ m_1 = (T_s,\sigma_s,\SCID,cert_s)$ \\
 $$\\ \vspace{5pt}
 $$\\
 $1.\ \ \peer = \SCID$ \\
 $2.\ \ pms=T_c^{t_s}$ \\
 $3.\ \ ms=\PRF(pms, \NONC)$ \\
 $4.\ \ trans = T_s \| T_c \| \SCID \| \NONC$ \\
 $5.\ \ k=\PRF(ms, trans)$ \\
 $6.\  \Lambda=\preaccept$ \\
 $7.\ \ t_s^{\ast} \xleftarrow{\$} \mathbb{Z}_q $ \\
 $8.\ \ T_s^{\ast} = g^{t_s^{\ast}} $ \\
 $9.\ \ c_s = \SE.\Enc(k, len, H, T_s^{\ast}, st_e)$ \\
 $10.\  m_3 = (c_s)$ \\
 $11.\  pms^{\ast} = T_c^{t_s^{\ast}}$ \\
 $12.\  ms^{\ast} = \PRF(pms^{\ast}, \NONC)$ \\
 $13.\  trans^{\ast} = T_s^{\ast} \| T_c \| \SCID \| \NONC$ \\
 $14.\  k=\PRF(ms^{\ast}, trans^{\ast})$ \\
 $15.\  \Lambda=\accept$ \\
 $16.\  \theta(\SCID) \leftarrow (T_s, t_s)$ \\
\end{tabular}
\end{minipage}%
} \vspace{10pt}

Resumption between a client and a server \vspace{10pt}\\

\fbox{
\begin{minipage}[t]{0.35\textwidth}
\begin{tabular}[t]{l}
 $1.\ \ (T_s, \SCID) \leftarrow \theta$ \\
 $2.\ \ t_c^{\prime} \xleftarrow{\$} \mathbb{Z}_q$ \\
 $3.\ \ T_c^{\prime}=g^{t_c^{\prime}}$ \\
 $4.\ \ pms^{\prime} = T_s^{t_c^{\prime}}$ \\
 $5.\ \ \NONC^{\prime} \leftarrow \$$ \\
 $6.\ \ ms^{\prime} = \PRF(pms^{\prime}, \NONC^{\prime})$ \\
 $7.\ \ trans^{\prime} = T_s \| T_c^{\prime} \| \SCID \| \NONC^{\prime}$ \\
 $8.\ \ k=\PRF(ms^{\prime}, trans^{\prime})$ \\
 $9.\ \ \Lambda=\preaccept$ \\
 $10.\  m_1^{\prime} = (T_c^{\prime},\SCID,\NONC^{\prime})$ \\
 $$\\
 $1.\ \ \text{If } \SE.\Dec(k,c_s,H,st_d) = \perp$ \\
 $2.\ \ \quad \Lambda=\text{'reject' and abort}$ \\
 $3.\ \ T_s^{\ast} = \SE.\Dec(k,c_s,H,st_d)$ \\
 $4.\ \ pms^{\ast} = T_s^{\ast t_c^{\prime}}$ \\
 $5.\ \ ms^{\ast} = \PRF(pms^{\ast}, \NONC^{\prime})$ \\
 $6.\ \ trans^{\ast} = T_s^{\ast} \| T_c^{\prime} \| \SCID \| \NONC^{\prime}$ \\
 $7.\ \ k=\PRF(ms^{\ast}, trans^{\ast})$ \\
 $8.\ \ \Lambda=\accept$ \\
 $9.\ \ \theta \leftarrow (T_s^{\ast}, \SCID)$ \\
\end{tabular}
\end{minipage}%
}
\begin{minipage}[t]{0.13\textwidth}
\begin{tabular}[t]{l}
 $$\\
 $$\\
 $$\\
 $$\\
 $$\\
 $$\\
 $\xrightarrow{\hspace{5pt}\text{CHLO}(m_1^{\prime})\hspace{5pt}}$\\
 $$\\
 $$\\
 $$\\
 $$\\
 $$\\
 $\xleftarrow{\hspace{5pt}\text{SHLO}(m_2^{\prime})\hspace{5pt}}$\\
\end{tabular}
\end{minipage}%
\fbox{
\begin{minipage}[t]{0.35\textwidth}
\begin{tabular}[t]{l}
 $$\vspace{3pt}\\
 $$\\
 $1.\ \ (T_s, t_s) \leftarrow \theta(\SCID)$ \\
 $2.\ \ pms^{\prime}=T_c^{\prime t_s}$ \\
 $3.\ \ ms^{\prime}=\PRF(pms^{\prime}, \NONC^{\prime})$ \\
 $4.\ \ trans^{\prime}= T_s \| T_c^{\prime} \| \SCID \| \NONC^{\prime}$ \\
 $5.\ \ k=\PRF(ms^{\prime}, trans^{\prime})$ \\
 $6.\ \ \Lambda=\preaccept$ \\
 $7.\ \ t_s^{\ast} \xleftarrow{\$} \mathbb{Z}_q $ \\
 $8.\ \ T_s^{\ast} = g^{t_s^{\ast}} $ \\
 $9.\ \ c_s = \SE.\Enc(k, len, H, T_s^{\ast}, st_e)$ \\
 $10.\  m_2^{\prime} = (c_s)$ \\
 $11.\  pms^{\ast} = T_c^{\prime t_s^{\ast}}$ \\
 $12.\  ms^{\ast} = \PRF(pms^{\ast}, \NONC^{\prime})$ \\
 $13.\  trans^{\ast} = T_s^{\ast} \| T_c^{\prime} \| \SCID \| \NONC^{\prime}$ \\
 $14.\  k=\PRF(ms^{\ast}, trans^{\ast})$ \\
 $15.\  \Lambda=\accept$ \\
 $16.\  \theta(\SCID) \leftarrow (T_s, t_s)$ \\
 $$\\
 $$\\
\end{tabular}
\end{minipage}
}

\caption{Abstract model of the QUIC handshake protocol}\label{fig:quic}
\end{center}
\end{figure*}
\begin{figure*}[!htp]
\begin{center}

Fullhandshake between a client and a server \vspace{10pt}\\

\fbox{
\begin{minipage}[t]{0.35\textwidth}
\begin{tabular}[t]{l}
 $\quad Client$ \\
 $$ \\
 $1.\ \ \peer = S, \text{$S$ is determined from $cert_s$}$ \\
 $2.\ \ pk_s = \getPK(cert_s)$ \\
 $3.\ \ \text{If } \SIG.\Vfy(pk_s, \sigma_s, T_s \| \SCID) = \perp$ \\
 $4.\ \ \quad \Lambda = \text{'reject' and abort}$ \\
 $5.\ \ t_c \xleftarrow{\$} \mathbb{Z}_q$ \\
 $6.\ \ T_c=g^{t_c}$ \\
 $7.\ \ pms = T_s^{t_c}$ \\
 $8.\ \ \NONC \leftarrow \$$ \\
 $9.\ \ ms = \PRF(pms, \NONC)$ \\
 $10.\  trans = T_s \| T_c \| \SCID \| \NONC$ \\
 $11.\  k=\PRF(ms, trans)$ \\
 $12.\  \Lambda=\preaccept$ \\
 $13.\  (pk_c,sk_c) = \SIG.\Gen(1^{\spar})$ \\
 $14.\  \sigma_c = \SIG.\Sign(sk_c, trans)$ \\
 $15.\  c_c = \SE.\Enc(k, len, H, pk_c \| \sigma_c, st_e)$ \\
 $16.\  m_2 = (T_c, \SCID, \NONC, c_c)$ \\
 $$\\
 $1.\ \ \text{If } \SE.\Dec(k,c_s,H,st_d) = \perp$ \\
 $2.\ \ \quad \Lambda=\text{'reject' and abort}$ \\
 $3.\ \ T_s^{\ast} = \SE.\Dec(k,c_s,H,st_d)$ \\
 $4.\ \ pms^{\ast} = T_s^{\ast t_c}$ \\
 $5.\ \ ms^{\ast} = \PRF(pms^{\ast}, \NONC)$ \\
 $6.\ \ trans^{\ast} = T_s^{\ast} \| T_c \| \SCID \| \NONC$ \\
 $7.\ \ k=\PRF(ms^{\ast}, trans^{\ast})$ \\
 $8.\ \ \Lambda=\accept$ \\
 $9.\ \ \theta \leftarrow (T_s^{\ast}, \SCID, pk_c, sk_c)$ \\
\end{tabular}
\end{minipage}%
}
\begin{minipage}[t]{0.13\textwidth}
\begin{tabular}[t]{l}
 $\xrightarrow{\text{inchoateCHLO}}$\\
 $$\\
 $$\\
 $$\\
 $\xleftarrow{\hspace{9pt}\text{REJ}(m_1)\hspace{9pt}}$\\
 $$\\
 $$\\
 $$\\
 $$\\
 $$\\
 $$\\
 $$\\
 $$\\
 $$\\
 $\xrightarrow{\hspace{5pt}\text{CHLO}(m_2)\hspace{5pt}}$\\
 $$\\
 $$\\
 $$\\
 $$\\
 $\xleftarrow{\hspace{5pt}\text{SHLO}(m_3)\hspace{5pt}}$\\
\end{tabular}
\end{minipage}%
\fbox{
\begin{minipage}[t]{0.35\textwidth}
\begin{tabular}[t]{l}
 $\quad Server$ \\
 $1.\ \ \SCID \leftarrow \{0,1\}^{\lambda} $ \\
 $2.\ \ t_s \xleftarrow{\$} \mathbb{Z}_q$ \\
 $3.\ \ T_s = g^{t_s}$ \\
 $4.\ \ \sigma_s = \SIG.\Sign(sk_s,T_s \| \SCID)$ \\
 $5.\ \ m_1 = (T_s,\sigma_s,\SCID,cert_s)$ \\
 $$\\ \vspace{6pt}
 $$\\
 $1.\ \ \peer = \SCID$ \\
 $2.\ \ pms=T_c^{t_s}$ \\
 $3.\ \ ms=\PRF(pms, \NONC)$ \\
 $4.\ \ trans = T_s \| T_c \| \SCID \| \NONC$ \\
 $5.\ \ k=\PRF(ms, trans)$ \\
 $6.\ \ \Lambda=\preaccept$ \\
 $7.\ \ (pk_c, \sigma_c) = \SE.\Dec(k,c_c,H,st_d)$ \\
 $8.\ \ \text{If } \SIG.\Vfy(pk_c, \sigma_c, trans) = \perp$ \\
 $9.\ \ \quad \Lambda = \text{'reject' and abort}$ \\
 $10.\  t_s^{\ast} \xleftarrow{\$} \mathbb{Z}_q $ \\
 $11.\  T_s^{\ast} = g^{t_s^{\ast}} $ \\
 $12.\  c_s = \SE.\Enc(k, len, H, T_s^{\ast}, st_e)$ \\
 $13.\  m_3 = (c_s)$ \\
 $14.\  pms^{\ast} = T_c^{t_s^{\ast}}$ \\
 $15.\  ms^{\ast} = \PRF(pms^{\ast}, \NONC)$ \\
 $16.\  trans^{\ast} = T_s^{\ast} \| T_c \| \SCID \| \NONC$ \\
 $17.\  k=\PRF(ms^{\ast}, trans^{\ast})$ \\
 $18.\  \Lambda=\accept$ \\
 $19.\  \theta(\SCID) \leftarrow (T_s^{\ast}, t_s^{\ast})$ \\
\end{tabular}
\end{minipage}%
} \vspace{10pt}

Resumption between a client and a server \vspace{10pt}\\

\fbox{
\begin{minipage}[t]{0.35\textwidth}
\begin{tabular}[t]{l}
 $1.\ \ (T_s^{\ast}, \SCID, pk_c, sk_c) \leftarrow \theta$ \\
 $2.\ \ t_c^{\prime} \xleftarrow{\$} \mathbb{Z}_q$ \\
 $3.\ \ T_c^{\prime}=g^{t_c^{\prime}}$ \\
 $4.\ \ pms^{\prime} = T_s^{\ast t_c^{\prime}}$ \\
 $5.\ \ \NONC^{\prime} \leftarrow \$$ \\
 $6.\ \ ms^{\prime} = \PRF(pms^{\prime}, \NONC^{\prime})$ \\
 $7.\ \ trans^{\prime} = T_s^{\ast} \| T_c^{\prime} \| \SCID \| \NONC^{\prime}$ \\
 $8.\ \ k=\PRF(ms^{\prime}, trans^{\prime})$ \\
 $9.\ \ \Lambda=\preaccept$ \\
 $10.\  \sigma_c^{\prime} = \SIG.\Sign(sk_c, trans^{\prime})$ \\
 $11.\  c_c = \SE.\Enc(k, len, H, pk_c \| \sigma_c^{\prime}, st_e)$ \\
 $12.\  m_1^{\prime} = (T_c^{\prime},\SCID,\NONC^{\prime},c_c)$ \\
 $$\\
 $1.\ \ \text{If } \SE.\Dec(k,c_s,H,st_d) = \perp$ \\
 $2.\ \ \quad \Lambda=\text{'reject' and abort}$ \\
 $3.\ \ T_s^{\ast} = \SE.\Dec(k,c_s,H,st_d)$ \\
 $4.\ \ pms^{\ast} = T_s^{\ast t_c^{\prime}}$ \\
 $5.\ \ ms^{\ast} = \PRF(pms^{\ast}, \NONC^{\prime})$ \\
 $6.\ \ trans^{\ast} = T_s^{\ast} \| T_c^{\prime} \| \SCID \| \NONC^{\prime}$ \\
 $7.\ \ k=\PRF(ms^{\ast}, trans^{\ast})$ \\
 $8.\ \ \Lambda=\accept$ \\
 $9.\ \ \theta \leftarrow (T_s^{\ast}, \SCID, pk_c, sk_c)$ \\
\end{tabular}
\end{minipage}%
}
\begin{minipage}[t]{0.13\textwidth}
\begin{tabular}[t]{l}
 $$\\
 $$\\
 $$\\
 $$\\
 $$\\
 $$\\
 $\xrightarrow{\hspace{5pt}\text{CHLO}(m_1^{\prime})\hspace{5pt}}$\\
 $$\\
 $$\\
 $$\\
 $$\\
 $$\\
 $\xleftarrow{\hspace{5pt}\text{SHLO}(m_2^{\prime})\hspace{5pt}}$\\
\end{tabular}
\end{minipage}%
\fbox{
\begin{minipage}[t]{0.35\textwidth}
\begin{tabular}[t]{l}
 $$\vspace{4pt}\\
 $$\\
 $1.\ \ (T_s^{\ast}, t_s^{\ast}) \leftarrow \theta(\SCID)$ \\
 $2.\ \ pms^{\prime}=T_c^{\prime t_s^{\ast}}$ \\
 $3.\ \ ms^{\prime}=\PRF(pms^{\prime}, \NONC^{\prime})$ \\
 $4.\ \ trans^{\prime}= T_s^{\ast} \| T_c^{\prime} \| \SCID \| \NONC^{\prime}$ \\
 $5.\ \ k=\PRF(ms^{\prime}, trans^{\prime})$ \\
 $6.\  \Lambda=\preaccept$ \\
 $7.\ \ (pk_c, \sigma_c^{\prime}) = \SE.\Dec(k,c_c,H,st_d)$ \\
 $8.\ \ \text{If } \SIG.\Vfy(pk_c, \sigma_c^{\prime}, trans^{\prime}) = \perp$ \\
 $9.\ \ \quad \Lambda = \text{'reject' and abort}$ \\
 $10.\  t_s^{\ast} \xleftarrow{\$} \mathbb{Z}_q $ \\
 $11.\  T_s^{\ast} = g^{t_s^{\ast}} $ \\
 $12.\  c_s = \SE.\Enc(k, len, H, T_s^{\ast}, st_e)$ \\
 $13.\  m_2^{\prime} = (c_s)$ \\
 $14.\  pms^{\ast} = T_c^{\prime t_s^{\ast}}$ \\
 $15.\  ms^{\ast} = \PRF(pms^{\ast}, \NONC^{\prime})$ \\
 $16.\  trans^{\ast} = T_s^{\ast} \| T_c^{\prime} \| \SCID \| \NONC^{\prime}$ \\
 $17.\  k=\PRF(ms^{\ast}, trans^{\ast})$ \\
 $18.\  \Lambda=\accept$ \\
 $19.\  \theta(\SCID) \leftarrow (T_s^{\ast}, t_s^{\ast})$ \\
 $$\\
\end{tabular}
\end{minipage}
}

\caption{Abstract model of the QUIC handshake protocol with CETV}\label{fig:quic_cetv}
\end{center}
\end{figure*}
\begin{figure*}[!htb]
\begin{center}

Fullhandshake between a client and a server \vspace{10pt}\\

\fbox{
\begin{minipage}[t]{0.35\textwidth}
\begin{tabular}[t]{l}
 $\quad Client$ \\
 $$ \\
 $1.\ \ \NONC \leftarrow \$$ \\
 $2.\ \ t_c \xleftarrow{\$} \mathbb{Z}_q$ \\
 $3.\ \ T_c=g^{t_c}$ \\
 $4.\ \ m_1 = (T_c, \NONC)$ \\
 $$\\
 $1.\ \ \peer = S, \text{$S$ is determined from $cert_s$}$ \\
 $2.\ \ pk_s = \getPK(cert_s)$ \\
 $3.\ \ pms = T_s^{t_c}$ \\
 $4.\ \ ms = \PRF(pms, \NONC)$ \\
 $5.\ \ trans = T_s \| T_c \| \SCID \| \NONC$ \\
 $6.\ \ k = \PRF(ms, trans)$ \\
 $7.\ \ \Lambda=\preaccept$ \\
 $8.\ \ T_s^{\ast} = \SE.\Dec(k,c_s,H,st_d)$ \\
 $9.\ \ \text{If } \SIG.\Vfy(pk_s, \sigma_s, trans \| T_s^{\ast}) = \perp$ \\
 $10.\  \quad \Lambda = \text{'reject' and abort}$ \\
 $11.\  pms^{\ast} = T_s^{\ast t_c}$ \\
 $12.\  ms^{\ast} = \PRF(pms^{\ast}, NONC)$ \\
 $13.\  trans^{\ast} = T_s^{\ast} \| T_c \| \SCID \| \NONC$ \\
 $14.\  k=\PRF(ms^{\ast}, trans^{\ast})$ \\
 $15.\  \Lambda=\accept$ \\
 $16.\  \theta \leftarrow (\SCID, T_s^{\ast})$ \\
\end{tabular}
\end{minipage}%
}
\begin{minipage}[t]{0.13\textwidth}
\begin{tabular}[t]{l}
 $$\\
 $$\\
 $$\\
 $$\\
 $\xrightarrow{\hspace{5pt}\text{CHLO}(m_1)\hspace{5pt}}$\\
 $$\\
 $$\\
 $$\\
 $\xleftarrow{\hspace{5pt}\text{SHLO}(m_2)\hspace{5pt}}$\\
 $$\\
 $$\\
 $$\\
 $$\\
\end{tabular}
\end{minipage}%
\fbox{
\begin{minipage}[t]{0.35\textwidth}
\begin{tabular}[t]{l}
 $\quad Server$ \\
 $$\\
 $1.\ \ \SCID \leftarrow \{0,1\}^{\lambda} $ \\
 $2.\ \ \peer = \SCID$ \\
 $3.\ \ t_s \xleftarrow{\$} \mathbb{Z}_q$ \\
 $4.\ \ T_s = g^{t_s}$ \\
 $5.\ \ trans= T_s \| T_c \| \SCID \| \NONC$ \\
 $6.\ \ pms=T_c^{t_s}$ \\
 $7.\ \ ms=\PRF(pms, \NONC)$ \\
 $8.\ \ k=\PRF(ms, trans)$ \\
 $9.\ \ \Lambda=\preaccept$ \\
 $10.\  t_s^{\ast} \xleftarrow{\$} \mathbb{Z}_q$ \\
 $11.\  T_s^{\ast} = g^{t_s^{\ast}}$ \\
 $12.\  c_s = \SE.\Enc(k,len,H,T_s^{\ast},st_e)$ \\
 $13.\  \sigma_s = \SIG.\Sign(sk_s,trans \| T_s^{\ast})$ \\
 $14.\  m_2 = (T_s, \sigma_s, \SCID, cert_s, c_s)$ \\
 $15.\  pms^{\ast} = T_c^{t_s^{\ast}}$ \\
 $16.\  ms^{\ast} = \PRF(pms^{\ast}, \NONC)$ \\
 $17.\  trans^{\ast} = T_s^{\ast} \| T_c \| \SCID \| \NONC$ \\
 $18.\  k=\PRF(ms^{\ast}, trans^{\ast})$ \\
 $19.\  \Lambda=\accept$ \\
 $20.\  \theta(\SCID) \leftarrow (T_s^{\ast},t_s^{\ast})$ \\
 $$
\end{tabular}
\end{minipage}%
} \vspace{10pt}

Resumption between a client and a server \vspace{10pt}\\

\fbox{
\begin{minipage}[t]{0.35\textwidth}
\begin{tabular}[t]{l}
 $1.\ \ (\SCID,T_s^{\ast}) \leftarrow \theta$ \\
 $2.\ \ t_c^{\prime} \xleftarrow{\$} \mathbb{Z}_q$ \\
 $3.\ \ T_c^{\prime}=g^{t_c^{\prime}}$ \\
 $4.\ \ pms^{\prime} = T_s^{\ast t_c^{\prime}}$ \\
 $5.\ \ \NONC^{\prime} \leftarrow \$$ \\
 $6.\ \ ms^{\prime} = \PRF(pms^{\prime}, \NONC^{\prime})$ \\
 $7.\ \ trans^{\prime} = T_s^{\ast} \| T_c^{\prime} \| \SCID \| \NONC^{\prime}$ \\
 $8.\ \ fin_c = \PRF(ms^{\prime}, trans^{\prime})$ \\
 $9.\ \ k=\PRF(ms^{\prime}, trans^{\prime})$ \\
 $10.\  \Lambda=\preaccept$ \\
 $11.\  m_1^{\prime} = (T_c^{\prime},\SCID,\NONC^{\prime},fin_c)$ \\
 $$\\
 $1.\ \ \text{If } \SE.\Dec(k,c_s,H,st_d) = \perp$ \\
 $2.\ \ \quad \Lambda=\text{'reject' and abort}$ \\
 $3.\ \ T_s^{\ast} = \SE.\Dec(k,c_s,H,st_d)$ \\
 $4.\ \ pms^{\ast} = T_s^{\ast t_c^{\prime}}$ \\
 $5.\ \ ms^{\ast} = \PRF(pms^{\prime}, \NONC^{\prime})$ \\
 $6.\ \ trans^{\ast} = T_s^{\ast} \| T_c^{\prime} \| \SCID \| \NONC^{\prime}$ \\
 $7.\ \ k=\PRF(ms^{\ast}, trans^{\ast})$ \\
 $8.\ \ \Lambda=\accept$ \\
 $9.\ \ \theta \leftarrow (\SCID,T_s^{\ast},fin_c)$ \\
\end{tabular}
\end{minipage}%
}
\begin{minipage}[t]{0.13\textwidth}
\begin{tabular}[t]{l}
 $$\\
 $$\\
 $$\\
 $$\\
 $$\\
 $$\\
 $\xrightarrow{\hspace{5pt}\text{CHLO}(m_1^{\prime})\hspace{5pt}}$\\
 $$\\
 $$\\
 $$\\
 $$\\
 $$\\
 $\xleftarrow{\hspace{5pt}\text{SHLO}(m_2^{\prime})\hspace{5pt}}$\\
\end{tabular}
\end{minipage}%
\fbox{
\begin{minipage}[t]{0.35\textwidth}
\begin{tabular}[t]{l}
 $$\\
 $1.\ \ (T_s^{\ast}, t_s^{\ast}) \leftarrow \theta(\SCID)$ \\
 $2.\ \ pms^{\prime}=T_c^{\prime t_s^{\ast}}$ \\
 $3.\ \ ms^{\prime}=\PRF(pms^{\prime}, \NONC^{\prime})$ \\
 $4.\ \ trans^{\prime}= T_s^{\ast} \| T_c^{\prime} \| \SCID \| \NONC^{\prime}$ \\
 $5.\ \ fin_c^{\ast} = \PRF(ms^{\prime}, trans^{\prime})$ \\
 $6.\ \ \text{If } fin_c^{\ast} \neq fin_c$ \\
 $7.\ \ \quad \Lambda=\text{'reject' and abort}$ \\
 $8.\ \ k=\PRF(ms^{\prime}, trans^{\prime})$ \\
 $9.\ \ \Lambda=\preaccept$ \\
 $10.\  t_s^{\ast} \xleftarrow{\$} \mathbb{Z}_q $ \\
 $11.\  T_s^{\ast} = g^{t_s^{\ast}} $ \\
 $12.\  c_s = \SE.\Enc(k, len, H, T_s^{\ast}, st_e)$ \\
 $13.\  pms^{\ast} = T_c^{t_s^{\ast}}$ \\
 $14.\  ms^{\ast} = \PRF(pms^{\ast}, \NONC^{\prime})$ \\
 $15.\  trans^{\ast} = T_s^{\ast} \| T_c^{\prime} \| \SCID \| \NONC^{\prime}$ \\
 $16.\  k=\PRF(ms^{\ast}, trans^{\ast})$ \\
 $17.\  m_2^{\prime} = (c_s)$ \\
 $18.\  \Lambda=\accept$ \\
 $19.\  \theta(\SCID) \leftarrow (T_s^{\ast}, t_s^{\ast})$ \\
 $$\\
\end{tabular}
\end{minipage}
}

\caption{Abstract model of our proposed scheme}\label{fig:quic_detail}
\end{center}
\end{figure*}

Quick UDP Internet Connections (QUIC) is a protocol developed by Google. This protocol is still under development. The abstract model of QUIC is described in Fig.~\ref{fig:quic}. The concepts of QUIC are (1) to reduce connectivity overheads before a client sends encrypted data and (2) to obtain better security guarantee than TLS.
To realize concept (1), QUIC is not defined over TCP but UDP, because TCP requires three-move handshake before initiating a cryptographic handshake, but UDP does not have. In addition, a client can send encrypted data concurrently with CHLO.
To realize concept (2), QUIC supports only secure cipher suites. Especially, a support algorithm of key exchange is only ephemeral elliptic curve Diffie-Hellman.

Unlike TLS, QUIC does not support client authentication.

%=====================================================
\subsection{Security of QUIC} \label{sec:quic_detail}
%=====================================================

\begin{theorem} \label{theorem:quic}
 Let $\mu$ be the output length of $\PRF$, let $\lambda$ be the length of $\SCID$, let $\nclient$ be the number of clients, let $\nserver$ be the number of servers, let $\noracle$ be the number of oracles of each parties, and let $n_{\ell}$ be the maximum number of resumptions. Assume that the $\PRF$ is $(t, \epsilon_{\prf})$-pseudo-random function family, the signature scheme $\SIG$ is $(t, \epsilon_{\sig})$-secure against existentially unforgeable under adaptive chosen-message attacks, the DDH problem on $G$ is $(t, \epsilon_{\ddh})$-hard, the PRF-ODH problem on $\PRF$ is $(t, \epsilon_{\prfodh})$-hard, the hash function family $\mathcal{H}$ is $(t,\epsilon_{H})$-collision-resistant (CR), the symmetric encryption scheme $\SE$ is $(t, \epsilon_{\sLHAE})$-secure.
 Then for all PPT adversaries, QUIC is RSACCE secure.
\end{theorem}%

The proof of Theorem~\ref{theorem:quic} is in Appendix~\ref{app:quic}.\\

We note that QUIC is not \textit{strong} RSACCE secure. Although QUIC has a mechanism to prevent forgery, whose mechanisms is called \textit{source-address token} (see below), the adversary can forge a resumption query as follows: A adversary can obtain SCID from $\REJ$ response, which is the first response from a server oracle. Then, the adversary make $(\overline{T}_c^{\prime}, \overline{\NONC}^{\prime})$ and send $(\overline{T}_c^{\prime}, \SCID, \overline{\NONC}^{\prime})$ to the server after a full handshake between the server and its intended partner is established. The server cannot distinguish whether the query in resumption comes from the intended partner or not. Because there is no authentication mechanism in this query. The server accepts this query and calculates the session key with the value of adversary. The adversary does not know the session key because $T_s^{\ast}$ is encrypted however a session key between the client and the server does not match.

\subsubsection{Source Address Token} \label{sec:source_address_token}
Source-address token ($\STK$) is introduced to QUIC in order to prevent IP address spoofing. A server generates and sends a new STK every time he sends a message to a client. The client updates STK when the client receives a new one from the server and sends it back along with his message. $\STK$ is an opaque byte string from the client's point of view. From the server's point of view it's an authenticated-encryption block that contains, at least, the client's IP address and a time stamp by the server. $\STK$ is encrypted except for the first server's query ($\REJ$). However, in our model the adversary has full control over the communication network and hence it can obtain $\STK$ and use it to forge a future query.

%--------------------------------------------------------
\subsection{Security of QUIC with CETV} \label{sec:quic_cetv}
%--------------------------------------------------------

We have already mentioned that QUIC does not meet strong RSACCE security. However, QUIC has an optional mechanism, called client encrypted tag-value (CETV), which is disabled in the default setting. This mechanism borrows from~\cite{MCBW12:TLS-OCB,ChannelID} and utilizes a part of its implementation. In this section, we prove that QUIC with CETV (i.e., QUIC with CETV enabled) satisfies strong RSACCE.
The abstract model of QUIC with CETV is described in Fig.~\ref{fig:quic_cetv}.

\begin{theorem} \label{theorem:quic_cetv}
 Let $\mu$ be the output length of $\PRF$, let $\lambda$ be the length of $\SCID$, let $\nclient$ be the number of clients, let $\nserver$ be the number of servers, let $\noracle$ be the number of oracles of each parties, and let $n_{\ell}$ be the maximum number of resumptions. Assume that the $\PRF$ is $(t, \epsilon_{\prf})$-pseudo-random function family, the signature scheme $\SIG$ is $(t, \epsilon_{\sig})$-secure against existentially unforgeable under adaptive chosen-message attacks, the DDH problem on $G$ is $(t, \epsilon_{\ddh})$-hard, the PRF-ODH problem on $\PRF$ is $(t, \epsilon_{\prfodh})$-hard, the hash function family $\mathcal{H}$ is $(t,\epsilon_{H})$-collision-resistant (CR), the symmetric encryption scheme $\SE$ is $(t, \epsilon_{\sLHAE})$-secure.
 Then for all PPT adversaries, QUIC with CETV is \textbf{strong} RSACCE secure.
\end{theorem}

The proof of Theorem~\ref{theorem:quic} is in Appendix~\ref{app:quic}.\\

Although we have proved that QUIC with CETV is strong RSACCE secure, many steps in CETV are useless for this purpose because the server does not cache its public key for the future sessions. The reason why QUIC with CETV satisfies strong RSACCE security is that a client sends ciphertext calculated using current session key. If the adversary cannot obtain the session key, the server does not accept the query from the adversary because the adversary cannot make the ciphertext $c^{\prime}$ without the session key which $\SE.\Dec$ returns a message $m \neq \perp$.
 %=====================================================
\section{Our proposed scheme} \label{sec:proposed_scheme}
%=====================================================

There are security concerns and redundant procedures in the original
QUIC.

As for the security concerns,
Lychev~\cite{LJBN15:QUIC} pointed out five attacks,
which may lead to the Distributed Denial of Service (DDoS) attacks.
As described above,
the DDoS attacks take advantage of either incomplete server-authentication or
lack of channel binding.
In the RSACCE model, the attacks in \cite{LJBN15:QUIC} do not work.

To improve efficiency of QUIC, we can learn from
the previous protocols such as SIGMA protocol~\cite{Kra03:SIGMA} and HMQV~\cite{Kra05:HQMV},
where a client sends its DH public value in the very first query.
This technique improves not only the number of interactions
but also makes assumptions weaker.

\if0
The main purpose of developing QUIC is to provide an
alternative equivalence of TLS wrapping TCP, with much
reduced latency and better SPDY and HTTP/2 support.
Considering the primary purpose, QUIC must be improved to stand against these DDoS attacks.
\fi

\if0
To solve above issues, we propose a new scheme which is more secure and more efficient
than the original QUIC and it satisfy RSACCE secure.
In our proposed scheme, a client sends its DH public key in
a first query.
% This scheme is big difference from original one and it sounds difficult to apply
% this modification even if QUIC is still under development.
% The handshake of QUIC will be replaced TLS1.3.
% However
%  and the handshake scheme of
% TLS1.3 is similar to our proposed scheme in terms of that a client sends its
% DH public key in a first query.
\fi

\if0
In~\cite{LJBN15:QUIC}, they found five attacks:
(1) Server Config Replay Attack,
(2) Source-Address Token Replay Attack,
(3) Connection ID Manipulation Attack,
(4) Source-Address Token Manipulation Attack,
(5) Crypto Stream Offset Attack.
In our proposed scheme, $\STK$ and $\cid$ are authenticated with the initial key
$\ik$ and other parameters are also authenticated.
This modification prevents (1)Server Config Replay Attack,
(3) Connection ID Manipulation Attack, and
(4) Source-Address Token Manipulation Attack.
In our proposed scheme, the server ensure the consistency of the client between
a 1-RTT connection and 0-RTT connections, in other words, only the client which
execute full handshake and has matching conversations for an initial key and forward
secure key with the server in 1-RTT connection can establish 0-RTT connection with
the server because the client sends MAC generated by $\key_{mac}$ in 0-RTT request.
The client receives a ciphertext which includes $\key_{mac}$ in 1-RTT connection.
This modification prevents (2) Source-Address Token Replay Attack.
\fi
%=====================================================
\subsection{1-RTT Connection Establishment} \label{sec:quic_prop_1rtt}
%=====================================================

The abstract model of our proposed scheme is in
Fig.~\ref{fig:quic_prop_1rtt}.

\begin{figure*}[!htp]
 \begin{center}

\begin{enumerate}
 \item{Initiate} \\
% client side
 \fbox{
  \begin{minipage}[t]{0.38\textwidth}
  \centering
   \begin{tabular}{c}
    $\quad Client$ \\
    $ $ \\
    $ $ \\
    $ $ \\
    $ $ \\
   \end{tabular}
  \end{minipage}%
 }
% middle
 \begin{minipage}[t]{0.13\textwidth}
  \centering
  \begin{tabular}{c}
   $ $ \\
   $ $ \\
   $ $ \\
  \end{tabular}
 \end{minipage}%
% server side
 \fbox{
  \begin{minipage}[t]{0.38\textwidth}
   \centering
   \begin{tabular}{c}
    $\quad Server$ \\
    $ $ \\
    $(pk_s, sk_s) = \SIG.\Gen()$ \\
    $(\SCFG_{pub}, t_s) = \scfgGen(sk_s)$ \\
    $k_{\STK} \xleftarrow{\$} \{0,1\}^{\lambda}$ \\
   \end{tabular}
  \end{minipage}%
 }
 \item{Key Agreement} \\
% client side
 \fbox{
  \begin{minipage}[t]{0.38\textwidth}
  \centering
   \begin{tabular}{c}
    $m_1 = \initialCHLO()$ \\
    $ $ \\
    $\checkSCFG(\SCFG_{pub}) $ \\
    $\shareInfo_c = (\NONC, \cid, T_s, t_c) $ \\
    $\ik = \getKey_c(\shareInfo_c, m_1, 1) $ \\
    $(T_s^{\prime}, \STK, \key_{MAC}) = \receiveSHLO(m_2, \ik)$ \\
    $\shareInfo_c^{\prime} = (\NONC, \cid, T_s^{\prime}, t_c)$ \\
    $m = m_1 \| m_2$ \\
    $\peer = S$ \\
    $k = \getKey_c(\shareInfo_c^{\prime}, m, 0)$ \\
    $\theta = (\SCFG_{pub}, \STK, k_{MAC})$ \\
    $\Lambda = \accept$ \\
   \end{tabular}
  \end{minipage}%
 }
% middle
 \begin{minipage}[t]{0.13\textwidth}
  \centering
  \begin{tabular}{c}
   $\xrightarrow{m_1}$ \\
   $ $ \\
   $\xleftarrow{m_2}$ \\
   $ $ \\
   $ $ \\
   $ $ \\
   $ $ \\
  \end{tabular}
 \end{minipage}%
% server side
 \fbox{
  \begin{minipage}[t]{0.38\textwidth}
   \centering
   \begin{tabular}{c}
    $ $ \\
    $ $ \\
    $ $ \\
    $\shareInfo_s = (\NONC, \cid, T_c, t_s)$ \\
    $\ik = \getKey_s(\shareInfo_s, m_1, 1)$ \\
    $ret = \SHLO(m_1, \ik, 0)$ \\
    $m_2 = ret \| \SCFG_{pub} $ \\
    $\shareInfo_s^{\prime} = (\NONC, \cid, T_c, t_s^{\prime}) $ \\
    $m = m_1 \| m_2$ \\
    $\peer = C$ \\
    $k=\getKey_s(\shareInfo_s^{\prime}, m, 0)$ \\
    $\Lambda = \accept$ \\
   \end{tabular}
  \end{minipage}%
 }
 \item{Data Exchange} \\
% client side
 \fbox{
  \begin{minipage}[t]{0.38\textwidth}
  \centering
   \begin{tabular}{c}
    $ $ \\
    $ $ \\
    $ $ \\
    $\text{for each } \alpha \in {0,...,\MsgCntC{0}}$ \\
    $\sqn_c = \alpha + 1$ \\
    $m_3^{\alpha} = \pak(k, sqn_c, M_c^{\alpha})$ \\
    $ $ \\
    $m_3 = (m_3^{0},...,m_3^{\MsgCntC{0}})$ \\
    $\processPacket(k, m_4)$ \\
   \end{tabular}
  \end{minipage}%
 }
% middle
 \begin{minipage}[t]{0.13\textwidth}
  \centering
  \begin{tabular}{c}
   $ $ \\
   $ $ \\
   $ $ \\
   $ $ \\
   $\xrightarrow{m_3}$ \\
   $\xleftarrow{m_4}$ \\
  \end{tabular}
 \end{minipage}%
% server side
 \fbox{
  \begin{minipage}[t]{0.38\textwidth}
   \centering
   \begin{tabular}{c}
    $\text{If the first message in $m_3$ does not come} $ \\
    $\text{in $\duration$ QUIC regards the first query as}$ \\
    $\text{DoS attack and this connection is closed.}$ \\
    $\text{for each } \beta \in {0,...,\MsgCntS{0}}$ \\
    $\sqn_s = \beta + 1$ \\
    $m_4^{\beta} = \pak(k, \sqn_s, M_s^{\beta})$ \\
    $ $ \\
    $m_4 = (m_4^{\MsgCntS{0}+1},...,m_4^{\MsgCntS{1}})$ \\
    $\processPacket(ik, m_3)$ \\
   \end{tabular}
  \end{minipage}%
 }
\end{enumerate}

 \caption{Abstract model of 1-RTT our proposed scheme}\label{fig:quic_prop_1rtt}
 \end{center}
\end{figure*}

We define three phases of our proposed scheme handshake in 1-RTT:
(1) \textbf{Initiate},
(2) \textbf{Key Agreement},
(3) \textbf{Data Exchange}.

%=====================================================
\subsubsection{Initiate}
%=====================================================
In this phase, a client do nothing and a server make
$k_{\STK}$ and run $\scfgGen$ and $\SIG.\Gen$ to
generate server config (SCFG) and long term secret
and public key.
SCFG is composed of seven parameters AEAD, SCID, PDMD,
PUBS, KEXS, and OBIT, and EXPY. The important parameters
are SCID which is an opaque 16 byte identifier for
this server config, PUBS which is server's
DH public key, and EXPY which is expiry time
for this server config. The details of other parameters
are described in~\cite{QUIC:Crypto}.
Our definition consider only important parameters.
\\
\noindent
\underline{$\scfgGen(sk_s)$:} \\
 \setcounter{nombre}{0}%
 $\prob.\quad t_s \xleftarrow{\$} \Zset_{q}^{\ast}$ \\
 $\prob.\quad T_s = g^{t_s}$ \\
 $\prob.\quad \SCID = \Hash(T_s \| \expy)$ \\
 $\prob.\quad str = \text{ QUIC server config signature }$ \\
 $\prob.\quad \doc = str \| 0x00 \| \SCID \| T_s \| \expy$ \\
 $\prob.\quad \sigma_s = \SIG.\Sign(sk_s, \doc)$ \\
 $\prob.\quad \SCFG_{pub} = (\SCID, T_s, \expy, \sigma_s, \cert_s)$ \\
 $\prob.\quad \return\ (\SCFG_{pub}, t_s)$ \\
%
Note that the generation of $\SCFG$ and the signing
of its public parameters are done independently of
client's connection requests.
%=====================================================
\subsubsection{Key Agreement}
%=====================================================
In this phase, the client sends an initial client
hello (initialCHLO) which contains connection id,
a client's Diffie-Hellman public value $T_c$, a client
nonce $\NONC$, and some information such as server name,
protocol version, and user agent id. In our definition,
the some informations are omitted.
\\
\noindent
\underline{$\initialCHLO()$:} \\
 \setcounter{nombre}{0}%
 $\prob.\quad \cid \xleftarrow{\$} \{0,1\}^{64} $ \\
 $\prob.\quad \pInfo = (IP_c, IP_s, port_c, port_s)$ \\
 $\prob.\quad t_c \xleftarrow{\$} \Zset_{q}^{\ast}$ \\
 $\prob.\quad T_c = g^{t_c}$ \\
 $\prob.\quad r \xleftarrow{\$} \{0,1\}^{160}$ \\
 $\prob.\quad \NONC = currentTime \| r$ \\
 $\prob.\quad \return\ (\pInfo, \cid, \NONC, T_c)$ \\
%
After the server receives initial client hello, it
sends a server hello (SHLO). The server hello contains
source address token (STK), server config (SCFG),
a certificate, a signature of server config generated
by the server long term secret key, ephemeral server's
Diffie-Hellman public value $T_s^{\prime}$. The client use
STK in future queries to demonstrate ownership of their
source IP address.
\\
\noindent
\underline{$\SHLO(m, \ik, \sqn)$:} \\
 \setcounter{nombre}{0}%
 $\prob.\quad (\pInfo, \cid, \NONC, T_c) = m$ \\
 $\prob.\quad \STK = \makeSTK()$ \\
 $\prob.\quad \pInfo = (IP_s, IP_c, port_s, port_c)$ \\
 $\prob.\quad (\ik_c, \ik_s, \iv_c, \iv_s) = \ik$ \\
 $\prob.\quad t_s^{\prime} \xleftarrow{\$} \Zset_{q}^{\ast}$ \\
 $\prob.\quad T_s^{\prime} = g^{t_s^{\prime}}$ \\
 $\prob.\quad \plaintext = T_{s}^{\prime} \| \STK \| k_{MAC} \| \SCID$\\
 $\prob.\quad H = (\cid, \sqn)$ \\
 $\prob.\quad c = \SE.\Enc(\ik_c, \iv_c \| \sqn, H, \plaintext)$ \\
 $\prob.\quad \return\ (\pInfo, \cid, \SCFG_{pub}, H, c)$ \\
\\
\underline{$\makeSTK()$:} \\
 \setcounter{nombre}{0}%
 $\prob.\quad \iv_{\STK} \xleftarrow{\$} \{0,1\}^{96}$ \\
 $\prob.\quad k_{MAC} \xleftarrow{\$} \{0,1\}^{\mu}$ \\
 $\prob.\quad \plaintext = IP_c \| currentTime \| k_{\MAC}$ \\
 $\prob.\quad \STK \leftarrow \iv_{\STK} \| \SE.\Enc(k_{\STK}, len, \iv_{\STK}, \plaintext)$ \\
 $\prob.\quad \return\ \STK$ \\
%
After the client receives a server hello, the client
checks server config and calculate a last key.
\\
\noindent
\underline{$\receiveSHLO(m, \ik)$:} \\
 \setcounter{nombre}{0}%
 $\prob.\quad (\pInfo, \cid, \SCFG_{pub}, H, c) = m$ \\
 $\prob.\quad (\ik_c, \ik_s, \iv_c, \iv_s) = \ik$ \\
 $\prob.\quad (\cid, \sqn) = H$ \\
 $\prob.\quad \plaintext = \SE.\Dec(\ik_c, \iv_c \| \sqn, H, c)$ \\
 $\prob.\quad \text{If }\plaintext = \perp$ \\
 $\prob.\quad \quad \Lambda = \text{'reject' and abort}$ \\
 $\prob.\quad T_{s}^{\prime} \| \STK \| \key_{MAC} \| \SCID^{\prime} = \plaintext $ \\
 $\prob.\quad \text{If }\SCID^{\prime} \neq \SCID$ \\
 $\prob.\quad \quad \Lambda = \text{'reject' and abort}$ \\
 $\prob.\quad \return\ (T_s^{\prime}, \STK, \key_{MAC})$ \\
\\
\underline{$\checkSCFG(\SCFG_{pub})$:} \\
 \setcounter{nombre}{0}%
 $\prob.\quad (\SCID, T_s, \expy, \sigma_s, \cert_s) = \SCFG_{pub}$ \\
 $\prob.\quad \text{If } \expy \leq currentTime$ \\
 $\prob.\quad \quad \Lambda = \text{'reject' and abort}$ \\
 $\prob.\quad pk_s = \getPK(cert_s)$ \\
 $\prob.\quad str = \text{ QUIC server config signature }$ \\
 $\prob.\quad \doc = str \| 0x00 \| \SCID \| T_s \| \expy$ \\
 $\prob.\quad \text{If } \SIG.\Vfy(pk_s, \sigma_s, \doc) = \perp$ \\
 $\prob.\quad \quad \Lambda = \text{'reject' and abort}$ \\
%
After the server sends a server hello or the
client validates a server hello, they calculate forward secure
key $\key$.
\noindent
\underline{$\getKey_c(\shareInfo, m, \init)$:} \\
 \setcounter{nombre}{0}%
 $\prob.\quad (\NONC, \cid ,T_s, t_c) = \shareInfo$ \\
 $\prob.\quad pms = T_s^{t_c}$ \\
 $\prob.\quad \return\ \extractKey(pms, \NONC, \cid, m, 40, \init)$ \\
\underline{$\getKey_s(\shareInfo, m, \init)$:} \\
 \setcounter{nombre}{0}%
 $\prob.\quad (\NONC, \cid ,T_c, t_s) = \shareInfo$ \\
 $\prob.\quad pms = T_c^{t_s}$ \\
 $\prob.\quad \return\ \extractKey(pms, \NONC, \cid, m, 40, \init)$ \\
\underline{$\extractKey(pms, \NONC, \cid, m, \ell, \init)$:}\\
 \setcounter{nombre}{0}%
 $\prob.\quad ms = \PRF(pms, \NONC)$ \\
 $\prob.\quad \text{If } \init = 1$ \\
 $\prob.\quad \quad str = \text{ QUIC key expansion }$ \\
 $\prob.\quad \text{Else }$ \\
 $\prob.\quad \quad str = \text{ QUIC forward secure expansion }$ \\
 $\prob.\quad \info = str \| 0x00 \| \cid \| m \| \SCFG_{pub}$ \\
 $\prob.\quad \return\ \text{the first $\ell$ octets (i.e. bytes) of T = }$ \\
 $\quad \quad \text{(T(1),T(2), ...), where for all $i \in \Nset$, $T(i) = $} $\\
 $\quad \quad \text{$\PRF(ms, T(i-1) \| \info \| 0x0i)$ and $T(0) = \epsilon$} $\\
%=====================================================
\subsubsection{Data Exchange}
%=====================================================
In this phase, the client and server exchange data
encrypted and authenticated using Length-Hiding
Authenticated Encryption $\SE$ with forward secure key $\key$.
They encrypt data using $\pak$ and decrypt data using
$\processPacket$.
\\
\noindent
\underline{$\getIV(H, \kappa, P)$:} \\
 \setcounter{nombre}{0}%
 $\prob.\quad (k_c, k_s, \iv_c, \iv_s) = \kappa$ \\
 $\prob.\quad \text{If } P \in \Client$ \\
 $\prob.\quad \quad src = c, dst = s$ \\
 $\prob.\quad \text{Else if} P \in \Server$ \\
 $\prob.\quad \quad src = s, dst = c$ \\
 $\prob.\quad (\cid, \sqn) = H$ \\
 $\prob.\quad \return\ (\iv_{dst}, \sqn)$ \\
\underline{$\pak(\kappa, \sqn, m)$:} \\
 \setcounter{nombre}{0}%
 $\prob.\quad (k_c, k_s, \iv_c, \iv_s) = \kappa$ \\
 $\prob.\quad \text{If } P \in \Client$ \\
 $\prob.\quad \quad src = c, dst = s$ \\
 $\prob.\quad \text{Else if } P \in \Server$ \\
 $\prob.\quad \quad src = s, dst = c$ \\
 $\prob.\quad \pInfo = (IP_{src}, IP_{dst}, port_{src}, port_{dst})$ \\
 $\prob.\quad H = (\cid, \sqn)$ \\
 $\prob.\quad \iv = \getIV(H, \kappa)$ \\
 $\prob.\quad \return\ (\pInfo, \SE.\Enc(k_{dst}, \iv, H, m) )$ \\
\underline{$\processPacket(\kappa, p_1,...,p_v)$:} \\
 \setcounter{nombre}{0}%
 $\prob.\quad (k_c, k_s, \iv_c, \iv_s) = \kappa$ \\
 $\prob.\quad \text{If } P \in \Client$ \\
 $\prob.\quad \quad src = c, dst = s$ \\
 $\prob.\quad \text{Else if} P \in \Server$ \\
 $\prob.\quad \quad src = s, dst = c$ \\
 $\prob.\quad \text{for each } \gamma \in [v]$ \\
 $\prob.\quad \quad (H_{\gamma}, c_{\gamma}) = p_{\gamma}$ \\
 $\prob.\quad \quad \iv_{\gamma} = \getIV(H_{\gamma}, \kappa)$ \\
 $\prob.\quad \quad m_{\gamma} = \SE.\Dec(k_{src}, \iv_{\gamma}, H_{\gamma}, c_{\gamma})$ \\

%=====================================================
\subsection{0-RTT Connection Establishment} \label{sec:quic_prop_0rtt}
%=====================================================

The abstract model of our proposed scheme is in
Fig.~\ref{fig:quic_prop_0rtt}.

\begin{figure*}[!htp]
 \begin{center}

\begin{enumerate}
 \item{Initial Key Agreement} \\
 \fbox{
  \begin{minipage}[t]{0.38\textwidth}
  \centering
   \begin{tabular}{c}
    $m_5 = \CHLO(\STK, \SCFG_{pub}, k_{MAC})$ \\
    $\shareInfo = (\NONC, \cid, T_s, t_c^{\ast})$ \\
    $\Lambda = \preaccept$ \\
    $\ik = \getKey_c(\shareInfo, m_5, 1)$ \\
   \end{tabular}
  \end{minipage}%
 }
 \begin{minipage}[t]{0.13\textwidth}
  \centering
  \begin{tabular}{c}
   $\xrightarrow{m_5}$ \\
   $ $ \\
  \end{tabular}
 \end{minipage}%
 \fbox{
  \begin{minipage}[t]{0.38\textwidth}
   \centering
   \begin{tabular}{c}
    $\checkQuery(m_5, k_{\STK}, IP_c)$ \\
    $\shareInfo = (\NONC, \cid, T_c^{\ast}, t_s)$ \\
    $\Lambda = \preaccept$ \\
    $\ik = \getKey_s(\shareInfo, m_5, 1)$ \\
   \end{tabular}
  \end{minipage}%
 }
 \item{Initial Data Exchange} \\
 \fbox{
  \begin{minipage}[t]{0.38\textwidth}
  \centering
   \begin{tabular}{c}
    $\text{for each } \alpha \in {\MsgCntC{0}+1,...,\MsgCntC{1}}$ \\
    $\sqn_c = \alpha + 2$ \\
    $m_{6}^{\alpha} = \pak(ik, sqn_c, M_c^{\alpha})$ \\
    $ $ \\
    $m_{6} = (m_{6}^{\MsgCntC{0}+1},...,m_{6}^{\MsgCntC{1}})$ \\
    $\processPacket(ik, m_{7})$ \\
   \end{tabular}
  \end{minipage}%
 }
 \begin{minipage}[t]{0.13\textwidth}
  \centering
  \begin{tabular}{c}
   $ $ \\
   $ $ \\
   $ $ \\
   $ $ \\
   $\xrightarrow{m_{6}}$ \\
   $\xleftarrow{m_{7}}$ \\
  \end{tabular}
 \end{minipage}%
 \fbox{
  \begin{minipage}[t]{0.38\textwidth}
   \centering
   \begin{tabular}{c}
    $\text{for each } \beta \in {\MsgCntS{0}+1,...,\MsgCntS{1}}$ \\
    $\sqn_s = \beta + 2$ \\
    $m_{7}^{\beta} = \pak(ik, \sqn_s, M_s^{\beta})$ \\
    $ $ \\
    $m_{7} = (m_{7}^{\MsgCntS{0}+1},...,m_{7}^{\MsgCntS{1}})$ \\
    $\processPacket(ik, m_{6})$ \\
   \end{tabular}
  \end{minipage}%
 }
 \item{Key Agreement} \\
 \fbox{
  \begin{minipage}[t]{0.38\textwidth}
  \centering
   \begin{tabular}{c}
    $ $ \\
    $T_s^{\prime} = \receiveSHLO(m_8)$ \\
    $\shareInfo = (\NONC, \cid, T_s^{\prime}, t_c^{\ast})$ \\
    $m = m_5 \| m_8$ \\
    $\Lambda = \accept$ \\
    $k = \getKey_c(\shareInfo, m, 0)$ \\
   \end{tabular}
  \end{minipage}%
 }
 \begin{minipage}[t]{0.13\textwidth}
  \centering
  \begin{tabular}{c}
   $ $ \\
   $\xleftarrow{m_{8}}$ \\
   $ $ \\
  \end{tabular}
 \end{minipage}%
 \fbox{
  \begin{minipage}[t]{0.38\textwidth}
   \centering
   \begin{tabular}{c}
    $ \sqn_s = \MsgCntS{1} + 3$ \\
    $m_{8} = \SHLO(m_5, \ik, \sqn_s)$ \\
    $\shareInfo = (\NONC, \cid, T_c^{\ast}, t_s^{\prime})$ \\
    $m = m_5 \| m_8$ \\
    $\Lambda = \accept$ \\
    $k = \getKey_s(\shareInfo, m, 0)$ \\
   \end{tabular}
  \end{minipage}%
 }
 \item{Data Exchange} \\
 \fbox{
  \begin{minipage}[t]{0.38\textwidth}
  \centering
   \begin{tabular}{c}
    $\text{for each } \alpha \in {\MsgCntC{1}+1,...,\MsgCntC{2}}$ \\
    $\sqn_c = \alpha + 3$ \\
    $m_{9}^{\alpha} = \pak(k, sqn_c, M_c^{\alpha})$ \\
    $ $ \\
    $m_{9} = (m_{9}^{\MsgCntC{1}+1},...,m_{9}^{\MsgCntC{2}})$ \\
    $\processPacket(k, m_{10})$ \\
   \end{tabular}
  \end{minipage}%
 }
 \begin{minipage}[t]{0.13\textwidth}
  \centering
  \begin{tabular}{c}
   $ $ \\
   $ $ \\
   $ $ \\
   $ $ \\
   $\xrightarrow{m_{9}}$ \\
   $\xleftarrow{m_{10}}$ \\
  \end{tabular}
 \end{minipage}%
 \fbox{
  \begin{minipage}[t]{0.38\textwidth}
   \centering
   \begin{tabular}{c}
    $\text{for each } \beta \in {\MsgCntS{1}+1,...,\MsgCntS{2}}$ \\
    $\sqn_s = \beta + 3$ \\
    $m_{10}^{\beta} = \pak(k, \sqn_s, M_s^{\beta})$ \\
    $ $ \\
    $m_{10} = (m_{10}^{\MsgCntS{1}+1},...,m_{10}^{\MsgCntS{2}})$ \\
    $\processPacket(ik, m_{9})$ \\
   \end{tabular}
  \end{minipage}%
 }
\end{enumerate}
 \caption{Abstract model of 0-RTT our proposed scheme}\label{fig:quic_prop_0rtt}
 \end{center}
\end{figure*}

We define four phases of our proposed scheme handshake in 0-RTT:
(1) \textbf{Initial Key Agreement},
(2) \textbf{Initial Data Exchange},
(3) \textbf{Key Agreement}, and
(4) \textbf{Data Exchange}.
The flow of (4) are the same as the 1-RTT
handshake.

%=====================================================
\subsubsection{Initial Key Agreement}
%=====================================================
In this phase, the client sends an client with source
address token $\STK$.
\\
\noindent
\underline{$\CHLO(\STK, \SCFG_{pub}, k_{MAC})$:} \\
 \setcounter{nombre}{0}%
 $\prob.\quad \cid \xleftarrow{\$} \{0,1\}^{64}$ \\
 $\prob.\quad r \xleftarrow{\$} \{0,1\}^{160}$ \\
 $\prob.\quad \NONC = currentTime \| r$ \\
 $\prob.\quad t_c^{\ast} \xleftarrow{\$} \Zset_{q}^{\ast}$ \\
 $\prob.\quad T_c^{\ast} = g^{t_c^{\ast}}$ \\
 $\prob.\quad \pInfo = (IP_c, IP_s, port_c, port_s)$ \\
 $\prob.\quad \doc = T_c^{\ast} \| \NONC \| \cid \| \STK \| \SCID$ \\
 $\prob.\quad \mac = \PRF(k_{MAC}, \doc) $ \\
 $\prob.\quad \return\ (\pInfo, \cid, \STK, \SCID, \NONC, T_c^{\ast}, \mac)$ \\
\\
\underline{$\checkQuery(m, k_{\STK}, IP_c)$:} \\
 \setcounter{nombre}{0}%
 $\prob.\quad (\pInfo, \cid, \STK, \SCID, \NONC, T_c^{\ast}, \mac) = m$ \\
 $\prob.\quad (\iv_{\STK}, c) = \STK$ \\
 $\prob.\quad IP_c^{\prime} \| currentTime \| k_{\MAC} = \SE.\Dec(k_{\STK}, \iv_{\STK}, c)$ \\
 $\prob.\quad \doc = T_c^{\ast} \| \NONC \| \cid \| \STK \| \SCID$ \\
 $\prob.\quad \text{If } \PRF(k_{MAC}, \doc) \neq \mac$ \\
 $\prob.\quad \quad \Lambda = \text{'reject' and abort}$ \\
 $\prob.\quad (time_{\NONC}, r) = \NONC$ \\
 $\prob.\quad \text{If } (IP_c^{\prime}, currentTime) = \perp$, or \\
 $\prob.\quad \quad IP_c^{\prime} \neq IP_c$, or $time_{\STK} \leq time_{allowed}$\\
 $\prob.\quad \quad r \in \strike$, or $time_{\NONC} \not\in \strike_{rng}$ \\
 $\prob.\quad \quad \quad \Lambda = \text{'reject' and abort}$ \\
\\
%=====================================================
\subsubsection{Initial Data Exchange}
%=====================================================
In this phase, the client and server exchange data
encrypted and authenticated using Length-Hiding
Authenticated Encryption $\SE$ with initial key $\ik$.
They encrypt data using $\pak$ and decrypt data using
$\processPacket$ defined previously.
\input{05_prop_02_03_key_agr}

%=====================================================
\subsection{Security of our proposed scheme} \label{sec:quic_proof}
%=====================================================

In this section we prove the security of our proposed scheme.

\begin{theorem} \label{theorem:proposed_scheme}
 Let $\mu$ be the output length of $\PRF$, let $\lambda$ be
 the length of $\SCID$, let $\nu$ be the length of mac, let $\nclient$ be the number of
 clients, let $\nserver$ be the number of servers, let
 $\noracle$ be the number of oracles of each parties, and
 let $n_{\ell}$ be the maximum number of 0-RTT connection. Assume
 that the $\PRF$ is $(t, \epsilon_{\prf})$-pseudo-random
 function family, the signature scheme
 $\SIG$ is $(t, \epsilon_{\sig})$-secure against existentially
 unforgeable under adaptive chosen-message attacks, the DDH
 problem on $G$ is $(t, \epsilon_{\ddh})$-hard, the hash
 function family $\mathcal{H}$ is
 $(t,\epsilon_{H})$-collision-resistant (CR), the symmetric
 authenticated encryption scheme $\SE$ is
 $(t, \epsilon_{\LHAE})$-secure.
 Then for all PPT adversaries, our proposed scheme is RSACCE secure.
\end{theorem}

We prove Theorem~\ref{theorem:proposed_scheme} by proving three lemmas.
Lemma~\ref{lemma:proposed_scheme_rsacce-sa} is about \textit{server authentication}.
In this proof, finally we randomize $\ik$ in the target server oracle.
$\ik$ is used to encrypt the server's ephemeral DH public
key. The adversary has to generate correct $\ik$ to make the client accept
the forged query. If the client accepts with the forged query, the adversary
break the server authentication.
Lemma~\ref{lemma:proposed_scheme_rsacce-cc} is about \textit{channel confidentiality}.
In this proof, we need to randomize not only $\ik$ and $\key$ of the target oracles
but also $\ik$ and $\key$ of oracles which execute 0-RTT request before the target
oracles. We repeat the game until randomizing all $\ik$ and $\key$ of the oracles.
Lemma~\ref{lemma:proposed_scheme_rsacce-cc} is about \textit{channel binding}.
This proof is similar to the proof of channel confidentiality. In this proof, we need
to randomize $\ik$ of the target oracles and $\ik$ and $\key$ of oracles which execute
0-RTT request before the target oracle.

\begin{lemma} \label{lemma:proposed_scheme_rsacce-sa}
 $\Adv^{\rsaccesa}_{P}(A)$ is at most
 \begin{eqnarray}
  \Adv^{\rsaccesa}_{P}(A) \leq \nclient \noracle \nserver \nresumption
  (\epsilon_{sig} + \noracle (\epsilon_{\ddh} + 2\epsilon_{\prf} + \epsilon_{\LHAE}))
 \end{eqnarray}
\end{lemma}
%
\begin{proof}
 The proof proceeds in a sequence of games. \vspace{10pt}\\
 {\bfseries Game 0.} This game equals the \textit{server authentication} experiment in Def.~\ref{def:rsacce-sa}.\\
 \begin{equation}
  \Adv_0 = \Adv^{\rsaccesa}_{P}(A)
 \end{equation}%
%
%
 \textbf{Game 1.} In this game we add an abort rule.
 The challenger aborts, if there exists any server oracle $\pi^s_{j, 0}$
  that chooses a SCID which is not unique.
 More precisely, the game is aborted if the adversary ever makes a first $\Send$ query to a server oracle $\pi^s_{j, 0}$, and the oracle replies with SCID such that there exists some other server oracle $\pi^{s^{\prime}}_{j^{\prime}, 0}$ which has previously sampled the same SCID.

 In total less than the number of $\nserver \noracle$ SCID are sampled, each uniformly random from $\{0,1\}^{\lambda}$.
 Thus, the probability that a collision occurs is bounded by $(\nserver \noracle)^2 2^{-\lambda}$
 \begin{equation}
  |\Adv_1 - \Adv_0| \leq \frac{(\nserver \noracle)^2}{2^{\lambda}}.
 \end{equation}%
%
%
 \textbf{Game 2.} We try to guess which client oracle will be the first oracle to break \textit{server authentication}. If our guess is wrong, i.e. if there is another client oracle that breaks \textit{server authentication} before, then we abort the game.

 Technically, the game is identical to Game 1, except for the following. The challenger guesses three random indices $(c^{\ast}, i^{\ast}, \ell^{\ast}) \xleftarrow{\$} [\nclient] \times [\noracle] \times [n_{\ell}]$. If there exists a client oracle $\pi^c_{i,\ell}$ that breaks server authentication, and $(c, i, \ell) \neq c^{\ast}, i^{\ast}, \ell^{\ast})$, then the challenger aborts the game. Note that if the first oracle $\pi^c_{i,\ell}$ that breaks server authentication, then with probability $1/(\nclient \noracle n_{\ell})$ we have $(c,i,\ell) = (c^{\ast}, i^{\ast}, \ell^{\ast})$, and thus
 \begin{equation}
  \Adv_1 \leq \nclient \noracle n_{\ell} \Adv_2.
 \end{equation}%
 Note that in this game the attacker can only break the security of the protocol, if oracle $\pi^{c^{\ast}}_{i^{\ast},\ell^{\ast}}$ is the first oracle that breaks server authentication, as otherwise the game is aborted.
\vspace{10pt}\\%
%
%
 \textbf{Game 3.} Again the challenger proceeds as before, but we add an abort rule. We want to make sure that $\pi^{c^\ast}_{i^{\ast},0}$ receives as input exactly the Diffie-Hellman value $T_s$ that was selected by some other uncorrupted oracle.

 Technically, we abort and raise event $\abort_\SIG$, if oracle $\pi^{c^{\ast}}_{i^{\ast},0}$ ever receives as input a message $\cert_s$ indicating intended partner $\peer = s$ and message $(T_s,\sigma_s,SCID)$ such taht $\sigma_s$ is a valid signature over $T_s\|SCID$, however there exists no oracle $\pi^s_{j,0}$ which has previously output $\sigma_s$. Clearly we have
 \begin{equation}
  |\Adv_3 - \Adv_2| = \Pr[\abort_{\SIG}].
 \end{equation}%

 Note that the experiment is aborted, if $\pi^{c^{\ast}}_{i^{\ast},0}$ satisfies server authentication, due to Game 2. This means that server $\Server_s$ must be $\tau_s$-corrupted with $\tau_s = \infty$ (i.e. not corrupted) when $\pi^{c^{\ast}}_{i^{\ast},0}$ accepts (as otherwise $\pi^{c^{\ast}}_{i^{\ast},0}$ satisfies server authentication). To show that $\Pr[\abort_{\SIG}] \leq \ell \epsilon_{\SIG}$, we construct a signature forger as follows. The forger receives as input a public key $pk^{\ast}$ and simulates the challenger for $\mathcal{A}$. It guesses an index $\phi \xleftarrow{\$}[\nserver]$, sets $pk_{\phi} = pk^{\ast}$, and generates all long-term public/secret keys as before. Then it proceeds as the challenger in Game 3, except that it uses its chosen message oracle to generate a signature under $pk_{\phi}$ when necessary.

 If $\phi = s$, which happens with probability $1/\nserver$, then the forger can use the signature received by $\pi^{c^{\ast}}_{i^{\ast},\ell^{\ast}}$ to break the EUF-CMA security of the signature scheme with success probability $\epsilon_{\SIG}$, so $\Pr[\abort_{\SIG}]/\ell \leq \epsilon_{\SIG}$. Therefore if $\Pr[\abort_{\SIG}]$ is not negligible, then $\epsilon_{\SIG}$ is not negligible as well and we have
 \begin{equation}
  |\Adv_3 - \Adv_2| = \nserver \epsilon_{\SIG}.
 \end{equation}%

 Note that in Game 3 oracle $\pi^{c^{\ast}}_{i^{\ast},0}$ receives as input a Diffie-Hellman value $T_s$ such that $T_s$ was chosen by another oracle, but not by the attacker. Note also that there is unique oracle that issued a signature $\sigma_s$ containing SCID.
\vspace{10pt}\\%
%
%
 \textbf{Game 4.} In this game we want to make sure that we know which oracle $\pi^s_{j,0}$ will issue the signature $\sigma_s$ that $\cOracleAstFull$ receives. Note that this signature includes SCID which is unique due to Game 1. Therefore the challenger in this game proceeds as before, however additionally guesses two indices $(s^{\ast}, j^{\ast}) \xleftarrow{\$} [\nserver] \times [\noracle]$.

 We know that there must exists at least one oracle that outputs $\sigma_s$ such that $\sigma_s$ is forwarded to $\cOracleAstFull$, due to Game 3. Thus we have
 \begin{equation}
  \Adv_3 \leq \nserver \noracle \Adv_4
 \end{equation}%
 Note that in this game we know exactly that oracle $\sOracleAstFull$ chooses the Diffie-Hellman share $T_s$ that $\cOracleAstFull$ uses to compute its premaster secret.
 \vspace{10pt}\\
%
%
 \textbf{Game 5.} Let $T_{\cIndexAstRes} = g^u$ denote the Diffie-Hellman share chosen by $\cOracleAstRes$, let $T_{\sIndexAstRes} = g^v$ denote the share chosen by its partner $\sOracleAstRes$, and let $k_{\cIndexAstRes}$ is the key computed by $\cOracleAstRes$. Thus, both oracles compute the premaster secret as $pms^{\prime} = g^{uv}$.

 The challenger in this game proceeds as before, however replaces the premaster secret $pms^{\prime}$ of $\cOracleAstRes$ and $\sOracleAstRes$ with a random group element $\widetilde{pms^{\prime}} = g^w$, $w \xleftarrow{\$} \Zset_p$. Note that both $g^u$ and $g^v$ are chosen by oracles $\cOracleAstRes$ and $\sOracleAstRes$, respectively, as otherwise $\cOracleAstRes$ would not have a matching conversation to $\sOracleAstRes$ and the game would be aborted.

 Distinguish Game 5 from Game 4 implies an algorithm solving the decisional Diffie-Hellman problem, thus
 \begin{equation}
  |\Adv_{5} - \Adv_{4}| \leq \epsilon_{\ddh}
 \end{equation}%
%
%
 \textbf{Game 6.} In this game we replace the value $ms^{\prime} = \PRF(\widetilde{pms^{\prime}}, \NONC)$ with a random value $\widetilde{ms}$.

 Distinguishing Game 6 from Game 5 implies an algorithm breaking the security of the pseudo random function $\PRF$, thus
 \begin{equation}
  |\Adv_{6} - \Adv_{5}| \leq \epsilon_{\prf}
 \end{equation}%
%
%
 \textbf{Game 7.} In this game we replace the function $\PRF(\widetilde{ms^{\prime}},\cdot)$ with a random function. If $\sOracleAstRes$ uses the same master secret $\widetilde{ms^{\prime}}$ as $\cOracleAstRes$, then the function $\PRF(\widetilde{ms^{\prime}},\cdot)$ used by $\sOracleAstRes$ is replaced as well. Of course the same random function is used for both oracles sharing the same $\widetilde{ms^{\prime}}$.

 Distinguishing Game 7 from Game 6 implies an algorithm breaking the security of the pseudo random function $\PRF$, thus
 \begin{equation}
  |\Adv_7 - \Adv_6| \leq \epsilon_{\prf}.
 \end{equation}%
%
%
 \textbf{Game 8.} Now we use that the key $k$ used by $\cOracleAstRes$ and $\sOracleAstRes$ in the stateful symmetric encryption scheme uniformly at random and independent of all QUIC handshake messages.

 The adversary have to make ciphertext $c$ such that $\SE$.$\Dec$ ( k , c , H , $st_d$ ) $\neq \perp$ without knowing the key $k$. It implies an algorithm breaking the sLHAE security of the symmetric encryption scheme, we have
 \begin{equation}
  \Adv_8 \leq 1/2 + \epsilon_{\sLHAE}.
 \end{equation}%
\end{proof}

\begin{lemma} \label{lemma:proposed_scheme_rsacce-cc}
 $\Adv^{\rsaccecc}_{P}(A)$ is at most
 \begin{eqnarray}
  \Adv^{\rsaccecc}_{P}(A) \leq \Adv^{\rsaccesa}_{P}(A) + \nonumber \\
  (\nclient + \noracle) \nserver \nresumption \{ \nresumption(\epsilon_{\ddh} + 2\epsilon_{\prf} + \epsilon_{\LHAE})
  + \epsilon_{\LHAE} \}
 \end{eqnarray}
\end{lemma}
%
\begin{proof}
 The proof proceeds in a sequence of games. \vspace{10pt}\\
 \textbf{Game 0.} This game equals the \textit{channel confidetility} security experiment.
 \begin{equation}
  \Adv_0 = \Adv^{\rsaccecc}_{P}(A)
 \end{equation}%
%
%
 \textbf{Game 1.} The challenger in this game proceeds as before, however it aborts and chooses $b^{\prime}$ uniformly random, if there exists any oracle that breaks server authentication. Thus we have
 \begin{equation}
  |\Adv_1 - \Adv_0| = \Adv^{\rsaccesa}_{P}(A).
 \end{equation}%
 Note that if there exists the oracle which breaks \textit{server authentication}, the adversary easily breaks \textit{channel confidentiality}. If the target oracle breaks \textit{server authentication}, the adversary or unrelated oracle (i.e. it is not an intended partner of the oracle) can establish the session with the oracle. The adversary can issue $\Reveal$-query to unrelated oracle which is out of the restriction of \textit{channel confidentiality}. Then the adversary can know the session key of the target oracle.
\vspace{10pt}\\%
%
%
 \textbf{Game 2.} The challenger in this game proceeds as before, however in addition guesses indices $(p^{\ast}, i^{\ast}, \ell^{\ast}) \xleftarrow{\$} [n_s + n_c] \times [n_o] \times [n_{\ell}]$. It aborts and chooses $b^{\prime}$ at random, if the attacker issues a $\Encrypt$-query with $(p,i,\ell) \neq (p^{\ast}, i^{\ast}, \ell^{\ast})$. With probability $1/((n_s+n_c)n_o n_{\ell})$ we have $(p,i,\ell) = (p^{\ast}, i^{\ast}, \ell^{\ast})$, and thus
 \begin{equation}
  \Adv_1 = (n_s + n_c) n_o n_{\ell}\Adv_2.
 \end{equation}%
 Note that in Game 2 we know that $\mathcal{A}$ will issue a $\Encrypt$-query to oracle $\pOracleAstRes$. Note also that $\pOracleAstRes$ has a unique partner due to Game 1. In the sequel we denote with $\qOracleAstRes$ the unique oracle such that $\pOracleAstRes$ has a matching conversations for an initial key or a final key with $\qOracleAstRes$, and say that $\qOracleAstRes$ is the intended partner of $\pOracleAstRes$.
\vspace{10pt}\\%
%
%
 \textbf{Game 3 + $4\ell$.} We repeat the game until randomizing the session between $\pOracleAstRes$ and $\qOracleAstRes$. Initially $\ell = 0$. The reason of repeating is that the adversary always return correct $b^{\prime}$ if the adversary can obtain $k_{mac}$ of $\pOracleAstEll$ or $\qOracleAstEll$. We need to prevent the adversary obtaining all $k_{mac}$ in $ 0 \leq \ell \leq \ell^{\ast}$.
 Let $T_{\pindexell} = g^u$ denote the DH public key chosen by $\pOracleAstEll$, let $T_{\qindexell} = g^v$ denote the share chosen by its partner $\qOracleAstEll$, and let $\ik_{\pindexell}$ and $k_{\pindexell}$ are the key computed by $\pOracleAstEll$, let $pms_{\ik}$ and $ms_{\ik}$ denote a premaster secret and master secret for an initial key, let $pms_{k}$ and $ms_{k}$ denote a premaster secret and master secret for session key. Thus, both oracles compute the premaster secret as $pms = g^{uv}$.

 The challenger in this game proceeds as before, however replaces the premaster secret $pms_{\ik}$ of $\pOracleAstEll$ and $\qOracleAstEll$ with a random group element $\widetilde{pms_{\ik}} = g^w$, $w \xleftarrow{\$} \Zset_p$. Note that both $g^u$ and $g^v$ are chosen by oracles $\pOracleAstEll$ and $\qOracleAstEll$, respectively, as otherwise $\pOracleAstEll$ would not have a matching conversations for an initial key with $\qOracleAstEll$ and the game would be aborted.

 Suppose that there exists an algorithm $\mathcal{A}$ distinguishes Game 3 + $4\ell$ from Game 2 + $4\ell$. Then we can construct an algorithm $\mathcal{B}$ solving the DDH problem as follows. $\mathcal{B}$ receives as input $(g,g^u,g^v,g^w)$. The challenger defines $T_{\pindexell} := g^u$ and $T_{\qindexell} := g^v$, and the premaster secret of $\pOracleAstEll$ and $\qOracleAstEll$ equal to $pms_{\ik} := g^w$. Note that $\mathcal{B}$ can simulate all messages exchanged between $\pOracleAstEll$ and $\qOracleAstEll$ properly. Since all other oracles are not modified, $\mathcal{B}$ can simulate these oracles properly as well.

 If $w=uv$, then the view of $\mathcal{A}$ when interacting with $\mathcal{B}$ is identical to Game 2 + $4\ell$, while if $w \xleftarrow{\$}\Zset_p$ then it is identical to Game 3 + $4\ell$. Thus, the DDH assumption implies that
 \begin{equation}
  |\Adv_{3 + 4\ell} - \Adv_{2 + 4\ell}| \leq \epsilon_{\ddh}
 \end{equation}%
%
%
 \textbf{Game 4 + $4\ell$.} In Game 4 + 4$\ell$ we make use of the fact that the premaster secret $\widetilde{pms_{\ik}}$ of $\pOracleAstEll$ and $\qOracleAstEll$ is chosen uniformly random. We thus replace the value $ms_{\ik} = \PRF(\widetilde{pms_{\ik}}, \NONC)$ with a random value $\widetilde{ms_{\ik}}$.

 Distinguish Game 4 + 4$\ell$ from Game 3 + 4$\ell$ implies an algorithm breaking the security of the pseudo random function $\PRF$, thus
 \begin{equation}
  |\Adv_{4 + 4\ell} - \Adv_{3 + 4\ell}| \leq \epsilon_{\prf}
 \end{equation}%
%
%
 \textbf{Game 5 + $4\ell$.} In this game we replace the function $\PRF(\widetilde{ms_{\ik}}, \cdot)$ used by $\pOracleAstEll$ and $\qOracleAstEll$ with a random function $F_{\widetilde{ms_{\ik}}}$. Of course the same random function is used for both oracles $\pOracleAstEll$ and $\qOracleAstEll$. Distinguishing Game 5 + $4\ell$ from Game 4 + $4\ell$ again implies an algorithm breaking the security of the pseudo random function $\PRF$.
 \begin{equation}
  |\Adv_{5 + 4\ell} - \Adv_{4 + 4\ell}| \leq \epsilon_{\prf}
 \end{equation}%

 Note that the adversary cannot obtain the DH public key and $k_{mac}$ because the adversary obtain nothing from the transcription due to randomization of the initial key $\ik$. These changes prevent trivially attack. If the adversary obtain the MAC key $k_{mac}$, the adversary can hijack the session following way: the adversary generate $T_c$, $\NONC$ by himself and calculate MAC generated by MAC key $k_{mac}$ and send these value with $\SCID$. The server cannot reject this query made by the adversary because MAC is valid. The adversary can share the secret key with the server and always return correct $b^{\prime}$.
\vspace{10pt}\\%
%
%
 \textbf{Game 6 + $4\ell$.} Now we use that the key $\ik_{\pindexell}$ and $\ik_{\qindexell}$ in the symmetric encryption scheme uniformly at random and independent of all QUIC handshake messages. In this we replace the value $k_{mac}$ with another random value $\widetilde{k_{mac}}$.

 Suppose that there exists an algorithm $\mathcal{A}$ distinguishes Game 6 + 4 $\ell$ from Game 5 + 4 $\ell$. Then we can construct an algorithm $\mathcal{B}$ breaking $\LHAE$ secure. By assumption, the simulator $\mathcal{B}$ is given access to an encryption oracle $\Encrypt$ and a decryption oracle $\Decrypt$.

 Since by assumption any attacker has advantage at most $\epsilon_{\LHAE}$ in breaking the $\LHAE$ security of the symmetric encryption scheme, we have
 \begin{equation}
  |\Adv_{6 + 4\ell} - \Adv_{5 + 4\ell}| \leq \epsilon_{\LHAE}.
 \end{equation}%
 After this game, we add $\ell = \ell + 1$. If $\ell \leq \ell^{\ast}$ we repeat the game.
\vspace{10pt}\\%
%
%
 For next step, there are two cases. The first case is that the adversary issue $\Encrypt$-query $\pOracleAstRes$ or $\qOracleAstRes$ whose state $\Lambda = \preaccept$. The second case is that the adversary issue $\Encrypt$-query $\pOracleAstRes$ or $\qOracleAstRes$ whose state $\Lambda = \accept$.

 We define the first case as Game$_{a}$ and the second case as Game$_{b}$
\vspace{10pt}\\%
%
%
 \textbf{Game$_a$ 7 + 4$\ell^{\ast}$.} Now we use that the key $\ik_{\pindexell}$ and $\ik_{\qindexell}$ which is independent of all QUIC handshake messages.

 In this game we construct a simulator $\mathcal{B}$ that uses a RSACCE attacker $\mathcal{A}$ to break the security of the underlying $\LHAE$ secure symmetric encryption scheme. By assumption, the simulator $\mathcal{B}$ is given access to an encryption oracle $\Encrypt$ and a decryption oracle $\Decrypt$. $\mathcal{B}$ embeds that $\LHAE$ experiment by simply forwarding all $\Encrypt(\pOracleAstRes,\cdot)$ queries to $\Encrypt$, and all $\Decrypt(\qOracleAstRes,\cdot)$ queries to $\Decrypt$. Otherwise it proceeds as the challenger in Game 6.

 Observe that the values generated in this game are exactly distributed as in the previous game. We thus have
 \begin{equation}
  \Adv_{7 + 4\ell^{\ast}} = \Adv_{6 + 4\ell^{\ast}}
 \end{equation}%
 If $\mathcal{A}$ outputs a quad $\pindexout$, then $\mathcal{B}$ forwards $b^{\prime}$ to the $\LHAE$ challenger. Otherwise it outputs a random bit. Since the simulator essentially relays all messages it is easy to see that an attacker $\mathcal{A}$ having advantage $\epsilon^{\prime}$ yields an attacker $\mathcal{B}$ against the $\LHAE$ security of the encryption scheme with success probability at least $1/2 + \epsilon^{\prime}$.

 Since by assumption any attacker has advantage at most $\epsilon_{\LHAE}$ in breaking the $\LHAE$ security of the symmetric encryption scheme, we have
 \begin{equation}
  \Adv_{7 + 4\ell^{\ast}} \leq \frac{1}{2} + \epsilon_{\LHAE}.
 \end{equation}%
%
%
 \textbf{Game$_b$ 7 + 4$\ell^{\ast}$.} The challenger in this game proceeds as before, however replaces the premaster secret $pms_{k}$ of $\pOracleAstRes$ and $\qOracleAstRes$ with a random group element $\widetilde{pms_{k}} = g^w$, $w \xleftarrow{\$} \Zset_p$. Note that both $g^u$ and $g^v$ are chosen by oracles $\pOracleAstRes$ and $\qOracleAstRes$, respectively, as otherwise $\pOracleAstRes$ would not have a matching conversations for a final key with $\qOracleAstRes$ and the game would be aborted.

 Distinguish Game$_b$ 7 + 4$\ell^{\ast}$ from Game 6 + 4$\ell^{\ast}$ implies an algorithm breaking the security of the decisional Diffie-Hellman problem, thus
 \begin{equation}
  |\Adv_{7 + 4\ell^{\ast}} - \Adv_{6 + 4\ell^{\ast}}| \leq \epsilon_{\ddh}
 \end{equation}%
%
%
 \textbf{Game$_b$ 8 + 4$\ell^{\ast}$.} In this game we make use of the fact that the premaster secret $\widetilde{pms_{k}}$ of $\pOracleAstRes$ and $\qOracleAstRes$ is chosen uniformly random. We thus replace the value $ms_{k} = \PRF(\widetilde{pms_{k}}, \NONC)$ with a random value $\widetilde{ms_{k}}$.

 Distinguish Game$_b$ 8 + 4$\ell^{\ast}$ from Game$_b$ 7 + 4$\ell^{\ast}$ implies an algorithm breaking the security of the pseudo random function $\PRF$, thus
 \begin{equation}
  |\Adv_{8 + 4\ell^{\ast}} - \Adv_{7 + 4\ell^{\ast}}| \leq \epsilon_{\prf}
 \end{equation}%
%
%
 \textbf{Game$_b$ 9 + 4$\ell^{\ast}$} In this game we replace the all function $\PRF(\widetilde{ms_{k}}, \cdot)$ used by $\pOracleAstRes$ and $\qOracleAstRes$ with a random function $F_{\widetilde{ms_{k}}}$. Of course the same random function is used for both oracles $\pOracleAstRes$ and $\qOracleAstRes$. Distinguishing Game$_b$ 9 + 4$\ell^{\ast}$ from Game$_b$ 8 + 4$\ell^{\ast}$ again implies an algorithm breaking the security of the pseudo random function $\PRF$.
 \begin{equation}
  |\Adv_{9 + 4\ell^{\ast}} - \Adv_{8 + 4\ell^{\ast}}| \leq \epsilon_{\prf}
 \end{equation}%
%
%
 \textbf{Game$_b$ 10 + 4$\ell^{\ast}$.} Now we use that the key $k_{\pindexell}$ and $k_{\qindexell}$ which is independent of all QUIC handshake messages.

 In this game we construct a simulator $\mathcal{B}$ that uses a RSACCE attacker $\mathcal{A}$ to break the security of the underlying $\sLHAE$ secure symmetric encryption scheme. By assumption, the simulator $\mathcal{B}$ is given access to an encryption oracle $\Encrypt$ and a decryption oracle $\Decrypt$. $\mathcal{B}$ embeds that $\sLHAE$ experiment by simply forwarding all $\Encrypt(\pOracleAstRes,\cdot)$ queries to $\Encrypt$, and all $\Decrypt(\qOracleAstRes,\cdot)$ queries to $\Decrypt$.

 Observe that the values generated in this game are exactly distributed as in the previous game. We thus have
 \begin{equation}
  \Adv_{10 + 4\ell^{\ast}} = \Adv_{9 + 4\ell^{\ast}}
 \end{equation}%
 If $\mathcal{A}$ outputs a quad $\pindexout$, then $\mathcal{B}$ forwards $b^{\prime}$ to the $\sLHAE$ challenger. Otherwise it outputs a random bit. Since the simulator essentially relays all messages it is easy to see that an attacker $\mathcal{A}$ having advantage $\epsilon^{\prime}$ yields an attacker $\mathcal{B}$ against the $\LHAE$ security of the encryption scheme with success probability at least $1/2 + \epsilon^{\prime}$.

 Since by assumption any attacker has advantage at most $\epsilon_{\LHAE}$ in breaking the $\LHAE$ security of the symmetric encryption scheme, we have
 \begin{equation}
  \Adv_{10 + 4\ell^{\ast}} \leq \frac{1}{2} + \epsilon_{\LHAE}.
 \end{equation}%
\end{proof}

\begin{lemma} \label{lemma:proposed_scheme_rsacce-cb}
 $\Adv^{\rsaccecb}_{P}(A)$ is at most
 \begin{eqnarray}
  \Adv^{\rsaccecb}_{P}(A) \leq \nonumber \\
  \nserver \noracle \nresumption \{ \nresumption(\epsilon_{\ddh} + 2\epsilon_{\prf} + \epsilon_{\LHAE}) + \frac{1}{2^{\nu}} \}
 \end{eqnarray}
\end{lemma}
%
Due to spaces, we omit the detailed proofs of the Lemmas.
This proof is similar to the proof of channel confidentiality.

%=====================================================
\subsection{Security concerns of our proposed scheme} \label{sec:prop_sec_concerns}
%=====================================================

Our proposed scheme prevents the attacks which the original QUIC has.
However, there are another security concerns in our proposed scheme.
This security concerns are the same as the problem DTLS\cite{DTLS12}
has and these are: (1) an adversary can consume excessive resources
on the server by transmitting a series of handshake initiation requests,
causing the server to allocate state and potentially to perform expensive
cryptographic operations, (2) an adversary can use the server as an
amplifier by sending connection initiation message with a forged source
of the victim. The server then sends its next message to the victim
machine.
About the first issue, in a first client request, the server cannot validate
client's IP address since there are no interactions.
For this property, the adversary who cannot see a
transcript and forge it can increase the server loads sending request
with the spoofed IP address.
The second issue is called Distributed Reflection Denial of Service
(DRDoS) and Christian~\cite{Ross14} proposed some countermeasures against it.
We can apply the countermeasure of DTLS to our proposed scheme.
However, this countermeasure needs additional interactions and eliminates
the advantage of our propose scheme which reduce interactions.

We suggest another solution for the first issue that add a restriction to our
proposed scheme.
If the server has more specific number connections, the server
responds to a first client's query with a rejection (REJ) which
is the same as a rejection in the original QUIC and
fall back to a protocol similar to the original QUIC that we call
modified QUIC.
The handshake of the modified QUIC is mixed the original QUIC and our
proposed scheme.
The handshake of modified QUIC in 0-RTT is the same as our proposed
scheme.
The difference in a 1-RTT connection between the original QUIC and modified one
is that the way to make $\STK$. In modified QUIC, $\STK$ is generated
by the same way as our proposed scheme. The client needs to send
MAC generated by $\key_{MAC}$ in 0-RTT request.
This modification prevents Source-Address Token Replay Attack
in the same way as our proposed scheme.
 %=====================================================
\section{Conclusion} \label{sec:conclusion}
%=====================================================

We propose a new security model, \textit{Resumable} SACCE (RSACCE) security and strong RSACCE.
This model consider forward secrecy among all independent sessions.
For resumption to be effective, we compromise but still require
some level of forward secrecy among related resumption sessions.
The original QUIC meets RSACCE security, however it does not meet strong RSACCE security.
Because the adversary can spoof the query in resumption and the server accepts this query.
QUIC has an optional mechanism, called client encrypted tag-value (CETV), which is disabled in the default setting.
If it is enabled, QUIC with CETV satisfies strong RSACCE.
Finally, we presented a more efficient protocol than QUIC, which satisfies strong RSACCE security.

% \newif\ifORIGINAL
%  \begin{figure*}[htb]
\begin{center}

1-RTT connection establishment \vspace{10pt}\\

\fbox{
\begin{minipage}[t]{0.39\textwidth}
\begin{tabular}[c]{l}
 $\quad Client$ \\
 $ $ \\
 $1.\ \ \cid \xleftarrow{\$} \{0,1\}^{64} $ \\
 $2.\ \ m_1 = \cid$ \\
 $ $\\
 $1.\ \ \checkSCFG(\SCFG_{pub})$ \\
 $2.\ \ t_c \xleftarrow{\$} \Zset_{q}^{\ast} $ \\
 $3.\ \ T_c = g^{t_c} $ \\
 $4.\ \ \NONC \xleftarrow{\$} \{0,1\}^{160} $ \\
 $5.\ \ m_3 = (T_c, \NONC, \STK, \SCFG.\SCID)$ \\
 $ $ \\
 $1.\ \ \shareInfo = (\NONC, \cid, \SCFG.T_s, t_c)$ \\
 $2.\ \ m = m_1 \| m_2 \| m_3$ \\
 $3.\ \ \ik = \getKey_c(\shareInfo, m, 1)$ \\
\end{tabular}
\end{minipage}%
}
% middle
 \begin{minipage}[t]{0.13\textwidth}
  \centering
  \begin{tabular}{c}
   $ $ \\
   $ $ \\
   $ $ \\
   $\xrightarrow{m_1}$ \\
   $ $ \\
   $\xleftarrow{m_2}$ \\
   $ $ \\
   $ $ \\
   $\xrightarrow{m_3}$ \\
   $ $ \\
   $ $ \\
   $ $ \\
   $ $ \\
   $ $ \\
  \end{tabular}
 \end{minipage}%
\fbox{
\begin{minipage}[t]{0.39\textwidth}
\begin{tabular}[c]{l}
 $\quad Server$ \\
 $ $ \\
 $ $ \\
 $1.\ \ \text{choose } \SCFG $\\
 $2.\ \ \STK = \SE.\Enc(\key_{\STK}, time\|IP_c)$ \\
 $3.\ \ m_2 = (\SCFG_{pub}, \STK)$ \\
 $ $ \\
 $ $ \\
 $ $ \\
 $1.\ \ \checkQuery(\STK, k_{\STK}, \NONC, IP_c)$ \\
 $ $ \\
 $ $ \\
 $1.\ \ \shareInfo = (\NONC, \cid, T_c, \SCFG.t_s)$ \\
 $2.\ \ m = m_1 \| m_2 \| m_3$ \\
 $3.\ \ \ik = \getKey_s(\shareInfo, m, 1)$ \\
\end{tabular}
\end{minipage}%
} \vspace{10pt}

1-RTT connection establishment for forward secure key
\vspace{10pt}\\

\fbox{
\begin{minipage}[t]{0.39\textwidth}
\begin{tabular}[c]{l}
 $ $ \\
 $ $ \\
 $ $ \\
 $ $ \\
 $1.\ \ T_s^{\prime} \| \STK = \SE.\Dec(\ik, m_4)$ \\
 $ $ \\
 $ $ \\
 $1.\ \ \shareInfo = (\NONC, \cid, T_s^{\prime}, t_c)$ \\
 $2.\ \ m = m_1 \| m_2 \| m_3 \| m_4 $ \\
 $3.\ \ \key = \getKey_c(\shareInfo, m, 0)$
\end{tabular}
\end{minipage}%
}
\begin{minipage}[t]{0.13\textwidth}
\centering
\begin{tabular}[c]{l}
 $ $\\
 $ $\\
 $ $\\
 $\xleftarrow{m_4}$\\
 $ $\\
 $ $\\
 $ $\\
 $ $\\
 $ $\\
\end{tabular}
\end{minipage}%
\fbox{
\begin{minipage}[t]{0.39\textwidth}
\begin{tabular}[c]{l}
 $1.\ \ t_s^{\prime} \xleftarrow{\$} \Zset_{q}^{\ast}$ \\
 $2.\ \ T_s^{\prime} = g^{t_s^{\prime}}$ \\
 $3.\ \ \STK = \SE.\Enc(\key_{\STK}, time\|IP_c)$ \\
 $4.\ \ \plaintext = T_s^{\prime} \| \STK$ \\
 $5.\ \ m_4 = \SE.\Enc(\ik, \plaintext)$ \\
 $ $ \\
 $1.\ \ \shareInfo = (\NONC, \cid, T_c, t_s^{\prime})$ \\
 $2.\ \ m = m_1 \| m_2 \| m_3 \| m_4 $ \\
 $3.\ \ \key = \getKey_s(\shareInfo, m, 0)$
\end{tabular}
\end{minipage}%
} \vspace{10pt}

0-RTT connection establishment
\vspace{10pt}\\

\fbox{
\begin{minipage}[t]{0.39\textwidth}
\begin{tabular}[c]{l}
 $1.\ \ \cid \xleftarrow{\$} \{0,1\}^{64} $ \\
 $2.\ \ t_c \xleftarrow{\$} \Zset_{q}^{\ast} $ \\
 $3.\ \ T_c = g^{t_c} $ \\
 $4.\ \ \NONC \xleftarrow{\$} \{0,1\}^{160} $ \\
 $5.\ \ m_5 = (T_c, \NONC, \cid, \STK, \SCFG.\SCID)$ \\
 $ $ \\
 $1.\ \ \shareInfo = (\NONC, \cid, \SCFG.T_s, t_c)$ \\
 $2.\ \ m = m_5$ \\
 $3.\ \ \ik = \getKey_c(\shareInfo, m, 1)$ \\
\end{tabular}
\end{minipage}%
}
% middle
 \begin{minipage}[t]{0.13\textwidth}
  \centering
  \begin{tabular}{c}
   $ $ \\
   $ $ \\
   $ $ \\
   $ $ \\
   $ $ \\
   $\xrightarrow{m_5}$ \\
   $ $ \\
   $ $ \\
   $ $ \\
   $ $ \\
  \end{tabular}
 \end{minipage}%
\fbox{
\begin{minipage}[t]{0.39\textwidth}
\begin{tabular}[c]{l}
 $ $\\
 $ $\\
 $ $\\
 $ $\\
 $ $\\
 $1.\ \ \checkQuery(\STK, \key_{\STK}, \NONC, IP_c) $\\
 $ $\\
 $1.\ \ \shareInfo = (\NONC, \cid, T_c, \SCFG.t_s)$ \\
 $2.\ \ m = m_1 \| m_2 \| m_3$ \\
 $3.\ \ \ik = \getKey_s(\shareInfo, m, 1)$ \\
\end{tabular}
\end{minipage}%
}

\caption{Abstract model of QUIC}\label{fig:quic_tls}
\end{center}
\end{figure*}
%  \ORIGINALfalse

\newcommand{\calcFSkey}[1]{%
 $\prob.\quad \key_{\mac} \xleftarrow{\$} \{0,1\}^{\lambda}$ \\
 $\prob.\quad \makeSTKQUIC$ \\
 $\prob.\quad t_s^{\prime} \xleftarrow{\$} \Zset_{q}^{\ast}$ \\
 $\prob.\quad T_s^{\prime} = g^{t_s^{\prime}}$ \\
 $\prob.\quad \plaintext = T_s^{\prime} \| \STK \| \key_{\mac} \| \SCID$ \\
 $\prob.\quad c = \SE.\Enc(\ik, \plaintext)$ \\
\ifONERTT
 $\prob.\quad m_{2} = (\SCFG_{pub}, c, \cid)$ \\
\else
 $\prob.\quad m_{4} = (\SCFG_{pub}, c, \cid)$ \\
\fi
 $\prob.\quad pms^{\prime} = T_s^{\prime t_c}$ \\
 $\prob.\quad ms^{\prime} = \PRF(pms^{\prime}, NONC)$ \\
\ifONERTT
 $\prob.\quad \key = \PRF(ms^{\prime}, m_1 \| m_2)$ \\
\else
 $\prob.\quad \key = \PRF(ms^{\prime}, m_1 \| m_2 \| m_3 \| m_4)$ \\
\fi
 $\prob.\quad \Lambda = \accept$ \\
}

\begin{figure*}[htb]
\begin{center}

1-RTT connection establishment \vspace{10pt}\\

\fbox{
\begin{minipage}[t]{0.39\textwidth}
\begin{tabular}[c]{l}
 \setcounter{nombre}{0}%
 $\quad Client$ \\
 $ $ \\
 $\prob.\quad \cid \xleftarrow{\$} \{0,1\}^{64} $ \\
 $\prob.\quad t_c \xleftarrow{\$} \Zset_{q}^{\ast} $ \\
 $\prob.\quad T_c = g^{t_c} $ \\
 $\prob.\quad \NONC \xleftarrow{\$} \{0,1\}^{160} $ \\
 $\prob.\quad m_1 = (\NONC, \cid, T_c)$ \\
 $ $ \\
 \setcounter{nombre}{0}%%
 $\prob.\quad (\SCID, T_s, \sigma_s) = \SCFG_{pub}$ \\
 $\prob.\quad \doc = \SCID \| T_s$ \\
 $\prob.\quad \text{If } \SIG.\Vfy(pk_s, \sigma_s, \doc)$ \\
 $\prob.\quad \quad \Lambda = \text{'reject' and abort}$ \\
 $\prob.\quad pms = T_s^{t_c}$ \\
 $\prob.\quad ms = \PRF(pms, NONC)$ \\
 $\prob.\quad \ik = \PRF(ms, m_1)$ \\
 $\prob.\quad \plaintext = \SE.\Dec(\ik, c)$ \\
 $\prob.\quad pms^{\prime} = T_s^{\prime t_c}$ \\
 $\prob.\quad ms^{\prime} = \PRF(pms^{\prime}, NONC)$ \\
 $\prob.\quad \key = \PRF(ms^{\prime}, m_1 \| m_2)$ \\
 $\prob.\quad \Lambda = \accept$ \\
 $\prob.\quad \theta = (\key_{mac}, \STK, \SCFG_{pub})$ \\
\end{tabular}
\end{minipage}%
}
% middle
 \begin{minipage}[t]{0.13\textwidth}
  \centering
  \begin{tabular}{c}
   $ $ \\
   $ $ \\
   $ $ \\
   $\xrightarrow{m_1}$ \\
   $ $ \\
   $ $ \\
   $ $ \\
   $ $ \\
   $ $ \\
   $\xleftarrow{m_2}$ \\
   $ $ \\
   $ $ \\
   $ $ \\
   $ $ \\
   $ $ \\
   $ $ \\
  \end{tabular}
 \end{minipage}%
\fbox{
\begin{minipage}[t]{0.39\textwidth}
\begin{tabular}[c]{l}
 \setcounter{nombre}{0}%
 $\quad Server$ \\
 $ $ \\
 $\prob.\quad \text{choose } \SCFG = (\SCFG_{pub}, t_s) $\\
 $\prob.\quad (\SCID, T_s, \sigma_s) = \SCFG_{pub}$ \\
 $\prob.\quad pms = T_c^{t_s}$ \\
 $\prob.\quad ms = \PRF(pms, NONC)$ \\
 $\prob.\quad \ik = \PRF(ms, m_1)$ \\
 \ONERTTtrue
 \calcFSkey{5}
\end{tabular}
\end{minipage}%
} \vspace{10pt}

0-RTT connection establishment
\vspace{10pt}\\

\fbox{
\begin{minipage}[t]{0.39\textwidth}
\begin{tabular}[c]{l}
 \setcounter{nombre}{0}%
 $\prob.\quad (\key_{mac}, \STK, \SCFG_{pub}) = \theta$ \\
 $\prob.\quad \cid \xleftarrow{\$} \{0,1\}^{64} $ \\
 $\prob.\quad t_c \xleftarrow{\$} \Zset_{q}^{\ast} $ \\
 $\prob.\quad T_c = g^{t_c} $ \\
 $\prob.\quad \NONC \xleftarrow{\$} \{0,1\}^{160} $ \\
 $\prob.\quad \doc = T_c \| \NONC \| \cid \| \STK \| \SCID$ \\
 $\prob.\quad \mac = \PRF(\key_{mac}, \doc)$ \\
 $\prob.\quad m_3 = (\doc, \mac)$ \\
 $\prob.\quad pms = T_s^{t_c}$ \\
 $\prob.\quad ms = \PRF(pms, NONC)$ \\
 $\prob.\quad \key = \PRF(ms, m_1 \| m_2 \| m_3)$ \\
 $\prob.\quad \Lambda = \preaccept$ \\
\end{tabular}
\end{minipage}%
}
% middle
 \begin{minipage}[t]{0.13\textwidth}
  \centering
  \begin{tabular}{c}
   $ $ \\
   $ $ \\
   $ $ \\
   $\xrightarrow{m_3}$ \\
   $ $ \\
   $ $ \\
   $ $ \\
   $ $ \\
   $ $ \\
   $ $ \\
   $ $ \\
  \end{tabular}
 \end{minipage}%
\fbox{
\begin{minipage}[t]{0.39\textwidth}
\begin{tabular}[c]{l}
 \setcounter{nombre}{0}%
 $ $\\
 $ $\\
 $\prob.\quad \text{search $\SCFG$ with $\SCID$}$ \\
 $\prob.\quad \key_{\mac} = \SE.\Dec(\key_{\STK}, \STK) $\\
 $\prob.\quad \doc = T_c \| \NONC \| \cid \| \STK \| \SCID$ \\
 $\prob.\quad \text{If }\mac \neq \PRF(\key_{\mac}, \doc)$ \\
 $\prob.\quad \quad \Lambda = \text{'reject' and abort}$ \\
 $\prob.\quad pms = T_c^{t_s}$ \\
 $\prob.\quad ms = \PRF(pms, NONC)$ \\
 $\prob.\quad \key = \PRF(ms, m_1 \| m_2 \| m_3)$ \\
 $\prob.\quad \Lambda = \preaccept$ \\
\end{tabular}
\end{minipage}%
} \vspace{10pt}

0-RTT connection establishment for forward secure key
\vspace{10pt}\\

\fbox{
\begin{minipage}[t]{0.39\textwidth}
\begin{tabular}[c]{l}
 \setcounter{nombre}{0}%
 $ $ \\
 $ $ \\
 $ $ \\
 $ $ \\
 $\prob.\quad T_s^{\prime} \| \STK = \SE.\Dec(\ik, c)$ \\
 $\prob.\quad pms^{\prime} = T_s^{\prime t_c} $ \\
 $\prob.\quad ms^{\prime} = \PRF(pms^{\prime}, \NONC) $ \\
 $\prob.\quad \key = \PRF(ms, m_1 \| m_2 \| m_3 \| m_4)$ \\
 $\prob.\quad \Lambda = \accept$ \\
 $\prob.\quad \theta = (\key_{mac}, \STK, \SCFG_{pub})$ \\
\end{tabular}
\end{minipage}%
}
\begin{minipage}[t]{0.13\textwidth}
\centering
\begin{tabular}[c]{l}
 $ $\\
 $ $\\
 $ $\\
 $\xleftarrow{m_4}$\\
 $ $\\
 $ $\\
 $ $\\
 $ $\\
 $ $\\
\end{tabular}
\end{minipage}%
\fbox{
\begin{minipage}[t]{0.39\textwidth}
\begin{tabular}[c]{l}
 \setcounter{nombre}{0}%
 \ORIGINALfalse
 \calcFSkey{1}
\end{tabular}
\end{minipage}%
} \vspace{10pt}

\caption{Abstract model of the propesed QUIC}\label{fig:quic_tls}
\end{center}
\end{figure*}
%  \begin{figure*}[htb]
\begin{center}

1-RTT connection establishment for initial key \vspace{10pt}\\

\fbox{
\begin{minipage}[t]{0.39\textwidth}
\begin{tabular}[c]{l}
 $\quad Client$ \\
 $ $ \\
 $1.\ \ \cid \xleftarrow{\$} \{0,1\}^{64} $ \\
 $2.\ \ m_1 = \cid$ \\
 $ $\\
 $1.\ \ \checkSCFG(\SCFG_{pub})$ \\
 $2.\ \ t_c \xleftarrow{\$} \Zset_{q}^{\ast} $ \\
 $3.\ \ T_c = g^{t_c} $ \\
 $4.\ \ \NONC \xleftarrow{\$} \{0,1\}^{160} $ \\
 $5.\ \ \shareInfo = (\NONC, \cid, \SCFG.T_s, t_c)$ \\
 $6.\ \ m = m_1 \| m_2 \| m_3$ \\
 $7.\ \ \ik = \getKey_c(\shareInfo, m, 1)$ \\
 $8.\ \ \doc = T_c, \NONC, \STK, \SCFG.\SCID$ \\
 $9.\ \ \mac = \PRF(\ik, \doc)$ \\
 $10.\  m_3 = (\doc, \mac)$ \\
\end{tabular}
\end{minipage}%
}
% middle
 \begin{minipage}[t]{0.13\textwidth}
  \centering
  \begin{tabular}{c}
   $ $ \\
   $ $ \\
   $ $ \\
   $\xrightarrow{m_1}$ \\
   $ $ \\
   $\xleftarrow{m_2}$ \\
   $ $ \\
   $ $ \\
   $\xrightarrow{m_3}$ \\
   $ $ \\
   $ $ \\
   $ $ \\
   $ $ \\
   $ $ \\
  \end{tabular}
 \end{minipage}%
\fbox{
\begin{minipage}[t]{0.39\textwidth}
\begin{tabular}[c]{l}
 $\quad Server$ \\
 $ $ \\
 $ $ \\
 $1.\ \ \text{choose } \SCFG $\\
 $2.\ \ \STK = \SE.\Enc(\key_{\STK}, time\|IP_c)$ \\
 $3.\ \ m_2 = (\SCFG_{pub}, \STK)$ \\
 $ $ \\
 $ $ \\
 $ $ \\
 $1.\ \ \checkQuery(\STK, k_{\STK}, \NONC, IP_c)$ \\
 $2.\ \ \shareInfo = (\NONC, \cid, T_c, \SCFG.t_s)$ \\
 $3.\ \ m = m_1 \| m_2 \| m_3$ \\
 $4.\ \ \ik = \getKey_s(\shareInfo, m, 1)$ \\
 $5.\ \ \doc = T_c, \NONC, \STK, \SCFG.\SCID$ \\
 $6.\ \ \text{If }\mac \neq \PRF(\key_{\mac}, \doc)$ \\
 $7.\ \ \quad \Lambda = \text{'reject' and abort}$ \\
\end{tabular}
\end{minipage}%
} \vspace{10pt}

1-RTT connection establishment for last key
\vspace{10pt}\\

\fbox{
\begin{minipage}[t]{0.39\textwidth}
\begin{tabular}[c]{l}
 $ $ \\
 $ $ \\
 $ $ \\
 $ $ \\
 $1.\ \ T_s^{\prime} \| \STK = \SE.\Dec(\ik, m_4)$ \\
 $ $ \\
 $ $ \\
 $1.\ \ \shareInfo = (\NONC, \cid, T_s^{\prime}, t_c)$ \\
 $2.\ \ m = m_1 \| m_2 \| m_3 \| m_4 $ \\
 $3.\ \ \key = \getKey_c(\shareInfo, m, 0)$ \\
\end{tabular}
\end{minipage}%
}
\begin{minipage}[t]{0.13\textwidth}
\centering
\begin{tabular}[c]{l}
 $ $\\
 $ $\\
 $ $\\
 $\xleftarrow{m_4}$\\
 $ $\\
 $ $\\
 $ $\\
 $ $\\
 $ $\\
\end{tabular}
\end{minipage}%
\fbox{
\begin{minipage}[t]{0.39\textwidth}
\begin{tabular}[c]{l}
 $1.\ \ t_s^{\prime} \xleftarrow{\$} \Zset_{q}^{\ast}$ \\
 $2.\ \ T_s^{\prime} = g^{t_s^{\prime}}$ \\
 $3.\ \ \key_{\mac} \xleftarrow{\$} \{0,1\}^{\lambda}$ \\
 $4.\ \ \STK = \SE.\Enc(\key_{\STK}, time\|IP_c\|\key_{\mac})$ \\
 $5.\ \ \plaintext = T_s^{\prime} \| \STK \| \key_{\mac}$ \\
 $6.\ \ m_4 = \SE.\Enc(\ik, \plaintext)$ \\
 $ $ \\
 $1.\ \ \shareInfo = (\NONC, \cid, T_c, t_s^{\prime})$ \\
 $2.\ \ m = m_1 \| m_2 \| m_3 \| m_4 $ \\
 $3.\ \ \key = \getKey_s(\shareInfo, m, 0)$ \\
\end{tabular}
\end{minipage}%
} \vspace{10pt}

0-RTT connection establishment
\vspace{10pt}\\

\fbox{
\begin{minipage}[t]{0.39\textwidth}
\begin{tabular}[c]{l}
 $1.\ \ \cid \xleftarrow{\$} \{0,1\}^{64} $ \\
 $2.\ \ t_c \xleftarrow{\$} \Zset_{q}^{\ast} $ \\
 $3.\ \ T_c = g^{t_c} $ \\
 $4.\ \ \NONC \xleftarrow{\$} \{0,1\}^{160} $ \\
 $5.\ \ \doc = T_c \| \NONC \| \cid \| \STK \| \SCFG.\SCID$ \\
 $6.\ \ \mac = \PRF(\key_{mac}, \doc)$ \\
 $7.\ \ m_5 = (\doc, \mac)$ \\
 $ $ \\
 $1.\ \ \shareInfo = (\NONC, \cid, \SCFG.T_s, t_c)$ \\
 $2.\ \ m = m_5$ \\
 $3.\ \ \ik = \getKey_c(\shareInfo, m, 1)$ \\
\end{tabular}
\end{minipage}%
}
% middle
 \begin{minipage}[t]{0.13\textwidth}
  \centering
  \begin{tabular}{c}
   $ $ \\
   $ $ \\
   $ $ \\
   $\xrightarrow{m_5}$ \\
   $ $ \\
   $ $ \\
   $ $ \\
   $ $ \\
   $ $ \\
   $ $ \\
  \end{tabular}
 \end{minipage}%
\fbox{
\begin{minipage}[t]{0.39\textwidth}
\begin{tabular}[c]{l}
 $ $\\
 $ $\\
 $ $\\
 $1.\ \ \checkQuery(\STK, k_{\STK}, \NONC, IP_c)$ \\
 $2.\ \ \key_{\mac} = \SE.\Dec(\key_{\STK}, \STK) $\\
 $3.\ \ \doc = T_c \| \NONC \| \cid \| \STK \| \SCFG.\SCID$ \\
 $4.\ \ \text{If }\mac \neq \PRF(\key_{\mac}, \doc)$ \\
 $5.\ \ \quad \Lambda = \text{'reject' and abort}$ \\
 $ $\\
 $1.\ \ \shareInfo = (\NONC, \cid, T_c, \SCFG.t_s)$ \\
 $2.\ \ m = m_1 \| m_2 \| m_3$ \\
 $3.\ \ \ik = \getKey_s(\shareInfo, m, 1)$ \\
\end{tabular}
\end{minipage}%
}

\caption{Abstract model of the proposed QUIC}\label{fig:quic_tls}
\end{center}
\end{figure*}

\bibliographystyle{abbrv}
\bibliography{mybib,confCryp,confComp}

\end{document}
