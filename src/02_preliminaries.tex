%=====================================================
\section{Preliminaries}\label{sec:pre}
%=====================================================
We let ${\negl}(\spar)$ to denote an unspecified
function $f(\spar)$ such that
$f(\spar) ={\spar}^{-\omega(1)}$, saying that such a
function is negligible in $\spar$.
We use the standard notations of digital signature,
collision-resistant hash functions, and pseudo random
functions.
We also use a length-Hiding authenticated encryption as well as a few
known assumptions.

% %=====================================================
% \subsection{Digital Signature Schemes} \label{sec:signature}
% %=====================================================
% A digital signature scheme consists of the following
% three algorithms $\SIG = (\Gen,\Sign,\Vfy)$, where
% \begin{itemize}
%  \item{$\Gen(1^{\spar})$ takes as input a security
%  parameter $\kappa$ and generates a key pair $(pk, sk)$
%  where $pk$ is a verification key and $sk$ is secret
%  key.}
%  \item{$\Sign(sk, m)$ takes as input a message $m$ and
%  $sk$ and outputs a signature $\sigma$.}
%  \item{$\Vfy(pk,\sigma,m)$ takes as input $pk$, $\sigma$
%  and $m$ and outputs 1 if $\sigma$ is valid and 0
%  otherwise.}
% \end{itemize}
% \begin{definition}[EUF-CMA]
% We say that $\SIG$ is $(t, \epsilon_{\sig})$-secure
% againt \textit{existentially unforgeable under
% adaptive chosen-message attacks}, if for all PPT
% adversaries $A$ that run in time $t$, the experiment
% $\Expt^{\sf EUF}_{A}$ in Fig.~\ref{fig:EUF-CMA} returns
% $1$ with at most probability $\epsilon_{\sf sig}$.

% \begin{figure*}[!htb]
% \begin{center}
% \fbox{
% \begin{minipage}[t]{0.38\textwidth}
% \begin{tabular}[t]{l}
%  $\Expt^{\sf EUF}_{A}\ \equiv$ \\
%  $\quad (pk,sk) \leftarrow \SIG.\Gen(1^k)$ \\
%  $\quad m^{\prime}, \sigma \leftarrow A^{\sf SIGN}(pk)$ \\
%  \quad returns $1$ iff \\
%  $(\quad m^{\prime} \not\in M) \, \wedge \, (\SIG.\Vfy(pk, m^{\prime}, \sigma) = 1)$ \\
% \end{tabular}
% \end{minipage}
% \vline \quad
% \begin{minipage}[t]{0.35\textwidth}
% \begin{tabular}[t]{l}
%  ${\sf Oracle\ SIGN(m)\ }\equiv$ \\
%  $\quad M = M \cup m$ \\
%  $\quad \text{returns }\SIG.\Sign(sk, m)$ \\
% \end{tabular}
% \end{minipage}
% }
% \caption{EUF-CMA security experiment}\label{fig:EUF-CMA}
% \end{center}
% \end{figure*}

% \end{definition}

% %=====================================================
% \subsection{Collision-Resistant Hash Functions} \label{sec:CRH}
% %=====================================================
% Let $\HashH\sets \{H\}$ be a hash function family from
% $\bits^*$ to $\bits^{\spar}$.
% We say that a hash-function family $\HashH$ is
% $(t,\epsilon_{H})$-collision-resistant (CR) if, for all
% adversaries $A$ that run in time $t$, it holds that
% \begin{equation}
%  \Pr[H \gets {\HashH}; \, (x,y) \gets A(H): \, x \neq y
%  \, \wedge \,(x) = H(y)] \leq \epsilon_{H}.
% \end{equation}

% %=====================================================
% \subsection{Pseudo-Random Functions} \label{sec:prf}
% %=====================================================
% Let $\{X_n,Y_n\}_{n \in \N}$ be a sequence of domains
% and let $\PRF = \{F_{s}\}_{s \in \Index}$ be a family
% of functions from $X_n$ to $Y_n$, indexed by $s \in
% \Index_{n}$ $(= \Index \cap \{0,1\}^{n})$, i.e.,
% $F_{s} : X_n \to Y_n$, where for every $n$ and
% $s \in \Index_n$, $\Index_n$ is efficiently samplable
% and $F_s$ is efficiently computable.
% We say that $\PRF$ is $(t,\epsilon_{\prf})$-pseudo-random
% function (PRF) familiy if for all adversaries $A$ that
% run in time $t\sets t(n)$,
% \begin{eqnarray}
%  |\Pr[s \gets {\Index}_{n}: \, A^{F_s(\cdot)}(1^n)=1] \, - \, \nonumber \\
%  \Pr[s \gets {\Index}_{n}: \, A^{R_n}(1^n)=1] |
%  \leq \epsilon_{\prf}(n),
% \end{eqnarray}
% where $R=\{R_n\}_{n\in \N}$ is the corresponding uniform
% function ensemble (i.e., $\forall n$, $R_n$ is uniformly
% distributed over the set of the functions from $X_n$ to
% $Y_n$).

% %=====================================================
% \subsection{Decisional Diffie-Hellman} \label{sec:assumption}
% %=====================================================
% Let $g$ be a generator of a finite cyclic group $G$ of
% prime order $q$, i.e.,
% $g \in G^{\times}(\sets G\backslash \{1\})$.

% \begin{definition}[Decisional Diffie-Hellman Assumption]
%  We say that the DDH problem is $(t,\epsilon_{\ddh})$-hard
%  in $G$, if for all PPT adversaries $A$ that runs time
%  in $t$, it holds that
%  \begin{equation}
%   |\Pr[A(p,g,g^a,g^b,g^{ab}) = 1] - \Pr[A(p,g,g^a,g^b,g^c)
%   = 1]| \leq \epsilon_{\ddh}
%  \end{equation}
%  where $g \getsr G^{\times}$ and
%  $a,b,c \getsr \mathbb{Z}_{q}$.
%  We say that the DDH assumption holds in $G$ if for all
%  polynomial (in $\spar$) $t(\spar)$, there exists
%  $\epsilon_{\ddh}(\spar)=\negl(\spar)$ such that the
%  DDH problem is $(t,\epsilon_{\ddh})$-hard in $G$.
% \end{definition}

%=====================================================
\subsection{Length-Hiding Authenticated Encryption} \label{sec:SE}
%=====================================================

A symmetric encryption scheme consists of three algorithm
$SE = (\Init, \Enc, \Dec)$.
\begin{itemize}
 \item{$\Enc(k,\iv,H,m)$ takes as input a secret key
 $k \in \{1,0\}^{\kappa}$, a initial vector $\iv$, an
 output ciphertext length $len \in \mathbb{N}$, some
 header data $H \in \{0,1\}^{\ast}$, and a plaintext
 $m \in \{0,1\}^{\ast}$.
 It outputs a ciphertext $c \in \{0,1\}^{len}$.}
 \item{$\Dec(k,\iv,H,c)$ takes as input a secret key
 $k \in \{1,0\}^{\kappa}$, a initial vector $\iv$,
 some header data $H \in \{0,1\}^{\ast}$, and a
 ciphertext $c \in \{0,1\}^{len}$. It outputs a
 message $m$.}
\end{itemize}

\begin{definition}
 We say that a symmetric encryption scheme
 $\SE = (\Enc,\Dec)$ is $(t, \epsilon_{\LHAE})$-secure,
 if for all PPT adversaries $A$ that running in time
 $t$, it holds that
 \begin{eqnarray}
  \Pr[b\xleftarrow{\$}\{0,1\}; k\xleftarrow{\$}\{0,1\}^{\kappa}; \nonumber \\
   b^{\prime} \xleftarrow{\$} A^{\Encrypt, \Decrypt} : b = b^{\prime}] \leq \epsilon_{\LHAE},
 \end{eqnarray}
 where Encrypt and Decrypt oracles are described in
 Fig.~\ref{fig:LHAE_Enc_Dec}.
 We say that $\SE$ is secure if for all polynomial
 (in $\spar$) $t(\spar)$, there exists
 $\epsilon_{\LHAE}(\spar)=\negl(\spar)$ such that $\SE$
 is $(t, \epsilon_{\LHAE})$-secure.
\end{definition}

\begin{figure*}[!htb]
\begin{center}
\fbox{
\begin{minipage}[t]{0.42\textwidth}
\begin{tabular}[t]{l}
 $\Encrypt(m_0,m_1,\iv,H)$: \\
 $\quad C_0 \xleftarrow{\$} \SE.\Enc(k, \iv, H,m_0)$ \\
 $\quad C_1 \xleftarrow{\$} \SE.\Enc(k, \iv, H,m_1)$ \\
 $\quad \text{If } C_0 = \perp \text{ or } C_1 = \perp$ \\
 $\quad \quad \text{return } \perp$ \\
 $\quad \mathcal{C} = \mathcal{C} \cup C_b$ \\
 $\quad \text{return } C_b$ \\
\end{tabular}
\end{minipage}
\vline \quad
\begin{minipage}[t]{0.42\textwidth}
\begin{tabular}[t]{l}
 $\Decrypt(C,\iv,H)$: \\
 $\quad \text{If } b = 1 \wedge C \not\in \mathcal{C}$ \\
 $\quad \quad \text{return } \SE.\Dec(k,\iv,H,C)$ \\
 $\quad \text{return } \perp$ \\
\end{tabular}
\end{minipage}
}
\caption{Encrypt and Decrypt oracle in the LHAE security experiment}\label{fig:LHAE_Enc_Dec}
\end{center}
\end{figure*}