%=====================================================
\section{Our proposed scheme} \label{sec:proposed_scheme}
%=====================================================

We note that QUIC is not RSACCE secure.
We propose a new scheme which satisfy RSACCE secure and
reduce interactions.

%=====================================================
\subsection{1-RTT Connection Establishment} \label{sec:quic_prop_1rtt}
%=====================================================

The abstract model of our proposed scheme is in
Fig.~\ref{fig:quic_prop_1rtt}.

\begin{figure*}[!htp]
 \begin{center}

\begin{enumerate}
 \item{Initiate} \\
% client side
 \fbox{
  \begin{minipage}[t]{0.38\textwidth}
  \centering
   \begin{tabular}{c}
    $\quad Client$ \\
    $ $ \\
    $ $ \\
    $ $ \\
    $ $ \\
   \end{tabular}
  \end{minipage}%
 }
% middle
 \begin{minipage}[t]{0.13\textwidth}
  \centering
  \begin{tabular}{c}
   $ $ \\
   $ $ \\
   $ $ \\
  \end{tabular}
 \end{minipage}%
% server side
 \fbox{
  \begin{minipage}[t]{0.38\textwidth}
   \centering
   \begin{tabular}{c}
    $\quad Server$ \\
    $ $ \\
    $(pk_s, sk_s) = \SIG.\Gen()$ \\
    $(\SCFG_{pub}, t_s) = \scfgGen(sk_s)$ \\
    $k_{\STK} \xleftarrow{\$} \{0,1\}^{\lambda}$ \\
   \end{tabular}
  \end{minipage}%
 }
 \item{Key Agreement} \\
% client side
 \fbox{
  \begin{minipage}[t]{0.38\textwidth}
  \centering
   \begin{tabular}{c}
    $m_1 = \initialCHLO()$ \\
    $ $ \\
    $\checkSCFG(\SCFG_{pub}) $ \\
    $\shareInfo_c = (\NONC, \cid, T_s, t_c) $ \\
    $\ik = \getKey_c(\shareInfo_c, m_1, 1) $ \\
    $(T_s^{\prime}, \STK, \key_{MAC}) = \receiveSHLO(m_2, \ik)$ \\
    $\shareInfo_c^{\prime} = (\NONC, \cid, T_s^{\prime}, t_c)$ \\
    $m = m_1 \| m_2$ \\
    $\peer = S$ \\
    $k = \getKey_c(\shareInfo_c^{\prime}, m, 0)$ \\
    $\theta = (\SCFG_{pub}, \STK, k_{MAC})$ \\
    $\Lambda = \accept$ \\
   \end{tabular}
  \end{minipage}%
 }
% middle
 \begin{minipage}[t]{0.13\textwidth}
  \centering
  \begin{tabular}{c}
   $\xrightarrow{m_1}$ \\
   $ $ \\
   $\xleftarrow{m_2}$ \\
   $ $ \\
   $ $ \\
   $ $ \\
   $ $ \\
  \end{tabular}
 \end{minipage}%
% server side
 \fbox{
  \begin{minipage}[t]{0.38\textwidth}
   \centering
   \begin{tabular}{c}
    $ $ \\
    $ $ \\
    $ $ \\
    $\shareInfo_s = (\NONC, \cid, T_c, t_s)$ \\
    $\ik = \getKey_s(\shareInfo_s, m_1, 1)$ \\
    $ret = \SHLO(m_1, \ik, 0)$ \\
    $m_2 = ret \| \SCFG_{pub} $ \\
    $\shareInfo_s^{\prime} = (\NONC, \cid, T_c, t_s^{\prime}) $ \\
    $m = m_1 \| m_2$ \\
    $\peer = C$ \\
    $k=\getKey_s(\shareInfo_s^{\prime}, m, 0)$ \\
    $\Lambda = \accept$ \\
   \end{tabular}
  \end{minipage}%
 }
 \item{Data Exchange} \\
% client side
 \fbox{
  \begin{minipage}[t]{0.38\textwidth}
  \centering
   \begin{tabular}{c}
    $ $ \\
    $ $ \\
    $ $ \\
    $\text{for each } \alpha \in {0,...,\MsgCntC{0}}$ \\
    $\sqn_c = \alpha + 1$ \\
    $m_3^{\alpha} = \pak(k, sqn_c, M_c^{\alpha})$ \\
    $ $ \\
    $m_3 = (m_3^{0},...,m_3^{\MsgCntC{0}})$ \\
    $\processPacket(k, m_4)$ \\
   \end{tabular}
  \end{minipage}%
 }
% middle
 \begin{minipage}[t]{0.13\textwidth}
  \centering
  \begin{tabular}{c}
   $ $ \\
   $ $ \\
   $ $ \\
   $ $ \\
   $\xrightarrow{m_3}$ \\
   $\xleftarrow{m_4}$ \\
  \end{tabular}
 \end{minipage}%
% server side
 \fbox{
  \begin{minipage}[t]{0.38\textwidth}
   \centering
   \begin{tabular}{c}
    $\text{If the first message in $m_3$ does not come} $ \\
    $\text{in $\duration$ QUIC regards the first query as}$ \\
    $\text{DoS attack and this connection is closed.}$ \\
    $\text{for each } \beta \in {0,...,\MsgCntS{0}}$ \\
    $\sqn_s = \beta + 1$ \\
    $m_4^{\beta} = \pak(k, \sqn_s, M_s^{\beta})$ \\
    $ $ \\
    $m_4 = (m_4^{\MsgCntS{0}+1},...,m_4^{\MsgCntS{1}})$ \\
    $\processPacket(ik, m_3)$ \\
   \end{tabular}
  \end{minipage}%
 }
\end{enumerate}

 \caption{Abstract model of 1-RTT our proposed scheme}\label{fig:quic_prop_1rtt}
 \end{center}
\end{figure*}

We define three phases of our proposed scheme handshake in 1-RTT.
(1) \textbf{Initiate},
(2) \textbf{Key Agreement},
(3) \textbf{Data Exchange}.
The flow of (1), (3) are the same as the same as the original QUIC
handshake in 1-RTT.

%=====================================================
\subsubsection{Key Agreement}
%=====================================================
In this phase, the client sends an initial client
hello (initialCHLO) which contains connection id,
a client's Diffie-Hellman public value $T_c$, a client
nonce $\NONC$, and some information such as server name,
protocol version, and user agent id. In our definition,
the some informations are omitted.
\\
\noindent
\underline{$\initialCHLO()$:} \\
 \setcounter{nombre}{0}%
 $\prob.\quad \cid \xleftarrow{\$} \{0,1\}^{64} $ \\
 $\prob.\quad \pInfo = (IP_c, IP_s, port_c, port_s)$ \\
 $\prob.\quad t_c \xleftarrow{\$} \Zset_{q}^{\ast}$ \\
 $\prob.\quad T_c = g^{t_c}$ \\
 $\prob.\quad r \xleftarrow{\$} \{0,1\}^{160}$ \\
 $\prob.\quad \NONC = currentTime \| r$ \\
 $\prob.\quad \return\ (\pInfo, \cid, \NONC, T_c)$ \\
%
After the server receives initial client hello, it
sends a server hello (SHLO). The server hello contains
source address token (STK), server config (SCFG),
a certificate, a signature of server config generated
by the server long term secret key, ephemeral server's
Diffie-Hellman public value $T_s^{\prime}$. The client use
STK in future queries to demonstrate ownership of their
source IP address.
\\
\noindent
\underline{$\SHLO(m, \ik, \sqn)$:} \\
 \setcounter{nombre}{0}%
 $\prob.\quad (\pInfo, \cid, \NONC, T_c) = m$ \\
 $\prob.\quad \STK = \makeSTK()$ \\
 $\prob.\quad \pInfo = (IP_s, IP_c, port_s, port_c)$ \\
 $\prob.\quad (\ik_c, \ik_s, \iv_c, \iv_s) = \ik$ \\
 $\prob.\quad t_s^{\prime} \xleftarrow{\$} \Zset_{q}^{\ast}$ \\
 $\prob.\quad T_s^{\prime} = g^{t_s^{\prime}}$ \\
 $\prob.\quad \plaintext = T_{s}^{\prime} \| \STK \| k_{MAC} \| \SCID$\\
 $\prob.\quad H = (\cid, \sqn)$ \\
 $\prob.\quad c = \SE.\Enc(\ik_c, \iv_c \| \sqn, H, \plaintext)$ \\
 $\prob.\quad \return\ (\pInfo, \cid, \SCFG_{pub}, H, c)$ \\
\\
\underline{$\makeSTK()$:} \\
 \setcounter{nombre}{0}%
 $\prob.\quad \iv_{\STK} \xleftarrow{\$} \{0,1\}^{96}$ \\
 $\prob.\quad k_{MAC} \xleftarrow{\$} \{0,1\}^{\mu}$ \\
 $\prob.\quad \plaintext = IP_c \| currentTime \| k_{\MAC}$ \\
 $\prob.\quad \STK \leftarrow \iv_{\STK} \| \SE.\Enc(k_{\STK}, len, \iv_{\STK}, \plaintext)$ \\
 $\prob.\quad \return\ \STK$ \\
%
After the client receives a server hello, the client
checks server config and calculate a last key.
\\
\noindent
\underline{$\receiveSHLO(m, \ik)$:} \\
 \setcounter{nombre}{0}%
 $\prob.\quad (\pInfo, \cid, \SCFG_{pub}, H, c) = m$ \\
 $\prob.\quad (\ik_c, \ik_s, \iv_c, \iv_s) = \ik$ \\
 $\prob.\quad (\cid, \sqn) = H$ \\
 $\prob.\quad \plaintext = \SE.\Dec(\ik_c, \iv_c \| \sqn, H, c)$ \\
 $\prob.\quad \text{If }\plaintext = \perp$ \\
 $\prob.\quad \quad \Lambda = \text{'reject' and abort}$ \\
 $\prob.\quad T_{s}^{\prime} \| \STK \| \key_{MAC} \| \SCID^{\prime} = \plaintext $ \\
 $\prob.\quad \text{If }\SCID^{\prime} \neq \SCID$ \\
 $\prob.\quad \quad \Lambda = \text{'reject' and abort}$ \\
 $\prob.\quad \return\ (T_s^{\prime}, \STK, \key_{MAC})$ \\
\\
\underline{$\checkSCFG(\SCFG_{pub})$:} \\
 \setcounter{nombre}{0}%
 $\prob.\quad (\SCID, T_s, \expy, \sigma_s, \cert_s) = \SCFG_{pub}$ \\
 $\prob.\quad \text{If } \expy \leq currentTime$ \\
 $\prob.\quad \quad \Lambda = \text{'reject' and abort}$ \\
 $\prob.\quad pk_s = \getPK(cert_s)$ \\
 $\prob.\quad str = \text{ QUIC server config signature }$ \\
 $\prob.\quad \doc = str \| 0x00 \| \SCID \| T_s \| \expy$ \\
 $\prob.\quad \text{If } \SIG.\Vfy(pk_s, \sigma_s, \doc) = \perp$ \\
 $\prob.\quad \quad \Lambda = \text{'reject' and abort}$ \\
%
After the server sends a server hello or the
client validates a server hello, they calculate forward secure
key $\key$.
\noindent
\underline{$\getKey_c(\shareInfo, m, \init)$:} \\
 \setcounter{nombre}{0}%
 $\prob.\quad (\NONC, \cid ,T_s, t_c) = \shareInfo$ \\
 $\prob.\quad pms = T_s^{t_c}$ \\
 $\prob.\quad \return\ \extractKey(pms, \NONC, \cid, m, 40, \init)$ \\
\underline{$\getKey_s(\shareInfo, m, \init)$:} \\
 \setcounter{nombre}{0}%
 $\prob.\quad (\NONC, \cid ,T_c, t_s) = \shareInfo$ \\
 $\prob.\quad pms = T_c^{t_s}$ \\
 $\prob.\quad \return\ \extractKey(pms, \NONC, \cid, m, 40, \init)$ \\
\underline{$\extractKey(pms, \NONC, \cid, m, \ell, \init)$:}\\
 \setcounter{nombre}{0}%
 $\prob.\quad ms = \PRF(pms, \NONC)$ \\
 $\prob.\quad \text{If } \init = 1$ \\
 $\prob.\quad \quad str = \text{ QUIC key expansion }$ \\
 $\prob.\quad \text{Else }$ \\
 $\prob.\quad \quad str = \text{ QUIC forward secure expansion }$ \\
 $\prob.\quad \info = str \| 0x00 \| \cid \| m \| \SCFG_{pub}$ \\
 $\prob.\quad \return\ \text{the first $\ell$ octets (i.e. bytes) of T = }$ \\
 $\quad \quad \text{(T(1),T(2), ...), where for all $i \in \Nset$, $T(i) = $} $\\
 $\quad \quad \text{$\PRF(ms, T(i-1) \| \info \| 0x0i)$ and $T(0) = \epsilon$} $\\

%=====================================================
\subsection{0-RTT Connection Establishment} \label{sec:quic_prop_0rtt}
%=====================================================

The abstract model of our proposed scheme is in
Fig.~\ref{fig:quic_prop_0rtt}.

\begin{figure*}[!htp]
 \begin{center}

\begin{enumerate}
 \item{Initial Key Agreement} \\
 \fbox{
  \begin{minipage}[t]{0.38\textwidth}
  \centering
   \begin{tabular}{c}
    $m_5 = \CHLO(\STK, \SCFG_{pub}, k_{MAC})$ \\
    $\shareInfo = (\NONC, \cid, T_s, t_c^{\ast})$ \\
    $\Lambda = \preaccept$ \\
    $\ik = \getKey_c(\shareInfo, m_5, 1)$ \\
   \end{tabular}
  \end{minipage}%
 }
 \begin{minipage}[t]{0.13\textwidth}
  \centering
  \begin{tabular}{c}
   $\xrightarrow{m_5}$ \\
   $ $ \\
  \end{tabular}
 \end{minipage}%
 \fbox{
  \begin{minipage}[t]{0.38\textwidth}
   \centering
   \begin{tabular}{c}
    $\checkQuery(m_5, k_{\STK}, IP_c)$ \\
    $\shareInfo = (\NONC, \cid, T_c^{\ast}, t_s)$ \\
    $\Lambda = \preaccept$ \\
    $\ik = \getKey_s(\shareInfo, m_5, 1)$ \\
   \end{tabular}
  \end{minipage}%
 }
 \item{Initial Data Exchange} \\
 \fbox{
  \begin{minipage}[t]{0.38\textwidth}
  \centering
   \begin{tabular}{c}
    $\text{for each } \alpha \in {\MsgCntC{0}+1,...,\MsgCntC{1}}$ \\
    $\sqn_c = \alpha + 2$ \\
    $m_{6}^{\alpha} = \pak(ik, sqn_c, M_c^{\alpha})$ \\
    $ $ \\
    $m_{6} = (m_{6}^{\MsgCntC{0}+1},...,m_{6}^{\MsgCntC{1}})$ \\
    $\processPacket(ik, m_{7})$ \\
   \end{tabular}
  \end{minipage}%
 }
 \begin{minipage}[t]{0.13\textwidth}
  \centering
  \begin{tabular}{c}
   $ $ \\
   $ $ \\
   $ $ \\
   $ $ \\
   $\xrightarrow{m_{6}}$ \\
   $\xleftarrow{m_{7}}$ \\
  \end{tabular}
 \end{minipage}%
 \fbox{
  \begin{minipage}[t]{0.38\textwidth}
   \centering
   \begin{tabular}{c}
    $\text{for each } \beta \in {\MsgCntS{0}+1,...,\MsgCntS{1}}$ \\
    $\sqn_s = \beta + 2$ \\
    $m_{7}^{\beta} = \pak(ik, \sqn_s, M_s^{\beta})$ \\
    $ $ \\
    $m_{7} = (m_{7}^{\MsgCntS{0}+1},...,m_{7}^{\MsgCntS{1}})$ \\
    $\processPacket(ik, m_{6})$ \\
   \end{tabular}
  \end{minipage}%
 }
 \item{Key Agreement} \\
 \fbox{
  \begin{minipage}[t]{0.38\textwidth}
  \centering
   \begin{tabular}{c}
    $ $ \\
    $T_s^{\prime} = \receiveSHLO(m_8)$ \\
    $\shareInfo = (\NONC, \cid, T_s^{\prime}, t_c^{\ast})$ \\
    $m = m_5 \| m_8$ \\
    $\Lambda = \accept$ \\
    $k = \getKey_c(\shareInfo, m, 0)$ \\
   \end{tabular}
  \end{minipage}%
 }
 \begin{minipage}[t]{0.13\textwidth}
  \centering
  \begin{tabular}{c}
   $ $ \\
   $\xleftarrow{m_{8}}$ \\
   $ $ \\
  \end{tabular}
 \end{minipage}%
 \fbox{
  \begin{minipage}[t]{0.38\textwidth}
   \centering
   \begin{tabular}{c}
    $ \sqn_s = \MsgCntS{1} + 3$ \\
    $m_{8} = \SHLO(m_5, \ik, \sqn_s)$ \\
    $\shareInfo = (\NONC, \cid, T_c^{\ast}, t_s^{\prime})$ \\
    $m = m_5 \| m_8$ \\
    $\Lambda = \accept$ \\
    $k = \getKey_s(\shareInfo, m, 0)$ \\
   \end{tabular}
  \end{minipage}%
 }
 \item{Data Exchange} \\
 \fbox{
  \begin{minipage}[t]{0.38\textwidth}
  \centering
   \begin{tabular}{c}
    $\text{for each } \alpha \in {\MsgCntC{1}+1,...,\MsgCntC{2}}$ \\
    $\sqn_c = \alpha + 3$ \\
    $m_{9}^{\alpha} = \pak(k, sqn_c, M_c^{\alpha})$ \\
    $ $ \\
    $m_{9} = (m_{9}^{\MsgCntC{1}+1},...,m_{9}^{\MsgCntC{2}})$ \\
    $\processPacket(k, m_{10})$ \\
   \end{tabular}
  \end{minipage}%
 }
 \begin{minipage}[t]{0.13\textwidth}
  \centering
  \begin{tabular}{c}
   $ $ \\
   $ $ \\
   $ $ \\
   $ $ \\
   $\xrightarrow{m_{9}}$ \\
   $\xleftarrow{m_{10}}$ \\
  \end{tabular}
 \end{minipage}%
 \fbox{
  \begin{minipage}[t]{0.38\textwidth}
   \centering
   \begin{tabular}{c}
    $\text{for each } \beta \in {\MsgCntS{1}+1,...,\MsgCntS{2}}$ \\
    $\sqn_s = \beta + 3$ \\
    $m_{10}^{\beta} = \pak(k, \sqn_s, M_s^{\beta})$ \\
    $ $ \\
    $m_{10} = (m_{10}^{\MsgCntS{1}+1},...,m_{10}^{\MsgCntS{2}})$ \\
    $\processPacket(ik, m_{9})$ \\
   \end{tabular}
  \end{minipage}%
 }
\end{enumerate}
 \caption{Abstract model of 0-RTT our proposed scheme}\label{fig:quic_prop_0rtt}
 \end{center}
\end{figure*}

We define four phases of our proposed scheme handshake in 0-RTT.
(1) \textbf{Initial Key Agreement},
(2) \textbf{Initial Data Exchange},
(3) \textbf{Key Agreement}, and
(4) \textbf{Data Exchange}.
The flow of (2), (3), (4) are the same as the 1-RTT
handshake.
\subsubsection{Handshake operations}
\underline{$\scfgGen(sk_s)$:} \\
 $1.\ \ t_s \xleftarrow{\$} \Zset_{q}^{\ast}$ \\
 $2.\ \ T_s = g^{t_s}$ \\
 $3.\ \ \expy = \tau_{t+1}$ \\
 $4.\ \ \SCID = \Hash(T_s \| \expy)$ \\
 $5.\ \ str = \text{ QUIC server config signature }$ \\
 $6.\ \ \doc = str \| 0x00 \| \SCID \| T_s \| \expy$ \\
 $7.\ \ \sigma_s = \SIG.\Sign(sk_s, \doc)$ \\
 $8.\ \ \SCFG_{pub} = (\SCID, T_s, \expy, \sigma_s, \cert_s)$ \\
 $9.\ \ \return\ (\SCFG_{pub}, t_s)$ \\
\\
\underline{$\initialCHLO()$:} \\
 $1.\ \ \cid \xleftarrow{\$} \{0,1\}^{64} $ \\
 $2.\ \ \pInfo = (IP_c, IP_s, port_c, port_s)$ \\
 $3.\ \ t_c \xleftarrow{\$} \Zset_{q}^{\ast}$ \\
 $4.\ \ T_c = g^{t_c}$ \\
 $5.\ \ r \xleftarrow{\$} \{0,1\}^{160}$ \\
 $6.\ \ \NONC = currentTime \| r$ \\
 $7.\ \ \return\ (\pInfo, \cid, \NONC, T_c)$ \\
\\
\underline{$\SHLO(m, \ik, \sqn)$:} \\
 $1.\ \ (\pInfo, \cid, \NONC, T_c) = m$ \\
 $2.\ \ \STK = \makeSTK()$ \\
 $3.\ \ \pInfo = (IP_s, IP_c, port_s, port_c)$ \\
 $4.\ \ (\ik_c, \ik_s, \iv_c, \iv_s) = \ik$ \\
 $5.\ \ t_s^{\prime} \xleftarrow{\$} \Zset_{q}^{\ast}$ \\
 $6.\ \ T_s^{\prime} = g^{t_s^{\prime}}$ \\
 $7.\ \ \plaintext = T_{s}^{\prime} \| \STK \| k_{MAC} \| \SCID$\\
 $8.\ \ H = (\cid, \sqn)$ \\
 $9.\ \ c = \SE.\Enc(\ik_c, \iv_c \| \sqn, H, \plaintext)$ \\
 $10.\  \return\ (\pInfo, \cid, \SCFG_{pub}, H, c)$ \\
\\
\underline{$\makeSTK()$:} \\
 $1.\ \ \iv_{\STK} \xleftarrow{\$} \{0,1\}^{96}$ \\
 $2.\ \ k_{MAC} \xleftarrow{\$} \{0,1\}^{\mu}$ \\
 $3.\ \ \plaintext = IP_c \| currentTime \| k_{\MAC}$ \\
 $4.\ \ \STK \leftarrow \iv_{\STK} \| \SE.\Enc(k_{\STK}, len, \iv_{\STK}, \plaintext)$ \\
 $5.\ \ \return\ \STK$ \\
\\
\underline{$\receiveSHLO(m, \ik)$:} \\
 $1.\ \ (\pInfo, \cid, \SCFG_{pub}, H, c) = m$ \\
 $2.\ \ (\ik_c, \ik_s, \iv_c, \iv_s) = \ik$ \\
 $3.\ \ (\cid, \sqn) = H$ \\
 $4.\ \ \plaintext = \SE.\Dec(\ik_c, \iv_c \| \sqn, H, c)$ \\
 $5.\ \ \text{If }\plaintext = \perp$ \\
 $6.\ \ \quad \Lambda = \text{'reject' and abort}$ \\
 $7.\ \ T_{s}^{\prime} \| \STK \| k_{MAC} \| \SCID^{\prime} = \plaintext $ \\
 $8.\ \ \text{If }\SCID^{\prime} \neq \SCID$ \\
 $9.\ \ \quad \Lambda = \text{'reject' and abort}$ \\
 $10.\  \return\ T_s^{\prime}$ \\
\\
\underline{$\CHLO(\STK, \SCFG_{pub}, k_{MAC})$:} \\
 $1.\ \ \cid \xleftarrow{\$} \{0,1\}^{64}$ \\
 $2.\ \ r \xleftarrow{\$} \{0,1\}^{160}$ \\
 $3.\ \ \NONC = currentTime \| r$ \\
 $4.\ \ t_c^{\ast} \xleftarrow{\$} \Zset_{q}^{\ast}$ \\
 $5.\ \ T_c^{\ast} = g^{t_c^{\ast}}$ \\
 $6.\ \ \pInfo = (IP_c, IP_s, port_c, port_s)$ \\
 $7.\ \ \doc = T_c^{\ast} \| \NONC \| \cid \| \STK \| \SCID$ \\
 $8.\ \ \mac = \PRF(k_{MAC}, \doc) $ \\
 $9.\ \ \return\ (\pInfo, \cid, \STK, \SCID, \NONC, T_c^{\ast}, \mac)$ \\
\\
\underline{$\checkQuery(m, k_{\STK}, IP_c)$:} \\
 $1.\ \ (\pInfo, \cid, \STK, \SCID, \NONC, T_c^{\ast}, \mac) = m$ \\
 $2.\ \ (\iv_{\STK}, c) = \STK$ \\
 $3.\ \ IP_c^{\prime} \| currentTime \| k_{\MAC} = \SE.\Dec(k_{\STK}, \iv_{\STK}, c)$ \\
 $4.\ \ \doc = \cid \| \NONC \| T_c^{\ast}$ \\
 $5.\ \ \text{If } \PRF(k_{MAC}, \doc) \neq \mac$ \\
 $6.\ \ \quad \Lambda = \text{'reject' and abort}$ \\
 $7.\ \ (time_{\NONC}, r) = \NONC$ \\
 $8.\ \ \text{If } (IP_c^{\prime}, currentTime) = \perp$, or \\
 $9.\ \ \quad IP_c^{\prime} \neq IP_c$, or $time_{\STK} \leq time_{allowed}$\\
 $10.\  \quad r \in \strike$, or $time_{\NONC} \not\in \strike_{rng}$ \\
 $11.\  \quad \quad \Lambda = \text{'reject' and abort}$ \\
\\
\underline{$\checkSCFG(\SCFG_{pub})$:} \\
 $1.\ \ (\SCID, T_s, \expy, \sigma_s, \cert_s) = \SCFG_{pub}$ \\
 $2.\ \ \text{If } \expy \leq currentTime$ \\
 $3.\ \ \quad \Lambda = \text{'reject' and abort}$ \\
 $4.\ \ pk_s = \getPK(cert_s)$ \\
 $5.\ \ \text{If } \SIG.\Vfy(pk_s, \sigma_s, \doc) = \perp$ \\
 $6.\ \ \quad \Lambda = \text{'reject' and abort}$ \\
\\
\underline{$\checkQuery(\STK, k_{\STK}, \NONC, IP_c)$:} \\
 $1.\ \ (\iv_{\STK}, c) = \STK$ \\
 $2.\ \ (IP_c^{\prime}, time_{\STK}) = \SE.\Dec(k_{\STK}, \iv_{\STK}, c)$ \\
 $3.\ \ (time_{\NONC}, r) = \NONC$ \\
 $4.\ \ \text{If } (IP_c^{\prime}, currentTime) = \perp$, or \\
 $5.\ \ \quad IP_c^{\prime} \neq IP_c$, or $time_{\STK} \leq time_{allowed}$\\
 $6.\ \ \quad r \in \strike$, or $time_{\NONC} \not\in \strike_{rng}$ \\
 $7.\ \ \quad \quad \Lambda = \text{'reject' and abort}$ \\

%=====================================================
\subsection{Security of our proposed scheme} \label{sec:quic_proof}
%=====================================================

\begin{theorem} \label{theorem:proposed_scheme}
 Let $\mu$ be the output length of $\PRF$, let $\lambda$ be
 the length of $\SCID$, let $\nclient$ be the number of
 clients, let $\nserver$ be the number of servers, let
 $\noracle$ be the number of oracles of each parties, and
 let $n_{\ell}$ be the maximum number of resumptions. Assume
 that the $\PRF$ is $(t, \epsilon_{\prf})$-pseudo-random
 function family, the signature shceme
 $\SIG$ is $(t, \epsilon_{\sig})$-secure against existentially
 unforgeable under adaptive chosen-message attacks, the DDH
 problem on $G$ is $(t, \epsilon_{\ddh})$-hard, the hash
 function family $\mathcal{H}$ is
 $(t,\epsilon_{H})$-collision-resistant (CR), the symmetric
 authenticated encryption scheme $\SE$ is
 $(t, \epsilon_{\sLHAE})$-secure.
 Then for all PPT adversaries, our proposed scheme is RSACCE secure.
\end{theorem}

Our proposed scheme can prevent the five attacks in~\cite{LJBN15:QUIC}.

\begin{itemize}
 \item{Server Config Replay Attack}
  In this attack, the adversary collect
 \item{Source-Address Token Replay Attack}
 \item{Connection ID Manipulation Attack}
 \item{Source-Address Token Manipulation Attack}
 \item{Crypto Stream Offset Attack}
\end{itemize}

We prove Theorem~\ref{theorem:proposed_scheme} by proving two lemmas.

\begin{lemma} \label{lemma:proposed_scheme_rsacce-sa}
 $\Adv^{\rsaccesa}_{P}(A)$ is at most
 \begin{equation}
  \Adv^{\rsaccesa}_{P}(A) \leq \frac{n_s n_c}{2^{\lambda}} m_s \epsilon_{\sig}
 \end{equation}
\end{lemma}
%
\begin{proof}
 The proof proceeds in a sequence of games. \vspace{10pt}\\
 {\bfseries Game 0.} This game equals the \textit{server authentication} experiment in Def.~\ref{def:rsacce-sa}.\\
 \begin{equation}
  \Adv_0 = \Adv^{\rsaccesa}_{P}(A)
 \end{equation}%
%
%
 \textbf{Game 1.} In this game we add an abort rule.
 The challenger aborts, if there exists any server oracle $\pi^s_{j, 0}$
  that chooses a SCID which is not unique.
 More precisely, the game is aborted if the adversary ever makes a first $\Send$ query to a server oracle $\pi^s_{j, 0}$, and the oracle replies with SCID such that there exists some other server oracle $\pi^{s^{\prime}}_{j^{\prime}, 0}$ which has previously sampled the same SCID.

 In total less than the number of $\nserver \noracle$ SCID are sampled, each uniformly random from $\{0,1\}^{\lambda}$.
 Thus, the probability that a collision occurs is bounded by $(\nserver \noracle)^2 2^{-\lambda}$
 \begin{equation}
  |\Adv_1 - \Adv_0| \leq \frac{(\nserver \noracle)^2}{2^{\lambda}}.
 \end{equation}%
%
%
 \textbf{Game 2.} We try to guess which client oracle will be the first oracle to break \textit{server authentication}. If our guess is wrong, i.e. if there is another client oracle that breaks \textit{server authentication} before, then we abort the game.

 Technically, the game is identical to Game 1, except for the following. The challenger guesses three random indices $(c^{\ast}, i^{\ast}, \ell^{\ast}) \xleftarrow{\$} [\nclient] \times [\noracle] \times [n_{\ell}]$. If there exists a client oracle $\pi^c_{i,\ell}$ that breaks server authentication, and $(c, i, \ell) \neq c^{\ast}, i^{\ast}, \ell^{\ast})$, then the challenger aborts the game. Note that if the first oracle $\pi^c_{i,\ell}$ that breaks server authentication, then with probability $1/(\nclient \noracle n_{\ell})$ we have $(c,i,\ell) = (c^{\ast}, i^{\ast}, \ell^{\ast})$, and thus
 \begin{equation}
  \Adv_1 \leq \nclient \noracle n_{\ell} \Adv_2.
 \end{equation}%
 Note that in this game the attacker can only break the security of the protocol, if oracle $\pi^{c^{\ast}}_{i^{\ast},\ell^{\ast}}$ is the first oracle that breaks server authentication, as otherwise the game is aborted.
\vspace{10pt}\\%
%
%
 \textbf{Game 3.} Again the challenger proceeds as before, but we add an abort rule. We want to make sure that $\pi^{c^\ast}_{i^{\ast},0}$ receives as input exactly the Diffie-Hellman value $T_s$ that was selected by some other uncorrupted oracle.

 Technically, we abort and raise event $\abort_\SIG$, if oracle $\pi^{c^{\ast}}_{i^{\ast},0}$ ever receives as input a message $\cert_s$ indicating intended partner $\peer = s$ and message $(T_s,\sigma_s,SCID)$ such taht $\sigma_s$ is a valid signature over $T_s\|SCID$, however there exists no oracle $\pi^s_{j,0}$ which has previously output $\sigma_s$. Clearly we have
 \begin{equation}
  |\Adv_3 - \Adv_2| = \Pr[\abort_{\SIG}].
 \end{equation}%

 Note that the experiment is aborted, if $\pi^{c^{\ast}}_{i^{\ast},0}$ satisfies server authentication, due to Game 2. This means that server $\Server_s$ must be $\tau_s$-corrupted with $\tau_s = \infty$ (i.e. not corrupted) when $\pi^{c^{\ast}}_{i^{\ast},0}$ accepts (as otherwise $\pi^{c^{\ast}}_{i^{\ast},0}$ satisfies server authentication). To show that $\Pr[\abort_{\SIG}] \leq \ell \epsilon_{\SIG}$, we construct a signature forger as follows. The forger receives as input a public key $pk^{\ast}$ and simulates the challenger for $\mathcal{A}$. It guesses an index $\phi \xleftarrow{\$}[\nserver]$, sets $pk_{\phi} = pk^{\ast}$, and generates all long-term public/secret keys as before. Then it proceeds as the challenger in Game 3, except that it uses its chosen message oracle to generate a signature under $pk_{\phi}$ when necessary.

 If $\phi = s$, which happens with probability $1/\nserver$, then the forger can use the signature received by $\pi^{c^{\ast}}_{i^{\ast},\ell^{\ast}}$ to break the EUF-CMA security of the signature scheme with success probability $\epsilon_{\SIG}$, so $\Pr[\abort_{\SIG}]/\ell \leq \epsilon_{\SIG}$. Therefore if $\Pr[\abort_{\SIG}]$ is not negligible, then $\epsilon_{\SIG}$ is not negligible as well and we have
 \begin{equation}
  |\Adv_3 - \Adv_2| = \nserver \epsilon_{\SIG}.
 \end{equation}%

 Note that in Game 3 oracle $\pi^{c^{\ast}}_{i^{\ast},0}$ receives as input a Diffie-Hellman value $T_s$ such that $T_s$ was chosen by another oracle, but not by the attacker. Note also that there is unique oracle that issued a signature $\sigma_s$ containing SCID.
\vspace{10pt}\\%
%
%
 \textbf{Game 4.} In this game we want to make sure that we know which oracle $\pi^s_{j,0}$ will issue the signature $\sigma_s$ that $\cOracleAstFull$ receives. Note that this signature includes SCID which is unique due to Game 1. Therefore the challenger in this game proceeds as before, however additionally guesses two indices $(s^{\ast}, j^{\ast}) \xleftarrow{\$} [\nserver] \times [\noracle]$.

 We know that there must exists at least one oracle that outputs $\sigma_s$ such that $\sigma_s$ is forwarded to $\cOracleAstFull$, due to Game 3. Thus we have
 \begin{equation}
  \Adv_3 \leq \nserver \noracle \Adv_4
 \end{equation}%
 Note that in this game we know exactly that oracle $\sOracleAstFull$ chooses the Diffie-Hellman share $T_s$ that $\cOracleAstFull$ uses to compute its premaster secret.
 \vspace{10pt}\\
%
%
 \textbf{Game 5.} Let $T_{\cIndexAstRes} = g^u$ denote the Diffie-Hellman share chosen by $\cOracleAstRes$, let $T_{\sIndexAstRes} = g^v$ denote the share chosen by its partner $\sOracleAstRes$, and let $k_{\cIndexAstRes}$ is the key computed by $\cOracleAstRes$. Thus, both oracles compute the premaster secret as $pms^{\prime} = g^{uv}$.

 The challenger in this game proceeds as before, however replaces the premaster secret $pms^{\prime}$ of $\cOracleAstRes$ and $\sOracleAstRes$ with a random group element $\widetilde{pms^{\prime}} = g^w$, $w \xleftarrow{\$} \Zset_p$. Note that both $g^u$ and $g^v$ are chosen by oracles $\cOracleAstRes$ and $\sOracleAstRes$, respectively, as otherwise $\cOracleAstRes$ would not have a matching conversation to $\sOracleAstRes$ and the game would be aborted.

 Distinguish Game 5 from Game 4 implies an algorithm solving the decisional Diffie-Hellman problem, thus
 \begin{equation}
  |\Adv_{5} - \Adv_{4}| \leq \epsilon_{\ddh}
 \end{equation}%
%
%
 \textbf{Game 6.} In this game we replace the value $ms^{\prime} = \PRF(\widetilde{pms^{\prime}}, \NONC)$ with a random value $\widetilde{ms}$.

 Distinguishing Game 6 from Game 5 implies an algorithm breaking the security of the pseudo random function $\PRF$, thus
 \begin{equation}
  |\Adv_{6} - \Adv_{5}| \leq \epsilon_{\prf}
 \end{equation}%
%
%
 \textbf{Game 7.} In this game we replace the function $\PRF(\widetilde{ms^{\prime}},\cdot)$ with a random function. If $\sOracleAstRes$ uses the same master secret $\widetilde{ms^{\prime}}$ as $\cOracleAstRes$, then the function $\PRF(\widetilde{ms^{\prime}},\cdot)$ used by $\sOracleAstRes$ is replaced as well. Of course the same random function is used for both oracles sharing the same $\widetilde{ms^{\prime}}$.

 Distinguishing Game 7 from Game 6 implies an algorithm breaking the security of the pseudo random function $\PRF$, thus
 \begin{equation}
  |\Adv_7 - \Adv_6| \leq \epsilon_{\prf}.
 \end{equation}%
%
%
 \textbf{Game 8.} Now we use that the key $k$ used by $\cOracleAstRes$ and $\sOracleAstRes$ in the stateful symmetric encryption scheme uniformly at random and independent of all QUIC handshake messages.

 The adversary have to make ciphertext $c$ such that $\SE$.$\Dec$ ( k , c , H , $st_d$ ) $\neq \perp$ without knowing the key $k$. It implies an algorithm breaking the sLHAE security of the symmetric encryption scheme, we have
 \begin{equation}
  \Adv_8 \leq 1/2 + \epsilon_{\sLHAE}.
 \end{equation}%
\end{proof}

\begin{lemma} \label{lemma:proposed_scheme_rsacce-cc}
 $\Adv^{\rsaccecc}_{P}(A)$ is at most
 \begin{equation}
  \Adv^{\rsaccecc}_{P}(A) \leq \Adv^{\rsaccesa}_{P}(A) + n_sn_c\ell(\epsilon_{\ddh} + 2\epsilon_{\prf} + 2\epsilon_{\sLHAE})
 \end{equation}
\end{lemma}
%
\begin{proof}
 The proof proceeds in a sequence of games. \vspace{10pt}\\
 \textbf{Game 0.} This game equals the \textit{channel confidetility} security experiment.
 \begin{equation}
  \Adv_0 = \Adv^{\rsaccecc}_{P}(A)
 \end{equation}%
%
%
 \textbf{Game 1.} The challenger in this game proceeds as before, however it aborts and chooses $b^{\prime}$ uniformly random, if there exists any oracle that breaks server authentication. Thus we have
 \begin{equation}
  |\Adv_1 - \Adv_0| = \Adv^{\rsaccesa}_{P}(A).
 \end{equation}%
 Note that if there exists the oracle which breaks \textit{server authentication}, the adversary easily breaks \textit{channel confidentiality}. If the target oracle breaks \textit{server authentication}, the adversary or unrelated oracle (i.e. it is not an intended partner of the oracle) can establish the session with the oracle. The adversary can issue $\Reveal$-query to unrelated oracle which is out of the restriction of \textit{channel confidentiality}. Then the adversary can know the session key of the target oracle.
\vspace{10pt}\\%
%
%
 \textbf{Game 2.} The challenger in this game proceeds as before, however in addition guesses indices $(p^{\ast}, i^{\ast}, \ell^{\ast}) \xleftarrow{\$} [n_s + n_c] \times [n_o] \times [n_{\ell}]$. It aborts and chooses $b^{\prime}$ at random, if the attacker issues a $\Encrypt$-query with $(p,i,\ell) \neq (p^{\ast}, i^{\ast}, \ell^{\ast})$. With probability $1/((n_s+n_c)n_o n_{\ell})$ we have $(p,i,\ell) = (p^{\ast}, i^{\ast}, \ell^{\ast})$, and thus
 \begin{equation}
  \Adv_1 = (n_s + n_c) n_o n_{\ell}\Adv_2.
 \end{equation}%
 Note that in Game 2 we know that $\mathcal{A}$ will issue a $\Encrypt$-query to oracle $\pOracleAstRes$. Note also that $\pOracleAstRes$ has a unique partner due to Game 1. In the sequel we denote with $\qOracleAstRes$ the unique oracle such that $\pOracleAstRes$ has a matching conversations for an initial key or a final key with $\qOracleAstRes$, and say that $\qOracleAstRes$ is the intended partner of $\pOracleAstRes$.
\vspace{10pt}\\%
%
%
 \textbf{Game 3 + $4\ell$.} We repeat the game until randomizing the session between $\pOracleAstRes$ and $\qOracleAstRes$. Initially $\ell = 0$. The reason of repeating is that the adversary always return correct $b^{\prime}$ if the adversary can obtain $k_{mac}$ of $\pOracleAstEll$ or $\qOracleAstEll$. We need to prevent the adversary obtaining all $k_{mac}$ in $ 0 \leq \ell \leq \ell^{\ast}$.
 Let $T_{\pindexell} = g^u$ denote the DH public key chosen by $\pOracleAstEll$, let $T_{\qindexell} = g^v$ denote the share chosen by its partner $\qOracleAstEll$, and let $\ik_{\pindexell}$ and $k_{\pindexell}$ are the key computed by $\pOracleAstEll$, let $pms_{\ik}$ and $ms_{\ik}$ denote a premaster secret and master secret for an initial key, let $pms_{k}$ and $ms_{k}$ denote a premaster secret and master secret for session key. Thus, both oracles compute the premaster secret as $pms = g^{uv}$.

 The challenger in this game proceeds as before, however replaces the premaster secret $pms_{\ik}$ of $\pOracleAstEll$ and $\qOracleAstEll$ with a random group element $\widetilde{pms_{\ik}} = g^w$, $w \xleftarrow{\$} \Zset_p$. Note that both $g^u$ and $g^v$ are chosen by oracles $\pOracleAstEll$ and $\qOracleAstEll$, respectively, as otherwise $\pOracleAstEll$ would not have a matching conversations for an initial key with $\qOracleAstEll$ and the game would be aborted.

 Suppose that there exists an algorithm $\mathcal{A}$ distinguishes Game 3 + $4\ell$ from Game 2 + $4\ell$. Then we can construct an algorithm $\mathcal{B}$ solving the DDH problem as follows. $\mathcal{B}$ receives as input $(g,g^u,g^v,g^w)$. The challenger defines $T_{\pindexell} := g^u$ and $T_{\qindexell} := g^v$, and the premaster secret of $\pOracleAstEll$ and $\qOracleAstEll$ equal to $pms_{\ik} := g^w$. Note that $\mathcal{B}$ can simulate all messages exchanged between $\pOracleAstEll$ and $\qOracleAstEll$ properly. Since all other oracles are not modified, $\mathcal{B}$ can simulate these oracles properly as well.

 If $w=uv$, then the view of $\mathcal{A}$ when interacting with $\mathcal{B}$ is identical to Game 2 + $4\ell$, while if $w \xleftarrow{\$}\Zset_p$ then it is identical to Game 3 + $4\ell$. Thus, the DDH assumption implies that
 \begin{equation}
  |\Adv_{3 + 4\ell} - \Adv_{2 + 4\ell}| \leq \epsilon_{\ddh}
 \end{equation}%
%
%
 \textbf{Game 4 + $4\ell$.} In Game 4 + 4$\ell$ we make use of the fact that the premaster secret $\widetilde{pms_{\ik}}$ of $\pOracleAstEll$ and $\qOracleAstEll$ is chosen uniformly random. We thus replace the value $ms_{\ik} = \PRF(\widetilde{pms_{\ik}}, \NONC)$ with a random value $\widetilde{ms_{\ik}}$.

 Distinguish Game 4 + 4$\ell$ from Game 3 + 4$\ell$ implies an algorithm breaking the security of the pseudo random function $\PRF$, thus
 \begin{equation}
  |\Adv_{4 + 4\ell} - \Adv_{3 + 4\ell}| \leq \epsilon_{\prf}
 \end{equation}%
%
%
 \textbf{Game 5 + $4\ell$.} In this game we replace the function $\PRF(\widetilde{ms_{\ik}}, \cdot)$ used by $\pOracleAstEll$ and $\qOracleAstEll$ with a random function $F_{\widetilde{ms_{\ik}}}$. Of course the same random function is used for both oracles $\pOracleAstEll$ and $\qOracleAstEll$. Distinguishing Game 5 + $4\ell$ from Game 4 + $4\ell$ again implies an algorithm breaking the security of the pseudo random function $\PRF$.
 \begin{equation}
  |\Adv_{5 + 4\ell} - \Adv_{4 + 4\ell}| \leq \epsilon_{\prf}
 \end{equation}%

 Note that the adversary cannot obtain the DH public key and $k_{mac}$ because the adversary obtain nothing from the transcription due to randomization of the initial key $\ik$. These changes prevent trivially attack. If the adversary obtain the MAC key $k_{mac}$, the adversary can hijack the session following way: the adversary generate $T_c$, $\NONC$ by himself and calculate MAC generated by MAC key $k_{mac}$ and send these value with $\SCID$. The server cannot reject this query made by the adversary because MAC is valid. The adversary can share the secret key with the server and always return correct $b^{\prime}$.
\vspace{10pt}\\%
%
%
 \textbf{Game 6 + $4\ell$.} Now we use that the key $\ik_{\pindexell}$ and $\ik_{\qindexell}$ in the symmetric encryption scheme uniformly at random and independent of all QUIC handshake messages. In this we replace the value $k_{mac}$ with another random value $\widetilde{k_{mac}}$.

 Suppose that there exists an algorithm $\mathcal{A}$ distinguishes Game 6 + 4 $\ell$ from Game 5 + 4 $\ell$. Then we can construct an algorithm $\mathcal{B}$ breaking $\LHAE$ secure. By assumption, the simulator $\mathcal{B}$ is given access to an encryption oracle $\Encrypt$ and a decryption oracle $\Decrypt$.

 Since by assumption any attacker has advantage at most $\epsilon_{\LHAE}$ in breaking the $\LHAE$ security of the symmetric encryption scheme, we have
 \begin{equation}
  |\Adv_{6 + 4\ell} - \Adv_{5 + 4\ell}| \leq \epsilon_{\LHAE}.
 \end{equation}%
 After this game, we add $\ell = \ell + 1$. If $\ell \leq \ell^{\ast}$ we repeat the game.
\vspace{10pt}\\%
%
%
 For next step, there are two cases. The first case is that the adversary issue $\Encrypt$-query $\pOracleAstRes$ or $\qOracleAstRes$ whose state $\Lambda = \preaccept$. The second case is that the adversary issue $\Encrypt$-query $\pOracleAstRes$ or $\qOracleAstRes$ whose state $\Lambda = \accept$.

 We define the first case as Game$_{a}$ and the second case as Game$_{b}$
\vspace{10pt}\\%
%
%
 \textbf{Game$_a$ 7 + 4$\ell^{\ast}$.} Now we use that the key $\ik_{\pindexell}$ and $\ik_{\qindexell}$ which is independent of all QUIC handshake messages.

 In this game we construct a simulator $\mathcal{B}$ that uses a RSACCE attacker $\mathcal{A}$ to break the security of the underlying $\LHAE$ secure symmetric encryption scheme. By assumption, the simulator $\mathcal{B}$ is given access to an encryption oracle $\Encrypt$ and a decryption oracle $\Decrypt$. $\mathcal{B}$ embeds that $\LHAE$ experiment by simply forwarding all $\Encrypt(\pOracleAstRes,\cdot)$ queries to $\Encrypt$, and all $\Decrypt(\qOracleAstRes,\cdot)$ queries to $\Decrypt$. Otherwise it proceeds as the challenger in Game 6.

 Observe that the values generated in this game are exactly distributed as in the previous game. We thus have
 \begin{equation}
  \Adv_{7 + 4\ell^{\ast}} = \Adv_{6 + 4\ell^{\ast}}
 \end{equation}%
 If $\mathcal{A}$ outputs a quad $\pindexout$, then $\mathcal{B}$ forwards $b^{\prime}$ to the $\LHAE$ challenger. Otherwise it outputs a random bit. Since the simulator essentially relays all messages it is easy to see that an attacker $\mathcal{A}$ having advantage $\epsilon^{\prime}$ yields an attacker $\mathcal{B}$ against the $\LHAE$ security of the encryption scheme with success probability at least $1/2 + \epsilon^{\prime}$.

 Since by assumption any attacker has advantage at most $\epsilon_{\LHAE}$ in breaking the $\LHAE$ security of the symmetric encryption scheme, we have
 \begin{equation}
  \Adv_{7 + 4\ell^{\ast}} \leq \frac{1}{2} + \epsilon_{\LHAE}.
 \end{equation}%
%
%
 \textbf{Game$_b$ 7 + 4$\ell^{\ast}$.} The challenger in this game proceeds as before, however replaces the premaster secret $pms_{k}$ of $\pOracleAstRes$ and $\qOracleAstRes$ with a random group element $\widetilde{pms_{k}} = g^w$, $w \xleftarrow{\$} \Zset_p$. Note that both $g^u$ and $g^v$ are chosen by oracles $\pOracleAstRes$ and $\qOracleAstRes$, respectively, as otherwise $\pOracleAstRes$ would not have a matching conversations for a final key with $\qOracleAstRes$ and the game would be aborted.

 Distinguish Game$_b$ 7 + 4$\ell^{\ast}$ from Game 6 + 4$\ell^{\ast}$ implies an algorithm breaking the security of the decisional Diffie-Hellman problem, thus
 \begin{equation}
  |\Adv_{7 + 4\ell^{\ast}} - \Adv_{6 + 4\ell^{\ast}}| \leq \epsilon_{\ddh}
 \end{equation}%
%
%
 \textbf{Game$_b$ 8 + 4$\ell^{\ast}$.} In this game we make use of the fact that the premaster secret $\widetilde{pms_{k}}$ of $\pOracleAstRes$ and $\qOracleAstRes$ is chosen uniformly random. We thus replace the value $ms_{k} = \PRF(\widetilde{pms_{k}}, \NONC)$ with a random value $\widetilde{ms_{k}}$.

 Distinguish Game$_b$ 8 + 4$\ell^{\ast}$ from Game$_b$ 7 + 4$\ell^{\ast}$ implies an algorithm breaking the security of the pseudo random function $\PRF$, thus
 \begin{equation}
  |\Adv_{8 + 4\ell^{\ast}} - \Adv_{7 + 4\ell^{\ast}}| \leq \epsilon_{\prf}
 \end{equation}%
%
%
 \textbf{Game$_b$ 9 + 4$\ell^{\ast}$} In this game we replace the all function $\PRF(\widetilde{ms_{k}}, \cdot)$ used by $\pOracleAstRes$ and $\qOracleAstRes$ with a random function $F_{\widetilde{ms_{k}}}$. Of course the same random function is used for both oracles $\pOracleAstRes$ and $\qOracleAstRes$. Distinguishing Game$_b$ 9 + 4$\ell^{\ast}$ from Game$_b$ 8 + 4$\ell^{\ast}$ again implies an algorithm breaking the security of the pseudo random function $\PRF$.
 \begin{equation}
  |\Adv_{9 + 4\ell^{\ast}} - \Adv_{8 + 4\ell^{\ast}}| \leq \epsilon_{\prf}
 \end{equation}%
%
%
 \textbf{Game$_b$ 10 + 4$\ell^{\ast}$.} Now we use that the key $k_{\pindexell}$ and $k_{\qindexell}$ which is independent of all QUIC handshake messages.

 In this game we construct a simulator $\mathcal{B}$ that uses a RSACCE attacker $\mathcal{A}$ to break the security of the underlying $\sLHAE$ secure symmetric encryption scheme. By assumption, the simulator $\mathcal{B}$ is given access to an encryption oracle $\Encrypt$ and a decryption oracle $\Decrypt$. $\mathcal{B}$ embeds that $\sLHAE$ experiment by simply forwarding all $\Encrypt(\pOracleAstRes,\cdot)$ queries to $\Encrypt$, and all $\Decrypt(\qOracleAstRes,\cdot)$ queries to $\Decrypt$.

 Observe that the values generated in this game are exactly distributed as in the previous game. We thus have
 \begin{equation}
  \Adv_{10 + 4\ell^{\ast}} = \Adv_{9 + 4\ell^{\ast}}
 \end{equation}%
 If $\mathcal{A}$ outputs a quad $\pindexout$, then $\mathcal{B}$ forwards $b^{\prime}$ to the $\sLHAE$ challenger. Otherwise it outputs a random bit. Since the simulator essentially relays all messages it is easy to see that an attacker $\mathcal{A}$ having advantage $\epsilon^{\prime}$ yields an attacker $\mathcal{B}$ against the $\LHAE$ security of the encryption scheme with success probability at least $1/2 + \epsilon^{\prime}$.

 Since by assumption any attacker has advantage at most $\epsilon_{\LHAE}$ in breaking the $\LHAE$ security of the symmetric encryption scheme, we have
 \begin{equation}
  \Adv_{10 + 4\ell^{\ast}} \leq \frac{1}{2} + \epsilon_{\LHAE}.
 \end{equation}%
\end{proof}