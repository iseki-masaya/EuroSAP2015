%=====================================================
\section{Our proposed scheme} \label{sec:quic_tls}
%=====================================================

\begin{figure*}[!htp]
 \begin{center}

\begin{enumerate}
 \item{Initiate} \\
 \fbox{
  \begin{minipage}[t]{0.38\textwidth}
  \centering
   \begin{tabular}{c}
    $ $ \\
    $ $ \\
    $ $ \\
   \end{tabular}
  \end{minipage}%
 }
 \begin{minipage}[t]{0.13\textwidth}
  \centering
  \begin{tabular}{c}
   $ $ \\
  \end{tabular}
 \end{minipage}%
 \fbox{
  \begin{minipage}[t]{0.38\textwidth}
   \centering
   \begin{tabular}{c}
    $(pk_s, sk_s) = \SIG.\Gen()$ \\
    $(\SCFG_{pub}, t_s) = \scfgGen(sk_s)$ \\
    $k_{\STK} \xleftarrow{\$} \{0,1\}^{\lambda}$ \\
   \end{tabular}
  \end{minipage}%
 }
 \item{Initial Key Agreement} \\
 \fbox{
  \begin{minipage}[t]{0.38\textwidth}
  \centering
   \begin{tabular}{c}
    $m_1 = \inchoateCHLO()$ \\
    $ $ \\
    $\shareInfo = (\NONC, \cid, T_s, t_c) $ \\
    $\ik = \getKey_c(\shareInfo, m_1, 1) $ \\
    $T_s^{\prime} = \receiveSHLO(m_2, \ik)$ \\
    $\shareInfo = (\NONC, \cid, T_s^{\prime}, t_c)$ \\
    $m = m_1 \| m_2$ \\
    $k = \getKey_c(\shareInfo, m, 0)$ \\
   \end{tabular}
  \end{minipage}%
 }
 \begin{minipage}[t]{0.13\textwidth}
  \centering
  \begin{tabular}{c}
   $\xrightarrow{m_1}$ \\
   $ $ \\
   $\xleftarrow{m_2}$ \\
   $ $ \\
   $ $ \\
   $ $ \\
   $ $ \\
  \end{tabular}
 \end{minipage}%
 \fbox{
  \begin{minipage}[t]{0.38\textwidth}
   \centering
   \begin{tabular}{c}
    $ $ \\
    $\shareInfo = (\NONC, \cid, T_c, t_s)$ \\
    $\ik = \getKey_s(\shareInfo, m_1, 1)$ \\
    $ret = \SHLO(m_1, \ik, 0)$ \\
    $m_2 = ret \| \SCFG_{pub} $ \\
    $\shareInfo = (\NONC, \cid, T_c, t_s^{\prime}) $ \\
    $m = m_1 \| m_2$ \\
    $k=\getKey_s(\shareInfo, m, 0)$ \\
   \end{tabular}
  \end{minipage}%
 }
 \item{Data Exchange} \\
 \fbox{
  \begin{minipage}[t]{0.38\textwidth}
  \centering
   \begin{tabular}{c}
    $ $ \\
    $ $ \\
    $ $ \\
    $\text{for each } \alpha \in {0,...,\MsgCntC{0}}$ \\
    $\sqn_c = \alpha + 1$ \\
    $m_3^{\alpha} = \pak(k, sqn_c, M_c^{\alpha})$ \\
    $ $ \\
    $m_3 = (m_3^{0},...,m_3^{\MsgCntC{0}})$ \\
    $\processPacket(k, m_4)$ \\
   \end{tabular}
  \end{minipage}%
 }
 \begin{minipage}[t]{0.13\textwidth}
  \centering
  \begin{tabular}{c}
   $ $ \\
   $ $ \\
   $ $ \\
   $ $ \\
   $\xrightarrow{m_3}$ \\
   $\xleftarrow{m_4}$ \\
  \end{tabular}
 \end{minipage}%
 \fbox{
  \begin{minipage}[t]{0.38\textwidth}
   \centering
   \begin{tabular}{c}
    $\text{If the first message in $m_3$ does not come} $ \\
    $\text{in $\duration$ QUIC regards the first query as}$ \\
    $\text{DoS attack and this connection is closed.}$ \\
    $\text{for each } \beta \in {0,...,\MsgCntS{0}}$ \\
    $\sqn_s = \beta + 1$ \\
    $m_4^{\beta} = \pak(k, \sqn_s, M_s^{\beta})$ \\
    $ $ \\
    $m_4 = (m_4^{\MsgCntS{0}+1},...,m_4^{\MsgCntS{1}})$ \\
    $\processPacket(ik, m_3)$ \\
   \end{tabular}
  \end{minipage}%
 }
 \item{Key Exchange} \\
 \fbox{
  \begin{minipage}[t]{0.38\textwidth}
  \centering
   \begin{tabular}{c}
    $m_5 = \CHLO(\STK, \SCFG_{pub}, k_{MAC})$ \\
    $\shareInfo = (\NONC, \cid, T_s, t_c^{\ast})$ \\
    $\ik = \getKey_c(\shareInfo, m_5, 1)$ \\
   \end{tabular}
  \end{minipage}%
 }
 \begin{minipage}[t]{0.13\textwidth}
  \centering
  \begin{tabular}{c}
   $\xrightarrow{m_5}$ \\
   $ $ \\
  \end{tabular}
 \end{minipage}%
 \fbox{
  \begin{minipage}[t]{0.38\textwidth}
   \centering
   \begin{tabular}{c}
    $\checkQuery(m_5, k_{\STK}, IP_c)$ \\
    $\shareInfo = (\NONC, \cid, T_c^{\ast}, t_s)$ \\
    $\ik = \getKey_s(\shareInfo, m_5, 1)$ \\
   \end{tabular}
  \end{minipage}%
 }
 \item{Initial Data Exchange after 0-RTT} \\
 \fbox{
  \begin{minipage}[t]{0.38\textwidth}
  \centering
   \begin{tabular}{c}
    $\text{for each } \alpha \in {\MsgCntC{0}+1,...,\MsgCntC{1}}$ \\
    $\sqn_c = \alpha + 2$ \\
    $m_{6}^{\alpha} = \pak(ik, sqn_c, M_c^{\alpha})$ \\
    $ $ \\
    $m_{6} = (m_{6}^{\MsgCntC{0}+1},...,m_{6}^{\MsgCntC{1}})$ \\
    $\processPacket(ik, m_{7})$ \\
   \end{tabular}
  \end{minipage}%
 }
 \begin{minipage}[t]{0.13\textwidth}
  \centering
  \begin{tabular}{c}
   $ $ \\
   $ $ \\
   $ $ \\
   $ $ \\
   $\xrightarrow{m_{6}}$ \\
   $\xleftarrow{m_{7}}$ \\
  \end{tabular}
 \end{minipage}%
 \fbox{
  \begin{minipage}[t]{0.38\textwidth}
   \centering
   \begin{tabular}{c}
    $\text{for each } \beta \in {\MsgCntS{0}+1,...,\MsgCntS{1}}$ \\
    $\sqn_s = \beta + 2$ \\
    $m_{7}^{\beta} = \pak(ik, \sqn_s, M_s^{\beta})$ \\
    $ $ \\
    $m_{7} = (m_{7}^{\MsgCntS{0}+1},...,m_{7}^{\MsgCntS{1}})$ \\
    $\processPacket(ik, m_{6})$ \\
   \end{tabular}
  \end{minipage}%
 }
 \item{Key Agreement after 0-RTT} \\
 \fbox{
  \begin{minipage}[t]{0.38\textwidth}
  \centering
   \begin{tabular}{c}
    $ $ \\
    $T_s^{\prime} = \receiveSHLO(m_8)$ \\
    $\shareInfo = (\NONC, \cid, T_s^{\prime}, t_c^{\ast})$ \\
    $m = m_5 \| m_8$ \\
    $k = \getKey_c(\shareInfo, m, 0)$ \\
   \end{tabular}
  \end{minipage}%
 }
 \begin{minipage}[t]{0.13\textwidth}
  \centering
  \begin{tabular}{c}
   $ $ \\
   $\xleftarrow{m_{8}}$ \\
   $ $ \\
  \end{tabular}
 \end{minipage}%
 \fbox{
  \begin{minipage}[t]{0.38\textwidth}
   \centering
   \begin{tabular}{c}
    $ \sqn_s = \MsgCntS{1} + 3$ \\
    $m_{8} = \SHLO(m_5, \ik, \sqn_s)$ \\
    $\shareInfo = (\NONC, \cid, T_c^{\ast}, t_s^{\prime})$ \\
    $m = m_5 \| m_8$ \\
    $k = \getKey_s(\shareInfo, m, 0)$ \\
   \end{tabular}
  \end{minipage}%
 }
 \item{Data Exchange after 0-RTT} \\
 \fbox{
  \begin{minipage}[t]{0.38\textwidth}
  \centering
   \begin{tabular}{c}
    $\text{for each } \alpha \in {\MsgCntC{1}+1,...,\MsgCntC{2}}$ \\
    $\sqn_c = \alpha + 3$ \\
    $m_{9}^{\alpha} = \pak(k, sqn_c, M_c^{\alpha})$ \\
    $ $ \\
    $m_{9} = (m_{9}^{\MsgCntC{1}+1},...,m_{9}^{\MsgCntC{2}})$ \\
    $\processPacket(k, m_{10})$ \\
   \end{tabular}
  \end{minipage}%
 }
 \begin{minipage}[t]{0.13\textwidth}
  \centering
  \begin{tabular}{c}
   $ $ \\
   $ $ \\
   $ $ \\
   $ $ \\
   $\xrightarrow{m_{9}}$ \\
   $\xleftarrow{m_{10}}$ \\
  \end{tabular}
 \end{minipage}%
 \fbox{
  \begin{minipage}[t]{0.38\textwidth}
   \centering
   \begin{tabular}{c}
    $\text{for each } \beta \in {\MsgCntS{1}+1,...,\MsgCntS{2}}$ \\
    $\sqn_s = \beta + 3$ \\
    $m_{10}^{\beta} = \pak(k, \sqn_s, M_s^{\beta})$ \\
    $ $ \\
    $m_{10} = (m_{10}^{\MsgCntS{1}+1},...,m_{10}^{\MsgCntS{2}})$ \\
    $\processPacket(ik, m_{9})$ \\
   \end{tabular}
  \end{minipage}%
 }
\end{enumerate}

 \end{center}
\end{figure*}
\subsubsection{Handshake operations}
\underline{$\scfgGen(sk_s)$:} \\
 $1.\ \ t_s \xleftarrow{\$} \Zset_{q}^{\ast}$ \\
 $2.\ \ T_s = g^{t_s}$ \\
 $3.\ \ \expy = \tau_{t+1}$ \\
 $4.\ \ \SCID = \Hash(T_s \| \expy)$ \\
 $5.\ \ str = \text{ QUIC server config signature }$ \\
 $6.\ \ \doc = str \| 0x00 \| \SCID \| T_s \| \expy$ \\
 $7.\ \ \sigma_s = \SIG.\Sign(sk_s, \doc)$ \\
 $8.\ \ \SCFG_{pub} = (\SCID, T_s, \expy, \sigma_s, \cert_s)$ \\
 $9.\ \ \return\ (\SCFG_{pub}, t_s)$ \\
\\
\underline{$\inchoateCHLO()$:} \\
 $1.\ \ \cid \xleftarrow{\$} \{0,1\}^{64} $ \\
 $2.\ \ \pInfo = (IP_c, IP_s, port_c, port_s)$ \\
 $3.\ \ t_c \xleftarrow{\$} \Zset_{q}^{\ast}$ \\
 $4.\ \ T_c = g^{t_c}$ \\
 $5.\ \ r \xleftarrow{\$} \{0,1\}^{160}$ \\
 $6.\ \ \NONC = currentTime \| r$ \\
 $7.\ \ \return\ (\pInfo, \cid, \NONC, T_c)$ \\
\\
\underline{$\SHLO(m, \ik, \sqn)$:} \\
 $1.\ \ (\pInfo, \cid, \NONC, T_c) = m$ \\
 $2.\ \ \STK = \makeSTK()$ \\
 $3.\ \ \pInfo = (IP_s, IP_c, port_s, port_c)$ \\
 $4.\ \ (\ik_c, \ik_s, \iv_c, \iv_s) = \ik$ \\
 $5.\ \ t_s^{\prime} \xleftarrow{\$} \Zset_{q}^{\ast}$ \\
 $6.\ \ T_s^{\prime} = g^{t_s^{\prime}}$ \\
 $7.\ \ \plaintext = T_{s}^{\prime} \| \STK \| k_{MAC} \| \SCID$\\
 $8.\ \ H = (\cid, \sqn)$ \\
 $9.\ \ c = \SE.\Enc(\ik_c, \iv_c \| \sqn, H, \plaintext)$ \\
 $10.\  \return\ (\pInfo, \cid, \SCFG_{pub}, H, c)$ \\
\\
\underline{$\makeSTK()$:} \\
 $1.\ \ \iv_{\STK} \xleftarrow{\$} \{0,1\}^{96}$ \\
 $2.\ \ k_{MAC} \xleftarrow{\$} \{0,1\}^{\mu}$ \\
 $3.\ \ \plaintext = IP_c \| currentTime \| k_{\MAC}$ \\
 $4.\ \ \STK \leftarrow \iv_{\STK} \| \SE.\Enc(k_{\STK}, len, \iv_{\STK}, \plaintext)$ \\
 $5.\ \ \return\ \STK$ \\
\\
\underline{$\receiveSHLO(m, \ik)$:} \\
 $1.\ \ (\pInfo, \cid, \SCFG_{pub}, H, c) = m$ \\
 $2.\ \ (\ik_c, \ik_s, \iv_c, \iv_s) = \ik$ \\
 $3.\ \ (\cid, \sqn) = H$ \\
 $4.\ \ \plaintext = \SE.\Dec(\ik_c, \iv_c \| \sqn, H, c)$ \\
 $5.\ \ \text{If }\plaintext = \perp$ \\
 $6.\ \ \quad \Lambda = \text{'reject' and abort}$ \\
 $7.\ \ T_{s}^{\prime} \| \STK \| k_{MAC} \| \SCID^{\prime} = \plaintext $ \\
 $8.\ \ \text{If }\SCID^{\prime} \neq \SCID$ \\
 $9.\ \ \quad \Lambda = \text{'reject' and abort}$ \\
 $10.\  \return\ T_s^{\prime}$ \\
\\
\underline{$\CHLO(\STK, \SCFG_{pub}, k_{MAC})$:} \\
 $1.\ \ \cid \xleftarrow{\$} \{0,1\}^{64}$ \\
 $2.\ \ r \xleftarrow{\$} \{0,1\}^{160}$ \\
 $3.\ \ \NONC = currentTime \| r$ \\
 $4.\ \ t_c^{\ast} \xleftarrow{\$} \Zset_{q}^{\ast}$ \\
 $5.\ \ T_c^{\ast} = g^{t_c^{\ast}}$ \\
 $6.\ \ \pInfo = (IP_c, IP_s, port_c, port_s)$ \\
 $7.\ \ \doc = T_c^{\ast}$ \| \NONC \| \cid \| \STK \| \SCID \\
 $8.\ \ \mac = \PRF(k_{MAC}, \doc) $ \\
 $9.\ \ \return\ (\pInfo, \cid, \STK, \SCID, \NONC, T_c^{\ast}, \mac)$ \\
\\
\underline{$\checkQuery(m, k_{\STK}, IP_c)$:} \\
 $1.\ \ (\pInfo, \cid, \STK, \SCID, \NONC, T_c^{\ast}, \mac) = m$ \\
 $2.\ \ (\iv_{\STK}, c) = \STK$ \\
 $3.\ \ IP_c^{\prime} \| currentTime \| k_{\MAC} = \SE.\Dec(k_{\STK}, \iv_{\STK}, c)$ \\
 $4.\ \ \doc = \cid \| \NONC \| T_c^{\ast}$ \\
 $5.\ \ \text{If } \PRF(k_{MAC}, \doc) \neq \mac$ \\
 $6.\ \ \quad \Lambda = \text{'reject' and abort}$ \\
 $7.\ \ (time_{\NONC}, r) = \NONC$ \\
 $8.\ \ \text{If } (IP_c^{\prime}, currentTime) = \perp$, or \\
 $9.\ \ \quad IP_c^{\prime} \neq IP_c$, or $time_{\STK} \leq time_{allowed}$\\
 $10.\  \quad r \in \strike$, or $time_{\NONC} \not\in \strike_{rng}$ \\
 $11.\  \quad \quad \Lambda = \text{'reject' and abort}$ \\
\\
\underline{$\checkSCFG(\SCFG_{pub})$:} \\
 $1.\ \ (\SCID, T_s, \expy, \sigma_s, \cert_s) = \SCFG_{pub}$ \\
 $2.\ \ \text{If } \expy \leq currentTime$ \\
 $3.\ \ \quad \Lambda = \text{'reject' and abort}$ \\
 $4.\ \ pk_s = \getPK(cert_s)$ \\
 $5.\ \ \text{If } \SIG.\Vfy(pk_s, \sigma_s, \doc) = \perp$ \\
 $6.\ \ \quad \Lambda = \text{'reject' and abort}$ \\
\\
\underline{$\checkQuery(\STK, k_{\STK}, \NONC, IP_c)$:} \\
 $1.\ \ (\iv_{\STK}, c) = \STK$ \\
 $2.\ \ (IP_c^{\prime}, time_{\STK}) = \SE.\Dec(k_{\STK}, \iv_{\STK}, c)$ \\
 $3.\ \ (time_{\NONC}, r) = \NONC$ \\
 $4.\ \ \text{If } (IP_c^{\prime}, currentTime) = \perp$, or \\
 $5.\ \ \quad IP_c^{\prime} \neq IP_c$, or $time_{\STK} \leq time_{allowed}$\\
 $6.\ \ \quad r \in \strike$, or $time_{\NONC} \not\in \strike_{rng}$ \\
 $7.\ \ \quad \quad \Lambda = \text{'reject' and abort}$ \\

As mentioned in Sec.~\ref{sec:quic_cetv}, many steps in CETV is useless to elevate the security of QUIC from RSACCE to strong RSACCE. These steps can be replaced with message authentication code keeping the security.
The abstract model of our proposed scheme is in Fig.~\ref{fig:quic_detail}.
% In contrast to original QUIC, the client send Diffie-Hellman public value in first query and cached $rms$ which is derived from master secret in this protocol.

\begin{theorem} \label{theorem:quic_tls}
 Let $\mu$ be the output length of $\PRF$, let $\lambda$ be the length of $\SCID$, let $\nclient$ be the number of clients, let $\nserver$ be the number of servers, let $\noracle$ be the number of oracles of each parties, and let $n_{\ell}$ be the maximum number of resumptions. Assume that the $\PRF$ is $(t, \epsilon_{\prf})$-pseudo-random function family, the signature shceme $\SIG$ is $(t, \epsilon_{\sig})$-secure against existentially unforgeable under adaptive chosen-message attacks, the DDH problem on $G$ is $(t, \epsilon_{\ddh})$-hard, the PRF-ODH problem on $\PRF$ is $(t, \epsilon_{\prfodh})$-hard, the hash function family $\mathcal{H}$ is $(t,\epsilon_{H})$-collision-resistant (CR), the stateful symmetric encryption scheme $\SE$ is $(t, \epsilon_{\sLHAE})$-secure.
 Then for all PPT adversaries, our proposed scheme is \textbf{strong} RSACCE secure.
\end{theorem}

Our proposed scheme can prevent the five attacks in~\cite{LJBN15:QUIC}.

\begin{itemize}
 \item{Server Config Replay Attack}
  In this attack, the adversary collect
 \item{Source-Address Token Replay Attack}
 \item{Connection ID Manipulation Attack}
 \item{Source-Address Token Manipulation Attack}
 \item{Crypto Stream Offset Attack}
\end{itemize}