\begin{proof}
 The proof proceeds in a sequence of games. \vspace{10pt}\\
 \textbf{Game 0.} This game equals the \textit{channel binding} security experiment.
 \begin{equation}
  \Adv_0 = \Adv^{\rsaccecb}_{P}(A)
 \end{equation}%
%
%
 \textbf{Game 1.} We try to guess which server oracle will be the first oracle to break \textit{channel binding}. If our guess is wrong, i.e. if there is another server oracle that breaks \textit{channel binding} before, then we abort the game.

 Technically, the game is identical to Game 1, except for the following. The challenger guesses three random indices $(s^{\ast}, j^{\ast}, \ell^{\ast}) \xleftarrow{\$} [\nserver] \times [\noracle] \times [n_{\ell}]$. If there exists a server oracle $\pi^s_{j,\ell}$ that breaks channel binding, and $(s, j, \ell) \neq s^{\ast}, j^{\ast}, \ell^{\ast})$, then the challenger aborts the game. Note that if the first oracle $\pi^s_{j,\ell}$ that breaks channel binding, then with probability $1/(\nserver \noracle n_{\ell})$ we have $(s,j,\ell) = (s^{\ast}, j^{\ast}, \ell^{\ast})$, and thus
 \begin{equation}
  \Adv_0 \leq \nserver \noracle n_{\ell} \Adv_1.
 \end{equation}%
 Note that in this game the attacker can only break the security of the protocol, if oracle $\pi^{s^{\ast}}_{j^{\ast},\ell^{\ast}}$ is the first oracle that breaks channel binding, as otherwise the game is aborted.
\vspace{10pt}\\%
%
%
 \textbf{Game 2 + $4\ell$.} We repeat the game until randomizing the session between $\pOracleAstRes$ and $\qOracleAstRes$. Initially $\ell = 0$. The reason of repeating is that the adversary always make the server accepts $\Lambda = \accept$ if the adversary can obtain $\key_{mac}$ of $\pOracleAstEll$ or $\qOracleAstEll$. We need to prevent the adversary obtaining all $\key_{mac}$ in $ 0 \leq \ell \leq \ell^{\ast}$.
 Let $T_{\pindexell} = g^u$ denote the Diffie-Hellman share chosen by $\pOracleAstEll$, let $T_{\qindexell} = g^v$ denote the share chosen by its partner $\qOracleAstEll$, and let $\ik_{\pindexell}$ and $k_{\pindexell}$ are the key computed by $\pOracleAstEll$, let $pms_{\ik}$ and $ms_{\ik}$ denote a premaster secret and master secret for initial key, let $pms_{\key}$ and $ms_{\key}$ denote a premaster secret and master secret for session key. Thus, both oracles compute the premaster secret as $pms = g^{uv}$.

 The challenger in this game proceeds as before, however replaces the premaster secret $pms_{\ik}$ of $\pOracleAstEll$ and $\qOracleAstEll$ with a random group element $\widetilde{pms_{\ik}} = g^w$, $w \xleftarrow{\$} \Zset_p$. Note that both $g^u$ and $g^v$ are chosen by oracles $\pOracleAstEll$ and $\qOracleAstEll$, respectively, as otherwise $\pOracleAstEll$ would not have a matching conversation to $\qOracleAstEll$ and the game would be aborted.

 Suppose that there exists an algorithm $\mathcal{A}$ distinguishes Game 2 + $4\ell$ from Game 1 + $4\ell$. Then we can construct an algorithm $\mathcal{B}$ solving the DDH problem as follows. $\mathcal{B}$ receives as input $(g,g^u,g^v,g^w)$. The challenger defines $T_{\pindexell} := g^u$ and $T_{\qindexell} := g^v$, and the premaster secret of $\pOracleAstEll$ and $\qOracleAstEll$ equal to $pms_{\ik} := g^w$. Note that $\mathcal{B}$ can simulate all messages exchanged between $\pOracleAstEll$ and $\qOracleAstEll$ properly. Since all other oracles are not modified, $\mathcal{B}$ can simulate these oracles properly as well.

 If $w=uv$, then the view of $\mathcal{A}$ when interacting with $\mathcal{B}$ is identical to Game 2 + $4\ell$, while if $w \xleftarrow{\$}\Zset_p$ then it is identical to Game 3 + $4\ell$. Thus, the DDH assumption implies that
 \begin{equation}
  |\Adv_{2 + 4\ell} - \Adv_{1 + 4\ell}| \leq \epsilon_{\ddh}
 \end{equation}%
%
%
 \textbf{Game 3 + $4\ell$.} In Game 3 + 4$\ell$ we make use of the fact that the premaster secret $\widetilde{pms_{\ik}}$ of $\pOracleAstEll$ and $\qOracleAstEll$ is chosen uniformly random. We thus replace the value $ms_{\ik} = \PRF(\widetilde{pms_{\ik}}, \NONC)$ with a random value $\widetilde{ms_{\ik}}$.

 Distinguish Game 3 + 4$\ell$ from Game 2 + 4$\ell$ implies an algorithm breaking the security of the pseudo random function $\PRF$, thus
 \begin{equation}
  |\Adv_{3 + 4\ell} - \Adv_{2 + 4\ell}| \leq \epsilon_{\prf}
 \end{equation}%
%
%
 \textbf{Game 4 + $4\ell$.} In this game we replace the function $\PRF(\widetilde{ms_{\ik}}, \cdot)$ used by $\pOracleAstEll$ and $\qOracleAstEll$ with a random function $F_{\widetilde{ms_{\ik}}}$. Of course the same random function is used for both oracles $\pOracleAstEll$ and $\qOracleAstEll$. Distinguishing Game 4 + $4\ell$ from Game 3 + $4\ell$ again implies an algorithm breaking the security of the pseudo random function $\PRF$.
 \begin{equation}
  |\Adv_{4 + 4\ell} - \Adv_{3 + 4\ell}| \leq \epsilon_{\prf}
 \end{equation}%

 Note that the adversary cannot obtain the Diffie-Hellman value $\key_{mac}$ because the adversary obtain nothing from the transcription due to randomization of the initial key $\ik$. These changes prevent trivially attack. If the adversary obtain the MAC key $\key_{mac}$, the adversary can hijack the session following way: the adversary generate $T_c$, $\NONC$ by himself and calculate MAC generated by MAC key $\key_{mac}$ and send these value with $\SCID$. The server cannot reject this query made by the adversary because MAC is valid. The adversary can share the secret key with the server and always return correct $b^{\prime}$.
\vspace{10pt}\\%
%
%
 \textbf{Game 5 + $4\ell$.} Now we use that the key $\ik_{\pindexell}$ and $\ik_{\qindexell}$ in the symmetric encryption scheme uniformly at random and independent of all QUIC handshake messages. In this we replace the value $k_{mac}$ with another random value $\widetilde{k_{mac}}$.

 Suppose that there exists an algorithm $\mathcal{A}$ distinguishes Game 5 + 4 $\ell$ from Game 4 + 4 $\ell$. Then we can construct an algorithm $\mathcal{B}$ breaking $\LHAE$ secure. By assumption, the simulator $\mathcal{B}$ is given access to an encryption oracle $\Encrypt$ and a decryption oracle $\Decrypt$.

 Since by assumption any attacker has advantage at most $\epsilon_{\LHAE}$ in breaking the $\LHAE$ security of the symmetric encryption scheme, we have
 \begin{equation}
  |\Adv_{5 + 4\ell} - \Adv_{4 + 4\ell}| \leq \epsilon_{\LHAE}.
 \end{equation}%
 After this game, we add $\ell = \ell + 1$. If $\ell \leq \ell^{\ast}$ we repeat the game.
\vspace{10pt}\\%
%
 \textbf{Game 6 + 4$\ell^{\ast}$.} Now the adversary does not have the way to let $\sOracleAstRes$ because the adversary cannot make mac generated by $\key_{mac}$. We thus have
 \begin{equation}
  |\Adv_{6 + 4\ell^{\ast}} - \Adv_{5 + 4\ell^{\ast}}| \leq \frac{1}{2^{\nu}}
 \end{equation}%
%
%
\end{proof}