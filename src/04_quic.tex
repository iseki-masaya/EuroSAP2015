%=====================================================
\section{Quick UDP Internet Connections} \label{sec:quic}
%=====================================================

Quick UDP Internet Connections (QUIC) is a protocol
developed by Google and this protocol is still under
development.
The concepts of QUIC are (1) to reduce connectivity
overheads before a client sends encrypted data and
(2) to obtain better security guarantee than TLS.
To realize concept (1), QUIC is not defined over
TCP but UDP, because TCP requires three-move handshake
before initiating a cryptographic handshake, but UDP
does not have. In addition, a client can send encrypted
data concurrently with CHLO which is second query of the
client.
To realize concept (2), QUIC supports only secure cipher
suites. Especially, a support algorithm of key exchange
is only ephemeral elliptic curve Diffie-Hellman.

There are two type connections, 1 round trip time (RTT)
connection and 0 round trip time connection, in QUIC.
The former case is called 1-RTT connection, the latter
case is called 0-RTT connection.
The first time a client connects to a server, the client
must perform 1-RTT handshake to acquire a necessary
information

%=====================================================
\subsection{1-RTT Connection Establishment} \label{sec:quic_1rtt}
%=====================================================

We provide the abstract model of QUIC for 1-RTT in
Fig.~\ref{fig:quic_abst_1rtt}.
%
\begin{figure*}[htb]
\begin{center}

\fbox{
\begin{minipage}[t]{0.39\textwidth}
\begin{tabular}[c]{l}
 $\quad Client$ \\
 $ $ \\
 \setcounter{nombre}{0}%
 $\prob.\quad \cid \xleftarrow{\$} \{0,1\}^{64} $ \\
 $\prob.\quad m_1 = \cid$ \\
 $\prob.\quad send\ m_1$ \\
 $ $\\
 \setcounter{nombre}{0}%
 $\prob.\quad receive\ m_2$ \\
 $\prob.\quad (\SCID, T_s, \sigma_s) = \SCFG_{pub}$ \\
 $\prob.\quad \doc = \SCID \| T_s$ \\
 $\prob.\quad \text{If } \SIG.\Vfy(pk_s, \sigma_s, \doc)$ \\
 $\prob.\quad \quad \Lambda = \text{'reject' and abort}$ \\
 $\prob.\quad t_c \xleftarrow{\$} \Zset_{q}^{\ast} $ \\
 $\prob.\quad T_c = g^{t_c} $ \\
 $\prob.\quad \NONC \xleftarrow{\$} \{0,1\}^{160} $ \\
 $\prob.\quad m_3 = (T_c, \NONC, \STK, \SCID, \cid)$ \\
 $\prob.\quad send\ m_3$ \\
 $\prob.\quad pms = T_s^{t_c}$ \\
 $\prob.\quad ms = \PRF(pms, NONC)$ \\
 $\prob.\quad \ik = \PRF(ms, m_1 \| m_2 \| m_3)$ \\
 $\prob.\quad \Lambda = \preaccept$ \\
\end{tabular}
\end{minipage}%
}
% middle
 \begin{minipage}[t]{0.13\textwidth}
  \centering
  \begin{tabular}{c}
   $ $ \\
   $ $ \\
   $ $ \\
   $\xrightarrow{m_1}$ \\
   $ $ \\
   $\xleftarrow{m_2}$ \\
   $ $ \\
   $ $ \\
   $ $ \\
   $ $ \\
   $ $ \\
   $ $ \\
   $\xrightarrow{m_3}$ \\
   $ $ \\
  \end{tabular}
 \end{minipage}%
\fbox{
\begin{minipage}[t]{0.39\textwidth}
\begin{tabular}[c]{l}
 $\quad Server$ \\
 $ $ \\
 $ $ \\
 $ $ \\
 \setcounter{nombre}{0}%
 $\prob.\quad receive\ m_1$ \\
 $\prob.\quad \text{choose } \SCFG = (\SCFG_{pub}, t_s) $\\
 $\prob.\quad (\SCID, T_s, \sigma_s) = \SCFG_{pub}$ \\
 $\prob.\quad \makeSTKQUIC$ \\
 $\prob.\quad m_2 = (\SCFG_{pub}, \STK, \cid)$ \\
 $\prob.\quad send\ m_2$ \\
 $ $ \\
 $ $ \\
 $ $ \\
 $ $ \\
 $ $ \\
 \setcounter{nombre}{0}%
 $\prob.\quad receive\ m_3$ \\
 $\prob.\quad \text{check $\STK$ using $\key_{\STK}$}$ \\
 $\prob.\quad \text{search $\SCFG$ with $\SCID$}$ \\
 $\prob.\quad pms = T_c^{t_s}$ \\
 $\prob.\quad ms = \PRF(pms, \NONC)$ \\
 $\prob.\quad \ik = \PRF(ms, m_1 \| m_2 \| m_3)$ \\
 $\prob.\quad \Lambda = \preaccept$ \\
\end{tabular}
\end{minipage}%
} \vspace{10pt}

% 1-RTT connection establishment for final key
% \vspace{10pt}\\

% \ONERTTtrue
% \ORIGINALtrue
% \fbox{
\begin{minipage}[t]{0.39\textwidth}
\begin{tabular}[c]{l}
 $ $ \\
 $ $ \\
 $ $ \\
 $ $ \\
 \setcounter{nombre}{0}%
\ifONERTT
 $\prob.\quad receive\ m_4$ \\
\else
 $\prob.\quad receive\ m_2$ \\
\fi
 $\prob.\quad T_s^{\prime} \| \STK = \SE.\Dec(\ik, c)$ \\
 $\prob.\quad pms^{\prime} = T_s^{\prime t_c} $ \\
 $\prob.\quad ms^{\prime} = \PRF(pms^{\prime}, \NONC) $ \\
\ifONERTT
 $\prob.\quad \key = \PRF(ms^{\prime}, m_1 \| m_2 \| m_3 \| m_4)$ \\
\else
 $\prob.\quad \key = \PRF(ms^{\prime}, m_1 \| m_2)$ \\
\fi
 $\prob.\quad \Lambda = \accept$ \\
 $\prob.\quad \theta = (\SCFG_{pub}, \STK)$ \\
\end{tabular}
\end{minipage}%
}
\begin{minipage}[t]{0.13\textwidth}
\centering
\begin{tabular}[c]{l}
 $ $\\
 $ $\\
 $ $\\
\ifONERTT
 $\xleftarrow{m_4}$\\
\else
 $\xleftarrow{m_2}$\\
\fi
 $ $\\
 $ $\\
 $ $\\
 $ $\\
 $ $\\
\end{tabular}
\end{minipage}%
\fbox{
\begin{minipage}[t]{0.39\textwidth}
\begin{tabular}[c]{l}
 \setcounter{nombre}{0}%
 $\prob.\quad t_s^{\prime} \xleftarrow{\$} \Zset_{q}^{\ast}$ \\
 $\prob.\quad T_s^{\prime} = g^{t_s^{\prime}}$ \\
 $\prob.\quad \makeSTKQUIC$ \\
 $\prob.\quad \plaintext = T_s^{\prime} \| \STK$ \\
 $\prob.\quad c = (\SE.\Enc(\ik, \plaintext)$ \\
\ifONERTT
 $\prob.\quad m_4 = (c, \cid)$ \\
 $\prob.\quad send\ m_4$ \\
\else
 $\prob.\quad m_2 = (c, \cid)$ \\
 $\prob.\quad send\ m_2$ \\
\fi
 $\prob.\quad pms^{\prime} = T_c{t_s^{\prime}} $ \\
 $\prob.\quad ms^{\prime} = \PRF(pms^{\prime}, \NONC) $ \\
\ifONERTT
 $\prob.\quad \key = \PRF(ms^{\prime}, m_1 \| m_2 \| m_3 \| m_4)$ \\
\else
 $\prob.\quad \key = \PRF(ms^{\prime}, m_1 \| m_2)$ \\
\fi
 $\prob.\quad \Lambda = \accept$ \\
\end{tabular}
\end{minipage}%
} \vspace{10pt}

\caption{Abstract model of 1-RTT connection establishment for an initial key in QUIC handshake}\label{fig:quic_abst_1rtt_init}
\end{center}
\end{figure*}
%
We define five phases of QUIC handshake in 1-RTT.
(1) \textbf{Initiate},
(2) \textbf{Initial Key Agreement},
(3) \textbf{Initial Data Exchange},
(4) \textbf{Key Agreement}, and
(5) \textbf{Data Exchange}.

%=====================================================
\subsubsection{Initiate}
%=====================================================
In this phase, a client do nothing and a server make
$k_{\STK}$ and run $\scfgGen$ and $\SIG.\Gen$ to
generate server config (SCFG) and long term secret
and public key.
SCFG is composed of seven parameters AEAD, SCID, PDMD,
PUBS, KEXS, and OBIT, and EXPY. The important parameters
are SCID which is an opaque 16 byte identifier for
this server config, PUBS which is server's
Diffie-Hellman public value, and EXPY which is expiry time
for this server config. The details of other parameters
are described in~\cite{QUIC:Crypto}.
Our definition consider only important parameters.
\\
\noindent
\underline{$\scfgGen(sk_s)$:} \\
 $1.\ \ t_s \xleftarrow{\$} \Zset_{q}^{\ast}$ \\
 $2.\ \ T_s = g^{t_s}$ \\
 $3.\ \ \SCID = \Hash(T_s \| \expy)$ \\
 $4.\ \ str = \text{ QUIC server config signature }$ \\
 $5.\ \ \doc = str \| 0x00 \| \SCID \| T_s \| \expy$ \\
 $6.\ \ \sigma_s = \SIG.\Sign(sk_s, \doc)$ \\
 $7.\ \ \SCFG_{pub} = (\SCID, T_s, \expy, \sigma_s, \cert_s)$ \\
 $8.\ \ \return\ (\SCFG_{pub}, t_s)$ \\
%
Note that the generation of $\SCFG$ and the signing
of its public parameters are done independently of
client's connection requests.
%=====================================================
\subsubsection{Initial Key Agreement}
%=====================================================
In this phase, the client sends an inchoate client
hello (inchoateCHLO) which contains connection id
and some information such as server name, protocol
version, and user agent id. In our definition,
the some informations are omitted.
\\
\noindent
\underline{$\inchoateCHLO()$:} \\
 \setcounter{nombre}{0}%
 $\prob.\quad \cid \xleftarrow{\$} \{0,1\}^{64} $ \\
 $\prob.\quad \pInfo = (IP_c, IP_s, port_c, port_s)$ \\
 $\prob.\quad \return\ (\pInfo, \cid)$ \\
%
After the server receives inchoate client hello, it
sends a rejection (REJ). The rejection (REJ) contains
source address token (STK), server config (SCFG),
a certificate, and a signature of server config generated
by the server long term secret key. The client use
STK in future queries to demonstrate ownership of their
source IP address.
\\
\noindent
\underline{$\REJ(m, \SCFG_{pub})$:} \\
 \setcounter{nombre}{0}%
 $\prob.\quad (\pInfo, \cid) = m$ \\
 $\prob.\quad \STK = \makeSTK()$ \\
 $\prob.\quad \pInfo = (IP_s, IP_c, port_s, port_c)$ \\
 $\prob.\quad \return\ (\pInfo, \cid, \SCFG_{pub}, \STK)$ \\
\underline{$\makeSTK()$:} \\
 \setcounter{nombre}{0}%
 $\prob.\quad \iv_{\STK} \xleftarrow{\$} \{0,1\}^{96}$ \\
 $\prob.\quad \plaintext = IP_c \| currentTime$ \\
 $\prob.\quad \STK \leftarrow \iv_{\STK} \|
        \SE.\Enc(k_{\STK}, len,\iv_{\STK},
        \plaintext)$ \\
 $\prob.\quad \return\ \STK$ \\
%
After the client receives a rejection (REJ), the client
checks the server config and generate NONC which consists
of a random value and current time, and ephemeral
Diffie-Hellman values.
The client sends a client hello (CHLO) to the server.
We define a client hello in 1-RTT connection as initialCHLO.
\\
\noindent
\underline{$\initialCHLO(m)$:} \\
 \setcounter{nombre}{0}%
 $\prob.\quad (\pInfo, \cid, \SCFG_{pub}, \STK) = m$ \\
 $\prob.\quad \pInfo = (IP_c, IP_s, port_c, port_s)$ \\
 $\prob.\quad t_c \xleftarrow{\$} \Zset_{q}^{\ast}$ \\
 $\prob.\quad T_c = g^{t_c}$ \\
 $\prob.\quad r \xleftarrow{\$} \{0,1\}^{160}$ \\
 $\prob.\quad \NONC \leftarrow currentTime \| r$ \\
 $\prob.\quad \return\ (\pInfo, \cid, \STK, \SCID, \NONC, T_c)$ \\
\underline{$\checkSCFG(\SCFG_{pub})$:} \\
 \setcounter{nombre}{0}%
 $\prob.\quad (\SCID, T_s, \expy, \sigma_s, \cert_s) = \SCFG_{pub}$ \\
 $\prob.\quad \text{If } \expy \leq currentTime$ \\
 $\prob.\quad \quad \Lambda = \text{'reject' and abort}$ \\
 $\prob.\quad pk_s = \getPK(cert_s)$ \\
 $\prob.\quad str = \text{ QUIC server config signature }$ \\
 $\prob.\quad \doc = str \| 0x00 \| \SCID \| T_s \| \expy$ \\
 $\prob.\quad \text{If } \SIG.\Vfy(pk_s, \sigma_s, \doc) = \perp$ \\
 $\prob.\quad \quad \Lambda = \text{'reject' and abort}$ \\
%
After the server receives a initial client hello, the
server check this query. $\STK$ is an opaque byte string
from the client's point of view. From the server's point
of view it's an authenticated-encryption block that
contains, at least, the client's IP address and a time
stamp by the server. The server decrypt $\STK$ and
validate time and source IP address. $\NONC$ is
concatenation of time and a random value. The server
register the time and random value and validate each
$\NONC$ to ensure that it does not process the same
connection twice. This checks prevent a part of replay
attacks.
\\
\noindent
\underline{$\checkQuery(\STK, k_{\STK}, \NONC, IP_c)$:} \\
 \setcounter{nombre}{0}%
 $\prob.\quad (\iv_{\STK}, c) = \STK$ \\
 $\prob.\quad (IP_c^{\prime}, time_{\STK}) = \SE.\Dec(k_{\STK}, \iv_{\STK}, c)$ \\
 $\prob.\quad (time_{\NONC}, r) = \NONC$ \\
 $\prob.\quad \text{If } (IP_c^{\prime}, currentTime) = \perp$, or \\
 $\prob.\quad \quad IP_c^{\prime} \neq IP_c$, or $time_{\STK} \leq time_{allowed}$\\
 $\prob.\quad \quad r \in \strike$, or $time_{\NONC} \not\in \strike_{rng}$ \\
 $\prob.\quad \quad \quad \Lambda = \text{'reject' and abort}$ \\
%
After the server validate initial client hello or the
client sends initial client hello, they calculate initial
key.
\noindent
\underline{$\getKey_c(\shareInfo, m, \init)$:} \\
 \setcounter{nombre}{0}%
 $\prob.\quad (\NONC, \cid ,T_s, t_c) = \shareInfo$ \\
 $\prob.\quad pms = T_s^{t_c}$ \\
 $\prob.\quad \return\ \extractKey(pms, \NONC, \cid, m, 40, \init)$ \\
\underline{$\getKey_s(\shareInfo, m, \init)$:} \\
 \setcounter{nombre}{0}%
 $\prob.\quad (\NONC, \cid ,T_c, t_s) = \shareInfo$ \\
 $\prob.\quad pms = T_c^{t_s}$ \\
 $\prob.\quad \return\ \extractKey(pms, \NONC, \cid, m, 40, \init)$ \\
\underline{$\extractKey(pms, \NONC, \cid, m, \ell, \init)$:}\\
 \setcounter{nombre}{0}%
 $\prob.\quad ms = \PRF(pms, \NONC)$ \\
 $\prob.\quad \text{If } \init = 1$ \\
 $\prob.\quad \quad str = \text{ QUIC key expansion }$ \\
 $\prob.\quad \text{Else }$ \\
 $\prob.\quad \quad str = \text{ QUIC forward secure expansion }$ \\
 $\prob.\quad \info = str \| 0x00 \| \cid \| m \| \SCFG_{pub}$ \\
 $\prob.\quad \return\ \text{the first $\ell$ octets (i.e. bytes) of T = }$ \\
 $\quad \quad \text{(T(1),T(2), ...), where for all $i \in \Nset$, $T(i) = $} $\\
 $\quad \quad \text{$\PRF(ms, T(i-1) \| \info \| 0x0i)$ and $T(0) = \epsilon$} $\\
%=====================================================
\subsubsection{Initial Data Exchange}
%=====================================================
In this phase, the client and server exchange data
encrypted and authenticated using Length-Hiding
Authenticated Encryption $\SE$ with initial key $\ik$.
They encrypt data using $\pak$ and decrypt data using
$\processPacket$.
\\
\noindent
\underline{$\getIV(H, \kappa, P)$:} \\
 $1.\ \ (k_c, k_s, \iv_c, \iv_s) = \kappa$ \\
 $2.\ \ \text{If } P \in \Client$ \\
 $3.\ \ \quad src = c, dst = s$ \\
 $4.\ \ \text{Else if} P \in \Server$ \\
 $5.\ \ \quad src = s, dst = c$ \\
 $6.\ \ (\cid, \sqn) = H$ \\
 $7.\ \ \return\ (\iv_{dst}, \sqn)$ \\
\underline{$\pak(\kappa, \sqn, m)$:} \\
 $1.\ \ (k_c, k_s, \iv_c, \iv_s) = \kappa$ \\
 $2.\ \ \text{If } P \in \Client$ \\
 $3.\ \ \quad src = c, dst = s$ \\
 $4.\ \ \text{Else if } P \in \Server$ \\
 $5.\ \ \quad src = s, dst = c$ \\
 $6.\ \ \pInfo = (IP_{src}, IP_{dst}, port_{src}, port_{dst})$ \\
 $7.\ \ H = (\cid, \sqn)$ \\
 $8.\ \ \iv = \getIV(H, \kappa)$ \\
 $9.\ \ \return\ (\pInfo, \SE.\Enc(k_{dst}, \iv, H, m) )$ \\
\underline{$\processPacket(\kappa, p_1,...,p_v)$:} \\
 $1.\ \ (k_c, k_s, \iv_c, \iv_s) = \kappa$ \\
 $2.\ \ \text{If } P \in \Client$ \\
 $3.\ \ \quad src = c, dst = s$ \\
 $4.\ \ \text{Else if} P \in \Server$ \\
 $5.\ \ \quad src = s, dst = c$ \\
 $6.\ \ \text{for each } \gamma \in [v]$ \\
 $7.\ \ \quad (H_{\gamma}, c_{\gamma}) = p_{\gamma}$ \\
 $8.\ \ \quad \iv_{\gamma} = \getIV(H_{\gamma}, \kappa)$ \\
 $9.\ \ \quad m_{\gamma} = \SE.\Dec(k_{src}, \iv_{\gamma}, H_{\gamma}, c_{\gamma})$ \\
%=====================================================
\subsubsection{Key Agreement}
%=====================================================
In this phase, the server run $\SHLO$ to share forward
secure key $\key$ and the client validate the query running
$\receiveSHLO$.
\\
\noindent
\underline{$\SHLO(m, \ik, \sqn)$:} \\
 $1.\ \ (\pInfo, \cid, \STK, \SCID, \NONC, T_c) = m$ \\
 $2.\ \ (\ik_c, \ik_s, \iv_c, \iv_s) = \ik$ \\
 $3.\ \ t_s^{\prime} \xleftarrow{\$} \Zset_{q}^{\ast}$ \\
 $4.\ \ T_s^{\prime} = g^{t_s^{\prime}}$ \\
 $5.\ \ \STK = \makeSTK()$ \\
 $6.\ \ H = (\cid, \sqn)$ \\
 $7.\ \ \plaintext = \SCFG_{pub} \| T_s^{\prime} \| \STK $\\
 $8.\ \ c = \SE.\Enc(\ik_c, \iv_c \| \sqn, H, \plaintext)$ \\
 $9.\ \ \return\ (\pInfo, H, c)$ \\
\underline{$\receiveSHLO(m, \ik)$:} \\
 $1.\ \ (\pInfo, H, c) = m$ \\
 $2.\ \ (\ik_c, \ik_s, \iv_c, \iv_s) = \ik$ \\
 $3.\ \ (\cid, \sqn) = H$ \\
 $4.\ \ \plaintext = \SE.\Dec(\ik_c, \iv_c \| \sqn, H, c)$ \\
 $5.\ \ \text{If }\plaintext = \perp$ \\
 $6.\ \ \quad \Lambda = \text{'reject' and abort}$ \\
 $7.\ \ \SCFG_{pub} \| T_s^{\prime} \| \STK  = \plaintext $ \\
 $8.\ \ \return\ T_s^{\prime}$ \\
%=====================================================
\subsubsection{Data Exchange}
%=====================================================
In this phase, the client and server exchange data
encrypted and authenticated using Length-Hiding
Authenticated Encryption $\SE$ with forward secure key
$\key$.
They encrypt data using $\pak$ and decrypt data using
$\processPacket$ defined previously.

%=====================================================
\subsection{0-RTT Connection Establishment} \label{sec:quic_0rtt}
%=====================================================

We provide the abstract model of QUIC for 0-RTT in
Fig.~\ref{fig:quic_abst_0rtt}.
%
\begin{figure*}[htb]
\begin{center}

0-RTT connection establishment for initial key
\vspace{10pt}\\

\fbox{
\begin{minipage}[t]{0.39\textwidth}
\begin{tabular}[c]{l}
 \setcounter{nombre}{0}%
 $\quad Client$ \\
 $ $ \\
 $\prob.\quad \theta = (\SCFG_{pub}, \STK)$ \\
 $\prob.\quad \cid \xleftarrow{\$} \{0,1\}^{64} $ \\
 $\prob.\quad t_c \xleftarrow{\$} \Zset_{q}^{\ast} $ \\
 $\prob.\quad T_c = g^{t_c} $ \\
 $\prob.\quad \NONC \xleftarrow{\$} \{0,1\}^{160} $ \\
 $\prob.\quad m_1 = (T_c, \NONC, \cid, \STK, \SCID)$ \\
 $\prob.\quad pms = T_s^{t_c}$ \\
 $\prob.\quad ms = \PRF(pms, \NONC)$ \\
 $\prob.\quad \ik = \PRF(ms, m_1)$ \\
 $\prob.\quad \Lambda = \preaccept$ \\
\end{tabular}
\end{minipage}%
}
% middle
 \begin{minipage}[t]{0.13\textwidth}
  \centering
  \begin{tabular}{c}
   $ $ \\
   $ $ \\
   $ $ \\
   $ $ \\
   $ $ \\
   $\xrightarrow{m_1}$ \\
   $ $ \\
   $ $ \\
   $ $ \\
   $ $ \\
  \end{tabular}
 \end{minipage}%
\fbox{
\begin{minipage}[t]{0.39\textwidth}
\begin{tabular}[c]{l}
 $\quad Server$ \\
 $ $ \\
 $ $\\
 $ $\\
 $ $\\
 $ $\\
 $ $\\
 $ $\\
 \setcounter{nombre}{0}%
 $\prob.\quad \text{search $\SCFG$ with $\SCID$}$ \\
 $\prob.\quad pms = T_c^{t_s}$ \\
 $\prob.\quad ms = \PRF(pms, \NONC)$ \\
 $\prob.\quad \ik = \PRF(ms, m_1)$ \\
 $\prob.\quad \Lambda = \preaccept$ \\
\end{tabular}
\end{minipage}%
} \vspace{10pt}

0-RTT connection establishment for forward secure key
\vspace{10pt}\\

\ONERTTfalse
\ORIGINALtrue
\fbox{
\begin{minipage}[t]{0.39\textwidth}
\begin{tabular}[c]{l}
 $ $ \\
 $ $ \\
 $ $ \\
 $ $ \\
 \setcounter{nombre}{0}%
\ifONERTT
 $\prob.\quad receive\ m_4$ \\
\else
 $\prob.\quad receive\ m_2$ \\
\fi
 $\prob.\quad T_s^{\prime} \| \STK = \SE.\Dec(\ik, c)$ \\
 $\prob.\quad pms^{\prime} = T_s^{\prime t_c} $ \\
 $\prob.\quad ms^{\prime} = \PRF(pms^{\prime}, \NONC) $ \\
\ifONERTT
 $\prob.\quad \key = \PRF(ms^{\prime}, m_1 \| m_2 \| m_3 \| m_4)$ \\
\else
 $\prob.\quad \key = \PRF(ms^{\prime}, m_1 \| m_2)$ \\
\fi
 $\prob.\quad \Lambda = \accept$ \\
 $\prob.\quad \theta = (\SCFG_{pub}, \STK)$ \\
\end{tabular}
\end{minipage}%
}
\begin{minipage}[t]{0.13\textwidth}
\centering
\begin{tabular}[c]{l}
 $ $\\
 $ $\\
 $ $\\
\ifONERTT
 $\xleftarrow{m_4}$\\
\else
 $\xleftarrow{m_2}$\\
\fi
 $ $\\
 $ $\\
 $ $\\
 $ $\\
 $ $\\
\end{tabular}
\end{minipage}%
\fbox{
\begin{minipage}[t]{0.39\textwidth}
\begin{tabular}[c]{l}
 \setcounter{nombre}{0}%
 $\prob.\quad t_s^{\prime} \xleftarrow{\$} \Zset_{q}^{\ast}$ \\
 $\prob.\quad T_s^{\prime} = g^{t_s^{\prime}}$ \\
 $\prob.\quad \makeSTKQUIC$ \\
 $\prob.\quad \plaintext = T_s^{\prime} \| \STK$ \\
 $\prob.\quad c = (\SE.\Enc(\ik, \plaintext)$ \\
\ifONERTT
 $\prob.\quad m_4 = (c, \cid)$ \\
 $\prob.\quad send\ m_4$ \\
\else
 $\prob.\quad m_2 = (c, \cid)$ \\
 $\prob.\quad send\ m_2$ \\
\fi
 $\prob.\quad pms^{\prime} = T_c{t_s^{\prime}} $ \\
 $\prob.\quad ms^{\prime} = \PRF(pms^{\prime}, \NONC) $ \\
\ifONERTT
 $\prob.\quad \key = \PRF(ms^{\prime}, m_1 \| m_2 \| m_3 \| m_4)$ \\
\else
 $\prob.\quad \key = \PRF(ms^{\prime}, m_1 \| m_2)$ \\
\fi
 $\prob.\quad \Lambda = \accept$ \\
\end{tabular}
\end{minipage}%
} \vspace{10pt}

\caption{Abstract model of 0-RTT QUIC handshake}\label{fig:quic_abst_0rtt}
\end{center}
\end{figure*}
%
We define four phases of QUIC handshake in 0-RTT.
(1) \textbf{Initial Key Agreement},
(2) \textbf{Initial Data Exchange},
(3) \textbf{Key Agreement}, and
(4) \textbf{Data Exchange}.
The flow of (2), (3), (4) is the same as the 1-RTT
handshake.

%=====================================================
\subsubsection{Initial Key Agreement}
%=====================================================
In this phase, the client sends an client with source
address token $\STK$.
\\
\noindent
\underline{$\CHLO(\STK, \SCFG_{pub})$:} \\
 \setcounter{nombre}{0}%
 $\prob.\quad \cid \xleftarrow{\$} \{0,1\}^{64}$ \\
 $\prob.\quad r \xleftarrow{\$} \{0,1\}^{160}$ \\
 $\prob.\quad \NONC \leftarrow currentTime \| r$ \\
 $\prob.\quad t_c^{\ast} \xleftarrow{\$} \Zset_{q}^{\ast}$ \\
 $\prob.\quad T_c^{\ast} = g^{t_c^{\ast}}$ \\
 $\prob.\quad \pInfo = (IP_{src}, IP_{dst}, port_{src}, port_{dst})$ \\
 $\prob.\quad \return\ (\pInfo, \cid, \STK, \SCID, \NONC, T_c^{\ast})$ \\

%=====================================================
\subsection{Security of QUIC} \label{sec:quic_detail}
%=====================================================

We note that QUIC is not RSACCE secure.
Although QUIC has a mechanism to prevent forgery,
whose mechanisms is called \textit{source-address token}
(see below), the adversary can forge an abbreviate handshake
query as follows: An adversary can obtain SCID from $\REJ$
response, which is the first response from a server oracle.
Then, the adversary make $(\overline{T}_c^{\prime},
\overline{\NONC}^{\prime})$ and send $(\overline{T}_c^{\prime},
\SCID, \overline{\NONC}^{\prime})$ to the server after a
full handshake between the server and its intended partner
is established.
The server cannot distinguish whether the query in resumption
comes from the intended partner or not. Because there is no
authentication mechanism in this query. The server accepts
this query and calculates the session key with the value
of adversary.
The adversary does not know the session key because
$T_s^{\ast}$ is encrypted however a session key between the
client and the server does not match.

\subsubsection{Source Address Token} \label{sec:source_address_token}
Source-address token ($\STK$) is introduced to QUIC in order to
prevent IP address spoofing.
A server generates and sends a new STK every time he sends a
message to a client.
The client updates STK when the client receives a new one from
the server and sends it back along with his message.
$\STK$ is an opaque byte string from the client's point of view.
From the server's point of view it's an authenticated-encryption
block that contains, at least, the client's IP address and a time
stamp by the server.
$\STK$ is encrypted except for the first server's query ($\REJ$).
However, in our model the adversary has full control over the
communication network and hence it can obtain $\STK$ and use it to
forge a future query.