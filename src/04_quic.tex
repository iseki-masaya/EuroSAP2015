%=====================================================
\section{Quick UDP Internet Connections} \label{sec:quic}
%=====================================================

\begin{figure*}[!htp]
\begin{center}

Fullhandshake between a client and a server \vspace{10pt}\\

\fbox{
\begin{minipage}[t]{0.35\textwidth}
\begin{tabular}[t]{l}
 $\quad Client$ \\
 $$ \\
 $1.\ \ \peer = S, \text{$S$ is determined from $cert_s$}$ \\
 $2.\ \ pk_s = \getPK(cert_s)$ \\
 $3.\ \ \text{If } \SIG.\Vfy(pk_s, \sigma_s, T_s \| \SCID) = \perp$ \\
 $4.\ \ \quad \Lambda = \text{'reject' and abort}$ \\
 $5.\ \ t_c \xleftarrow{\$} \mathbb{Z}_q$ \\
 $6.\ \ T_c=g^{t_c}$ \\
 $7.\ \ pms = T_s^{t_c}$ \\
 $8.\ \ \NONC \leftarrow \$$ \\
 $9.\ \ ms = \PRF(pms, \NONC)$ \\
 $10.\  trans = T_s \| T_c \| \SCID \| \NONC$ \\
 $11.\  k=\PRF(ms, trans)$ \\
 $12.\  \Lambda=\preaccept$ \\
 $13.\  m_2 = (T_c, \SCID, \NONC)$ \\
 $$\\
 $1.\ \ \text{If } \SE.\Dec(k,c_s,H,st_d) = \perp$ \\
 $2.\ \ \quad \Lambda=\text{'reject' and abort}$ \\
 $3.\ \ T_s^{\ast} = \SE.\Dec(k,c_s,H,st_d)$ \\
 $4.\ \ pms^{\ast} = T_s^{\ast t_c}$ \\
 $5.\ \ ms^{\ast} = \PRF(pms^{\ast}, \NONC)$ \\
 $6.\ \ trans^{\ast} = T_s^{\ast} \| T_c \| \SCID \| \NONC$ \\
 $7.\ \ k=\PRF(ms^{\ast}, trans^{\ast})$ \\
 $8.\ \ \Lambda=\accept$ \\
 $9.\ \ \theta \leftarrow (T_s, \SCID)$ \\
\end{tabular}
\end{minipage}%
}
\begin{minipage}[t]{0.13\textwidth}
\begin{tabular}[t]{l}
 $\xrightarrow{\text{inchoateCHLO}}$\\
 $$\\
 $$\\
 $$\\
 $\xleftarrow{\hspace{9pt}\text{REJ}(m_1)\hspace{9pt}}$\\
 $$\\
 $$\\
 $$\\
 $$\\
 $$\\
 $$\\
 $\xrightarrow{\hspace{5pt}\text{CHLO}(m_2)\hspace{5pt}}$\\
 $$\\
 $$\\
 $$\\
 $$\\
 $\xleftarrow{\hspace{5pt}\text{SHLO}(m_3)\hspace{5pt}}$\\
\end{tabular}
\end{minipage}%
\fbox{
\begin{minipage}[t]{0.35\textwidth}
\begin{tabular}[t]{l}
 $\quad Server$ \\
 $1.\ \ \SCID \leftarrow \{0,1\}^{\lambda} $ \\
 $2.\ \ t_s \xleftarrow{\$} \mathbb{Z}_q$ \\
 $3.\ \ T_s = g^{t_s}$ \\
 $4.\ \ \sigma_s = \SIG.\Sign(sk_s,T_s \| \SCID)$ \\
 $5.\ \ m_1 = (T_s,\sigma_s,\SCID,cert_s)$ \\
 $$\\ \vspace{5pt}
 $$\\
 $1.\ \ \peer = \SCID$ \\
 $2.\ \ pms=T_c^{t_s}$ \\
 $3.\ \ ms=\PRF(pms, \NONC)$ \\
 $4.\ \ trans = T_s \| T_c \| \SCID \| \NONC$ \\
 $5.\ \ k=\PRF(ms, trans)$ \\
 $6.\  \Lambda=\preaccept$ \\
 $7.\ \ t_s^{\ast} \xleftarrow{\$} \mathbb{Z}_q $ \\
 $8.\ \ T_s^{\ast} = g^{t_s^{\ast}} $ \\
 $9.\ \ c_s = \SE.\Enc(k, len, H, T_s^{\ast}, st_e)$ \\
 $10.\  m_3 = (c_s)$ \\
 $11.\  pms^{\ast} = T_c^{t_s^{\ast}}$ \\
 $12.\  ms^{\ast} = \PRF(pms^{\ast}, \NONC)$ \\
 $13.\  trans^{\ast} = T_s^{\ast} \| T_c \| \SCID \| \NONC$ \\
 $14.\  k=\PRF(ms^{\ast}, trans^{\ast})$ \\
 $15.\  \Lambda=\accept$ \\
 $16.\  \theta(\SCID) \leftarrow (T_s, t_s)$ \\
\end{tabular}
\end{minipage}%
} \vspace{10pt}

Resumption between a client and a server \vspace{10pt}\\

\fbox{
\begin{minipage}[t]{0.35\textwidth}
\begin{tabular}[t]{l}
 $1.\ \ (T_s, \SCID) \leftarrow \theta$ \\
 $2.\ \ t_c^{\prime} \xleftarrow{\$} \mathbb{Z}_q$ \\
 $3.\ \ T_c^{\prime}=g^{t_c^{\prime}}$ \\
 $4.\ \ pms^{\prime} = T_s^{t_c^{\prime}}$ \\
 $5.\ \ \NONC^{\prime} \leftarrow \$$ \\
 $6.\ \ ms^{\prime} = \PRF(pms^{\prime}, \NONC^{\prime})$ \\
 $7.\ \ trans^{\prime} = T_s \| T_c^{\prime} \| \SCID \| \NONC^{\prime}$ \\
 $8.\ \ k=\PRF(ms^{\prime}, trans^{\prime})$ \\
 $9.\ \ \Lambda=\preaccept$ \\
 $10.\  m_1^{\prime} = (T_c^{\prime},\SCID,\NONC^{\prime})$ \\
 $$\\
 $1.\ \ \text{If } \SE.\Dec(k,c_s,H,st_d) = \perp$ \\
 $2.\ \ \quad \Lambda=\text{'reject' and abort}$ \\
 $3.\ \ T_s^{\ast} = \SE.\Dec(k,c_s,H,st_d)$ \\
 $4.\ \ pms^{\ast} = T_s^{\ast t_c^{\prime}}$ \\
 $5.\ \ ms^{\ast} = \PRF(pms^{\ast}, \NONC^{\prime})$ \\
 $6.\ \ trans^{\ast} = T_s^{\ast} \| T_c^{\prime} \| \SCID \| \NONC^{\prime}$ \\
 $7.\ \ k=\PRF(ms^{\ast}, trans^{\ast})$ \\
 $8.\ \ \Lambda=\accept$ \\
 $9.\ \ \theta \leftarrow (T_s^{\ast}, \SCID)$ \\
\end{tabular}
\end{minipage}%
}
\begin{minipage}[t]{0.13\textwidth}
\begin{tabular}[t]{l}
 $$\\
 $$\\
 $$\\
 $$\\
 $$\\
 $$\\
 $\xrightarrow{\hspace{5pt}\text{CHLO}(m_1^{\prime})\hspace{5pt}}$\\
 $$\\
 $$\\
 $$\\
 $$\\
 $$\\
 $\xleftarrow{\hspace{5pt}\text{SHLO}(m_2^{\prime})\hspace{5pt}}$\\
\end{tabular}
\end{minipage}%
\fbox{
\begin{minipage}[t]{0.35\textwidth}
\begin{tabular}[t]{l}
 $$\vspace{3pt}\\
 $$\\
 $1.\ \ (T_s, t_s) \leftarrow \theta(\SCID)$ \\
 $2.\ \ pms^{\prime}=T_c^{\prime t_s}$ \\
 $3.\ \ ms^{\prime}=\PRF(pms^{\prime}, \NONC^{\prime})$ \\
 $4.\ \ trans^{\prime}= T_s \| T_c^{\prime} \| \SCID \| \NONC^{\prime}$ \\
 $5.\ \ k=\PRF(ms^{\prime}, trans^{\prime})$ \\
 $6.\ \ \Lambda=\preaccept$ \\
 $7.\ \ t_s^{\ast} \xleftarrow{\$} \mathbb{Z}_q $ \\
 $8.\ \ T_s^{\ast} = g^{t_s^{\ast}} $ \\
 $9.\ \ c_s = \SE.\Enc(k, len, H, T_s^{\ast}, st_e)$ \\
 $10.\  m_2^{\prime} = (c_s)$ \\
 $11.\  pms^{\ast} = T_c^{\prime t_s^{\ast}}$ \\
 $12.\  ms^{\ast} = \PRF(pms^{\ast}, \NONC^{\prime})$ \\
 $13.\  trans^{\ast} = T_s^{\ast} \| T_c^{\prime} \| \SCID \| \NONC^{\prime}$ \\
 $14.\  k=\PRF(ms^{\ast}, trans^{\ast})$ \\
 $15.\  \Lambda=\accept$ \\
 $16.\  \theta(\SCID) \leftarrow (T_s, t_s)$ \\
 $$\\
 $$\\
\end{tabular}
\end{minipage}
}

\caption{Abstract model of the QUIC handshake protocol}\label{fig:quic}
\end{center}
\end{figure*}
\begin{figure*}[!htp]
\begin{center}

Fullhandshake between a client and a server \vspace{10pt}\\

\fbox{
\begin{minipage}[t]{0.35\textwidth}
\begin{tabular}[t]{l}
 $\quad Client$ \\
 $$ \\
 $1.\ \ \peer = S, \text{$S$ is determined from $cert_s$}$ \\
 $2.\ \ pk_s = \getPK(cert_s)$ \\
 $3.\ \ \text{If } \SIG.\Vfy(pk_s, \sigma_s, T_s \| \SCID) = \perp$ \\
 $4.\ \ \quad \Lambda = \text{'reject' and abort}$ \\
 $5.\ \ t_c \xleftarrow{\$} \mathbb{Z}_q$ \\
 $6.\ \ T_c=g^{t_c}$ \\
 $7.\ \ pms = T_s^{t_c}$ \\
 $8.\ \ \NONC \leftarrow \$$ \\
 $9.\ \ ms = \PRF(pms, \NONC)$ \\
 $10.\  trans = T_s \| T_c \| \SCID \| \NONC$ \\
 $11.\  k=\PRF(ms, trans)$ \\
 $12.\  \Lambda=\preaccept$ \\
 $13.\  (pk_c,sk_c) = \SIG.\Gen(1^{\spar})$ \\
 $14.\  \sigma_c = \SIG.\Sign(sk_c, trans)$ \\
 $15.\  c_c = \SE.\Enc(k, len, H, pk_c \| \sigma_c, st_e)$ \\
 $16.\  m_2 = (T_c, \SCID, \NONC, c_c)$ \\
 $$\\
 $1.\ \ \text{If } \SE.\Dec(k,c_s,H,st_d) = \perp$ \\
 $2.\ \ \quad \Lambda=\text{'reject' and abort}$ \\
 $3.\ \ T_s^{\ast} = \SE.\Dec(k,c_s,H,st_d)$ \\
 $4.\ \ pms^{\ast} = T_s^{\ast t_c}$ \\
 $5.\ \ ms^{\ast} = \PRF(pms^{\ast}, \NONC)$ \\
 $6.\ \ trans^{\ast} = T_s^{\ast} \| T_c \| \SCID \| \NONC$ \\
 $7.\ \ k=\PRF(ms^{\ast}, trans^{\ast})$ \\
 $8.\ \ \Lambda=\accept$ \\
 $9.\ \ \theta \leftarrow (T_s^{\ast}, \SCID, pk_c, sk_c)$ \\
\end{tabular}
\end{minipage}%
}
\begin{minipage}[t]{0.13\textwidth}
\begin{tabular}[t]{l}
 $\xrightarrow{\text{inchoateCHLO}}$\\
 $$\\
 $$\\
 $$\\
 $\xleftarrow{\hspace{9pt}\text{REJ}(m_1)\hspace{9pt}}$\\
 $$\\
 $$\\
 $$\\
 $$\\
 $$\\
 $$\\
 $$\\
 $$\\
 $$\\
 $\xrightarrow{\hspace{5pt}\text{CHLO}(m_2)\hspace{5pt}}$\\
 $$\\
 $$\\
 $$\\
 $$\\
 $\xleftarrow{\hspace{5pt}\text{SHLO}(m_3)\hspace{5pt}}$\\
\end{tabular}
\end{minipage}%
\fbox{
\begin{minipage}[t]{0.35\textwidth}
\begin{tabular}[t]{l}
 $\quad Server$ \\
 $1.\ \ \SCID \leftarrow \{0,1\}^{\lambda} $ \\
 $2.\ \ t_s \xleftarrow{\$} \mathbb{Z}_q$ \\
 $3.\ \ T_s = g^{t_s}$ \\
 $4.\ \ \sigma_s = \SIG.\Sign(sk_s,T_s \| \SCID)$ \\
 $5.\ \ m_1 = (T_s,\sigma_s,\SCID,cert_s)$ \\
 $$\\ \vspace{6pt}
 $$\\
 $1.\ \ \peer = \SCID$ \\
 $2.\ \ pms=T_c^{t_s}$ \\
 $3.\ \ ms=\PRF(pms, \NONC)$ \\
 $4.\ \ trans = T_s \| T_c \| \SCID \| \NONC$ \\
 $5.\ \ k=\PRF(ms, trans)$ \\
 $6.\ \ \Lambda=\preaccept$ \\
 $7.\ \ (pk_c, \sigma_c) = \SE.\Dec(k,c_c,H,st_d)$ \\
 $8.\ \ \text{If } \SIG.\Vfy(pk_c, \sigma_c, trans) = \perp$ \\
 $9.\ \ \quad \Lambda = \text{'reject' and abort}$ \\
 $10.\  t_s^{\ast} \xleftarrow{\$} \mathbb{Z}_q $ \\
 $11.\  T_s^{\ast} = g^{t_s^{\ast}} $ \\
 $12.\  c_s = \SE.\Enc(k, len, H, T_s^{\ast}, st_e)$ \\
 $13.\  m_3 = (c_s)$ \\
 $14.\  pms^{\ast} = T_c^{t_s^{\ast}}$ \\
 $15.\  ms^{\ast} = \PRF(pms^{\ast}, \NONC)$ \\
 $16.\  trans^{\ast} = T_s^{\ast} \| T_c \| \SCID \| \NONC$ \\
 $17.\  k=\PRF(ms^{\ast}, trans^{\ast})$ \\
 $18.\  \Lambda=\accept$ \\
 $19.\  \theta(\SCID) \leftarrow (T_s^{\ast}, t_s^{\ast})$ \\
\end{tabular}
\end{minipage}%
} \vspace{10pt}

Resumption between a client and a server \vspace{10pt}\\

\fbox{
\begin{minipage}[t]{0.35\textwidth}
\begin{tabular}[t]{l}
 $1.\ \ (T_s^{\ast}, \SCID, pk_c, sk_c) \leftarrow \theta$ \\
 $2.\ \ t_c^{\prime} \xleftarrow{\$} \mathbb{Z}_q$ \\
 $3.\ \ T_c^{\prime}=g^{t_c^{\prime}}$ \\
 $4.\ \ pms^{\prime} = T_s^{\ast t_c^{\prime}}$ \\
 $5.\ \ \NONC^{\prime} \leftarrow \$$ \\
 $6.\ \ ms^{\prime} = \PRF(pms^{\prime}, \NONC^{\prime})$ \\
 $7.\ \ trans^{\prime} = T_s^{\ast} \| T_c^{\prime} \| \SCID \| \NONC^{\prime}$ \\
 $8.\ \ k=\PRF(ms^{\prime}, trans^{\prime})$ \\
 $9.\ \ \Lambda=\preaccept$ \\
 $10.\  \sigma_c^{\prime} = \SIG.\Sign(sk_c, trans^{\prime})$ \\
 $11.\  c_c = \SE.\Enc(k, len, H, pk_c \| \sigma_c^{\prime}, st_e)$ \\
 $12.\  m_1^{\prime} = (T_c^{\prime},\SCID,\NONC^{\prime},c_c)$ \\
 $$\\
 $1.\ \ \text{If } \SE.\Dec(k,c_s,H,st_d) = \perp$ \\
 $2.\ \ \quad \Lambda=\text{'reject' and abort}$ \\
 $3.\ \ T_s^{\ast} = \SE.\Dec(k,c_s,H,st_d)$ \\
 $4.\ \ pms^{\ast} = T_s^{\ast t_c^{\prime}}$ \\
 $5.\ \ ms^{\ast} = \PRF(pms^{\ast}, \NONC^{\prime})$ \\
 $6.\ \ trans^{\ast} = T_s^{\ast} \| T_c^{\prime} \| \SCID \| \NONC^{\prime}$ \\
 $7.\ \ k=\PRF(ms^{\ast}, trans^{\ast})$ \\
 $8.\ \ \Lambda=\accept$ \\
 $9.\ \ \theta \leftarrow (T_s^{\ast}, \SCID, pk_c, sk_c)$ \\
\end{tabular}
\end{minipage}%
}
\begin{minipage}[t]{0.13\textwidth}
\begin{tabular}[t]{l}
 $$\\
 $$\\
 $$\\
 $$\\
 $$\\
 $$\\
 $\xrightarrow{\hspace{5pt}\text{CHLO}(m_1^{\prime})\hspace{5pt}}$\\
 $$\\
 $$\\
 $$\\
 $$\\
 $$\\
 $\xleftarrow{\hspace{5pt}\text{SHLO}(m_2^{\prime})\hspace{5pt}}$\\
\end{tabular}
\end{minipage}%
\fbox{
\begin{minipage}[t]{0.35\textwidth}
\begin{tabular}[t]{l}
 $$\vspace{4pt}\\
 $$\\
 $1.\ \ (T_s^{\ast}, t_s^{\ast}) \leftarrow \theta(\SCID)$ \\
 $2.\ \ pms^{\prime}=T_c^{\prime t_s^{\ast}}$ \\
 $3.\ \ ms^{\prime}=\PRF(pms^{\prime}, \NONC^{\prime})$ \\
 $4.\ \ trans^{\prime}= T_s^{\ast} \| T_c^{\prime} \| \SCID \| \NONC^{\prime}$ \\
 $5.\ \ k=\PRF(ms^{\prime}, trans^{\prime})$ \\
 $6.\  \Lambda=\preaccept$ \\
 $7.\ \ (pk_c, \sigma_c^{\prime}) = \SE.\Dec(k,c_c,H,st_d)$ \\
 $8.\ \ \text{If } \SIG.\Vfy(pk_c, \sigma_c^{\prime}, trans^{\prime}) = \perp$ \\
 $9.\ \ \quad \Lambda = \text{'reject' and abort}$ \\
 $10.\  t_s^{\ast} \xleftarrow{\$} \mathbb{Z}_q $ \\
 $11.\  T_s^{\ast} = g^{t_s^{\ast}} $ \\
 $12.\  c_s = \SE.\Enc(k, len, H, T_s^{\ast}, st_e)$ \\
 $13.\  m_2^{\prime} = (c_s)$ \\
 $14.\  pms^{\ast} = T_c^{\prime t_s^{\ast}}$ \\
 $15.\  ms^{\ast} = \PRF(pms^{\ast}, \NONC^{\prime})$ \\
 $16.\  trans^{\ast} = T_s^{\ast} \| T_c^{\prime} \| \SCID \| \NONC^{\prime}$ \\
 $17.\  k=\PRF(ms^{\ast}, trans^{\ast})$ \\
 $18.\  \Lambda=\accept$ \\
 $19.\  \theta(\SCID) \leftarrow (T_s^{\ast}, t_s^{\ast})$ \\
 $$\\
\end{tabular}
\end{minipage}
}

\caption{Abstract model of the QUIC handshake protocol with CETV}\label{fig:quic_cetv}
\end{center}
\end{figure*}
\begin{figure*}[!htb]
\begin{center}

Fullhandshake between a client and a server \vspace{10pt}\\

\fbox{
\begin{minipage}[t]{0.35\textwidth}
\begin{tabular}[t]{l}
 $\quad Client$ \\
 $$ \\
 $1.\ \ \NONC \leftarrow \$$ \\
 $2.\ \ t_c \xleftarrow{\$} \mathbb{Z}_q$ \\
 $3.\ \ T_c=g^{t_c}$ \\
 $4.\ \ m_1 = (T_c, \NONC)$ \\
 $$\\
 $1.\ \ \peer = S, \text{$S$ is determined from $cert_s$}$ \\
 $2.\ \ pk_s = \getPK(cert_s)$ \\
 $3.\ \ pms = T_s^{t_c}$ \\
 $4.\ \ ms = \PRF(pms, \NONC)$ \\
 $5.\ \ trans = T_s \| T_c \| \SCID \| \NONC$ \\
 $6.\ \ k = \PRF(ms, trans)$ \\
 $7.\ \ \Lambda=\preaccept$ \\
 $8.\ \ T_s^{\ast} = \SE.\Dec(k,c_s,H,st_d)$ \\
 $9.\ \ \text{If } \SIG.\Vfy(pk_s, \sigma_s, trans \| T_s^{\ast}) = \perp$ \\
 $10.\  \quad \Lambda = \text{'reject' and abort}$ \\
 $11.\  pms^{\ast} = T_s^{\ast t_c}$ \\
 $12.\  ms^{\ast} = \PRF(pms^{\ast}, NONC)$ \\
 $13.\  trans^{\ast} = T_s^{\ast} \| T_c \| \SCID \| \NONC$ \\
 $14.\  k=\PRF(ms^{\ast}, trans^{\ast})$ \\
 $15.\  \Lambda=\accept$ \\
 $16.\  \theta \leftarrow (\SCID, T_s^{\ast})$ \\
\end{tabular}
\end{minipage}%
}
\begin{minipage}[t]{0.13\textwidth}
\begin{tabular}[t]{l}
 $$\\
 $$\\
 $$\\
 $$\\
 $\xrightarrow{\hspace{5pt}\text{CHLO}(m_1)\hspace{5pt}}$\\
 $$\\
 $$\\
 $$\\
 $\xleftarrow{\hspace{5pt}\text{SHLO}(m_2)\hspace{5pt}}$\\
 $$\\
 $$\\
 $$\\
 $$\\
\end{tabular}
\end{minipage}%
\fbox{
\begin{minipage}[t]{0.35\textwidth}
\begin{tabular}[t]{l}
 $\quad Server$ \\
 $$\\
 $1.\ \ \SCID \leftarrow \{0,1\}^{\lambda} $ \\
 $2.\ \ \peer = \SCID$ \\
 $3.\ \ t_s \xleftarrow{\$} \mathbb{Z}_q$ \\
 $4.\ \ T_s = g^{t_s}$ \\
 $5.\ \ trans= T_s \| T_c \| \SCID \| \NONC$ \\
 $6.\ \ pms=T_c^{t_s}$ \\
 $7.\ \ ms=\PRF(pms, \NONC)$ \\
 $8.\ \ k=\PRF(ms, trans)$ \\
 $9.\ \ \Lambda=\preaccept$ \\
 $10.\  t_s^{\ast} \xleftarrow{\$} \mathbb{Z}_q$ \\
 $11.\  T_s^{\ast} = g^{t_s^{\ast}}$ \\
 $12.\  c_s = \SE.\Enc(k,len,H,T_s^{\ast},st_e)$ \\
 $13.\  \sigma_s = \SIG.\Sign(sk_s,trans \| T_s^{\ast})$ \\
 $14.\  m_2 = (T_s, \sigma_s, \SCID, cert_s, c_s)$ \\
 $15.\  pms^{\ast} = T_c^{t_s^{\ast}}$ \\
 $16.\  ms^{\ast} = \PRF(pms^{\ast}, \NONC)$ \\
 $17.\  trans^{\ast} = T_s^{\ast} \| T_c \| \SCID \| \NONC$ \\
 $18.\  k=\PRF(ms^{\ast}, trans^{\ast})$ \\
 $19.\  \Lambda=\accept$ \\
 $20.\  \theta(\SCID) \leftarrow (T_s^{\ast},t_s^{\ast})$ \\
 $$
\end{tabular}
\end{minipage}%
} \vspace{10pt}

Resumption between a client and a server \vspace{10pt}\\

\fbox{
\begin{minipage}[t]{0.35\textwidth}
\begin{tabular}[t]{l}
 $1.\ \ (\SCID,T_s^{\ast}) \leftarrow \theta$ \\
 $2.\ \ t_c^{\prime} \xleftarrow{\$} \mathbb{Z}_q$ \\
 $3.\ \ T_c^{\prime}=g^{t_c^{\prime}}$ \\
 $4.\ \ pms^{\prime} = T_s^{\ast t_c^{\prime}}$ \\
 $5.\ \ \NONC^{\prime} \leftarrow \$$ \\
 $6.\ \ ms^{\prime} = \PRF(pms^{\prime}, \NONC^{\prime})$ \\
 $7.\ \ trans^{\prime} = T_s^{\ast} \| T_c^{\prime} \| \SCID \| \NONC^{\prime}$ \\
 $8.\ \ fin_c = \PRF(ms^{\prime}, trans^{\prime})$ \\
 $9.\ \ k=\PRF(ms^{\prime}, trans^{\prime})$ \\
 $10.\  \Lambda=\preaccept$ \\
 $11.\  m_1^{\prime} = (T_c^{\prime},\SCID,\NONC^{\prime},fin_c)$ \\
 $$\\
 $1.\ \ \text{If } \SE.\Dec(k,c_s,H,st_d) = \perp$ \\
 $2.\ \ \quad \Lambda=\text{'reject' and abort}$ \\
 $3.\ \ T_s^{\ast} = \SE.\Dec(k,c_s,H,st_d)$ \\
 $4.\ \ pms^{\ast} = T_s^{\ast t_c^{\prime}}$ \\
 $5.\ \ ms^{\ast} = \PRF(pms^{\prime}, \NONC^{\prime})$ \\
 $6.\ \ trans^{\ast} = T_s^{\ast} \| T_c^{\prime} \| \SCID \| \NONC^{\prime}$ \\
 $7.\ \ k=\PRF(ms^{\ast}, trans^{\ast})$ \\
 $8.\ \ \Lambda=\accept$ \\
 $9.\ \ \theta \leftarrow (\SCID,T_s^{\ast},fin_c)$ \\
\end{tabular}
\end{minipage}%
}
\begin{minipage}[t]{0.13\textwidth}
\begin{tabular}[t]{l}
 $$\\
 $$\\
 $$\\
 $$\\
 $$\\
 $$\\
 $\xrightarrow{\hspace{5pt}\text{CHLO}(m_1^{\prime})\hspace{5pt}}$\\
 $$\\
 $$\\
 $$\\
 $$\\
 $$\\
 $\xleftarrow{\hspace{5pt}\text{SHLO}(m_2^{\prime})\hspace{5pt}}$\\
\end{tabular}
\end{minipage}%
\fbox{
\begin{minipage}[t]{0.35\textwidth}
\begin{tabular}[t]{l}
 $$\\
 $1.\ \ (T_s^{\ast}, t_s^{\ast}) \leftarrow \theta(\SCID)$ \\
 $2.\ \ pms^{\prime}=T_c^{\prime t_s^{\ast}}$ \\
 $3.\ \ ms^{\prime}=\PRF(pms^{\prime}, \NONC^{\prime})$ \\
 $4.\ \ trans^{\prime}= T_s^{\ast} \| T_c^{\prime} \| \SCID \| \NONC^{\prime}$ \\
 $5.\ \ fin_c^{\ast} = \PRF(ms^{\prime}, trans^{\prime})$ \\
 $6.\ \ \text{If } fin_c^{\ast} \neq fin_c$ \\
 $7.\ \ \quad \Lambda=\text{'reject' and abort}$ \\
 $8.\ \ k=\PRF(ms^{\prime}, trans^{\prime})$ \\
 $9.\ \ \Lambda=\preaccept$ \\
 $10.\  t_s^{\ast} \xleftarrow{\$} \mathbb{Z}_q $ \\
 $11.\  T_s^{\ast} = g^{t_s^{\ast}} $ \\
 $12.\  c_s = \SE.\Enc(k, len, H, T_s^{\ast}, st_e)$ \\
 $13.\  pms^{\ast} = T_c^{t_s^{\ast}}$ \\
 $14.\  ms^{\ast} = \PRF(pms^{\ast}, \NONC^{\prime})$ \\
 $15.\  trans^{\ast} = T_s^{\ast} \| T_c^{\prime} \| \SCID \| \NONC^{\prime}$ \\
 $16.\  k=\PRF(ms^{\ast}, trans^{\ast})$ \\
 $17.\  m_2^{\prime} = (c_s)$ \\
 $18.\  \Lambda=\accept$ \\
 $19.\  \theta(\SCID) \leftarrow (T_s^{\ast}, t_s^{\ast})$ \\
 $$\\
\end{tabular}
\end{minipage}
}

\caption{Abstract model of our proposed scheme}\label{fig:quic_detail}
\end{center}
\end{figure*}

Quick UDP Internet Connections (QUIC) is a protocol developed by Google. This protocol is still under development. The abstract model of QUIC is described in Fig.~\ref{fig:quic}. The concepts of QUIC are (1) to reduce connectivity overheads before a client sends encrypted data and (2) to obtain better security guarantee than TLS.
To realize concept (1), QUIC is not defined over TCP but UDP, because TCP requires three-move handshake before initiating a cryptographic handshake, but UDP does not have. In addition, a client can send encrypted data concurrently with CHLO.
To realize concept (2), QUIC supports only secure cipher suites. Especially, a support algorithm of key exchange is only ephemeral elliptic curve Diffie-Hellman.

Unlike TLS, QUIC does not support client authentication.

%=====================================================
\subsection{Security of QUIC} \label{sec:quic_detail}
%=====================================================

\begin{theorem} \label{theorem:quic}
 Let $\mu$ be the output length of $\PRF$, let $\lambda$ be the length of $\SCID$, let $\nclient$ be the number of clients, let $\nserver$ be the number of servers, let $\noracle$ be the number of oracles of each parties, and let $n_{\ell}$ be the maximum number of resumptions. Assume that the $\PRF$ is $(t, \epsilon_{\prf})$-pseudo-random function family, the signature scheme $\SIG$ is $(t, \epsilon_{\sig})$-secure against existentially unforgeable under adaptive chosen-message attacks, the DDH problem on $G$ is $(t, \epsilon_{\ddh})$-hard, the PRF-ODH problem on $\PRF$ is $(t, \epsilon_{\prfodh})$-hard, the hash function family $\mathcal{H}$ is $(t,\epsilon_{H})$-collision-resistant (CR), the symmetric encryption scheme $\SE$ is $(t, \epsilon_{\sLHAE})$-secure.
 Then for all PPT adversaries, QUIC is RSACCE secure.
\end{theorem}%

The proof of Theorem~\ref{theorem:quic} is in Appendix~\ref{app:quic}.\\

We note that QUIC is not \textit{strong} RSACCE secure. Although QUIC has a mechanism to prevent forgery, whose mechanisms is called \textit{source-address token} (see below), the adversary can forge a resumption query as follows: A adversary can obtain SCID from $\REJ$ response, which is the first response from a server oracle. Then, the adversary make $(\overline{T}_c^{\prime}, \overline{\NONC}^{\prime})$ and send $(\overline{T}_c^{\prime}, \SCID, \overline{\NONC}^{\prime})$ to the server after a full handshake between the server and its intended partner is established. The server cannot distinguish whether the query in resumption comes from the intended partner or not. Because there is no authentication mechanism in this query. The server accepts this query and calculates the session key with the value of adversary. The adversary does not know the session key because $T_s^{\ast}$ is encrypted however a session key between the client and the server does not match.

\subsubsection{Source Address Token} \label{sec:source_address_token}
Source-address token ($\STK$) is introduced to QUIC in order to prevent IP address spoofing. A server generates and sends a new STK every time he sends a message to a client. The client updates STK when the client receives a new one from the server and sends it back along with his message. $\STK$ is an opaque byte string from the client's point of view. From the server's point of view it's an authenticated-encryption block that contains, at least, the client's IP address and a time stamp by the server. $\STK$ is encrypted except for the first server's query ($\REJ$). However, in our model the adversary has full control over the communication network and hence it can obtain $\STK$ and use it to forge a future query.

%--------------------------------------------------------
\subsection{Security of QUIC with CETV} \label{sec:quic_cetv}
%--------------------------------------------------------

We have already mentioned that QUIC does not meet strong RSACCE security. However, QUIC has an optional mechanism, called client encrypted tag-value (CETV), which is disabled in the default setting. This mechanism borrows from~\cite{MCBW12:TLS-OCB,ChannelID} and utilizes a part of its implementation. In this section, we prove that QUIC with CETV (i.e., QUIC with CETV enabled) satisfies strong RSACCE.
The abstract model of QUIC with CETV is described in Fig.~\ref{fig:quic_cetv}.

\begin{theorem} \label{theorem:quic_cetv}
 Let $\mu$ be the output length of $\PRF$, let $\lambda$ be the length of $\SCID$, let $\nclient$ be the number of clients, let $\nserver$ be the number of servers, let $\noracle$ be the number of oracles of each parties, and let $n_{\ell}$ be the maximum number of resumptions. Assume that the $\PRF$ is $(t, \epsilon_{\prf})$-pseudo-random function family, the signature scheme $\SIG$ is $(t, \epsilon_{\sig})$-secure against existentially unforgeable under adaptive chosen-message attacks, the DDH problem on $G$ is $(t, \epsilon_{\ddh})$-hard, the PRF-ODH problem on $\PRF$ is $(t, \epsilon_{\prfodh})$-hard, the hash function family $\mathcal{H}$ is $(t,\epsilon_{H})$-collision-resistant (CR), the symmetric encryption scheme $\SE$ is $(t, \epsilon_{\sLHAE})$-secure.
 Then for all PPT adversaries, QUIC with CETV is \textbf{strong} RSACCE secure.
\end{theorem}

The proof of Theorem~\ref{theorem:quic} is in Appendix~\ref{app:quic}.\\

Although we have proved that QUIC with CETV is strong RSACCE secure, many steps in CETV are useless for this purpose because the server does not cache its public key for the future sessions. The reason why QUIC with CETV satisfies strong RSACCE security is that a client sends ciphertext calculated using current session key. If the adversary cannot obtain the session key, the server does not accept the query from the adversary because the adversary cannot make the ciphertext $c^{\prime}$ without the session key which $\SE.\Dec$ returns a message $m \neq \perp$.