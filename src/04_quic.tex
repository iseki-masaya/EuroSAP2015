%=====================================================
\section{Quick UDP Internet Connections} \label{sec:quic}
%=====================================================

Quick UDP Internet Connections (QUIC) is a protocol
developed by Google. This protocol is still under
development.
The abstract model of QUIC is described in
Fig.~\ref{fig:quic_abst}.
The concepts of QUIC are (1) to reduce connectivity
overheads before a client sends encrypted data and
(2) to obtain better security guarantee than TLS.
To realize concept (1), QUIC is not defined over
TCP but UDP, because TCP requires three-move handshake
before initiating a cryptographic handshake, but UDP
does not have. In addition, a client can send encrypted
data concurrently with CHLO.
To realize concept (2), QUIC supports only secure cipher
suites.
Especially, a support algorithm of key exchange is only
ephemeral elliptic curve Diffie-Hellman.

Unlike TLS, QUIC does not support client authentication.

%=====================================================
\subsection{Security of QUIC} \label{sec:quic_detail}
%=====================================================

\begin{figure*}[!htp]
 \begin{center}

\begin{enumerate}
 \item{Initiate} \\
 \fbox{
  \begin{minipage}[t]{0.38\textwidth}
  \centering
   \begin{tabular}{c}
    $ $ \\
    $ $ \\
    $ $ \\
   \end{tabular}
  \end{minipage}%
 }
 \begin{minipage}[t]{0.13\textwidth}
  \centering
  \begin{tabular}{c}
   $ $ \\
  \end{tabular}
 \end{minipage}%
 \fbox{
  \begin{minipage}[t]{0.38\textwidth}
   \centering
   \begin{tabular}{c}
    $(pk_s, sk_s) = \SIG.\Gen()$ \\
    $(\SCFG_{pub}, t_s) = \scfgGen(sk_s)$ \\
    $k_{\STK} \xleftarrow{\$} \{0,1\}^{\lambda}$ \\
   \end{tabular}
  \end{minipage}%
 }
 \item{Initial Key Agreement} \\
 \fbox{
  \begin{minipage}[t]{0.38\textwidth}
  \centering
   \begin{tabular}{c}
    $m_1 = \inchoateCHLO()$ \\
    $m_3 = \initialCHLO(m_2)$ \\
    $\shareInfo = (\NONC, \cid, T_s, t_c)$ \\
    $m = m_1 \| m_2 \| m_3$ \\
    $\ik = \getKey_c(\shareInfo, m, 1)$ \\
    $\peer = S$ \\
    $\Lambda = \preaccept$ \\
   \end{tabular}
  \end{minipage}%
 }
 \begin{minipage}[t]{0.13\textwidth}
  \centering
  \begin{tabular}{c}
   $\xrightarrow{m_1}$ \\
   $\xleftarrow{m_2}$ \\
   $\xrightarrow{m_3}$ \\
   $ $ \\
  \end{tabular}
 \end{minipage}%
 \fbox{
  \begin{minipage}[t]{0.38\textwidth}
   \centering
   \begin{tabular}{c}
    $m_2 = \REJ(m_1, \SCFG_{pub})$ \\
    $\checkQuery(\STK, k_{\STK}, \NONC, IP_c)$ \\
    $\shareInfo = (\NONC, \cid, T_c, t_s) $ \\
    $m = m_1 \| m_2 \| m_3$ \\
    $\ik=\getKey_s(\shareInfo, m, 1)$ \\
    $\peer = C$ \\
    $\Lambda = \preaccept$ \\
   \end{tabular}
  \end{minipage}%
 }
 \item{Initial Data Exchange} \\
 \fbox{
  \begin{minipage}[t]{0.38\textwidth}
  \centering
   \begin{tabular}{c}
    $\text{for each } \alpha \in [\MsgCntC{0}]$ \\
    $\sqn_c = \alpha + 2$ \\
    $m_4^{\alpha} = \pak(ik, sqn_c, M_c^{\alpha})$ \\
    $m_4 = (m_4^{1},...,m_4^{\MsgCntC{0}})$ \\
    $\processPacket(ik, m_5)$ \\
   \end{tabular}
  \end{minipage}%
 }
 \begin{minipage}[t]{0.13\textwidth}
  \centering
  \begin{tabular}{c}
   $ $ \\
   $ $ \\
   $\xrightarrow{m_4}$ \\
   $\xleftarrow{m_5}$ \\
  \end{tabular}
 \end{minipage}%
 \fbox{
  \begin{minipage}[t]{0.38\textwidth}
   \centering
   \begin{tabular}{c}
    $\text{for each } \beta \in [\MsgCntS{0}]$ \\
    $\sqn_s = \beta + 1$ \\
    $m_5^{\beta} = \pak(\ik, \sqn_s, M_s^{\beta})$ \\
    $m_5 = (m_5^{1},...,m_5^{\MsgCntS{0}})$ \\
    $\processPacket(\ik, m_4)$ \\
   \end{tabular}
  \end{minipage}%
 }
 \item{Key Agreement} \\
 \fbox{
  \begin{minipage}[t]{0.38\textwidth}
  \centering
   \begin{tabular}{c}
    $ $ \\
    $T_s^{\prime} = \receiveSHLO(m_6, \ik)$ \\
    $\shareInfo = (\NONC, \cid ,T_s^{\prime}, t_c) $ \\
    $m = m_1 \| m_2 \| m_3 \| m_6$ \\
    $k = \getKey_c(\shareInfo, m, 0)$ \\
    $\Lambda = \accept$ \\
   \end{tabular}
  \end{minipage}%
 }
 \begin{minipage}[t]{0.13\textwidth}
  \centering
  \begin{tabular}{c}
   $ $ \\
   $\xleftarrow{m_6}$ \\
   $ $ \\
  \end{tabular}
 \end{minipage}%
 \fbox{
  \begin{minipage}[t]{0.38\textwidth}
   \centering
   \begin{tabular}{c}
    $ \sqn_s = \MsgCntS{0} + 2$ \\
    $m_6 = \SHLO(m_3, \ik, \sqn)$ \\
    $\shareInfo = (\NONC, \cid, T_c, t_s^{\prime})$ \\
    $m = m_1 \| m_2 \| m_3 \| m_6$ \\
    $k = \getKey_s(\shareInfo, m, 0)$ \\
    $\Lambda = \accept$ \\
   \end{tabular}
  \end{minipage}%
 }
 \item{Data Exchange} \\
 \fbox{
  \begin{minipage}[t]{0.38\textwidth}
  \centering
   \begin{tabular}{c}
    $\text{for each } \alpha \in {\MsgCntC{0}+1,...,\MsgCntC{1}}$ \\
    $\sqn_c = \alpha + 2$ \\
    $m_7^{\alpha} = \pak(k, sqn_c, M_c^{\alpha})$ \\
    $m_7 = (m_7^{\MsgCntC{0}+1},...,\MsgCntC{1})$ \\
    $\processPacket(k, m_8)$ \\
   \end{tabular}
  \end{minipage}%
 }
 \begin{minipage}[t]{0.13\textwidth}
  \centering
  \begin{tabular}{c}
   $ $ \\
   $\xrightarrow{m_7}$ \\
   $\xleftarrow{m_8}$ \\
  \end{tabular}
 \end{minipage}%
 \fbox{
  \begin{minipage}[t]{0.38\textwidth}
   \centering
   \begin{tabular}{c}
    $\text{for each } \beta \in {\MsgCntS{0}+1,...,\MsgCntS{1}}$ \\
    $\sqn_s = \beta + 2$ \\
    $m_8^{\beta} = \pak(k, \sqn_s, M_s^{\beta})$ \\
    $m_8 = (m_8^{\MsgCntS{0}+1},...,m_8^{\MsgCntS{1}})$ \\
    $\processPacket(\ik, m_7)$ \\
   \end{tabular}
  \end{minipage}%
 }
 \item{Key Exchange} \\
 \fbox{
  \begin{minipage}[t]{0.38\textwidth}
  \centering
   \begin{tabular}{c}
    $m_9 = \CHLO(\STK, \SCFG_{pub})$ \\
    $\shareInfo = (\NONC, \cid, T_s, t_c^{\ast})$ \\
    $\ik = \getKey_c(\shareInfo, m_9, 1)$ \\
    $\Lambda = \preaccept$ \\
   \end{tabular}
  \end{minipage}%
 }
 \begin{minipage}[t]{0.13\textwidth}
  \centering
  \begin{tabular}{c}
   $\xrightarrow{m_9}$ \\
   $ $ \\
  \end{tabular}
 \end{minipage}%
 \fbox{
  \begin{minipage}[t]{0.38\textwidth}
   \centering
   \begin{tabular}{c}
    $\checkQuery(\STK, k_{\STK}, \NONC, IP_c)$ \\
    $\shareInfo = (\NONC, \cid, T_c^{\ast}, t_s) $ \\
    $\ik=\getKey_s(\shareInfo, m_9, 1)$ \\
    $\Lambda = \preaccept$ \\
   \end{tabular}
  \end{minipage}%
 }
 \item{Initial Data Exchange after 0-RTT} \\
 \fbox{
  \begin{minipage}[t]{0.38\textwidth}
  \centering
   \begin{tabular}{c}
    $\text{for each } \alpha \in {\MsgCntC{1}+1,...,\MsgCntC{2}}$ \\
    $\sqn_c = \alpha + 3$ \\
    $m_{10}^{\alpha} = \pak(\ik, sqn_c, M_c^{\alpha})$ \\
    $m_{10} = (m_{10}^{\MsgCntC{1}+1},...,m_{10}^{\MsgCntC{2}})$ \\
    $\processPacket(\ik, m_{11})$ \\
   \end{tabular}
  \end{minipage}%
 }
 \begin{minipage}[t]{0.13\textwidth}
  \centering
  \begin{tabular}{c}
   $ $ \\
   $ $ \\
   $\xrightarrow{m_{10}}$ \\
   $\xleftarrow{m_{11}}$ \\
  \end{tabular}
 \end{minipage}%
 \fbox{
  \begin{minipage}[t]{0.38\textwidth}
   \centering
   \begin{tabular}{c}
    $\text{for each } \beta \in {\MsgCntS{1}+1,...,\MsgCntS{2}}$ \\
    $\sqn_s = \beta + 2$ \\
    $m_{11}^{\beta} = \pak(\ik, \sqn_s, M_s^{\beta})$ \\
    $m_{11} = (m_{11}^{\MsgCntS{1}+1},...,m_{11}^{\MsgCntS{2}})$ \\
    $\processPacket(\ik, m_{10})$ \\
   \end{tabular}
  \end{minipage}%
 }
 \item{Key Agreement after 0-RTT} \\
 \fbox{
  \begin{minipage}[t]{0.38\textwidth}
  \centering
   \begin{tabular}{c}
    $ $ \\
    $T_s^{\prime} = \receiveSHLO(m_{12}, \ik)$ \\
    $\shareInfo = (\NONC, \cid ,T_s^{\prime}, t_c^{\ast}) $ \\
    $m = m_9 \| m_{12}$ \\
    $k = \getKey_c(\shareInfo, m_9, 0)$ \\
    $\Lambda = \accept$ \\
   \end{tabular}
  \end{minipage}%
 }
 \begin{minipage}[t]{0.13\textwidth}
  \centering
  \begin{tabular}{c}
   $ $ \\
   $\xleftarrow{m_{12}}$ \\
   $ $ \\
  \end{tabular}
 \end{minipage}%
 \fbox{
  \begin{minipage}[t]{0.38\textwidth}
   \centering
   \begin{tabular}{c}
    $ \sqn_s = \MsgCntS{2} + 3$ \\
    $m_{12} = \SHLO(m_9, \ik, \sqn)$ \\
    $\shareInfo = (\NONC, \cid, T_c^{\ast}, t_s^{\prime})$ \\
    $m = m_9 \| m_{12}$ \\
    $k = \getKey_s(\shareInfo, m, 0)$ \\
    $\Lambda = \accept$ \\
   \end{tabular}
  \end{minipage}%
 }
 \item{Data Exchange after 0-RTT} \\
 \fbox{
  \begin{minipage}[t]{0.38\textwidth}
  \centering
   \begin{tabular}{c}
    $\text{for each } \alpha \in {\MsgCntC{2}+1,...,\MsgCntC{3}}$ \\
    $\sqn_c = \alpha + 3$ \\
    $m_{13}^{\alpha} = \pak(k, sqn_c, M_c^{\alpha})$ \\
    $m_{13} = (m_{13}^{\MsgCntC{2}+1},...,m_{13}^{\MsgCntC{3}})$ \\
    $\processPacket(k, m_{14})$ \\
   \end{tabular}
  \end{minipage}%
 }
 \begin{minipage}[t]{0.13\textwidth}
  \centering
  \begin{tabular}{c}
   $ $ \\
   $ $ \\
   $\xrightarrow{m_{13}}$ \\
   $\xleftarrow{m_{14}}$ \\
  \end{tabular}
 \end{minipage}%
 \fbox{
  \begin{minipage}[t]{0.38\textwidth}
   \centering
   \begin{tabular}{c}
    $\text{for each } \beta \in {\MsgCntS{2}+1,...,\MsgCntS{3}}$ \\
    $\sqn_s = \beta + 3$ \\
    $m_{14}^{\beta} = \pak(k, \sqn_s, M_s^{\beta})$ \\
    $m_{14} = (m_{14}^{\MsgCntS{2}+1},...,m_{14}^{\MsgCntS{3}})$ \\
    $\processPacket(\ik, m_{13})$ \\
   \end{tabular}
  \end{minipage}%
 }
\end{enumerate}
 \caption{Abstract model of the QUIC handshake}\label{fig:quic_abst}
 \end{center}
\end{figure*}
%MACのkeyサイズを記号化する
%提案法のreceiveCHLOが未定義

%MACを
% mを鍵生成に入れる
%  NONCの改ざんも可能
% Encにはk_mac,T_s*のみを入れる。改ざんされてもikが不一致になるので、Client側では改ざんを検出できる
% SCRA、CIDMA、SATMAはkを共有するためのクエリをサーバは待たなければならないが、提案法はそれをまたなくて良い。

\subsubsection{Handshake operations}
\noindent
\underline{$\scfgGen(sk_s)$:} \\
 $1.\ \ t_s \xleftarrow{\$} \Zset_{q}^{\ast}$ \\
 $2.\ \ T_s = g^{t_s}$ \\
 $3.\ \ \SCID = \Hash(T_s \| \expy)$ \\
 $4.\ \ str = \text{ QUIC server config signature }$ \\
 $5.\ \ \doc = str \| 0x00 \| \SCID \| T_s \| \expy$ \\
 $6.\ \ \sigma_s = \SIG.\Sign(sk_s, \doc)$ \\
 $7.\ \ \SCFG_{pub} = (\SCID, T_s, \expy, \sigma_s, \cert_s)$ \\
 $8.\ \ \return\ (\SCFG_{pub}, t_s)$ \\
\\
\underline{$\inchoateCHLO()$:} \\
 $1.\ \ \cid \xleftarrow{\$} \{0,1\}^{64} $ \\
 $2.\ \ \pInfo = (IP_c, IP_s, port_c, port_s)$ \\
 $3.\ \ \return\ (\pInfo, \cid)$ \\
\\
\underline{$\REJ(m, \SCFG_{pub})$:} \\
 $1.\ \ (\pInfo, \cid) = m$ \\
 $2.\ \ \STK = \makeSTK()$ \\
 $3.\ \ \pInfo = (IP_s, IP_c, port_s, port_c)$ \\
 $4.\ \ \return\ (\pInfo, \cid, \SCFG_{pub}, \STK)$ \\
\\
\underline{$\initialCHLO(m)$:} \\
 $1.\ \ (\pInfo, \cid, \SCFG_{pub}, \STK) = m$ \\
 $2.\ \ \pInfo = (IP_c, IP_s, port_c, port_s)$ \\
 $3.\ \ \checkSCFG(\SCFG_{pub})$ \\
 $4.\ \ t_c \xleftarrow{\$} \Zset_{q}^{\ast}$ \\
 $5.\ \ T_c = g^{t_c}$ \\
 $6.\ \ r \xleftarrow{\$} \{0,1\}^{160}$ \\
 $7.\ \ \NONC \leftarrow currentTime \| r$ \\
 $8.\ \ \return\ (\pInfo, \cid, \STK, \SCID, \NONC, T_c)$ \\
\\
\underline{$\SHLO(m, \ik, \sqn)$:} \\
 $1.\ \ (\pInfo, \cid, \STK, \SCID, \NONC, T_c) = m$ \\
 $2.\ \ (\ik_c, \ik_s, \iv_c, \iv_s) = \ik$ \\
 $3.\ \ t_s^{\prime} \xleftarrow{\$} \Zset_{q}^{\ast}$ \\
 $4.\ \ T_s^{\prime} = g^{t_s^{\prime}}$ \\
 $5.\ \ \STK = \makeSTK()$ \\
 $6.\ \ H = (\cid, \sqn)$ \\
 $7.\ \ \plaintext = \SCFG_{pub} \| T_s^{\prime} \| \STK $\\
 $8.\ \ c = \SE.\Enc(\ik_c, \iv_c \| \sqn, H, \plaintext)$ \\
 $9.\ \ \return\ (\pInfo, H, c)$ \\
\\
\underline{$\receiveSHLO(m, \ik)$:} \\
 $1.\ \ (\pInfo, H, c) = m$ \\
 $2.\ \ (\ik_c, \ik_s, \iv_c, \iv_s) = \ik$ \\
 $3.\ \ (\cid, \sqn) = H$ \\
 $4.\ \ \plaintext = \SE.\Dec(\ik_c, \iv_c \| \sqn, H, c)$ \\
 $5.\ \ \text{If }\plaintext = \perp$ \\
 $6.\ \ \quad \Lambda = \text{'reject' and abort}$ \\
 $7.\ \ \SCFG_{pub} \| T_s^{\prime} \| \STK  = \plaintext $ \\
 $8.\ \ \return\ T_s^{\prime}$ \\
\\
\underline{$\CHLO(\STK, \SCFG_{pub})$:} \\
 $1.\ \ \cid \xleftarrow{\$} \{0,1\}^{64}$ \\
 $2.\ \ r \xleftarrow{\$} \{0,1\}^{160}$ \\
 $3.\ \ \NONC \leftarrow currentTime \| r$ \\
 $4.\ \ t_c^{\ast} \xleftarrow{\$} \Zset_{q}^{\ast}$ \\
 $5.\ \ T_c^{\ast} = g^{t_c^{\ast}}$ \\
 $6.\ \ \pInfo = (IP_{src}, IP_{dst}, port_{src}, port_{dst})$ \\
 $7.\ \ \return\ (\pInfo, \cid, \STK, \SCID, \NONC, T_c^{\ast})$ \\
\\
\underline{$\makeSTK()$:} \\
 $1.\ \ \iv_{\STK} \xleftarrow{\$} \{0,1\}^{96}$ \\
 $2.\ \ \plaintext = IP_c \| currentTime$ \\
 $3.\ \ \STK \leftarrow \iv_{\STK} \| \SE.\Enc(k_{\STK}, len,\iv_{\STK},\plaintext)$
\\
 $4.\ \ \return\ \STK$ \\
\\
\underline{$\checkSCFG(\SCFG_{pub})$:} \\
 $1.\ \ (\SCID, T_s, \expy, \sigma_s, \cert_s) = \SCFG_{pub}$ \\
 $2.\ \ \text{If } \expy \leq currentTime$ \\
 $3.\ \ \quad \Lambda = \text{'reject' and abort}$ \\
 $4.\ \ pk_s = \getPK(cert_s)$ \\
 $5.\ \ \text{If } \SIG.\Vfy(pk_s, \sigma_s, \doc) = \perp$ \\
 $6.\ \ \quad \Lambda = \text{'reject' and abort}$ \\
\\
\underline{$\checkQuery(\STK, k_{\STK}, \NONC, IP_c)$:} \\
 $1.\ \ (\iv_{\STK}, c) = \STK$ \\
 $2.\ \ (IP_c^{\prime}, time_{\STK}) = \SE.\Dec(k_{\STK}, \iv_{\STK}, c)$ \\
 $3.\ \ (time_{\NONC}, r) = \NONC$ \\
 $4.\ \ \text{If } (IP_c^{\prime}, currentTime) = \perp$, or \\
 $5.\ \ \quad IP_c^{\prime} \neq IP_c$, or $time_{\STK} \leq time_{allowed}$\\
 $6.\ \ \quad r \in \strike$, or $time_{\NONC} \not\in \strike_{rng}$ \\
 $7.\ \ \quad \quad \Lambda = \text{'reject' and abort}$ \\
\\
\subsubsection{Key operations}
\noindent
\underline{$\getKey_c(\shareInfo, m, \init)$:} \\
 $1.\ \ (\NONC, \cid ,T_s, t_c) = \shareInfo$ \\
 $2.\ \ pms = T_s^{t_c}$ \\
 $3.\ \ \return\ \extractKey(pms, \NONC, \cid, m, 40, \init)$ \\
\\
\underline{$\getKey_s(\shareInfo, m, \init)$:} \\
 $1.\ \ (\NONC, \cid ,T_c, t_s) = \shareInfo$ \\
 $2.\ \ pms = T_c^{t_s}$ \\
 $3.\ \ \return\ \extractKey(pms, \NONC, \cid, m, 40, \init)$ \\
\\
\underline{$\extractKey(pms, \NONC, \cid, m, \ell, \init)$:}\\
 $1.\ \ ms = \PRF(pms, \NONC)$ \\
 $2.\ \ \text{If } \init = 1$ \\
 $3.\ \ \quad str = \text{ QUIC key expansion }$ \\
 $4.\ \ \text{Else }$ \\
 $5.\ \ \quad str = \text{ QUIC forward secure expansion }$ \\
 $6.\ \ \info = str \| 0x00 \| \cid \| m \| \SCFG_{pub}$ \\
 $7.\ \ \return\ \text{the first $\ell$ octets (i.e. bytes) of T = }$ \\
 $\quad \quad \text{(T(1),T(2), ...), where for all $i \in \Nset$, $T(i) = $} $\\
 $\quad \quad \text{$\PRF(ms, T(i-1) \| \info \| 0x0i)$ and $T(0) = \epsilon$} $\\
\\
\subsubsection{Packet operations}
\noindent
\underline{$\getIV(H, \kappa, P)$:} \\
 $1.\ \ (k_c, k_s, \iv_c, \iv_s) = \kappa$ \\
 $2.\ \ \text{If } P \in \Client$ \\
 $3.\ \ \quad src = c, dst = s$ \\
 $4.\ \ \text{Else if} P \in \Server$ \\
 $5.\ \ \quad src = s, dst = c$ \\
 $6.\ \ (\cid, \sqn) = H$ \\
 $7.\ \ \return\ (\iv_{dst}, \sqn)$ \\
\\
\underline{$\pak(\kappa, \sqn, m)$:} \\
 $1.\ \ (k_c, k_s, \iv_c, \iv_s) = \kappa$ \\
 $2.\ \ \text{If } P \in \Client$ \\
 $3.\ \ \quad src = c, dst = s$ \\
 $4.\ \ \text{Else if } P \in \Server$ \\
 $5.\ \ \quad src = s, dst = c$ \\
 $6.\ \ \pInfo = (IP_{src}, IP_{dst}, port_{src}, port_{dst})$ \\
 $7.\ \ H = (\cid, \sqn)$ \\
 $8.\ \ \iv = \getIV(H, \kappa)$ \\
 $9.\ \ \return\ (\pInfo, \SE.\Enc(k_{dst}, \iv, H, m) )$ \\
\\
\underline{$\processPacket(\kappa, p_1,...,p_v)$:} \\
 $1.\ \ (k_c, k_s, \iv_c, \iv_s) = \kappa$ \\
 $2.\ \ \text{If } P \in \Client$ \\
 $3.\ \ \quad src = c, dst = s$ \\
 $4.\ \ \text{Else if} P \in \Server$ \\
 $5.\ \ \quad src = s, dst = c$ \\
 $6.\ \ \text{for each } \gamma \in [v]$ \\
 $7.\ \ \quad (H_{\gamma}, c_{\gamma}) = p_{\gamma}$ \\
 $8.\ \ \quad \iv_{\gamma} = \getIV(H_{\gamma}, \kappa)$ \\
 $9.\ \ \quad m_{\gamma} = \SE.\Dec(k_{src}, \iv_{\gamma}, H_{\gamma}, c_{\gamma})$ \\

We note that QUIC is not RSACCE secure.
Although QUIC has a mechanism to prevent forgery,
whose mechanisms is called \textit{source-address token}
(see below), the adversary can forge an abbreviate handshake
query as follows: An adversary can obtain SCID from $\REJ$
response, which is the first response from a server oracle.
Then, the adversary make $(\overline{T}_c^{\prime},
\overline{\NONC}^{\prime})$ and send $(\overline{T}_c^{\prime},
\SCID, \overline{\NONC}^{\prime})$ to the server after a
full handshake between the server and its intended partner
is established.
The server cannot distinguish whether the query in resumption
comes from the intended partner or not. Because there is no
authentication mechanism in this query. The server accepts
this query and calculates the session key with the value
of adversary.
The adversary does not know the session key because
$T_s^{\ast}$ is encrypted however a session key between the
client and the server does not match.

\subsubsection{Source Address Token} \label{sec:source_address_token}
Source-address token ($\STK$) is introduced to QUIC in order to
prevent IP address spoofing.
A server generates and sends a new STK every time he sends a
message to a client.
The client updates STK when the client receives a new one from
the server and sends it back along with his message.
$\STK$ is an opaque byte string from the client's point of view.
From the server's point of view it's an authenticated-encryption
block that contains, at least, the client's IP address and a time
stamp by the server.
$\STK$ is encrypted except for the first server's query ($\REJ$).
However, in our model the adversary has full control over the
communication network and hence it can obtain $\STK$ and use it to
forge a future query.