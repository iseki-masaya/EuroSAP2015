%=====================================================
\section{Quick UDP Internet Connections} \label{sec:quic}
%=====================================================

Quick UDP Internet Connections (QUIC) is a protocol developed by Google. This protocol is still under development. The abstract model of QUIC is described in Fig.~\ref{fig:quic}. The concepts of QUIC are (1) to reduce connectivity overheads before a client sends encrypted data and (2) to obtain better security guarantee than TLS.
To realize concept (1), QUIC is not defined over TCP but UDP, because TCP requires three-move handshake before initiating a cryptographic handshake, but UDP does not have. In addition, a client can send encrypted data concurrently with CHLO.
To realize concept (2), QUIC supports only secure cipher suites. Especially, a support algorithm of key exchange is only ephemeral elliptic curve Diffie-Hellman.

Unlike TLS, QUIC does not support client authentication.

%=====================================================
\subsection{Security of QUIC} \label{sec:quic_detail}
%=====================================================

\input{04_z1_protocol_overview}
\input{04_z2_protocol_operations}

We note that QUIC is not RSACCE secure.
Although QUIC has a mechanism to prevent forgery,
whose mechanisms is called \textit{source-address token} (see below),
the adversary can forge an abbreviate handshake query as follows:
An adversary can obtain SCID from $\REJ$ response,
which is the first response from a server oracle.
Then, the adversary make $(\overline{T}_c^{\prime}, \overline{\NONC}^{\prime})$
and send $(\overline{T}_c^{\prime}, \SCID, \overline{\NONC}^{\prime})$ to the
server after a full handshake between the server and its intended partner is established.
The server cannot distinguish whether the query in resumption comes from
the intended partner or not. Because there is no authentication
mechanism in this query. The server accepts this query and calculates
the session key with the value of adversary. The adversary does not
know the session key because $T_s^{\ast}$ is encrypted however a
session key between the client and the server does not match.

\subsubsection{Source Address Token} \label{sec:source_address_token}
Source-address token ($\STK$) is introduced to QUIC in order to prevent IP address spoofing. A server generates and sends a new STK every time he sends a message to a client. The client updates STK when the client receives a new one from the server and sends it back along with his message. $\STK$ is an opaque byte string from the client's point of view. From the server's point of view it's an authenticated-encryption block that contains, at least, the client's IP address and a time stamp by the server. $\STK$ is encrypted except for the first server's query ($\REJ$). However, in our model the adversary has full control over the communication network and hence it can obtain $\STK$ and use it to forge a future query.