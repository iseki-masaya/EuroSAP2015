%=====================================================
\section{``Resumable" Server-only Authenticated and Confidential Channel Establishment (RSACCE)} \label{sec:rsacce}
%=====================================================
We consider the security of QUIC and its variants.
Our abstract model to capture the cryptographic core
of them is a server-only authenticated and confidential
channel establishment (SACCE) protocol~\cite{KPW13:SACCE},
the server-only authentication version of an ACCE
protocol~\cite{JKSS12:ACCE}.
In our SACCE protocol, a client initiates an instance
of the protocol and a server always sends the last
message of the instance.
We then extend the security notion of
SACCE~\cite{KPW13:SACCE} to the notion of
\textit{resumable} SACCE (RSACCE).
The main difference is that this model treats
abbreviate handshakes.
As mentioned above, an abbreviate handshake takes
advantage of a prior established full handshake session
between the same client and server.
To treat abbreviate handshakes in the server-only
authenticated setting, we should take into account the
attack
\textbf{where an adversarial client fools a
server and impersonates the initial client in an
abbreviate handshake session.}
Depending on the extent of the attack, we define two
security requirements, denoted
\textit{server authentications} and
\textit{channel confidentiality}.
The channel confidentiality implies that, in contrast
to SACCE, an adversary is allowed to submit an
encryption query
\textit{not only to a client but to a server} in an
abbreviate handshake session.
We note that if an adversary can make a handshake with
a server with the same session key in some abbreviate
handshake session, then it can indeed break message
confidentiality of a ciphertext sent by the server.

%=====================================================
\subsection{Execution Environment} \label{sec:exec_env_party}
%=====================================================
We basically borrow the notations
from~\cite{JKSS12:ACCE,KPW13:SACCE}.
We denote by $\Client$ and $\Server$ the set of honest
clients and servers, respectively.
We assume for simplicity that each client $\Client$
and server $\Server$ has a \textit{unique} identity
number $c \in \Nset$
\footnote{Since a client has no certified identity,
the numbering is just conceptual.}
and $s \in \Nset$.
For the case we don't need to specify clients or
servers, we also assume that each party
$P \in \Server \cup \Client$ has a \textit{unique}
identity number $p \in \Nset$.
We denoted by $P_p$ a party with identity number $p$.
In particular, $\Server_s$ has a unique key pair
$(\pk_s, \sk_s)$, along with a certificate
$\cert_s=(s,\pk_s)_{\text{CA}}$ signed by a certificate
authority CA.
A party $P_p$ maintains a collection of oracles
$\{\pi^p_{i,\ell }\}_{i,\ell}$ where oracle
$\pi^p_{i, \ell}$ models party $P_p$ executing a single
instance of a protocol in the $\ell$-th abbreviate
handshake session derived from  the $i$-th full
handshake session.
When $\ell=0$, $\pi^p_{i,\ell}$ refers to the $i$-th
full handshake session of $P_p$.
The oracle $\pi^p_{i, \ell}$ maintains as internal
state the following variables:

\begin{itemize}
 \item{$\Lambda \in \{\accept, \preaccept, \reject,
 \emptyset\}$, the state of a handshake.}

 \item{$\ik, \key \in \mathcal{K}$, the session key
 where $\mathcal{K}$ is the key space of the protocol.
 $\ik$ is called initial key which calculated with
 ephemeral client's Diffie-Hellman value and static
 server's Diffie-Hellman value, and $\key$ is called
 forward-secure key which calculated with ephemeral
 client's Diffie-Hellman value and ephemeral server's
 Diffie-Hellman value.}

 \item{$\peer$, the intended partner. If $P_p$ is
 $\Client_c$, then $\pi^c_{i,\ell}$ maintains the
 identity of intended partner $\peer \in \Server$.
 Otherwise (if $P_p$ is $\Server_s$), it maintains
 the identity of intended partner $\peer \in \Client$.}

 \item{$b$, the challenge bit chosen uniformly.}
\end{itemize}
The inner state of each oracle is initialized to
($\Lambda$, $\ik$, $\key$, $\peer$) = ($\emptyset$,
$\emptyset$, $\emptyset$, $\emptyset$), where
variable $V=\emptyset$ denotes that variable $V$
is undefined.
On one hand, and $b$ is chosen uniformly and fixed.

The adversary issues the following queries to the
oracles:
\begin{itemize}
 \item {$\Send(\pi^p_{i, \ell}, m)$:
 The adversary can use this query to send message
 $m$ to the oracle $\pi^p_{i, \ell}$.
 The oracle will respond with an outgoing message
 according to the protocol specification and its
 internal state.
 The oracle $\pi^p_{i, \ell}$ replies with $\bot$,
 either (a) if $\ell \geq 1$ and
 $\pi^p_{i, {\ell-1}}$ has state
 $\Lambda\not\in \{\accept, \preaccept\}$, or (b)
 if $\pi^p_{i,\ell}$ has reached state
 $\Lambda = \accept$.
 Otherwise, it does the following: If the adversary
 asks the first $\Send$-query to oracle
 $\pi^p_{i, \ell}$, then the oracle checks whether
 $m = \top$ consists of a special
 ``initiate client session'' symbol $\top$.
 If so, it responds with the first protocol message.
 If $\ell \geq 1$ and $\pi^p_{i,\ell-1}$ accepts,
 then $\pi^p_{i,\ell}$ \textit{inherits} $\peer$ from
 $\pi^p_{i, \ell-1}$ (modeling an abbreviate handshake
 session).
 The adversary can send this returned message to any
 oracle even if this oracle is not intended partner of
 the oracle $\pi^p_{i, \ell}$.}

 \item {$\Reveal(\pi^p_{i,\ell})$:
 The oracle $\pi^p_{i,\ell}$ returns session keys
 $\ik$, $\key$, and other secrets such as a master
 secret or source-address token to respond this query.}

 \item {$\Corrupt(P_p)$:
 If $P_p$ is $\Client_c$, then returns $\bot$.
 Otherwise, $P_p$ is $\Server_s$ and returns long-term
 secret key $\sk_s$ and other secrets which are kept
 for a long time such as static Diffie-Hellman secret
 value.
 If $P_p$ is $\Server_s$ and $\Corrupt(P_p)$ is the
 $\tau$-th query issued by the adversary, $P_p$ is said
 $\tau$-\textit{corrupted}.
 For parties that are not corrupted we define
 $\tau = \infty$.}

 \item {$\Encrypt(\pi^p_{i,\ell}, m_0, m_1, \iv, H)$:
 If $\pi^p_{i,\ell}$ has state
 $\Lambda$ $\not\in$ \\ $\{\preaccept, \accept\}$,
 it returns $\bot$.
 Otherwise, it makes a challenge ciphertext according to
 the procedure in Fig.~\ref{fig:LHAE_rsacce}.}

 \item {$\Decrypt(\pi^p_{i, \ell}, c, \iv, H)$:
 The oracle $\pi^p_{i, \ell}$ replies according to the
 procedure in Fig.~\ref{fig:LHAE_rsacce}.}
\end{itemize}

\begin{figure*}[!htb]
\begin{center}
\fbox{
\begin{minipage}[t]{0.42\textwidth}
\begin{tabular}[t]{l}
 $\Encrypt(\pi^p_{i,\ell},m_0,m_1,\iv,H)$: \\
 $\quad C_0 \xleftarrow{\$} \SE.\Enc(k, \iv, H,m_0)$ \\
 $\quad C_1 \xleftarrow{\$} \SE.\Enc(k, \iv, H,m_1)$ \\
 $\quad \text{If } C_0 = \perp \text{ or } C_1 = \perp$ \\
 $\quad \quad \text{return } \perp$ \\
 $\quad \mathcal{C} = \mathcal{C} \cup C_b$ \\
 $\quad \text{return } C_b$ \\
\end{tabular}
\end{minipage}
\vline \quad
\begin{minipage}[t]{0.42\textwidth}
\begin{tabular}[t]{l}
 $\Decrypt(\pi^p_{i,\ell},C,\iv,H)$: \\
 $\quad \text{If } b = 1 \wedge C \not\in \mathcal{C}$ \\
 $\quad \quad \text{return } \SE.\Dec(k,\iv,H,C)$ \\
 $\quad \text{return } \perp$ \\
\end{tabular}
\end{minipage}
}
\caption{Encrypt and Decrypt oracle in the RSACCE security experiment}
 \label{fig:LHAE_rsacce}
\end{center}
\end{figure*}

%=====================================================
\subsection{Security Definition} \label{sec:sec_def}
%=====================================================

We define the security model of resumable server-only
authenticated confidential channel establishment (RSACCE).

\subsubsection{Matching Conversations}
In our SACCE protocol, a client always initiates an
instance of the protocol and a server always sends the
last message of the instance.
We define matching conversations as in~\cite{JKSS12:ACCE}.
We denote by $T^p_{i,\ell}$ the transcript of
$\pi^p_{i,\ell}$, i.e., the history of all messages sent
and received by $\pi^p_{i,\ell}$ in chronological order
(not including the initialization-symbol $\top$).
For two transcripts, $T^p_{i,\ell}$ and
$T^{p^{\prime}}_{j,\ell'}$ we say that $T^p_{i,\ell}$
is a \textit{prefix} of $T^{p^{\prime}}_{j,\ell'}$ if
$T^p_{i,\ell}$ contains at least one message, and
$T^p_{i,\ell}$ is identical to
$T^{p^{\prime}}_{j,\ell'}$ except the last message sent
by $\pi^{p^{\prime}}_{j,\ell'}$.

\begin{definition}[Matching conversations]
 We say that $\pi^c_{i,\ell}$ and $\pi^s_{j,\ell'}$ have
 a matching conversation with each other if  in addition
 to $\ell=\ell'$,
 \begin{itemize}
  \item{Both oracles, $\pi^c_{i, \ell}$ and
  $\pi^s_{j,\ell'}$, accept ($\Lambda \in \{\accept, \preaccept\}$) and
  $T^c_{i,\ell} = T^s_{j,\ell'}$; or}

  \item{The server oracle $\pi^s_{j, \ell'}$ accepts
  ($\Lambda = \accept$),
  and $T^c_{i,\ell}$ is a prefix of $T^s_{j,\ell'}$.}
 \end{itemize}
\end{definition}
\begin{remark}
 The second condition is necessary because in our SACCE
 protocol a server accepts a session before sending the
 last message to a client and cannot know whether a
 client indeed receives it.
\end{remark}

\subsubsection{RSACCE Game}
We define the RSACCE game between an adversary $A$ and
a challenger.
In this game, the challenger firstly instantiates the
collection of oracles $\{\pi^p_{i,\ell}\}$.
Then the challenger generates the certificate's
signing/verification keys; generates long-term keys
$(\pk_s, \sk_s)$ for all servers; and issues the
certificates for all public keys.
The adversary receives all public keys $\pk_s$ with
identity $s$ as input.
Now the adversary may start by issuing $\Send$,
$\Reveal$, $\Corrupt$, $\Encrypt$ and $\Decrypt$ queries.
Finally, the adversary outputs
$(p, i, \ell, b^{\prime})$ and terminates.

\begin{definition}[Correctness]
 Assume a ``benign" adversary $A$, which picks two
 arbitrary oracles, $\pi^c_{i, \ell}$ and
 $\pi^s_{j, \ell^{\prime}}$, and performs a sequence of
 $\Send$-queries by faithfully forwarding all messages
 between $\pi^c_{i, \ell}$ and $\pi^s_{j, \ell^{\prime}}$.
 Let $\ik^c_{i, \ell}$ and $\key^c_{i, \ell}$ denote the
 key computed by $\pi^c_{i, \ell}$ and let
 $\ik^s_{j, \ell^{\prime}}$ and $\key^s_{j, \ell^{\prime}}$
 denote the key computed by $\pi^s_{j, \ell^{\prime}}$.
 We say that RSACCE protocol is \textit{correct}, if two
 arbitrary server and client oracles, $\pi^c_{i, \ell}$
 and $\pi^s_{j, \ell^{\prime}}$, always hold that:
 \begin{itemize}
  \item{Both oracles, $\pi^c_{i, \ell}$ and
  $\pi^s_{j, \ell^{\prime}}$, have
  \textit{a matching conversation} with each other; and}

  \item{Both initial keys, $\ik^c_{i, \ell}$ and
  $\ik^s_{j, \ell^{\prime}}$, are the same
  ($\ik^c_{i, \ell} = \ik^s_{j, \ell^{\prime}}$).}

  \item{Both forward-secure keys, $\key^c_{i, \ell}$ and
  $\key^s_{j, \ell^{\prime}}$, are the same
  ($\key^c_{i, \ell} = \key^s_{j, \ell^{\prime}}$).}
 \end{itemize}
\end{definition}

\subsubsection{Security Requirements}
We now define the following advantage measures.

\begin{definition}[Server Authentication] \label{def:rsacce-sa}
 $\Adv^{\rsaccesa}_{\Pi}$ (A) is the probability that when
 $A$ terminates, there is a client oracle $\pi^c_{i, \ell}$
 such that the following conditions hold:
 \begin{itemize}
  \item{$\pi^c_{i, \ell}$ accepts
  ($\Lambda \in \{\accept, \preaccept\}$) when $A$ issues its
  $\tau_0$-th query with intended partner $\peer=s$, }

  \item{$\Server_s$ is $\tau_{s}$-corrupted with
  $\tau_0 < \tau_{s}$ and}

  \item{There is no server oracle $\pi^s_{j, \ell}$ such
  that $\pi^c_{i,\ell}$ has a matching conversation with
  $\pi^s_{j,\ell}$ or there exist plural oracles that have
  a matching conversation with $\pi^c_{i,\ell}$.}
 \end{itemize}
 We say that protocol $\Pi$ has \textit{server authentication},
 if $\Adv^{\rsaccesa}_{\Pi}(A)$ is negligible in $\kappa$.
\end{definition}

\begin{remark}
 The \textit{server authentication} is a natural extension
 of the counter part of~\cite{KPW13:SACCE}, which implies
 that if a client accepts, protocol $\Pi$ guarantees that
 the client has a matching conversation with the intended
 parter (server). In contrast to the original
 definition~\cite{KPW13:SACCE}, we allow an adversary to
 submit $\Corrupt$ queries.
 In our definition, the second condition ensure the
 consistency of the server handshake sessions.
 If the adversary can issue $\Corrupt$ query to the server
 $\Server_s$ before $\pi^s_{j, \ell}$ reaches accept, she can
 impersonate the server because she can make valid response
 (SHLO) using secret values such as static server's
 Diffie-Hellman secret value or master secret.
 \end{remark}

\begin{definition}[Channel Confidentiality] \label{def:rsacce-cc}
 $\Adv^{\rsaccecc}_{\Pi}$ (A) is defined to be
 $x$ - $\frac{1}{2}$ where $x$ is the probability that
 the adversary $A$ outputs $(p, i, \ell, b^{\prime})$,
 with $0\leq \ell$, such that $b = b^{\prime}$ where
 $b^{\prime} \in \bits$ is set during the
 $\Encrypt(\pi^p_{i,\ell},\dots)$ query and we define
 $b=\bot$ unless the following conditions hold:
 \begin{itemize}
  \item{$\pi^p_{i,\ell}$ accepts
  ($\Lambda \in \{\accept, \preaccept\}$) when $A$
  issues its $\tau_0$-th query. (It implies by
  definition that for all $0\leq k < \ell$,
  $\pi^p_{i,k}$ has already accepted.)}

  \item{If $\pi^p_{i,\ell}$ is a client oracle, the
  intended pater $\Server_s$ is $\tau_s$-corrupted
  with $\tau_0 < \tau_s$. }

  \item{If $P_p$ is a server $\Server_s$, then there
  is a client $\Client_c$ maintaining oracle
  $\pi^c_{j,0}$ that has a matching conversation with
  $\pi^s_{i,0}$.}

  \item{The adversary does not issue a $\Reveal$ query
  to either $\pi^p_{i,\ell}$, nor to
  $\pi^{p^{\prime}}_{j,\ell}$ such that $\pi^p_{i,\ell}$
  has a matching conversation with
  $\pi^{p^{\prime}}_{j,\ell}$.}
 \end{itemize}
 We say that protocol $\Pi$ has
 \textit{channel confidentiality} if
 $\Adv^{\rsaccecc}_{\Pi}(A)$ is negligible in $\kappa$.
\end{definition}

\begin{remark}
We extend the security requirement of
\textit{channel confidentiality} of SACCE~\cite{KPW13:SACCE}.
The original security notion takes care of message
confidentiality of ciphertexts sent by a (honest)
client, only. It is because an adversary can trivially
play a role of an honest client and have a matching
conversation with a (honest) server with the same
session key. In our definition, the third condition
indicates that an adversary may send the $\Encrypt$
query to even \textit{server} $\Server_s$ (to break
message confidentiality of the challenge ciphertext
sent by \textit{server oracle} $\pi^s_{i,\ell}$) as
long as the initial full handshake is established
between $\Server_s$ and an \textit{honest} client
$\Client_c$ (not the adversary). Later, if the
adversary can hijack some abbreviate handshake session
and share a new session key with server $\Server_s$,
then it can break our channel confidentiality.
The second condition specifies forward-secrecy in the
sense that message confidentiality of $\pi^p_{i,\ell}$
is preserved, as long as the $\Server_s$ does not
corrupt their long-term secret.
\end{remark}

\subsubsection{RSACCE security}
We define the RSACCE security.

\begin{definition}[RSACCE secure]
 We say that the protocol $\Pi$ is RSACCE secure
 if $\Pi$ satisfies \textbf{correctness},
 \textbf{server authentication},
 \textbf{channel confidentiality}.
\end{definition}