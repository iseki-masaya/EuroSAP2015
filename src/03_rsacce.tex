%=====================================================
\section{``Resumable" Server-only Authenticated and Confidential Channel Establishment (RSACCE)} \label{sec:rsacce}
%=====================================================
\textcolor{blue}{
We consider the security of QUIC and its variants.
Our abstract model to capture the cryptographic core of them is a server-only
authenticated and confidential channel establishment (SACCE) protocol~\cite{KPW13:SACCE},
the server-only authentication version of an ACCE protocol~\cite{JKSS12:ACCE}.
In our SACCE protocol, a client initiates an instance of the protocol and
a server always sends the last message of the instance.
We then extend the security notion of SACCE~\cite{KPW13:SACCE}
to the notion of \textit{resumable} SACCE (RSACCE).
The main difference is that this model treats ``resumption".
As mentioned above, a resumption is an abbreviated handshake session
taking advantage of a prior established full handshake session
between the same client and server.
To treat resumptions in the server-only authenticated setting,
we should take into account the attack \textbf{where an adversarial client fools a server
and impersonates the initial client in a resumption session.}
Depending on the extent of the attack, we define two security requirements,
denoted \textit{channel confidentiality} and \textit{channel binding}.
The channel confidentiality implies that, in contrast to SACCE,
an adversary is allowed to submit
an encryption query \textit{not only to a client but to a server} in a resumption session.
We note that if an adversary can make a handshake with a server with the same session
key in some resumption session, then it can indeed break message confidentiality of
a ciphertext sent by the server.
We also consider channel binding in order to make a protocol robust,
in which a server should accept
only if a resumed session is started by the initial client
in the prior established full handshake session.
}



%--------------------------------------------------------
\subsection{Execution Environment} \label{sec:exec_env_party}
%--------------------------------------------------------
We basically borrow the notations from~\cite{JKSS12:ACCE,KPW13:SACCE}.
We denote by $\Client$ and $\Server$ the set of honest clients and servers, respectively.
We assume for simplicity that each client $\Client$ and server $\Server$ has a \textit{unique} identity number $c \in \Nset$~\footnote{Since a client has
no certified identity, the numbering is just conceptual.}
and $s \in \Nset$.
For the case we don't need to specify clients or servers, we also assume that each party $P \in \Server \cup \Client$ has a \textit{unique} identity number $p \in \Nset$.
%~\footnote{Since a client has
% no certified identity, the numbering is just conceptual.}.
We denoted by $P_p$ a party with identity number $p$.
In particular, $\Server_s$ has a unique key pair $(pk_s, sk_s)$, along with a certificate $cert_s=(s,pk_s)_{\text{CA}}$
signed by a certificate authority CA.
A party $P_p$ maintains a collection of oracles $\{\pi^p_{i,\ell }\}_{i,\ell}$
where oracle $\pi^p_{i, \ell}$ models party $P_p$ executing a single instance of a protocol in the $\ell$-th abbreviate handshake session
derived from  the $i$-th full handshake session.
When $\ell=0$, $\pi^p_{i,\ell}$ refers to the $i$-th full handshake session of $P_p$.
The oracle $\pi^p_{i, \ell}$ maintains as internal state the following variables:

\begin{itemize}
 \item{$\Lambda \in \{\accept, \preaccept, \reject, \emptyset\}$, the state of a handshake.}
 \item{$ik, k \in \mathcal{K}$, the session key where $\mathcal{K}$ is the key space of the protocol. $ik$ is a initial key and $k$ is a forward secure key.}
 \item{$\peer$, the intended partner. If $P_p$ is $\Client_c$, then $\pi^c_{i,\ell}$ maintains the identity of intended partener $\peer \in \Server$. Otherwise (if $P_p$ is $\Server_s$), it maintains $\peer:=SCID$ where $SCID$ is a random string chosen by $\pi^s_{i,\ell}$.}
 \item{$b$, the challenge bit chosen uniformly.}

% \item{$(b, C, u, v)$ are the variable to reply the adversary query. The detail about the adversary query is at (Sec.~\ref{sec:exec_env_adv}). $b \in \{0,1\}$ is a bit. $C$ is the array of ciphertexts. $u$ is the counter of Encrypt query, $v$ is the counter of Decrypt query.}
\end{itemize}
The inner state of each oracle is initialized to
 ( $\Lambda$ , ik, k, $\peer$) = ( $\emptyset$, $\emptyset$, $\emptyset$, $\emptyset$) ,
where variable $V=\emptyset$ denotes that variable $V$ is undefined.
On one hand, and
$b$ is chosen uniformly and fixed.

%--------------------------------------------------------
%\subsection{Execution Environment: Adversary} \label{sec:exec_env_adv}
%--------------------------------------------------------
The adversary issues the following queries to the oracles:
\begin{itemize}
 \item {$\Send(\pi^p_{i, \ell}, m)$:
    The adversary can use this query to send message $m$ to oracle $\pi^p_{i, \ell}$.
    The oracle will respond with an outgoing message according
    to the protocol specification and its internal state.
    The oracle $\pi^p_{i, \ell}$ replies with $\bot$, either
    (a) if $\ell \geq 1$ and $\pi^p_{i, {\ell-1}}$ has state $\Lambda\neq\accept$,
    or
    (b) if $\pi^p_{i,\ell}$ has reached state $\Lambda=\accept$.
    Otherwise, it does the following:
    If the adversary asks the first $\Send$-query to oracle $\pi^p_{i, \ell}$,
    then the oracle checks whether $m = \top$ consists of a special ``initiate client session'' symbol $\top$.
    If so, it responds with the first protocol message.
    If $\ell \geq 1$ and $\pi^p_{i,\ell-1}$ accepts, then $\pi^p_{i,\ell}$ \textit{inherits} variables
    $(\theta,\peer)$
    from $\pi^p_{i, \ell-1}$ (modeling a resumption session).
    The adversary can send this returned message to any oracle even if this oracle is not intended partner of the oracle $\pi^p_{i, \ell}$.}

 \item {$\Reveal(\pi^p_{i,\ell})$ :
    The oracle $\pi^p_{i,\ell}$ returns key $k$ and cache data $\theta$
    to respond this query. }

 \item {$\Corrupt(P_p)$:
    If $P_p$ is $\Client_c$, then returns $\bot$. Otherwise, $P_p$ is $\Server_s$ and returns
    long-term secret key $sk_s$.
    If $P_p$ is $\Server_s$ and $\Corrupt(P_p)$ is the $\tau$-th query issued by the adversary,
    $P_p$ is said $\tau$-\textit{corrupted}.
    For parties that are not corrupted we define $\tau = \infty$.}

 \item {$\Encrypt(\pi^p_{i,\ell}, m_0, m_1, len, H)$:
    If $\pi^p_{i,\ell}$ has state
    $\Lambda$ $\not\in$ \\ $\{\preaccept, \accept\}$, it returns $\bot$.
    Otherwise, it makes a challenge ciphertext according to the procedure in Fig.~\ref{fig:LHAE_rsacce}.}

 \item {$\Decrypt(\pi^p_{i, \ell}, c, H)$:
    The oracle $\pi^p_{i, \ell}$ replies according to the procedure in Fig.~\ref{fig:LHAE_rsacce}.}
\end{itemize}

\begin{figure*}[!htb]
\begin{center}
\fbox{
\begin{minipage}[t]{0.42\textwidth}
\begin{tabular}[t]{l}
 $\Encrypt(\pi^p_{i,\ell},m_0,m_1,len,H)$: \\
 $\quad C_0 \xleftarrow{\$} \SE.\Enc(k, len, H,m_0)$ \\
 $\quad C_1 \xleftarrow{\$} \SE.\Enc(k, len, H,m_1)$ \\
 $\quad \text{If } C_0 = \perp \text{ or } C_1 = \perp$ \\
 $\quad \quad \text{return } \perp$ \\
 $\quad \mathcal{C} = \mathcal{C} \cup C_b$ \\
 $\quad \text{return } C_b$ \\
\end{tabular}
\end{minipage}
\vline \quad
\begin{minipage}[t]{0.42\textwidth}
\begin{tabular}[t]{l}
 $\Decrypt(\pi^p_{i,\ell},C,H)$: \\
 $\quad \text{If } b = 1 \wedge C \not\in \mathcal{C}$ \\
 $\quad \quad \text{return } \SE.\Dec(k,H,C)$ \\
 $\quad \text{return } \perp$ \\
\end{tabular}
\end{minipage}
}
\caption{Encrypt and Decrypt oracle in the RSACCE security experiment}
 \label{fig:LHAE_rsacce}
\end{center}
\end{figure*}


 In this model, $\Reveal$ query exposes not only session key $k$ but also secret cache data $\theta$.

%--------------------------------------------------------
\subsection{Security Definition} \label{sec:sec_def}
%--------------------------------------------------------

We define the security model of resumable server-only authenticated confidential channel establishment
(RSACCE).


\subsubsection{Matching Conversations}
In our SACCE protocol, a client always initiates an instance of the protocol and
a server always sends the last message of the instance.
We define matching conversations as in~\cite{JKSS12:ACCE}.
%Let $P_i$ be a client and $P_j$ be a server.
We denote by $T^p_{i,\ell}$ the transcript of $\pi^p_{i,\ell}$, i.e., the history of
all messages sent and received by $\pi^p_{i,\ell}$ in chronological order (not including the initialization-symbol $\top$).
For two transcripts, $T^p_{i,\ell}$ and $T^{p^{\prime}}_{j,\ell'}$
We say that $T^p_{i,\ell}$ is a \textit{prefix} of $T^{p^{\prime}}_{j,\ell'}$ if
$T^p_{i,\ell}$ contains at least one message, and
$T^p_{i,\ell}$ is identical to $T^{p^{\prime}}_{j,\ell'}$ except the last message sent
by $\pi^{p^{\prime}}_{j,\ell'}$.

\begin{definition}[Matching conversations]
 We say that $\pi^c_{i,\ell}$ and $\pi^s_{j,\ell'}$ have a matching conversation with each other if  in addition to $\ell=\ell'$,
 \begin{itemize}
  \item{Both oracles, $\pi^c_{i, \ell}$ and $\pi^s_{j,\ell'}$, accept and
  $T^c_{i,\ell} = T^s_{j,\ell'}$; or }
  \item{The server oracle $\pi^s_{j, \ell'}$ accepts,
  and $T^c_{i,\ell}$ is a prefix of $T^s_{j,\ell'}$.}
 \end{itemize}
\end{definition}
\begin{remark}
 The second condition is necessary because in our SACCE protocol
 a server accepts a session before sending the last message to a client
 and cannot know whether a client indeed receives it.
\end{remark}

\subsubsection{RSACCE Game}
We define the RSACCE game between an adversary $A$ and a challenger.
In this game, the challenger firstly instantiates the collection of oracles
$\{\pi^p_{i,\ell}\}$.
Then the challenger generates the certificate's signing/verification keys;
generates long-term keys $(pk_s, sk_s)$ for all servers;
and issues the certificates for all public keys.
The adversary receives all public keys $pk_s$ with identity $s$ as input.
Now the adversary may start by issuing $\Send$, $\Reveal$, $\Corrupt$, $\Encrypt$ and $\Decrypt$ queries.
Finally, the adversary outputs $(p, i, \ell, b^{\prime})$ and terminates.

\begin{definition}[Correctness]
 Assume a ``benign" adversary $A$, which picks two arbitrary oracles, $\pi^c_{i, \ell}$ and $\pi^s_{j, \ell^{\prime}}$,
 and performs a sequence of $\Send$-queries by faithfully forwarding all messages
 between $\pi^c_{i, \ell}$ and $\pi^s_{j, \ell^{\prime}}$.
 Let $k^c_{i, \ell}$ denote the key computed by $\pi^c_{i, \ell}$
 and let $k^s_{j, \ell^{\prime}}$ denote the key computed by $\pi^s_{j, \ell^{\prime}}$.
 We say that RSACCE protocol is \textit{correct}, if two arbitrary server and client oracles, $\pi^c_{i, \ell}$ and $\pi^s_{j, \ell^{\prime}}$, always hold that:
 \begin{itemize}
  \item{Both oracles, $\pi^c_{i, \ell}$ and $\pi^s_{j, \ell^{\prime}}$, have \textit{a matching conversation} with each other; and}
  \item{Both session keys, $k^c_{i, \ell}$ and $k^s_{j, \ell^{\prime}}$, are the same ($k^c_{i, \ell} = k^s_{j, \ell^{\prime}}$).}
 \end{itemize}
\end{definition}

\subsubsection{Security Requirements}
We now define the following advantage measures.

\begin{definition}[Server Authentication] \label{def:rsacce-sa}
 $\Adv^{\rsaccesa}_{\Pi}$ (A) is the probability that when $A$ terminates, there is
 a client oracle $\pi^c_{i, \ell}$ such that the following conditions hold:
 \begin{itemize}
  \item{$\pi^c_{i, \ell}$ accepts when $A$ issues its $\tau_0$-th query with intended partner $\peer=s$, }
  \item{$\Server_s$ is $\tau_{s}$-corrupted with $\tau_0 < \tau_{s}$ and}
  \item{There is no server oracle $\pi^s_{j, \ell}$ such that $\pi^c_{i,\ell}$ has a matching conversation
  with $\pi^s_{j,\ell}$ or there exist plural oracles that have a matching conversation with $\pi^c_{i,\ell}$.}
 \end{itemize}
 We say that protocol $\Pi$ has \textit{server authentication}, if
 $\Adv^{\rsaccesa}_{\Pi}(A)$ is negligible in $\kappa$.
\end{definition}

\begin{remark}
 The \textit{server authentication}  is a natural extension of the counter part of~\cite{KPW13:SACCE},
 which implies that if a client accepts, protocol $\Pi$ guarantees that the client has a matching conversation with
 the intended parter (server). In contrast to the original definition~\cite{KPW13:SACCE},
 we allow an adversary to submit $\Corrupt$ queries.
 \end{remark}

\begin{definition}[Channel Confidentiality] \label{def:rsacce-cc}
 $\Adv^{\rsaccecc}_{\Pi}$ (A) is defined to be $x$ - $\frac{1}{2}$ where $x$ is the probability that the adversary $A$ outputs $(p, i, \ell, b^{\prime})$, with $0\leq \ell$,
 such that $b = b^{\prime}$ where $b^{\prime} \in \bits$ is set during the $\Encrypt(\pi^p_{i,\ell},\dots)$ query and we define $b=\bot$ unless the following conditions hold:
 \begin{itemize}
%  \item{$\pi_{s,0}^i$ reaches state $\Lambda=\accept$.}
  \item{$\pi^p_{i,\ell}$ reaches state $\Lambda=\preaccept$ when $A$ issues
  its $\tau_0$-th query. (It implies by definition that for all $0\leq k < \ell$,
  $\pi^p_{i,k}$ has already accepted.)}

  \item{If $\pi^p_{i,\ell}$ is a client oracle,
  the intended pater $\Server_s$ is $\tau_s$-corrupted with $\tau_0 < \tau_s$. }

  \item{If $P_p$ is a server $\Server_s$, then
  there is a client $\Client_c$ maintaining oracle $\pi^c_{j,0}$ that has a matching
  conversation with $\pi^s_{i,0}$.}

  \item{For all $0\leq k \leq \ell$, the adversary does not issue a $\Reveal$ query to
  $\pi^p_{i,k}$, nor to $\pi^{p^{\prime}}_{j,k}$ such that $\pi^p_{i,k}$ has a matching conversation
  with $\pi^{p^{\prime}}_{j,k}$.}

 \end{itemize}
 We say that protocol $\Pi$ has \textit{channel confidentiality}
 if $\Adv^{\rsaccecc}_{\Pi}(A)$ is negligible in $\kappa$.
\end{definition}

\begin{remark}
We extend the security requirement of \textit{channel confidentiality} of SACCE~\cite{KPW13:SACCE}.
The original security notion takes care of message confidentiality of ciphertexts
sent by a (honest) client, only.  It is because an adversary can trivially play a role of an honest client and
have a matching conversation with a (honest) server with the same session key.
In our definition, the third condition indicates that an adversary may send
the $\Encrypt$ query to even \textit{server} $\Server_s$
(to break message confidentiality of the challenge ciphertext sent by \textit{server oracle} $\pi^s_{i,\ell}$)
as long as the initial full handshake is established between
$\Server_s$ and an \textit{honest} client $\Client_c$ (not the adversary).
Later, if the adversary can hijack some resumption session and
share a new session key with server $\Server_s$,
then it can break our channel confidentiality.
The forth condition specifies forward-secrecy in the sense that message confidentiality of $\pi^p_{i,\ell}$
is preserved, as long as the prior related oracles (including itself), $\pi^p_{i,k}$,
with $0\leq k\leq \ell$, or their peer oracles do not reveal their cached data and session keys.
\end{remark}

We now present an additional advantage measure.

\begin{definition}[Channel Binding] \label{def:rsacce-cb}
 $\Adv^{\rsaccecb}_{\Pi}$ (A) is the probability that when $A$ terminates, there exists
 a server oracle $\pi^s_{j, \ell}$, with $0< \ell$, such that the following conditions hold:

 \begin{itemize}
  \item{$\pi^s_{j,\ell}$ reaches state $\Lambda=\accept$ when $A$ issues
  its $\tau_0$-th query. (It implies by definition that for all $0\leq k < \ell$,
  $\pi^s_{j,k}$ has already accepted.)}

  \item{There is a client $\Client_c$ maintaining an oracle $\pi^c_{i,0}$ with $\peer=s$
  that has a matching conversation with $\pi^s_{j,0}$.}

  \item{For all $0\leq k \leq \ell$, the adversary does not issue a $\Reveal$ query either to
  $\pi^c_{i,k}$ nor to $\pi^s_{j,k}$,
  such that $\pi^c_{i,k}$ has a matching conversation with $\pi^s_{j,k}$.}

  \item{Client oracle $\pi^c_{i, \ell}$  does not have a matching conversation
  with $\pi^s_{j,\ell}$.}
%  \item{Client oracle $\pi^j_{t, \ell}$ has state $\peer \neq i$.}

 \end{itemize}
 We say that protocol $\Pi$ has \textit{channel confidentiality}
 if $\Adv^{\rsaccecc}_{\Pi}$ (A) is negligible in $\kappa$.
\end{definition}

\begin{remark}
Channel binding is a security property to protect something like client authentication.
While clients do not have certificates, one would need a guarantee that if a server accepts
in a resumption session, the server should have a matching conversation only with the client
in the initially established full handshake session.
We note that to break channel binding, it is unnecessary for an adversary to share a session key with a server
(if so, it breaks channel confidentiality).
If an adversary breaks channel binding, then the client and server could not resume a new resumption session.
\end{remark}

\subsubsection{RSACCE security}
We define the two levels of RSACCE security.

\begin{definition}[RSACCE secure]
 We say that the protocol $\Pi$ is RSACCE secure if $\Pi$ satisfies \textbf{correctness},
 \textbf{server authentication}, and \textbf{channel confidentiality}.
\end{definition}

\begin{definition}[Strong RSACCE secure]
 We say that the protocol $P$ is \textbf{strong} RSACCE secure if $\Pi$ is RSACCE secure
 and also satisfies \textbf{channel binding}.
\end{definition}